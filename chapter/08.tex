\chapter{定积分的应用}
定积分作为微积分的核心工具,在数学、物理、工程、经济和生物等多个领域有广泛应用。它通过累积微小变化来计算总量,体现了极限思想的实际价值。

在几何方面,定积分常用于计算不规则形状的几何量,如弧长、面积、体积和曲面面积。

在物理中,定积分用于计算连续变化的量,如功、力矩、质心和流体静力。

在工程学、经济学、 生物学与医学等方面都有广泛的应用。



\section{曲线的弧长}
曲线的弧长,简单来说,就是曲线上两点之间那段路径的精确长度。

要计算平面曲线 $y = f(x)$ 从 $x=a$ 到 $x=b$ 之间的弧长 $s$,我们需要使用定积分。

\begin{figure}[htbp] 
    \centering
    \includegraphics[width=0.6\textwidth]{images/08/8.1.png} 
    \caption{\textbf{弧长微分}}
\end{figure}

\textbf{弧长公式的推导}

(1)区间微分:
\begin{align*}
    \delta &= \frac{b-a}{\infty} = (b-a)\epsilon \\
    dx &= (x+\delta) - x = \delta \\
    dy &= df(x) = f'(x) \,dx
\end{align*}

(2)弧长微分 $ds$:根据勾股定理
\begin{align*}
    d^2 s &= d^2 x + d^2 y = d^2 x + [f'(x)]^2 \cdot d^2 x \\
    ds &= \sqrt{d^2 x + [f'(x)]^2 \cdot d^2 x} = \sqrt{1 + [f'(x)]^2 } \,dx
\end{align*}

(3)定积分求和:将这些微小的弧长 $ds$ 从 $x=a$ 累加到 $x=b$,得到总弧长 $s$
\begin{align}
    s = \int_{a}^{b} \,ds &= \int_{a}^{b} \sqrt{1 + [f'(x)]^2} \,dx \\
    &= \int_{a}^{b} \sqrt{1 + \left(\frac{dy}{dx}\right)^2} \,dx
\end{align}

\medskip

\textbf{曲线 $x = g(y)$ 的弧长}

类似地,如果曲线由函数 $x = g(y)$ 定义,且 $g'(y)$ 在闭区间 $[c, d]$ 上连续,则从 $y=c$ 到 $y=d$ 的弧长 $s$ 为:
\begin{align}
    s &= \int_{c}^{d} \sqrt{1 + [g'(y)]^2} \,dy \\
    s &= \int_{c}^{d} \sqrt{1 + \left(\frac{dx}{dy}\right)^2} \,dy
\end{align}

\medskip

\textbf{弧长计算步骤}

确定曲线的函数形式($y=f(x)$ 或 $x=g(y)$),并求出它的导数 $\frac{dy}{dx}$ 或 $\frac{dx}{dy}$。

将导数代入相应的弧长积分公式。

确定积分的上下限($x$ 或 $y$ 的范围)。

计算定积分以得出最终的弧长值。

\medskip

\begin{example} 计算曲线 $y = \frac{2}{3}x^{3/2}$ 从 $x=0$ 到 $x=3$ 的弧长。

    求导:$$\frac{dy}{dx} = \frac{2}{3} \cdot \frac{3}{2} x^{3/2 - 1} = x^{1/2} = \sqrt{x}$$

    代入公式:
    \begin{align*}
        \left(\frac{dy}{dx}\right)^2 &= (\sqrt{x})^2 = x \\
        \sqrt{1 + \left(\frac{dy}{dx}\right)^2} &= \sqrt{1 + x}
    \end{align*}

    设置限值并构建积分:$$s = \int_{0}^{3} \sqrt{1 + x} \,dx$$

    计算积分:
    \begin{align*}
        \int_{0}^{3} \sqrt{1 + x} \,dx &= \int_{0}^{3} (1 + x)^{\frac 1 2} \,d(1+x) \\
        &= \left[\frac{1}{1 + 1/2}(1 + x)^{1 + 1/2} \right]_0^3\\
        &= \left[\frac{2}{3}(1 + x)^{\frac 3 2} \right]_0^3 \\
        &= \frac{2}{3}(1 + 3)^{\frac 3 2} - \frac{2}{3}(1 + 0)^{\frac 3 2} \\
        &= \frac{2}{3} \cdot 8 - \frac{2}{3} =\frac{14}{3}
    \end{align*}

    因此,该曲线在 $x=0$ 到 $x=3$ 之间的弧长是 $\frac{14}{3}$。
\end{example}

弧长计算的难点通常在于积分的计算,因为 $\sqrt{1 + [f'(x)]^2}$ 形式的被积函数往往会产生难以求出反导数的表达式。

\medskip

\textbf{弧长函数}

弧长函数定义为从曲线上的固定起点 $A(a, f(a))$ 到曲线上任意一点 $P(x, f(x))$ 的弧长。

它是一个以 $x$ 为变量的变限积分函数:
\begin{align}
    s(x) = \int_{a}^{x} \sqrt{1 + [f'(t)]^2} \,dt
\end{align}

其中:

$a$ 是起点的横坐标。

$x$ 是终点的横坐标(也是函数的变量)。

$t$ 是积分中的哑变量(dummy variable)。

\medskip

弧长函数最主要的用途之一是作为曲线的自然参数。当曲线的参数是其弧长 $s$ 时,曲线的参数化被称为弧长参数化。在这种参数化下,曲线的切向量的模(长度)恒为 1,这极大地简化了微分几何中的许多公式和计算,例如曲率和挠率的计算。

在物理学中,弧长函数可以用来确定一个物体沿着给定路径(曲线)移动的距离。

\medskip

\textbf{参数方程表示的曲线弧长}

假设一条曲线由参数方程定义:
$$\begin{cases} x = x(t) \\ y = y(t) \end{cases} \quad (\alpha \le t \le \beta)$$

那么
\begin{align*}
    d^2 s &= d^2 x + d^2 y \\
    d^2 s &= d^2 x(t) + d^2 y(t) \\
    &= [x'(t)]^2 \cdot d^2 t + [y'(t)]^2 \cdot d^2 t \\
    \\
    ds &= \sqrt{[x'(t)]^2 + [y'(t)]^2} \,dt
\end{align*}

那么该曲线从 $t=\alpha$ 到 $t=\beta$ 的弧长 $s$ 可以用以下定积分表示:
\begin{align}
    s = \int_{\alpha}^{\beta} \sqrt{[x'(t)]^2 + [y'(t)]^2} \, dt
\end{align}

或者写成更常见的形式:
\begin{align}
    s = \int_{\alpha}^{\beta} \sqrt{\left(\frac{dx}{dt}\right)^2 + \left(\frac{dy}{dt}\right)^2} \, dt
\end{align}

\medskip

\begin{example} 求半径为 $R$ 的圆的周长:圆的参数方程为:$$\begin{cases} x = R\cos t \\ y = R\sin t \end{cases} \quad (0 \le t \le 2\pi)$$
    
    求导:$\frac{dx}{dt} = -R\sin t$,$\frac{dy}{dt} = R\cos t$

    圆的周长:
    \begin{align*}
        &\sqrt{(-R\sin t)^2 + (R\cos t)^2} = \sqrt{R^2(\sin^2 t + \cos^2 t)} = R \\
        \\
        &s = \int_{0}^{2\pi} R \, dt = Rt\Bigg |_{0}^{2\pi} = 2\pi R
    \end{align*}
\end{example}

在实际计算中,被积函数 $\sqrt{[x'(t)]^2 + [y'(t)]^2}$ 往往带有根号,计算可能会比较复杂,有时需要用到三角代换或特殊的积分技巧。


\section{曲线之间的面积}
如果想计算两条曲线 $y = f(x)$ 和 $y = g(x)$ 在区间 $x=a$ 到 $x=b$ 之间所围成的面积 $A$,且在这个区间内 $f(x) \ge g(x)$(即 $f(x)$ 的曲线始终在 $g(x)$ 的曲线上方或与它相交),如图所示

\begin{figure}[htbp] 
    \centering
    \includegraphics[width=0.4\textwidth]{images/08/8.2.png} 
    \caption{\textbf{曲线之间的面积}}
\end{figure}

那么这个面积 $A$ 可以通过以下定积分来计算:
\begin{align*}
    dA_1 &= f(x) \,dx \\
    dA_2 &= g(x) \,dx \\
    dA &= dA_1 - dA_2 = f(x) \,dx - g(x) \,dx \\
    dA &= [f(x) - g(x)] \,dx \\
    \text{等式两边进行区间积分:} &\\
    A &= \int_a^b [f(x) - g(x)] \,dx
\end{align*}

\medskip

\textbf{面积计算步骤}

(1)找到交点(确定上下限 $a$ 和 $b$):

将两个函数 $f(x)$ 和 $g(x)$ 设置相等,$f(x) = g(x)$,解出 $x$ 的值。这些 $x$ 值就是曲线的交点,它们通常作为积分的上下限 $a$ 和 $b$。如果问题已经给出了区间,则使用给定的区间端点。

(2)确定哪个函数在上方:

在积分区间 $(a, b)$ 内,选择一个测试点 $c$,计算 $f(c)$ 和 $g(c)$ 的值。值较大的函数就是上方的函数 $f_{top}(x)$,值较小的函数是下方的函数 $f_{bottom}(x)$。

如果在整个区间内,两条曲线有交叉(即哪个函数在上方发生了变化),您需要将积分分成多个部分,并在每个部分重新确定上下函数。面积 $A$ 总是 $\int |f(x) - g(x)| \, dx$。

(3)设置并计算定积分:

使用正确的上方函数 $f_{top}(x)$ 和下方函数 $f_{bottom}(x)$ 设置积分
\begin{align}
    A = \int_{a}^{b} [f_{top}(x) - f_{bottom}(x)] \, dx
\end{align}

然后计算这个定积分,得到面积。

\medskip

\begin{example} 计算曲线 $y = x^2$ 和 $y = x$ 之间的面积。

    \begin{figure}[htbp] 
        \centering
        \includegraphics[width=0.4\textwidth]{images/08/8.3.png} 
        \caption{\textbf{面积计算}}
    \end{figure}

    找交点:$x^2 = x \iff x = 0,\; x = 1$。因此,$a = 0$, $b = 1$。

    确定上下函数:
    
    $(0.5)^2 < (0.5)$,所以 $y = x$ 是上方的函数 ($f_{top}(x) = x$),而 $y = x^2$ 是下方的函数 ($f_{bottom}(x) = x^2$)。

    计算面积:
    \begin{align*}
        A &= \int_{0}^{1} (x - x^2) \, dx \\
        &= \left[\frac{1}{2}x^2 - \frac{1}{3}x^3 \right]_0^1 = \left(\frac{1}{2} - \frac{1}{3} \right) \\
        &= \frac 1 6
    \end{align*}
    曲线之间的面积是 $1/6$ 平方单位。
\end{example}

\medskip

在某些情况下,如果曲线方程是 $x$ 作为 $y$ 的函数(即 $x = f(y)$ 和 $x = g(y)$),或者如果用垂直矩形(对 $x$ 积分)计算面积会要求将积分分解成多段,那么使用水平矩形(对 $y$ 积分)会更简单。
$$A = \int_{c}^{d} [f(y) - g(y)] \, dy$$

$f(y)$: 右侧的函数(曲线)。

$g(y)$: 左侧的函数(曲线)。

$d$ 和 $c$: 积分的上下限,$y$ 的值。

\medskip

\begin{example} 计算由抛物线 $x = 4 - y^2$ 和直线 $x = y$ 所围成的面积。

    \begin{figure}[htbp] 
        \centering
        \includegraphics[width=0.6\textwidth]{images/08/8.4.png} 
        \caption{\textbf{水平矩形}}
    \end{figure}

    找到交点(确定上下限 $c$ 和 $d$):
    
    将两个函数的 $x$ 设置相等,以找到它们相交的 $y$ 值:
    $$4 - y^2 = y $$ $$y^2 + y - 4 = 0$$

    使用求根公式 $y = \frac{-b \pm \sqrt{b^2 - 4ac}}{2a}$:
    \begin{align*}
        y &= \frac{-1 \pm \sqrt{1^2 - 4(1)(-4)}}{2(1)} \\
        &= \frac{-1 \pm \sqrt{1 + 16}}{2} \\
        &= \frac{-1 \pm \sqrt{17}}{2}
    \end{align*}
    所以,积分的下限 $c = \frac{-1 - \sqrt{17}}{2}$ (约 $-2.56$),上限 $d = \frac{-1 + \sqrt{17}}{2}$ (约 $1.56$)。

    确定哪个函数在右侧($f_{right}(y)$):
    
    在 $y$ 的区间 $(c, d)$ 内,选择一个测试点,例如 $y=0$。
    
    对于 $x = 4 - y^2$ (抛物线):$x(0) = 4 - 0^2 = 4$
    
    对于 $x = y$ (直线):$x(0) = 0$

    设置并计算定积分:
    \begin{align*}
        A &= \int_{c}^{d} [f_{right}(y) - f_{left}(y)] \, dy \\
        &= \int_{c}^{d} [(4 - y^2) - y] \, dy \\
        &= \int_{c}^{d} (4 - y - y^2) \, dy \\
        &= \left[ 4y - \frac{y^2}{2} - \frac{y^3}{3} \right]_{c}^{d}
    \end{align*}

    这个具体的代入计算会非常复杂,但我们可以利用公式来简化。对于二次方程 $ay^2+by+c=0$ 所围成的面积,如果 $y_1$ 和 $y_2$ 是它的两个根,则面积 $A$ 为:
    $$A = \frac{|\Delta|^{3/2}}{6a^2}$$

    在这个例子中,被积函数是 $-(y^2 + y - 4)$。我们的原始二次方程是 $y^2 + y - 4 = 0$。这里 $a=1, b=1, c=-4$。判别式 $\Delta = b^2 - 4ac = 1^2 - 4(1)(-4) = 17$。

    因此,面积 $A$ 为:$$A = \frac{(17)^{3/2}}{6(1)^2} = \frac{17\sqrt{17}}{6}$$
\end{example}

\medskip

\textbf{椭圆的面积}

考虑中心在原点的标准椭圆方程:$$\frac{x^2}{a^2} + \frac{y^2}{b^2} = 1$$其中 $a$ 是长半轴, $b$ 是短半轴。

\begin{figure}[htbp] 
    \centering
    \includegraphics[width=0.6\textwidth]{images/08/8.5.png} 
    \caption{\textbf{椭圆的面积}}
\end{figure}

椭圆在四个象限内是完全对称的。因此,我们只需要计算第一象限(即 $x$ 从 $0$ 到 $a$)的面积,然后乘以 $4$ 即可。

在第二章我们学习到椭圆的参数方程
$$\begin{cases} x = a \cos \theta \\ y = b \sin \theta \end{cases} \quad (\theta \in [0, 2\pi])$$

在第一象限,注意到 $x$ 从 $0$ 到 $a$ 时,$\theta$ 由 $\frac \pi 2$ 变化到 $0$。

椭圆的面积:
\begin{align*}
    dA &= y \,dx = b\sin\theta \,da\cos\theta \\
    &= b\sin\theta \cdot (-a\sin\theta \,d\theta) = -ab\sin^2 \theta \,d\theta \\
    A &= 4\int_{\frac \pi 2}^0 (-ab\sin^2 \theta) \,d\theta = 4ab \int_0^{\frac \pi 2} \sin^2 \theta \,d\theta \\
    &= 4ab  \int_0^{\frac \pi 2} \frac{1 - \cos 2\theta}{2} \,d\theta \\
    &= 2ab \int_0^{\frac \pi 2} \,d\theta - ab \int_0^{\frac \pi 2} \cos 2\theta \,d2\theta \\
    &= \pi ab
\end{align*}

椭圆面积公式:$S = \pi ab$

如果 $a = b = r$(圆的情况),公式变为 $S = \pi r^2$,这验证了公式的准确性。

直观理解:椭圆可以看作是将半径为 $a$ 的圆在 $y$ 轴方向上压缩(或拉伸)了 $b/a$ 倍。


\section{极坐标下的面积和弧长}
极坐标系(Polar Coordinate System)是一种不同于我们常用的直角坐标系(笛卡尔坐标系)的定位方式。它不使用“左右、上下”来定位,而是通过“距离和角度”来确定点的位置。

这种坐标系在处理圆形、螺旋线或者与中心点相关的运动(如雷达、卫星轨道)时非常高效。

\medskip

\textbf{1. 基本概念}

在极坐标系中,平面上的任何一点 $P$ 都可以用一对坐标 $(r, \theta)$ 来表示:

极点 (Pole): 相当于直角坐标系的原点 $(0,0)$。

极轴 (Polar Axis): 从极点引出的一条水平射线,通常指向右侧(相当于 $x$ 轴的正方向)。

极径 ($r$): 点 $P$ 到极点的距离。

极角 ($\theta$): 极轴按逆时针方向旋转到极径 $OP$ 所成的角度。

\medskip

\textbf{2. 坐标表示法}

通常极角 $\theta$ 使用弧度 (Radians) 表示,但在初学者教程中也常用角度 (°)。

逆时针旋转:角度为正。

顺时针旋转:角度为负。

示例:

$(3, 45^\circ)$ 表示距离中心 3 个单位,角度为 45 度。

$(2, \pi)$ 表示距离中心 2 个单位,角度为 180 度。


\medskip

\textbf{3. 极坐标与直角坐标的转换}

如果需要在两种系统之间切换,可以使用以下三角函数公式:

从极坐标 $(r, \theta)$ 转为直角坐标 $(x, y)$:
\begin{align}
    x &= r \cos(\theta) \\
    y &= r \sin(\theta)
\end{align}

从直角坐标 $(x, y)$ 转为极坐标 $(r, \theta)$:
\begin{align}
    r &= \sqrt{x^2 + y^2} \\
    \theta &= \arctan\left(\frac{y}{x}\right)
\end{align}

计算 $\theta$ 时需根据 $(x, y)$ 所在的象限调整角度。

\medskip

\textbf{4. 常见的极坐标方程图形}

极坐标能用非常简单的公式画出极其复杂的曲线:
$$
\begin{array}{lll}
    \text{方程类型} &\text{公式示例} &\text{图形特征} \\
    \hline
    \text{圆} & r = a &\text{以原点为圆心,半径为 $a$ 的圆} \\
    \text{玫瑰线} & r = a \sin(n\theta) &\text{像花瓣一样的图形,$n$ 决定花瓣数量} \\
    \text{阿基米德螺旋线} & r = a \theta &\text{随着角度增大,距离匀速增加的螺旋} \\
    \text{心形线} & r = a(1 - \sin\theta) &\text{形状像一颗心} \\
    \hline    
\end{array}
$$

在计算某些圆形的面积或重积分时,使用极坐标可以极大减少计算的复杂度。

\medskip

\textbf{5. 面积公式}

在极坐标系中推导面积公式,其核心思想与直角坐标系下的定积分一致:将不规则图形切割成无数个微小的“基本单元”,求和后再取极限。

在直角坐标系中,我们使用的是“小矩形”;而在极坐标系中,最基本的单元是“小扇形”。

\medskip

(1)基本单元:扇形的面积

首先,我们需要回顾几何学中扇形的面积公式。对于一个半径为 $r$,圆心角为 $\theta$(弧度制)的扇形,其面积 $A$ 为:
$$A = \frac{1}{2}r^2\theta$$

假设我们要计算由曲线 $r = f(\theta)$ 以及射线 $\theta = \alpha$ 和 $\theta = \beta$ 所围成的图形面积。

(2)区间微分

将角度区间 $[\alpha, \beta]$ 等分成 $\infty$ 个小区间,微分小扇形的面积 $dA$:

\begin{align*}
    \delta &= \frac{\beta - \alpha}{\infty} = (\beta - \alpha) \epsilon \\
    d\theta &= (\theta + \delta) - \theta = \theta - (\theta - \delta) = \delta \\
    dA &= \frac{1}{2} f^2(\theta) \,d\theta
\end{align*}

(3)区间积分

将所有小扇形的面积相加,得到总面积 $A$:
\begin{align*}
    A = \sum_{n = 1}^\infty \frac{1}{2} f^2(\alpha + n\delta) \,d\theta = \int_\alpha^\beta \frac{1}{2} f^2(\theta) \,d\theta 
\end{align*}

(3)极坐标下面积公式

极坐标下曲线 $r = f(\theta)$ 围成的面积公式为:
\begin{align}
    A = \frac{1}{2} \int_\alpha^\beta f^2(\theta) \,d\theta = \frac{1}{2} \int_\alpha^\beta r^2 \,d\theta
\end{align}

\medskip

\begin{example} 计算心形线 $r = a(1 + \cos\theta)$ 的面积

    由于心形线在 $\theta$ 从 $0$ 变到 $2\pi$ 时刚好绕原点一周,形成一个封闭图形,因此积分限为:
    
    下限 $\alpha = 0$
    
    上限 $\beta = 2\pi$

    观察心形线的图形可以发现它关于极轴($x$轴)对称。因此,我们可以先计算上半部分($0$ 到 $\pi$),再将结果乘以 $2$。

    根据公式 $A = \frac{1}{2} \int_{\alpha}^{\beta} r^2 \, d\theta$,代入 $r = a(1 + \cos\theta)$:
    \begin{align*}
        A &= \frac{1}{2} \int_{0}^{2\pi} [a(1 + \cos\theta)]^2 \, d\theta \\
        &= \frac{a^2}{2} \int_{0}^{2\pi} (1 + 2\cos\theta + \cos^2\theta) \, d\theta
    \end{align*}

    用三角恒等式降次:$\cos^2\theta = \frac{1 + \cos(2\theta)}{2}$:
    \begin{align*}
        A &= \frac{a^2}{2} \int_{0}^{2\pi} \left( 1 + 2\cos\theta + \frac{1 + \cos(2\theta)}{2} \right) \, d\theta \\
        &= \frac{a^2}{2} \int_{0}^{2\pi} \left( \frac{3}{2} + 2\cos\theta + \frac{1}{2}\cos(2\theta) \right) \, d\theta
    \end{align*}

    现在逐项积分:
    \begin{align*}
        \int_{0}^{2\pi} \frac{3}{2} \, d\theta &= \frac{3}{2}\theta \,\Bigg |_0^{2\pi} = 3\pi \\
        \int_{0}^{2\pi} 2\cos\theta \, d\theta &= 2\sin\theta \,\Bigg |_0^{2\pi} = 0 \\
        \int_{0}^{2\pi} \frac{1}{2}\cos(2\theta) \, d\theta &= \frac{1}{4}\int_{0}^{2\pi} \cos(2\theta) \, d2\theta = \frac{1}{4}\sin(2\theta) \,\Bigg |_0^{2\pi} = 0
    \end{align*}

    将上述结果加总并乘以系数:$$A = \frac{a^2}{2} \cdot (3\pi + 0 + 0) = \frac{3}{2}\pi a^2$$

    心形线 $r = a(1 + \cos\theta)$ 所围成的面积为 $\frac{3}{2}\pi a^2$。
\end{example}

\medskip

\textbf{6. 弧长公式}

在直角坐标系中,微元长度 $ds = \sqrt{dx^2 + dy^2}$。在极坐标系中,我们有转换关系:
$$x = r \cos\theta, \quad y = r \sin\theta$$

其中 $r$ 是 $\theta$ 的函数 $r = f(\theta)$。对 $x, y$ 关于 $\theta$ 求导:
\begin{align*}
    x &= r \cos\theta, \quad y = r \sin\theta \\
    dx &= dr\cos\theta = \cos\theta \,dr + r \,d\cos\theta \\
    &= \cos\theta \,dr - r\sin\theta \,d\theta \\
    dy &= dr\sin\theta = \sin\theta \,dr + r \,d\sin\theta \\
    &= \sin\theta \,dr + r\cos\theta \,d\theta \\
    d^2 x + d^2 y &= (\cos\theta \,dr - r\sin\theta \,d\theta)^2 + (\sin\theta \,dr + r\cos\theta \,d\theta)^2 \\
    &= d^2 r + r^2 \,d^2 \theta \\
    &= \left[r^2 +\left(\frac{dr}{d\theta} \right)^2 \right] \cdot d^2 \theta \\
    ds &= \sqrt{dx^2 + dy^2} = \sqrt{r^2 + \left(\frac{dr}{d\theta}\right)^2} \, d\theta
\end{align*}

如果曲线 $r = f(\theta)$ 在区间 $[\alpha, \beta]$ 上连续可导,则其弧长 $s$ 为:
\begin{align}
    s = \int_{\alpha}^{\beta} \sqrt{r^2 + \left(\frac{dr}{d\theta}\right)^2} \, d\theta = \int_{\alpha}^{\beta} \sqrt{r^2 + (r')^2} d\theta
\end{align}

\medskip

\begin{example} 计算心形线 $r = a(1 + \cos\theta)$ 的周长。
    \begin{align*}
        \frac{dr}{d\theta} &= -a\sin\theta \\
        r^2 + \left(\frac{dr}{d\theta}\right)^2 &= [a(1+\cos\theta)]^2 + (-a\sin\theta)^2 \\
        &= a^2(1 + 2\cos\theta + \cos^2\theta + \sin^2\theta) \\
        &= a^2(2 + 2\cos\theta) = 2a^2(1 + \cos\theta)
    \end{align*}

    利用半角公式 $1 + \cos\theta = 2\cos^2\frac{\theta}{2}$:
    $$\sqrt{r^2 + \left(\frac{dr}{d\theta}\right)^2} = \sqrt{4a^2\cos^2\frac{\theta}{2}} = 2a\left|\cos\frac{\theta}{2}\right|$$

    考虑到对称性,我们计算上半部分($0$ 到 $\pi$)再乘以 $2$。在 $0 \le \theta \le \pi$ 时,$\cos\frac{\theta}{2} \ge 0$,所以:
    \begin{align*}
        s &= 2 \int_{0}^{\pi} 2a\cos\frac{\theta}{2} \, d\theta \\
        &= 4a \cdot \left[ 2\sin\frac{\theta}{2} \right]_0^{\pi} \\
        &= 4a \cdot (2 - 0) = 8a
    \end{align*}

    心形线 $r = a(1+\cos\theta)$ 的总弧长(周长)为 $8a$。
\end{example}


\section{旋转体的体积}

旋转体的体积,简单来说,就是将一个平面图形绕着某条轴旋转 $360^{\circ}$,求所形成的几何体的体积。

最常用的方法有两种:磁盘法(Disk Method) 和 外壳法(Shell Method)。

\medskip

\textbf{1. 磁盘法}

这种方法适用于切片垂直于旋转轴的情况。可以把旋转体想象成由无数个极薄的圆柱体(圆盘)堆叠而成。

如果曲线 $y = f(x)$ 绕 $x$ 轴旋转,从 $x=a$ 到 $x=b$:
\begin{align}
    V = \int_{a}^{b} \pi f^2(x) \, dx
\end{align}

每个圆盘的半径是 $R = f(x)$,圆盘面积是 $A = \pi R^2$,厚度是 $dx$。

如果是绕 $y$ 轴旋转: 公式变为
\begin{align}
    V = \int_{c}^{d} \pi g^2(x) \, dy
\end{align}

当平面图形是由两条曲线围成,且旋转轴不在图形边缘时,中间会产生空心。这就像是在大圆盘中间挖掉了一个小圆盘。
\begin{align}
V = \int_{a}^{b} \pi \left[f^2(x) - g^2(x) \right] \, dx
\end{align}

$R$: 外半径(距离旋转轴较远的函数)。

$r$: 内半径(距离旋转轴较近的函数)。

\medskip

\begin{example} 计算由曲线 $y = \sin(x)$ 在区间 $[0, \pi/2]$ 上,以及 $x = 0$($y$ 轴)和 $y = 1$ 围成的图形,绕 $x$ 轴旋转一周所得旋转体的体积。

    因为绕 $x$ 轴旋转,切片垂直于 $x$ 轴,变量为 $x$。

    这个图形在旋转时,上方由 $y=1$ 限制,下方由 $y=\sin(x)$ 限制。
    
    外半径 $R(x)$: $1$
    
    内半径 $r(x)$: $\sin(x)$
    
    区间: $x \in [0, \pi/2]$

    代入公式计算体积:
    \begin{align*}
        V &= \int_{0}^{\pi/2} \pi (1^2 - \sin^2 x) \, dx = \pi \int_{0}^{\pi/2} \cos^2 x \, dx \\
        &= \frac{\pi}{2} \int_{0}^{\pi/2} (1 + \cos(2x)) \, dx \\
        &= \frac{\pi}{2} \left[ x + \frac{1}{2}\sin(2x) \right]_{0}^{\pi/2} \\
        &= \frac{\pi}{2} \cdot \frac{\pi}{2} = \frac{\pi^2}{4}
    \end{align*}
\end{example}

\medskip

\textbf{2. 圆柱外壳法}

这种方法适用于切片平行于旋转轴的情况。可以把几何体想象成由一层层“洋葱皮”(圆柱壳)嵌套而成。

如果曲线 $y = f(x)$ 绕 $y$ 轴旋转:
\begin{align}
    V = \int_{a}^{b} 2\pi x f(x) \, dx
\end{align}

每个外壳的半径是 $x$,高度是 $f(x)$,展开后的面积是 $2\pi \cdot \text{radius} \cdot \text{height}$。\

当函数 $y = f(x)$ 很难反解成 $x = g(y)$ 时,绕 $y$ 轴旋转用外壳法通常更容易。

\medskip

\begin{example} 求曲线 $y = 3x - x^2$ 与 $x$ 轴围成的区域,绕 $y$ 轴旋转一周所得的体积。

    分析图形
    
    边界: 这是一个开口向下的抛物线,交 $x$ 轴于 $(0, 0)$ 和 $(3, 0)$。
    
    旋转轴: $y$ 轴。

    在外壳法中,想象我们在图形内部取一条垂直于 $x$ 轴的“细条”进行旋转。
    
    半径 (Radius): 细条到旋转轴($y$ 轴)的距离,即 $r = x$。
    
    高度 (Height): 细条的高度,即曲线的纵坐标,$h = y = 3x - x^2$。
    
    积分区间: 从图形的左端点 $x=0$ 到右端点 $x=3$。

    代入公式计算体积:
    \begin{align*}
        V &= \int_{a}^{b} 2\pi x f(x) \, dx \\
        &= \int_{0}^{3} 2\pi \cdot x \cdot (3x - x^2) \, dx = 2\pi \int_{0}^{3} (3x^2 - x^3) \, dx \\
        &= 2\pi \left[ x^3 - \frac{1}{4}x^4 \right]_{0}^{3} \\
        &= 2\pi \left( (3^3 - \frac{1}{4} \cdot 3^4) - 0 \right) = 2\pi \left( 27 - \frac{81}{4} \right) \\
        &= 2\pi \left( \frac{108 - 81}{4} \right) = 2\pi \cdot \frac{27}{4} \\
        & = \frac{27\pi}{2}
    \end{align*}
\end{example}

旋转轴是 $y$ 轴,但函数是 $y = f(x)$: 这种“交叉配合”通常意味着外壳法更简单。
    
\medskip

\textbf{3. 磁盘法和外壳法的比较}

在数学本质上,它们是同一种思想的不同表现形式;但在计算实践中,它们是“切蛋糕”的两种不同刀法。

磁盘法:

切法: 垂直于旋转轴切。

微元形状: 像薄圆片(或带孔的垫圈)。

外壳法:

切法: 平行于旋转轴切。

微元形状: 像薄圆柱壳(类似洋葱圈或俄罗斯套娃)。

\medskip

\begin{example} 计算由曲线 $y = x^2$、$x = 1$ 和 $y = 0$(即 $x$ 轴)围成的图形,绕 $y$ 轴旋转一周所得旋转体的体积。

    方法 1:磁盘法

    在磁盘法中,切片是垂直于旋转轴($y$ 轴)的,所以切片是水平的,变量必须为 $y$。
    
    因为图形绕 $y$ 轴旋转,且左侧有空隙,这会形成一个“带孔的垫圈”:

    外半径 $R$: 固定的,最右边是 $x = 1$。
    
    内半径 $r$: 由曲线决定。既然 $y = x^2$,那么 $x = \sqrt{y}$。所以内半径 $r = \sqrt{y}$。
    
    积分区间: 从 $y = 0$ 到 $y = 1$(当 $x=1$ 时,$y=1^2=1$)。

    代入公式计算体积:
    \begin{align*}
        V &= \int_{0}^{1} \pi (R^2 - r^2) \, dy \\
        &= \int_{0}^{1} \pi (1^2 - (\sqrt{y})^2) \, dy = \pi \int_{0}^{1} (1 - y) \, dy \\
        &= \pi \left[ y - \frac{1}{2}y^2 \right]_{0}^{1} = \pi (1 - \frac{1}{2}) = \frac{\pi}{2}
    \end{align*}

    方法 2:圆柱外壳法

    在外壳法中,切片是平行于旋转轴($y$ 轴)的,所以切片是垂直的,变量为 $x$。
    
    半径 $r$: 离旋转轴的距离,即 $x$。
    
    高度 $h$: 函数的高度,即 $f(x) = x^2$。
    
    积分区间: 从 $x = 0$ 到 $x = 1$。

    代入公式计算体积:
    \begin{align*}
        V &= \int_{a}^{b} 2\pi x f(x) \, dx \\
        &= \int_{0}^{1} 2\pi \cdot x \cdot x^2 \, dx = 2\pi \int_{0}^{1} x^3 \, dx \\
        &= 2\pi \left[ \frac{1}{4}x^4 \right]_{0}^{1} = 2\pi \cdot \frac{1}{4} = \frac{\pi}{2}
    \end{align*}    
\end{example}