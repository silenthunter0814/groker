\chapter{定积分的应用}
定积分作为微积分的核心工具,在数学、物理、工程、经济和生物等多个领域有广泛应用。它通过累积微小变化来计算总量,体现了极限思想的实际价值。

在几何方面,定积分常用于计算不规则形状的几何量,如弧长、面积、体积和曲面面积。

在物理中,定积分用于计算连续变化的量,如功、力矩、质心和流体静力。

在工程学、经济学、 生物学与医学等方面都有广泛的应用。



\section{曲线的弧长}
曲线的弧长,简单来说,就是曲线上两点之间那段路径的精确长度。

要计算平面曲线 $y = f(x)$ 从 $x=a$ 到 $x=b$ 之间的弧长 $L$,我们需要使用定积分。

\begin{figure}[htbp] 
    \centering
    \includegraphics[width=0.6\textwidth]{images/07/7.3.png} 
    \caption{\textbf{弧长微分}}
\end{figure}

\textbf{弧长公式的推导}

(1)区间微分:
\begin{align*}
    \delta &= \frac{b-a}{\infty} = (b-a)\epsilon \\
    dx &= (x+\delta) - x = \delta \\
    dy &= df(x) = f'(x) \,dx
\end{align*}

(2)弧长微分 $ds$:根据勾股定理
\begin{align*}
    d^2 s &= d^2 x + d^2 y = d^2 x + [f'(x)]^2 \cdot d^2 x \\
    ds &= \sqrt{d^2 x + [f'(x)]^2 \cdot d^2 x} = \sqrt{1 + [f'(x)]^2 } \,dx
\end{align*}

(3)定积分求和:将这些微小的弧长 $ds$ 从 $x=a$ 累加到 $x=b$,得到总弧长 $L$
\begin{align}
    L = \int_{a}^{b} \,ds &= \int_{a}^{b} \sqrt{1 + [f'(x)]^2} \,dx \\
    &= \int_{a}^{b} \sqrt{1 + \left(\frac{dy}{dx}\right)^2} \,dx
\end{align}

\medskip

\textbf{曲线 $x = g(y)$ 的弧长}

类似地,如果曲线由函数 $x = g(y)$ 定义,且 $g'(y)$ 在闭区间 $[c, d]$ 上连续,则从 $y=c$ 到 $y=d$ 的弧长 $L$ 为:
\begin{align}
    L &= \int_{c}^{d} \sqrt{1 + [g'(y)]^2} \,dy \\
    L &= \int_{c}^{d} \sqrt{1 + \left(\frac{dx}{dy}\right)^2} \,dy
\end{align}

\medskip

\textbf{弧长计算步骤}

确定曲线的函数形式($y=f(x)$ 或 $x=g(y)$),并求出它的导数 $\frac{dy}{dx}$ 或 $\frac{dx}{dy}$。

将导数代入相应的弧长积分公式。

确定积分的上下限($x$ 或 $y$ 的范围)。

计算定积分以得出最终的弧长值。

\medskip

\begin{example} 计算曲线 $y = \frac{2}{3}x^{3/2}$ 从 $x=0$ 到 $x=3$ 的弧长。

    求导:$$\frac{dy}{dx} = \frac{2}{3} \cdot \frac{3}{2} x^{3/2 - 1} = x^{1/2} = \sqrt{x}$$

    代入公式:
    \begin{align*}
        \left(\frac{dy}{dx}\right)^2 &= (\sqrt{x})^2 = x \\
        \sqrt{1 + \left(\frac{dy}{dx}\right)^2} &= \sqrt{1 + x}
    \end{align*}

    设置限值并构建积分:$$L = \int_{0}^{3} \sqrt{1 + x} \,dx$$

    计算积分:
    \begin{align*}
        \int_{0}^{3} \sqrt{1 + x} \,dx &= \int_{0}^{3} (1 + x)^{\frac 1 2} \,d(1+x) \\
        &= \left[\frac{1}{1 + 1/2}(1 + x)^{1 + 1/2} \right]_0^3\\
        &= \left[\frac{2}{3}(1 + x)^{\frac 3 2} \right]_0^3 \\
        &= \frac{2}{3}(1 + 3)^{\frac 3 2} - \frac{2}{3}(1 + 0)^{\frac 3 2} \\
        &= \frac{2}{3} \cdot 8 - \frac{2}{3} =\frac{14}{3}
    \end{align*}

    因此,该曲线在 $x=0$ 到 $x=3$ 之间的弧长是 $\frac{14}{3}$。
\end{example}

弧长计算的难点通常在于积分的计算,因为 $\sqrt{1 + [f'(x)]^2}$ 形式的被积函数往往会产生难以求出反导数的表达式。

\medskip

\textbf{弧长函数}

弧长函数定义为从曲线上的固定起点 $A(a, f(a))$ 到曲线上任意一点 $P(x, f(x))$ 的弧长。

它是一个以 $x$ 为变量的变限积分函数:
\begin{align}
    s(x) = \int_{a}^{x} \sqrt{1 + [f'(t)]^2} \,dt
\end{align}

其中:

$a$ 是起点的横坐标。

$x$ 是终点的横坐标(也是函数的变量)。

$t$ 是积分中的哑变量(dummy variable)。

\medskip

弧长函数最主要的用途之一是作为曲线的自然参数。当曲线的参数是其弧长 $s$ 时,曲线的参数化被称为弧长参数化。在这种参数化下,曲线的切向量的模(长度)恒为 1,这极大地简化了微分几何中的许多公式和计算,例如曲率和挠率的计算。

在物理学中,弧长函数可以用来确定一个物体沿着给定路径(曲线)移动的距离。



\section{导数的定义与几何意义}
函数 $f(x)$ 在点 $x_0$ 处的导数定义为:
$$ f'(x_0) = \lim_{\Delta x \to 0} \frac{f(x_0 + \Delta x) - f(x_0)}{\Delta x} $$
导数表示函数在某点变化率的精确度量。