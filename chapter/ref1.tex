\chapter{多变量函数}

二元函数的图像(曲面): 建立 3D 空间感。

偏导数: 仅作为“固定其他方向、只看一个方向”的变化率。

全微分(线性近似): 强调 $df \approx f_x dx + f_y dy$。这是积分应用中 $dA$、$ds$ 元素分解的理论根基。

针对“积分应用”的极简前置清单

A. 投影与区域描述 (Domain Description)

重积分最难的不是积分本身,而是确定积分限。

裁剪重点: 教会读者如何把空间中的曲面投影到 $xy$ 平面上,并用不等式描述这个区域。

前置意义: 这是二重积分、三重积分的直接基础。

B. 局部线性化 (Local Linearity)

裁剪重点: 不要讲二阶导数判别法,重点讲“切平面”的近似。

前置意义: 在计算曲面积分时,我们需要把弯曲的表面切割成无数个微小的“小平面($dS$)”。如果读者理解了全微分,理解 $dS = \sqrt{1+f_x^2+f_y^2} dA$ 就会容易得多。



\section{从平面到空间:定义域与可视化}

核心目标: 建立 3D 空间感,学会“描述区域”。

关键内容:二元函数的直观理解:$z = f(x, y)$ 是空间中的一张曲面。

等高线 (Level Curves): 重点讲解,因为它是后续理解“梯度垂直于等值线”的直观来源。

区域描述(核心必备): 学习用不等式描述 $xy$ 平面上的区域(如圆、矩形、三角形),这是重积分定限的直接前置。

\medskip

在单变量微积分中,我们研究的是 $y = f(x)$,它在平面上是一条线。当我们引入第二个变量 $y$,研究 $z = f(x, y)$ 时,数学世界从“线”跃迁到了“面”。

1. 二元函数:空间中的“地形图”

想象你正站在一座山上,你所处的海拔高度 $z$ 取决于你所在的经度 $x$ 和纬度 $y$。

单变量函数: 像是一条过山车的轨道。

二元函数: 则是整座山的表面。

每一个坐标点 $(x, y)$ 都对应唯一的高度 $z$。这种直观的物理对应,是我们理解后续所有偏导数和积分应用的基础。

当我们谈论 $z = f(x, y)$ 时,你可以把它想象成在 $xy$ 平面(地面)上每一个点 $(x, y)$ 处,都垂直竖起一根高度为 $z$ 的柱子。所有这些柱子的顶端连在一起,就形成了一张曲面。


2. 定义域:积分的“地基”

在单变量微积分里,定义域通常只是 $x$ 轴上的一段区间 $[a, b]$。但在多变量微积分里,定义域变成了 $xy$ 平面上的一块区域 (Region)。

用数学不等式准确地描述这块区域:

矩形区域: $a \le x \le b, c \le y \le d$。

圆盘区域: $x^2 + y^2 \le R^2$。

一般区域: 例如由两条曲线 $y=g(x)$ 和 $y=h(x)$ 围成的部分。

3. 等高线 (Level Curves)

理解 3D 曲面最有效的手段是“等高线”——就像地理地图上标注海拔高度的线一样。

令 $f(x, y) = C$(常数),在 $xy$ 平面上画出的曲线就是等高线。

密集的等高线意味着山势陡峭(变化率大,即梯度大)。

稀疏的等高线意味着地势平缓。



\section{偏导数:固定维度的变化}

核心目标: 降维理解变化率。

关键内容:

偏导的计算:把不相关的变量看作常数(算法层)。

几何意义: 切线的斜率。只需强调:偏导数只关心沿坐标轴方向的变化。

裁剪掉: 高阶导数、克莱罗定理(除非涉及极值判定)。


\section{全微分与切平面}

核心目标: 为积分中的“微元法”和向量分析中的“切平面”打底。

关键内容:

全微分公式: $dz = f_x dx + f_y dy = \parital f(x) + \parital f(y)$。

线性近似的直观: “以直代曲”。解释为什么在极小尺度下,曲面可以看作平面。

切平面方程: 为后续向量分析中寻找“法向量”埋下伏笔。

\section{复合函数与雅可比初步}

核心目标: 解决积分中变元替换的“倍率”问题。

关键内容:

链式法则(路径图法): 侧重于 $z=f(x,y)$ 且 $x(t), y(t)$ 的情况,这与向量路径相关。

坐标转换的缩放感: 介绍从 $(x,y)$ 到 $(r, \theta)$ 的转换,不讲严谨推导,只讲为什么需要一个“修正系数”(即雅可比行列式的几何直观)。