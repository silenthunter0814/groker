\chapter{多变量函数}

二元函数的图像(曲面): 建立 3D 空间感。

偏导数: 仅作为“固定其他方向、只看一个方向”的变化率。

全微分(线性近似): 强调 $df \approx f_x dx + f_y dy$。这是积分应用中 $dA$、$ds$ 元素分解的理论根基。

针对“积分应用”的极简前置清单

A. 投影与区域描述 (Domain Description)

重积分最难的不是积分本身,而是确定积分限。

裁剪重点: 教会读者如何把空间中的曲面投影到 $xy$ 平面上,并用不等式描述这个区域。

前置意义: 这是二重积分、三重积分的直接基础。

B. 局部线性化 (Local Linearity)

裁剪重点: 不要讲二阶导数判别法,重点讲“切平面”的近似。

前置意义: 在计算曲面积分时,我们需要把弯曲的表面切割成无数个微小的“小平面($dS$)”。如果读者理解了全微分,理解 $dS = \sqrt{1+f_x^2+f_y^2} dA$ 就会容易得多。



\section{从平面到空间:定义域与可视化}

核心目标: 建立 3D 空间感,学会“描述区域”。

关键内容:二元函数的直观理解:$z = f(x, y)$ 是空间中的一张曲面。

等高线 (Level Curves): 重点讲解,因为它是后续理解“梯度垂直于等值线”的直观来源。

区域描述(核心必备): 学习用不等式描述 $xy$ 平面上的区域(如圆、矩形、三角形),这是重积分定限的直接前置。

\medskip

在单变量微积分中,我们研究的是 $y = f(x)$,它在平面上是一条线。当我们引入第二个变量 $y$,研究 $z = f(x, y)$ 时,数学世界从“线”跃迁到了“面”。

1. 二元函数:空间中的“地形图”

想象你正站在一座山上,你所处的海拔高度 $z$ 取决于你所在的经度 $x$ 和纬度 $y$。

单变量函数: 像是一条过山车的轨道。

二元函数: 则是整座山的表面。

每一个坐标点 $(x, y)$ 都对应唯一的高度 $z$。这种直观的物理对应,是我们理解后续所有偏导数和积分应用的基础。

当我们谈论 $z = f(x, y)$ 时,你可以把它想象成在 $xy$ 平面(地面)上每一个点 $(x, y)$ 处,都垂直竖起一根高度为 $z$ 的柱子。所有这些柱子的顶端连在一起,就形成了一张曲面。

例子 A:平面 (The Plane) —— 坡度恒定的斜坡

方程: $z = ax + by + c$

视觉特征: 就像一张斜铺在空中的无限大的硬纸板。

教学意义: 这是最简单的地形。无论你在哪一点,坡度都是恒定的。它是以后理解切平面(全微分)的基石。

积分关联: 如果我们要对这个平面进行积分,其实就是在求一个棱柱体的体积。

例子 B:抛物面 (The Paraboloid) —— 碗状山谷或山峰

方程: $z = x^2 + y^2$

视觉特征: 一个完美的圆底大碗。

特征分析:在原点 $(0, 0)$ 处,它是平坦的(导数为 0)。

离原点越远,高度增加得越快(坡度越来越陡)。

坐标衔接: 引导读者观察这个形状的轮廓:如果从上方俯视,它的等高线是同心圆。

例子 C:马鞍面 (The Saddle Surface) —— 复杂的交汇点

方程: $z = x^2 - y^2$

视觉特征: 就像一个马鞍,或者山口(Pass)。

特征分析: 沿着 $x$ 轴看,它是向上弯曲的谷底;沿着 $y$ 轴看,它是向下弯曲的山脊。

教学意义: 这告诉读者,多元函数的变化极其复杂,仅仅看一个方向(偏导数)是不够的,同一个点在不同方向可能有完全相反的变化趋势。

如何从方程“读出”地形?

固定 $y$: 想象用一把大刀沿着平行于 $x$ 轴的方向切下去。剩下的曲线就是一个关于 $x$ 的单变量函数。这正是偏导数正在做的事情。

固定 $z$: 想象用水平面去拦截这个地形。截出来的轮廓线就是等高线。这正是重积分定限时寻找边界的方法。







2. 定义域:积分的“地基”

在单变量微积分里,定义域通常只是 $x$ 轴上的一段区间 $[a, b]$。但在多变量微积分里,定义域变成了 $xy$ 平面上的一块区域 (Region)。

用数学不等式准确地描述这块区域:

矩形区域: $a \le x \le b, c \le y \le d$。

圆盘区域: $x^2 + y^2 \le R^2$。

一般区域: 例如由两条曲线 $y=g(x)$ 和 $y=h(x)$ 围成的部分。


在处理二元函数 $z = f(x, y)$ 时,我们不仅关心高度,更关心高度下面的那块地基。在微积分里,这块地基被称为积分区域 $D$。

复杂的区域归纳为两类:

A. X-型区域(垂直切割)

如果一个区域的左、右边界是两条直线,而上、下边界是两条曲线,我们称之为 X-型区域。

特征: $x$ 的范围是常数,而 $y$ 的范围取决于 $x$。

数学描述:

$a \leq x \leq b$

$g_1(x) \leq y \leq g_2(x)$

B. Y-型区域(水平切割)

如果区域的上、下边界是直线,而左、右边界是曲线。

特征: $y$ 的范围是常数,而 $x$ 的范围取决于 $y$。

数学描述:

$c \leq y \leq d$

$h_1(y) \leq x \leq h_2(y)$

经典案例:圆盘区域 (The Disk)

方程: $x^2 + y^2 \leq R^2$

直角坐标描述(麻烦但必须懂):

$-R \leq x \leq R$

$-\sqrt{R^2 - x^2} \leq y \leq \sqrt{R^2 - x^2}$




3. 等高线 (Level Curves)

理解 3D 曲面最有效的手段是“等高线”——就像地理地图上标注海拔高度的线一样。

令 $f(x, y) = C$(常数),在 $xy$ 平面上画出的曲线就是等高线。

密集的等高线意味着山势陡峭(变化率大,即梯度大)。

稀疏的等高线意味着地势平缓。


\section{格洛克自微分法则}

\begin{definition} 自微分

    如果变量 $u, v$ 表示一个或多个变量的表达式,那么 $u, v$ 以变量替换的方式进行微分,并根据链式法则展开,即:
    \begin{align}
        d(u + v) &= du + dv \\
        duv &= v\cdot du + u\cdot dv \\
        d\frac{u}{v} &= \frac{v\cdot du - u\cdot dv}{v^2} \\
        du(v) &= u'(v)\cdot dv
    \end{align}
\end{definition}

对于单变量函数 $y = f(x)$,$y$ 的微分总能表示为导函数和 $dx$ 的乘积表达式,即 $dy = f'(x) \,dx$。

对于多变量函数如 $z = f(x, y)$,我们无法确知 $x, y$ 的约束关系,因此需要以独立自变量的形态进行微分,这也是格洛克自微分名称的由来,虽然微分法则本质上没有改变,但确实改变了思考微分问题的角度。同时,对于这里的因变量 $z$,我们也改变思考问题的角度,把函数看作等号两端相等的方程形式,那么 $z$ 也可以看作是自变量,也就是所谓的隐函数形态,那么微分方法就得到了统一。

\begin{example} 表达式的自微分。
    \begin{align*}
        dxy &= y \,dx + x \,dy \\
        d(xy^2 + 2x + y^2) &= dxy^2 + d2x + dy^2 = y^2 \,dx + x \,dy^2 + 2 \,dx + 2y \,dy \\
        &= (y^2 + 2) \,dx + (2xy + 2y) \,dy \\
        d\frac{1}{x + y} &= d(x + y)^{-1} = -1\cdot (x + y)^{-2} \,d(x + y) \\
        &= -\frac{dx + dy}{(x + y)^2}
    \end{align*}
\end{example}










\section{偏导数:固定维度的变化}

核心目标: 降维理解变化率。

关键内容:

偏导的计算:把不相关的变量看作常数(算法层)。

几何意义: 切线的斜率。只需强调:偏导数只关心沿坐标轴方向的变化。

裁剪掉: 高阶导数、克莱罗定理(除非涉及极值判定)。


当你站在山坡上,想要描述地势的倾斜程度时,问题来了:你可以向四面八方走,每一个方向的坡度可能都不一样。

为了简化问题,微积分采取了“分而治之”的策略:先只看**正东($x$轴)方向的变化,再只看正北($y$轴)**方向的变化。

符号表示: 我们用 $\frac{\partial f}{\partial x}$ 或 $f_x$ 来表示对 $x$ 的偏导数。

例子:设海拔高度方程为 $z = x^2 + 3xy + y^2$

求 $f_x$: 把 $y$ 当成常数。$x^2$ 变成 $2x$,$3xy$ 变成 $3y$(因为 $x$ 被导掉了),$y^2$ 变成 0。结果:$f_x = 2x + 3y$

求 $f_y$: 把 $x$ 当成常数。$x^2$ 变成 0,$3xy$ 变成 $3x$,$y^2$ 变成 $2y$。结果:$f_y = 3x + 2y$

几何意义:切线的斜率

这是最关键的可视化步骤。

$f_x$ 的直观: 想象用一把平行于 $x$ 轴的刀,垂直切开山体。切口是一条曲线,这条曲线在某一点的切线斜率就是 $f_x$。

$f_y$ 的直观: 同理,用平行于 $y$ 轴的刀切开,得到的切线斜率就是 $f_y$。

偏导数告诉我们,如果 $x$ 挪动了一点点($dx$),高度 $z$ 会变化多少。

变化量 $\approx f_x \cdot dx$

偏导数本质上是“假装其他变量不存在”的单变量微积分。

哪些不讲?

不要讲:复杂的偏导数定义(极限形式)。只需让读者会算、懂几何意义即可。

不要讲:高阶偏导数的混合求导顺序证明。只需口头告知“通常情况下 $f_{xy} = f_{yx}$”。

不要讲:隐函数求导的复杂公式。在应用层,直接两边求导更符合直觉。


\section{全微分与切平面}

核心目标: 为积分中的“微元法”和向量分析中的“切平面”打底。

关键内容:

全微分公式: $dz = f_x dx + f_y dy = \partial f(x) + \partial f(y)$。

线性近似的直观: “以直代曲”。解释为什么在极小尺度下,曲面可以看作平面。

切平面方程: 为后续向量分析中寻找“法向量”埋下伏笔。

在单变量微积分中,我们学会了用切线来近似一条曲线。到了多变量世界,我们要用一个切平面来近似一块曲面。

从偏导数到全微分

偏导数 $f_x$ 和 $f_y$ 分别告诉了我们沿两个轴方向的“坡度”。如果我们把这两个变化结合起来,当 $x$ 变动了 $dx$,同时 $y$ 也变动了 $dy$ 时,函数高度 $z$ 的总变化量是多少?

全微分公式:
$$dz = \frac{\partial f}{\partial x}dx + \frac{\partial f}{\partial y}dy$$

直观理解:

总高度的变化,等于“$x$ 方向贡献的变化”加上“$y$ 方向贡献的变化”。

这里的 $dz$ 是切平面上的高度变化,而 $\Delta z$ 是真实曲面上的高度变化。当 $dx, dy$ 极小时,这两者几乎相等。这就是线性近似。

切平面

想象你把一个足球无限放大,当你站在足球表面的一点时,你会觉得地面是平的。这个“平地”就是切平面。

切平面方程:利用全微分公式,如果我们已知点 $(x_0, y_0, z_0)$,切平面的方程就是:

$$z - z_0 = f_x(x_0, y_0)(x - x_0) + f_y(x_0, y_0)(y - y_0)$$

观察切平面方程,如果我们把它重新排列成:
$$f_x(x - x_0) + f_y(y - y_0) - (z - z_0) = 0$$

这完全符合向量代数中平面的点法式方程。这意味着,向量 $\mathbf{n} = (f_x, f_y, -1)$ 恰好是垂直于该平面的法向量。

现在,我们不需要通过复杂的几何观察去寻找法向量了。只要你会求偏导数,你就拥有了描述空间中任何曲面方向的能力。这种能力,我们称之为梯度。

1. 核心直觉:什么是全微分?

想象你在一座山上,你的高度 $z$ 由坐标 $(x, y)$ 决定,即 $z = f(x, y)$。当你移动了一小步 $(dx, dy)$ 时,高度的变化量 $dz$ 必然是:
$$dz = \frac{\partial f}{\partial x}dx + \frac{\partial f}{\partial y}dy$$

这个 $dz$ 就是全微分。它描述了一个由函数 $f$ 产生的“完美增量”。

2. 问题的提出

现在我们拿到一个方程:
$$P(x, y)dx + Q(x, y)dy = 0$$

我们要问:这个式子是否也是某个函数 $f(x, y)$ 的全微分?

如果是,那么原方程就可以写成 $df = 0$,其解就是高度恒定的等高线:$f(x, y) = C$。

3. 简洁的推导:利用“混合偏导相等”

如果上述方程是全微分,那么必然存在一个 $f$ 使得:

$P = \frac{\partial f}{\partial x}$ ($P$ 是 $f$ 对 $x$ 的斜率)

$Q = \frac{\partial f}{\partial y}$ ($Q$ 是 $f$ 对 $y$ 的斜率)

关键逻辑:由于一个光滑函数的二阶混合偏导数与求导顺序无关(克莱罗定理):
$$\frac{\partial}{\partial y}\left( \frac{\partial f}{\partial x} \right) = \frac{\partial}{\partial x}\left( \frac{\partial f}{\partial y} \right)$$

将 $P$ 和 $Q$ 代入上式,立即得到:
$$\frac{\partial P}{\partial y} = \frac{\partial Q}{\partial x}$$

这就是全微分方程的判定条件。

4. 如何求出原函数 $f$ ?

一旦满足上述条件,找 $f$ 就像“拼图”:

第一步: 对 $P$ 凑 $x$ 的积分:$f(x, y) = \int P dx + g(y)$。

注意: 因为是对 $x$ 积分,所以会多出一个只跟 $y$ 有关的常数项 $g(y)$。

第二步: 求这个 $f$ 对 $y$ 的偏导,让它等于 $Q$:
$$\frac{\partial}{\partial y} \left( \int P dx \right) + g'(y) = Q$$

第三步: 解出 $g(y)$,拼回 $f(x, y) = C$。

总结

物理意义: 全微分方程代表一个保守场(就像重力场)。

判定准则: “交叉求导”相等。

直观理解: 如果把 $P$ 看作力的 $x$ 分量,$Q$ 看作 $y$ 分量,判定条件本质上是在说这个力的旋度为零,即做功与路径无关。

\begin{example} 解方程 $(2xy + e^y)dx + (x^2 + xe^y)dy = 0$

    第一步:判定(交叉验证)
    
    首先,我们要确认这个方程是不是“完美”的全微分。
    
    设 $P = 2xy + e^y$
    
    设 $Q = x^2 + xe^y$

    对 $P$ 求 $y$ 的偏导,对 $Q$ 求 $x$ 的偏导:
    
    $\frac{\partial P}{\partial y} = 2x + e^y$
    
    $\frac{\partial Q}{\partial x} = 2x + e^y$

    结论: 两者相等,这是一个全微分方程。这意味着背后一定隐藏着一个函数 $f(x,y)$。

    第二步:还原原函数(拼图法)
    
    我们要找的 $f$ 必须满足 $\frac{\partial f}{\partial x} = P$。
    
    1. 先对 $x$ 积分:
    $$f(x,y) = \int (2xy + e^y) dx = x^2y + xe^y + g(y)$$

    注意: 这里的 $g(y)$ 是“积分常数”。因为在对 $x$ 求导时,任何只含 $y$ 的项都会消失,所以积分回去时必须补上。

    2. 利用 $Q$ 来定位 $g(y)$:我们知道 $\frac{\partial f}{\partial y}$ 必须等于 $Q$。把我们刚才求出的 $f$ 对 $y$ 求偏导:
    $$\frac{\partial}{\partial y}(x^2y + xe^y + g(y)) = x^2 + xe^y + g'(y)$$

    对照题目中的 $Q = x^2 + xe^y$,发现:
    $$x^2 + xe^y + g'(y) = x^2 + xe^y$$

    消掉相同项,得到 $g'(y) = 0$。这意味着 $g(y)$ 只是一个常数(我们可以取 $0$)。
    
    第三步:写出通解将 $g(y)$ 填回第一步的公式中,并令其等于常数 $C$:
    $$x^2y + xe^y = C$$
\end{example}



\section{链式法则:变量之间的传递链}

在实际应用中,变量往往是层层嵌套的。例如:

空间中某点的温度 $T$ 取决于坐标 $(x, y, z)$。

而由于你正在运动,你的坐标 $(x, y, z)$ 又取决于时间 $t$。那么,随着时间的推移,你感受到的温度变化率 $\frac{dT}{dt}$ 是多少?

1. 核心工具:路径图法 (Tree Diagram)

不要试图背诵公式。处理链式法则最直观的方法是画变量关系图。

场景: 设 $z = f(x, y)$,而 $x = x(t), y = y(t)$。

第一层: 画出 $z$。

第二层: 从 $z$ 连出两条线,指向 $x$ 和 $y$(代表偏导数 $\frac{\partial z}{\partial x}$ 和 $\frac{\partial z}{\partial y}$)。

第三层: 从 $x$ 连出一条线指向 $t$(代表 $\frac{dx}{dt}$),从 $y$ 连出一条线指向 $t$(代表 $\frac{dy}{dt}$)。

求导法则: 走到底: 沿着每一条路径走到终点 $t$,并把路径上的导数相乘。

加总: 把所有到达 $t$ 的路径结果相加。

$$\frac{dz}{dt} = \frac{\partial z}{\partial x} \cdot \frac{dx}{dt} + \frac{\partial z}{\partial y} \cdot \frac{dy}{dt}$$

坐标变换
在二重积分中,我们经常需要从直角坐标 $(x, y)$ 切换到极坐标 $(r, \theta)$。这时,$x = r\cos\theta, y = r\sin\theta$。如果我们想知道函数 $f$ 随着半径 $r$ 如何变化:

$$\frac{\partial f}{\partial r} = \frac{\partial f}{\partial x}\frac{\partial x}{\partial r} + \frac{\partial f}{\partial y}\frac{\partial y}{\partial r}$$

裁剪说明:

裁剪掉: 隐函数求导公式(如 $dy/dx = -F_x/F_y$)。直接告诉读者用全微分或链式法则对等式两边同时求导即可,没必要让读者多背一个容易记混正负号的公式。

裁剪掉: 复杂的二阶偏导链式法则。除非读者要研究高深的物理波动方程,否则一阶链式法则已经涵盖了 90% 的应用场景。

