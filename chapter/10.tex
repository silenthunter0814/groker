\chapter{向量微分}

\section{微分向量分析}

16. 向量的微分。 让我们考虑如下向量场:
$$\mathbf{u} = \alpha(x, y, z, t)\mathbf{i} + \beta(x, y, z, t)\mathbf{j} + \gamma(x, y, z, t)\mathbf{k} \quad (42)$$

在任意点 $P(x, y, z)$ 及任意时间 $t$,式 (42) 定义了一个向量。如果我们固定点 $P$,由于分量 $\alpha, \beta, \gamma$ 具有时间相关性,向量 $\mathbf{u}$ 仍会发生变化。如果我们固定时间,我们会注意到在点 $P(x, y, z)$ 处的向量通常与在以下点处的向量不同:

$$Q(x + dx, y + dy, z + dz)$$

在微积分中,学生已经学过如何计算 $x, y, z, t$ 的单函数变化。那么在向量的情况下,我们会遇到什么困难吗?实际上完全没有,因为我们很容易注意到,当且仅当向量的分量发生变化时,$\mathbf{u}$ 才会发生变化。因此,$\alpha(x, y, z, t)$ 的变化会产生 $\mathbf{u}$ 在 $x$ 方向上的变化,同理,$\beta$ 和 $\gamma$ 的变化分别产生 $\mathbf{u}$ 在 $y$ 和 $z$ 方向上的变化。由此,我们得出以下定义:

$$\mathbf{du} = d\alpha\,\mathbf{i} + d\beta\,\mathbf{j} + d\gamma\,\mathbf{k} \quad (43)$$

$$\begin{aligned} \mathbf{du} = &\left( \frac{\partial \alpha}{\partial x} dx + \frac{\partial \alpha}{\partial y} dy + \frac{\partial \alpha}{\partial z} dz + \frac{\partial \alpha}{\partial t} dt \right) \mathbf{i} \\ &+ \left( \frac{\partial \beta}{\partial x} dx + \frac{\partial \beta}{\partial y} dy + \frac{\partial \beta}{\partial z} dz + \frac{\partial \beta}{\partial t} dt \right) \mathbf{j} \\ &+ \left( \frac{\partial \gamma}{\partial x} dx + \frac{\partial \gamma}{\partial y} dy + \frac{\partial \gamma}{\partial z} dz + \frac{\partial \gamma}{\partial t} dt \right) \mathbf{k} \end{aligned}$$

例如,设 $\mathbf{r} = x\mathbf{i} + y\mathbf{j} + z\mathbf{k}$ 为三维空间中运动质点 $P(x, y, z)$ 的位置向量。则:
$$\mathbf{dr} = dx\,\mathbf{i} + dy\,\mathbf{j} + dz\,\mathbf{k}$$

以及
$$\mathbf{v} = \frac{d\mathbf{r}}{dt} = \frac{dx}{dt}\mathbf{i} + \frac{dy}{dt}\mathbf{j} + \frac{dz}{dt}\mathbf{k} \quad (44)$$

$$\mathbf{a} = \frac{d^2\mathbf{r}}{dt^2} = \frac{d^2x}{dt^2}\mathbf{i} + \frac{d^2y}{dt^2}\mathbf{j} + \frac{d^2z}{dt^2}\mathbf{k} \quad (45)$$

根据定义,方程 (44) 和 (45) 分别是质点的速度和加速度。我们假设向量 $\mathbf{i, j, k}$ 在空间中保持固定。

如果向量 $\mathbf{u}$ 仅取决于单个变量 $t$,我们可以定义:
$$\frac{d\mathbf{u}}{dt} = \lim_{\Delta t \to 0} \frac{\mathbf{u}(t + \Delta t) - \mathbf{u}(t)}{\Delta t} \quad (46)$$

很容易验证 (46) 与 (43) 是等价的。

例 14。 考虑一个质点 $P$ 在半径为 $r$ 的圆上以恒定角速度 $\omega = \frac{d\theta}{dt}$ 运动(见图 28)。我们注意到:

$$\mathbf{r} = r \cos \theta\,\mathbf{i} + r \sin \theta\,\mathbf{j}$$

因此:
$$\mathbf{v} = \frac{d\mathbf{r}}{dt} = (-r \sin \theta\,\mathbf{i} + r \cos \theta\,\mathbf{j}) \frac{d\theta}{dt}$$

以及:
$$\mathbf{a} = \frac{d\mathbf{v}}{dt} = \frac{d^2\mathbf{r}}{dt^2} = (-r \cos \theta\,\mathbf{i} - r \sin \theta\,\mathbf{j}) \left( \frac{d\theta}{dt} \right)^2$$

所以加速度为:
$$\mathbf{a} = -\omega^2\mathbf{r} \quad (47)$$

点 $P$ 具有指向原点的加速度,其大小为恒定的 $\omega^2r$。这个加速度是由于速度向量以恒定速率改变方向而产生的;它被称为向心加速度。

例 15。 设 $P$ 为空间曲线上的任意一点(见图 29):

$$\begin{aligned} x &= x(s) \\ y &= y(s) \\ z &= z(s) \end{aligned} \text{}$$

其中 $s$ 是从某个固定点 $Q$ 开始测量的弧长。那么:

$$\mathbf{r} = x(s)\mathbf{i} + y(s)\mathbf{j} + z(s)\mathbf{k} \quad (48) \text{}$$

因此:
$$\frac{d\mathbf{r}}{ds} = \frac{dx}{ds}\mathbf{i} + \frac{dy}{ds}\mathbf{j} + \frac{dz}{ds}\mathbf{k} \quad (49) \text{}$$

并且由微积分可知:
$$\begin{aligned} \frac{d\mathbf{r}}{ds} \cdot \frac{d\mathbf{r}}{ds} &= \left( \frac{dx}{ds} \right)^2 + \left( \frac{dy}{ds} \right)^2 + \left( \frac{dz}{ds} \right)^2 \\ &= \frac{dx^2 + dy^2 + dz^2}{ds^2} \equiv 1 \end{aligned} \text{}$$

因此 $\frac{d\mathbf{r}}{ds}$ 是一个单位向量。当 $\Delta s \to 0$ 时,$\frac{\Delta\mathbf{r}}{\Delta s}$ 的位置趋近于点 $P$ 处的切线。因此,(49) 代表了空间曲线 (48) 的单位切向量。

17. 微分规则。 考虑:

$$\begin{aligned} \varphi(t) &= \mathbf{u}(t) \cdot \mathbf{v}(t) \\ \varphi(t + \Delta t) - \varphi(t) &= \mathbf{u}(t + \Delta t) \cdot \mathbf{v}(t + \Delta t) - \mathbf{u}(t) \cdot \mathbf{v}(t) \end{aligned}$$

现在:
$$\begin{aligned} \mathbf{u}(t + \Delta t) &= \mathbf{u}(t) + \Delta\mathbf{u} \\ \mathbf{v}(t + \Delta t) &= \mathbf{v}(t) + \Delta\mathbf{v} \end{aligned}$$

(见图 27),所以:
$$\frac{\varphi(t + \Delta t) - \varphi(t)}{\Delta t} = \mathbf{u} \cdot \frac{\Delta\mathbf{v}}{\Delta t} + \frac{\Delta\mathbf{u}}{\Delta t} \cdot \mathbf{v} + \frac{\Delta\mathbf{u}}{\Delta t} \cdot \Delta\mathbf{v}$$

取极限,我们得到:
$$\frac{d(\mathbf{u} \cdot \mathbf{v})}{dt} = \mathbf{u} \cdot \frac{d\mathbf{v}}{dt} + \frac{d\mathbf{u}}{dt} \cdot \mathbf{v} \quad (50)$$

类似地:
$$\frac{d(\mathbf{u} \times \mathbf{v})}{dt} = \mathbf{u} \times \frac{d\mathbf{v}}{dt} + \frac{d\mathbf{u}}{dt} \times \mathbf{v} \quad (51)$$

$$\frac{d(f\mathbf{u})}{dt} = f\frac{d\mathbf{u}}{dt} + \frac{df}{dt}\mathbf{u} \quad (52)$$

请注意这些公式与微积分规则的一致性。

例 16。 设 $\mathbf{u}(t)$ 为一个模(大小)恒定的向量。因此:

$$\mathbf{u} \cdot \mathbf{u} = u^2 = \text{常数}$$

通过微分,我们得到:

$$\begin{aligned} \mathbf{u} \cdot \frac{d\mathbf{u}}{dt} + \frac{d\mathbf{u}}{dt} \cdot \mathbf{u} &= 0 \\ 
\mathbf{u} \cdot \frac{d\mathbf{u}}{dt} &= 0 \end{aligned}$$

因此,要么 $\frac{d\mathbf{u}}{dt} = 0$,要么 $\frac{d\mathbf{u}}{dt}$ 与 $\mathbf{u}$ 垂直。这是一个重要的结果,学生应当充分理解。读者应给出该定理的几何证明。

例 17。在所有情况下,$\mathbf{u} \cdot \mathbf{u} = u^2$,其中 $u$ 是 $\mathbf{u}$ 的长度。微分得:
$$2\mathbf{u} \cdot \frac{d\mathbf{u}}{dt} = 2u \frac{du}{dt}$$

以及:
$$\mathbf{u} \cdot \frac{d\mathbf{u}}{dt} = u \frac{du}{dt} \quad (53)$$

这一结果并非显而易见,因为 $|d\mathbf{u}| \neq du$。

例 18:平面运动。现在 $\mathbf{r} = r\mathbf{R}$,其中 $\mathbf{R}$ 是一个单位向量(见图 30)。因此:
$$\mathbf{v} = \frac{d\mathbf{r}}{dt} = \frac{dr}{dt}\mathbf{R} + r\frac{d\mathbf{R}}{dt}$$

由于 $\mathbf{R}$ 是单位向量,根据例 16,$\frac{d\mathbf{R}}{dt}$ 与 $\mathbf{R}$ 垂直。同时,通过对 $\mathbf{R} = \cos \theta\,\mathbf{i} + \sin \theta\,\mathbf{j}$ 进行微分,我们可以很容易验证 $|\frac{d\mathbf{R}}{dt}| = \frac{d\theta}{dt}$。因此 $\mathbf{v} = \frac{dr}{dt}\mathbf{R} + r\frac{d\theta}{dt}\mathbf{P}$,其中 $\mathbf{P}$ 是垂直于 $\mathbf{R}$ 的单位向量。再次微分得:
$$\mathbf{a} = \frac{d\mathbf{v}}{dt} = \frac{d^2r}{dt^2}\mathbf{R} + \frac{dr}{dt}\frac{d\mathbf{R}}{dt} + \frac{dr}{dt}\frac{d\theta}{dt}\mathbf{P} + r\frac{d^2\theta}{dt^2}\mathbf{P} + r\frac{d\theta}{dt}\frac{d\mathbf{P}}{dt}$$

或:
$$\mathbf{a} = \left[ \frac{d^2r}{dt^2} - r\left( \frac{d\theta}{dt} \right)^2 \right] \mathbf{R} + \left[ 2\frac{dr}{dt}\frac{d\theta}{dt} + r\frac{d^2\theta}{dt^2} \right] \mathbf{P}$$

因为:
$$\frac{d\mathbf{P}}{dt} = -\frac{d\theta}{dt}\mathbf{R} \quad (54)$$

因此:
$$\mathbf{a} = \left[ \frac{d^2r}{dt^2} - r\left( \frac{d\theta}{dt} \right)^2 \right] \mathbf{R} + \frac{1}{r} \frac{d}{dt} \left( r^2 \frac{d\theta}{dt} \right) \mathbf{P} \quad (55)$$

18. 梯度

设 $\varphi(x, y, z)$ 为任意连续可微的空间函数。根据微积分:
$$d\varphi = \frac{\partial \varphi}{\partial x} dx + \frac{\partial \varphi}{\partial y} dy + \frac{\partial \varphi}{\partial z} dz \quad (56)$$

现在设 $\mathbf{r}$ 为指向点 $P(x, y, z)$ 的位置向量:
$$\mathbf{r} = x\mathbf{i} + y\mathbf{j} + z\mathbf{k}$$

如果我们移动到点 $Q(x + dx, y + dy, z + dz)$(见图 32),则:
$$d\mathbf{r} = dx\,\mathbf{i} + dy\,\mathbf{j} + dz\,\mathbf{k}$$

现在注意到式 (56) 包含了项 $dx, dy, dz$ 以及项 $\frac{\partial \varphi}{\partial x}, \frac{\partial \varphi}{\partial y}, \frac{\partial \varphi}{\partial z}$。我们定义一个由 $\varphi$ 的三个偏导数组成的新向量。令 $\text{del}\,\varphi \equiv \nabla\varphi \equiv \text{gradient}\,\varphi$($\varphi$ 的梯度)定义为:

$$\nabla\varphi = \frac{\partial \varphi}{\partial x}\mathbf{i} + \frac{\partial \varphi}{\partial y}\mathbf{j} + \frac{\partial \varphi}{\partial z}\mathbf{k} \quad (57)$$

我们立即可以看到:
$$d\varphi = d\mathbf{r} \cdot \nabla\varphi \quad (58)$$

梯度的几何解释

我们现在给出 $\nabla\varphi$ 的几何解释。在点 $P(x_0, y_0, z_0)$ 处,$\varphi$ 的值为 $\varphi(x_0, y_0, z_0)$,因此:
$$\varphi(x, y, z) = \varphi(x_0, y_0, z_0)$$

表示一个显然包含点 $P(x_0, y_0, z_0)$ 的曲面(等值面)。

只要我们沿着这个曲面移动,$\varphi$ 就具有恒定值 $\varphi(x_0, y_0, z_0)$,且 $d\varphi = 0$。因此,根据式 (58):
$$d\mathbf{r} \cdot \nabla\varphi = 0 \quad (59)$$

由于 $\nabla\varphi$ 是一个在 $\varphi$ 被求导后就完全确定的向量,而式 (59) 表明,只要 $d\mathbf{r}$ 表示从 $P$ 到 $Q$ 的位移且 $Q$ 仍在 $\varphi = \text{常数}$ 的曲面上,$\nabla\varphi$ 就与 $d\mathbf{r}$ 垂直。

因此,$\nabla\varphi$ 垂直于曲面在 $P$ 点处所有可能的切线,所以 $\nabla\varphi$ 必然正交(垂直)于曲面 $\varphi(x, y, z) = \text{常数}$。

(接上页图 33)。现在让我们回到公式 $d\varphi = d\mathbf{r} \cdot \nabla\varphi$。向量 $\nabla\varphi$ 在任何给定点 $P(x, y, z)$ 都是固定的,因此 $d\varphi$($\varphi$ 的变化量)在很大程度上取决于 $d\mathbf{r}$。

显然,当 $d\mathbf{r}$ 与 $\nabla\varphi$ 平行时,$d\varphi$ 取得最大值,因为 $d\mathbf{r} \cdot \nabla\varphi = |d\mathbf{r}||\nabla\varphi| \cos \theta$,而 $\cos \theta$ 在 $\theta = 0^\circ$ 时达到最大值。因此,$\nabla\varphi$ 的方向是函数 $\varphi(x, y, z)$ 增加最快的方向。

设 $|d\mathbf{r}| = ds$,则有:
$$\frac{d\varphi}{ds} = \mathbf{u} \cdot \nabla\varphi \quad (60)$$

其中 $\mathbf{u}$ 是沿 $d\mathbf{r}$ 方向的单位向量。因此,$\varphi$ 在任何方向上的变化率,就是 $\nabla\varphi$ 在该方向单位向量上的投影。

示例解析例 

19: 求曲面 $x^2 + y^2 - z = 1$ 在点 $P(1, 1, 1)$ 处的单位法向量。

这里,$\varphi(x, y, z) = x^2 + y^2 - z$。

$\nabla\varphi = 2x\mathbf{i} + 2y\mathbf{j} - \mathbf{k}$。

在点 $P(1, 1, 1)$ 处,$\nabla\varphi = 2\mathbf{i} + 2\mathbf{j} - \mathbf{k}$。

因此,单位法向量为:

$$\mathbf{N} = \frac{2\mathbf{i} + 2\mathbf{j} - \mathbf{k}}{3}$$

(注:分母 3 是向量长度 $\sqrt{2^2 + 2^2 + (-1)^2}$)

例 20: 若 $r = (x^2 + y^2 + z^2)^{\frac{1}{2}}$,求 $\nabla r$。

曲面 $r = \text{常数}$ 是一个球面。

因此 $\nabla r$ 与球面正交,也就是说它与位置向量 $\mathbf{r}$ 平行。

故 $\nabla r = k\mathbf{r}$。根据公式 (53) 有 $dr = d\mathbf{r} \cdot \nabla r = k d\mathbf{r} \cdot \mathbf{r} = kr\,dr$。

因此 $k = \frac{1}{r}$,且:
$$\nabla r = \frac{\mathbf{r}}{r} = \mathbf{R}$$

(此处 $\mathbf{R}$ 表示径向单位向量)

例 21(链式法则证明):
$$\nabla f(u) = \frac{\partial f}{\partial x}\mathbf{i} + \frac{\partial f}{\partial y}\mathbf{j} + \frac{\partial f}{\partial z}\mathbf{k}$$

$$= f'(u)\frac{\partial u}{\partial x}\mathbf{i} + f'(u)\frac{\partial u}{\partial y}\mathbf{j} + f'(u)\frac{\partial u}{\partial z}\mathbf{k} = f'(u)\nabla u \quad$$

22(多元复合函数):
$$\nabla f(u_1, u_2, \dots, u_n) = \sum_{\alpha=1}^n \frac{\partial f}{\partial u_\alpha} \nabla u_\alpha \quad (62) \quad$$

梯度的几何应用:

椭圆例 23: 考虑由 $r_1 + r_2 = \text{常数}$ 定义的椭圆(见图 34)。此时 $\nabla(r_1 + r_2)$ 与椭圆正交。设 $\mathbf{T}$ 为椭圆的单位切向量,则有:
$$\nabla(r_1 + r_2) \cdot \mathbf{T} = 0, \quad \text{即} \quad \nabla r_1 \cdot \mathbf{T} = -\nabla r_2 \cdot \mathbf{T} \quad (63) \quad$$

根据例 20,$\nabla r_1$ 是平行于向量 $\vec{AP}$ 的单位向量,而 $\nabla r_2$ 是平行于向量 $\vec{BP}$ 的单位向量。这表明向量 $\vec{AP}$ 和 $\vec{BP}$ 与椭圆切线的夹角相等。

19. 向量算子 $\nabla$。我们定义:
$$\nabla \equiv \mathbf{i} \frac{\partial}{\partial x} + \mathbf{j} \frac{\partial}{\partial y} + \mathbf{k} \frac{\partial}{\partial z} \quad (64)$$

请注意,$\nabla$ 是一个算子,正如 $\frac{d}{dx}$ 是微分学中的一个算子一样。因此:

$$\begin{aligned} 
    \nabla \varphi &= \left( \mathbf{i} \frac{\partial}{\partial x} + \mathbf{j} \frac{\partial}{\partial y} + \mathbf{k} \frac{\partial}{\partial z} \right) \varphi \\ 
    &= \mathbf{i} \frac{\partial \varphi}{\partial x} + \mathbf{j} \frac{\partial \varphi}{\partial y} + \mathbf{k} \frac{\partial \varphi}{\partial z} 
\end{aligned}$$

我们称 $\nabla$ (读作 del) 为向量算子,因为它的分量是 $\frac{\partial}{\partial x}, \frac{\partial}{\partial y}, \frac{\partial}{\partial z}$。在未来,记住 $\nabla$ 既表现为微分算子,又表现为向量,将对我们有所帮助。

例 24
$$\begin{aligned} 
    \nabla(uv) &= \mathbf{i} \frac{\partial(uv)}{\partial x} + \mathbf{j} \frac{\partial(uv)}{\partial y} + \mathbf{k} \frac{\partial(uv)}{\partial z} \\ 
    &= \left( \mathbf{i} \frac{\partial v}{\partial x} + \mathbf{j} \frac{\partial v}{\partial y} + \mathbf{k} \frac{\partial v}{\partial z} \right) u + \left( \mathbf{i} \frac{\partial u}{\partial x} + \mathbf{j} \frac{\partial u}{\partial y} + \mathbf{k} \frac{\partial u}{\partial z} \right) v 
\end{aligned}$$

$$\nabla(uv) = u \nabla v + v \nabla u \quad (65)$$

如果我们记住 $\nabla$ 是一个微分算子,从而可以应用普通的微积分法则,那么这个结果就很容易记住了。

20. 向量的散度。 

让我们考虑密度为 $\rho(x, y, z)$ 的流体的运动。我们假设其速度场由 $\mathbf{f} = u(x, y, z)\mathbf{i} + v(x, y, z)\mathbf{j} + w(x, y, z)\mathbf{k}$ 给出。由于 $\rho$ 和 $\mathbf{f}$ 显式地独立于时间(不随时间变化),这种类型的运动被称为定常运动(steady motion)。我们现在集中研究流经一个尺寸为 $dx, dy, dz$ 的微小平行六面体 $ABCDEFGH$(图 35)的流量。

\begin{figure}[htbp] 
    \centering
    \includegraphics[width=0.8\textwidth]{images/ref/35.png} 
    \caption{\textbf{向量的散度}}
\end{figure}

1. 核心逻辑:空间的变化率

假设流体沿着 $y$ 轴方向流动。在进入面 $ABCD$ 时,单位时间内流过的质量流量(密度 $\times$ 速度 $\times$ 截面积)可以表示为一个关于坐标的函数 $F(y)$,其中 $F = \rho v$。

当流体经过一段极小的距离 $dy$ 到达出口面 $EFGH$ 时,由于流场是不均匀的,出口处的流量 $F(y + dy)$ 通常不等于入口处的流量 $F(y)$。

2. 数学上的推导根据导数的定义,函数在 $y + dy$ 处的值可以通过其在 $y$ 处的值加上变化量来表示:
$$F(y + dy) \approx F(y) + \frac{dF}{dy} \cdot dy$$
$$F(y + dy) = F(y) + dF(y) = \frac{dF}{dy} \cdot dy$$

将我们的物理量代入:

入口流量 ($y$ 处):$\rho v$ (这里省略了截面积 $dx \, dz$)。

变化率:$\frac{\partial(\rho v)}{\partial y}$,这表示单位长度上质量流量的变化量。

总变化量:变化率 $\times$ 距离 $dy$,即 $\frac{\partial(\rho v)}{\partial y} dy$。

所以,出口处的流量就等于:
$$\text{入口量} + \text{这一段路程中产生的增量} = \rho v + \frac{\partial(\rho v)}{\partial y} dy$$

3. 为什么这个“增量”导致了“损失”?

理解这个问题的关键在于净流量的概念:

净流出量 = 流出 - 流入。

代入公式:$\left[ \rho v + \frac{\partial(\rho v)}{\partial y} dy \right] - \rho v = \frac{\partial(\rho v)}{\partial y} dy$。

如果这个结果是正数,说明“流出的比流入的多”。因为质量守恒,多出来的流体只能来自于这个小方块内部存储的流体,所以对这个方块本身来说,它损失了质量。

4. 举个直观的例子

想象一条高速公路:

入口 (y):每分钟进来 100 辆车。

路段 (dy):在这段路内,车速变快了或者车流变稀疏了。

出口 (y+dy):每分钟出去了 110 辆车。

结果:出口比入口多了 10 辆。这多出来的 10 辆车(增量)必然导致了这段公路上的车辆总数在减少。这 10 辆车就是该路段的“车辆损失”。


首先,让我们计算单位时间内通过面 $ABCD$ 的流体量。速度 $\mathbf{f}$ 的 $x$ 和 $z$ 分量对流经 $ABCD$ 的流量没有贡献。单位时间内进入面 $ABCD$ 的流体质量由 $\rho v \, dx \, dz$ 给出。单位时间内离开面 $EFGH$ 的流体质量为:

$$\left[ \rho v + \frac{\partial(\rho v)}{\partial y} dy \right] dx \, dz$$

因此,单位时间内的质量损失等于:
$$\frac{\partial(\rho v)}{\partial y} dx \, dy \, dz$$

如果我们也考虑其他两个面,我们会发现单位时间内的总质量损失为:
$$\left[ \frac{\partial(\rho u)}{\partial x} + \frac{\partial(\rho v)}{\partial y} + \frac{\partial(\rho w)}{\partial z} \right] dx \, dy \, dz$$

因此
$$\frac{\partial(\rho u)}{\partial x} + \frac{\partial(\rho v)}{\partial y} + \frac{\partial(\rho w)}{\partial z} \quad (66)$$

表示单位时间、单位体积内的质量损失。这个量被称为向量 $\rho \mathbf{f}$ 的散度。我们立刻可以看到:
$$\nabla \cdot (\rho \mathbf{f}) = \text{div } (\rho \mathbf{f}) = \frac{\partial(\rho u)}{\partial x} + \frac{\partial(\rho v)}{\partial y} + \frac{\partial(\rho w)}{\partial z} = \frac{1}{V} \frac{dM}{dt} \quad (67)$$

由于 $\mathbf{i}, \mathbf{j}, \mathbf{k}$ 是常向量。$M$ 和 $V$ 分别代表流体的质量和体积。

公式最后提到的 $\frac{1}{V} \frac{dM}{dt}$ 实际上表达了连续性方程:
$$\nabla \cdot (\rho \mathbf{f}) = -\frac{\partial \rho}{\partial t}$$

这里的散度代表了单位体积内质量流出的速率。

任意向量 $\mathbf{f}$ 的散度定义为 $\nabla \cdot \mathbf{f}$。我们现在计算 $\varphi(x, y, z)\mathbf{f}$ 的散度:

$$\begin{aligned} 
    \nabla \cdot (\varphi\mathbf{f}) &= \frac{\partial(\varphi u)}{\partial x} + \frac{\partial(\varphi v)}{\partial y} + \frac{\partial(\varphi w)}{\partial z} \\ 
    &= \varphi \left( \frac{\partial u}{\partial x} + \frac{\partial v}{\partial y} + \frac{\partial w}{\partial z} \right) + \left( u \frac{\partial \varphi}{\partial x} + v \frac{\partial \varphi}{\partial y} + w \frac{\partial \varphi}{\partial z} \right) 
\end{aligned}$$

由此得到:
$$\nabla \cdot (\varphi\mathbf{f}) = \varphi \nabla \cdot \mathbf{f} + \mathbf{f} \cdot \nabla \varphi \quad (68)$$

如果我们把 $\nabla$ 看作一个向量微分算子,就能很容易地记住这个结果。因此,当对 $\varphi\mathbf{f}$ 进行运算时,我们首先保持 $\varphi$ 不变让 $\nabla$ 作用于 $\mathbf{f}$,然后保持 $\mathbf{f}$ 不变让 $\nabla$ 作用于 $\varphi$(注意:$\nabla \cdot \varphi$ 是没有意义的),由于 $\mathbf{f}$ 和 $\nabla \varphi$ 都是向量,我们通过取它们的点积(内积)来完成乘法。

例 25 计算 $\nabla \cdot \mathbf{f}$,其中 $\mathbf{f} = \mathbf{r}/r^3$(平方反比力)。

$$\begin{aligned} 
    \nabla \cdot (r^{-3}\mathbf{r}) &= r^{-3} \nabla \cdot \mathbf{r} + \mathbf{r} \cdot \nabla r^{-3} \\ &= 3r^{-3} + \mathbf{r} \cdot (-3r^{-4} \nabla r) \\ &= 3r^{-3} - 3r^{-5} \mathbf{r} \cdot \mathbf{r} = 3r^{-3} - 3r^{-3} = 0 
\end{aligned}$$

$$\nabla \cdot (r^{-3}\mathbf{r}) = 0 \quad (69)$$

这是一个重要的结果:平方反比力的散度为零。我们注意到:
$$\nabla \cdot \mathbf{r} = \frac{\partial x}{\partial x} + \frac{\partial y}{\partial y} + \frac{\partial z}{\partial z} = 3$$

公式 (69) 说明对于像引力或静电力这样的平方反比力场,在除源点($r=0$)之外的任何地方,场线的“流出”与“流入”是平衡的。

第一步:套用乘法法则 (公式 68)

根据公式 (68):$\nabla \cdot (\varphi\mathbf{f}) = \varphi \nabla \cdot \mathbf{f} + \mathbf{f} \cdot \nabla \varphi$。在这里,我们令标量 $\varphi = r^{-3}$,向量 $\mathbf{f} = \mathbf{r}$。带入后得到:

$$\nabla \cdot (r^{-3}\mathbf{r}) = r^{-3} (\nabla \cdot \mathbf{r}) + \mathbf{r} \cdot (\nabla r^{-3})$$

第二步:计算两个关键项

我们需要分别算出括号里的两个部分:

计算 $\nabla \cdot \mathbf{r}$ :由于 $\mathbf{r} = x\mathbf{i} + y\mathbf{j} + z\mathbf{k}$,根据散度定义:

$$\nabla \cdot \mathbf{r} = \frac{\partial x}{\partial x} + \frac{\partial y}{\partial y} + \frac{\partial z}{\partial z} = 1 + 1 + 1 = 3$$

所以第一项变成了 $3r^{-3}$。

计算 $\nabla r^{-3}$ (标量的梯度):这用到复合函数求导(链式法则)。由于 $r = (x^2 + y^2 + z^2)^{1/2}$:
$$\nabla r^n = n r^{n-1} \nabla r$$

当 $n = -3$ 时:
$$\nabla r^{-3} = -3 r^{-4} \nabla r$$

而 $\nabla r$(距离函数的梯度)等于单位向量 $\frac{\mathbf{r}}{r}$。所以:$\nabla r^{-3} = -3 r^{-4} (\frac{\mathbf{r}}{r}) = -3 r^{-5} \mathbf{r}$。

在向量运算中,一个向量与其自身的点积等于其模长的平方,即 $\mathbf{r} \cdot \mathbf{r} = r^2$。


例 26 梯度的散度是什么?
$$\begin{aligned} 
    \nabla \cdot (\nabla \varphi) &= \nabla \cdot \left( \frac{\partial \varphi}{\partial x}\mathbf{i} + \frac{\partial \varphi}{\partial y}\mathbf{j} + \frac{\partial \varphi}{\partial z}\mathbf{k} \right) \\ &= \frac{\partial^2 \varphi}{\partial x^2} + \frac{\partial^2 \varphi}{\partial y^2} + \frac{\partial^2 \varphi}{\partial z^2} 
\end{aligned}$$

这个重要的量(梯度的散度)被称为 $\varphi$ 的拉普拉斯算子(Laplacian):
$$\text{Lap } \varphi = \nabla \cdot (\nabla \varphi) = \nabla^2 \varphi = \frac{\partial^2 \varphi}{\partial x^2} + \frac{\partial^2 \varphi}{\partial y^2} + \frac{\partial^2 \varphi}{\partial z^2} \quad (70)$$


21. 向量的旋度。 

我们暂时搁置旋度的物理含义,直接给出定义:
$$\text{curl } \mathbf{f} = \nabla \times \mathbf{f} = \begin{vmatrix} \mathbf{i} & \mathbf{j} & \mathbf{k} \\ \frac{\partial}{\partial x} & \frac{\partial}{\partial y} & \frac{\partial}{\partial z} \\ u & v & w \end{vmatrix}$$

$$\nabla \times \mathbf{f} = \mathbf{i} \left( \frac{\partial w}{\partial y} - \frac{\partial v}{\partial z} \right) + \mathbf{j} \left( \frac{\partial u}{\partial z} - \frac{\partial w}{\partial x} \right) + \mathbf{k} \left( \frac{\partial v}{\partial x} - \frac{\partial u}{\partial y} \right) \quad (71)$$

例 27 径向量
$$\nabla \times \mathbf{r} = \begin{vmatrix} \mathbf{i} & \mathbf{j} & \mathbf{k} \\ \frac{\partial}{\partial x} & \frac{\partial}{\partial y} & \frac{\partial}{\partial z} \\ x & y & z \end{vmatrix} = 0$$

(注:这说明向径向量场是无旋的)

例 28
$$\begin{aligned} 
    \nabla \times (\varphi \mathbf{f}) &= \begin{vmatrix} \mathbf{i} & \mathbf{j} & \mathbf{k} \\ \frac{\partial}{\partial x} & \frac{\partial}{\partial y} & \frac{\partial}{\partial z} \\ \varphi u & \varphi v & \varphi w \end{vmatrix} \\ 
    &= \mathbf{i} \left[ \frac{\partial(\varphi w)}{\partial y} - \frac{\partial(\varphi v)}{\partial z} \right] + \mathbf{j} \left[ \frac{\partial(\varphi u)}{\partial z} - \frac{\partial(\varphi w)}{\partial x} \right] + \mathbf{k} \left[ \frac{\partial(\varphi v)}{\partial x} - \frac{\partial(\varphi u)}{\partial y} \right] \\ 
    &= \varphi \left[ \mathbf{i} \left( \frac{\partial w}{\partial y} - \frac{\partial v}{\partial z} \right) + \dots \right] + \begin{vmatrix} \mathbf{i} & \mathbf{j} & \mathbf{k} \\ \frac{\partial \varphi}{\partial x} & \frac{\partial \varphi}{\partial y} & \frac{\partial \varphi}{\partial z} \\ u & v & w \end{vmatrix} 
\end{aligned}$$

最终得到公式:
$$\nabla \times (\varphi \mathbf{f}) = \varphi \nabla \times \mathbf{f} + \nabla \varphi \times \mathbf{f} \quad (72)$$

通过将 $\nabla$ 视为一个向量微分算子,可以很容易地得到这个结果。

例 29. 证明梯度的旋度为零。

$$\nabla \times (\nabla \varphi) = \begin{vmatrix} \mathbf{i} & \mathbf{j} & \mathbf{k} \\ \frac{\partial}{\partial x} & \frac{\partial}{\partial y} & \frac{\partial}{\partial z} \\ \frac{\partial \varphi}{\partial x} & \frac{\partial \varphi}{\partial y} & \frac{\partial \varphi}{\partial z} \end{vmatrix} = \mathbf{i} \left( \frac{\partial^2 \varphi}{\partial y \partial z} - \frac{\partial^2 \varphi}{\partial z \partial y} \right) + \mathbf{j} \left( \frac{\partial^2 \varphi}{\partial z \partial x} - \frac{\partial^2 \varphi}{\partial x \partial z} \right) + \mathbf{k} \left( \frac{\partial^2 \varphi}{\partial x \partial y} - \frac{\partial^2 \varphi}{\partial y \partial x} \right)$$

因此:
$$\nabla \times \nabla \varphi = 0 \quad (73)$$

前提是 $\varphi$ 具有连续的二阶导数。

(注:根据偏导数无关次序的原则,交叉偏导数相等,故结果为零)

例 30. 证明旋度的散度为零。
$$\nabla \cdot (\nabla \times \mathbf{f}) = \frac{\partial}{\partial x} \left( \frac{\partial w}{\partial y} - \frac{\partial v}{\partial z} \right) + \frac{\partial}{\partial y} \left( \frac{\partial u}{\partial z} - \frac{\partial w}{\partial x} \right) + \frac{\partial}{\partial z} \left( \frac{\partial v}{\partial x} - \frac{\partial u}{\partial y} \right)$$

$$= \frac{\partial^2 w}{\partial y \partial x} - \frac{\partial^2 v}{\partial z \partial x} + \frac{\partial^2 u}{\partial z \partial y} - \frac{\partial^2 w}{\partial x \partial y} + \frac{\partial^2 v}{\partial x \partial z} - \frac{\partial^2 u}{\partial y \partial z}$$

由此得到:
$$\nabla \cdot (\nabla \times \mathbf{f}) = 0 \quad (74)$$

例 31. $(\mathbf{u} \cdot \nabla)\mathbf{v}$ 是什么意思?我们首先计算 $\mathbf{u}$ 与 $\nabla$ 的点积。这产生了一个标量微分算子:

$$u_x \frac{\partial}{\partial x} + u_y \frac{\partial}{\partial y} + u_z \frac{\partial}{\partial z}$$

然后我们用它作用于向量 $\mathbf{v}$,得到:
$$(\mathbf{u} \cdot \nabla)\mathbf{v} = u_x \frac{\partial \mathbf{v}}{\partial x} + u_y \frac{\partial \mathbf{v}}{\partial y} + u_z \frac{\partial \mathbf{v}}{\partial z}$$

因此:
$$d\mathbf{f} = \frac{\partial \mathbf{f}}{\partial x} dx + \frac{\partial \mathbf{f}}{\partial y} dy + \frac{\partial \mathbf{f}}{\partial z} dz$$

$$= dx \frac{\partial \mathbf{f}}{\partial x} + dy \frac{\partial \mathbf{f}}{\partial y} + dz \frac{\partial \mathbf{f}}{\partial z}$$

(注:此处 $d\mathbf{f}$ 描述了向量场 $\mathbf{f}$ 随位移 $d\mathbf{r}$ 的全微分变化量)

重点概念解析

两个为零的恒等式:

$\nabla \times \nabla \varphi = 0$:梯度的场一定是“无旋”的。在物理上,这意味着保守场(如重力场)没有涡流。

$\nabla \cdot (\nabla \times \mathbf{f}) = 0$:旋度的场一定是“无源”的。这意味着涡旋线既没有起点也没有终点,总是闭合的。

方向导数算子 $(\mathbf{u} \cdot \nabla)$:

这个算子在流体力学中极其重要,它描述了物理量随流体运动而产生的变化(即对流项)。例如在纳维-斯托克斯方程中,速度场的自对流项就是用这种形式表示的。

23. 曲线坐标系 (Curvilinear Coordinates)

数学家、物理学家或工程师经常发现,使用除了熟悉的直角笛卡尔坐标系以外的其他坐标系会更加方便。如果他正在处理球体问题,他可能会发现用球面坐标 $r, \theta, \varphi$ 来描述空间中点的位置更为得当(见图 31)。

让我们注意以下定义:

球面:$x^2 + y^2 + z^2 = r^2$

圆锥面:$z / (x^2 + y^2 + z^2)^{1/2} = \cos \theta$

平面:$y/x = \tan \varphi$

这些面都通过点 $P(r, \theta, \varphi)$。我们可以考虑如下从 $x-y-z$ 坐标系到 $r-\theta-\varphi$ 坐标系的变换:

$r = (x^2 + y^2 + z^2)^{1/2}$

$\theta = \cos^{-1} \frac{z}{(x^2 + y^2 + z^2)^{1/2}}$

$\varphi = \tan^{-1} \frac{y}{x}$

表面 $r = c_1$, $\theta = c_2$, $\varphi = c_3$ 分别代表球面、圆锥面和平面。空间中除了原点外的任何点 $P$,都恰好会有每种类型的一个面通过它,点 $P$ 的坐标由常数 $c_1, c_2, c_3$ 决定。

球面与圆锥面的交线是一个圆,即纬线圈,在该圆上点 $P$ 具有单位切向量 $\mathbf{e}_\varphi$。这个圆被称为 $\varphi$-曲线,因为在该曲线上 $r$ 和 $\theta$ 保持不变,只有坐标 $\varphi$ 随着我们沿曲线移动而改变。

球面与平面的交线产生 $\theta$-曲线(经线圈);而圆锥面与平面的交线产生从原点通过 $P$ 点的直线,即 $r$-曲线。

在 $P$ 点的三个单位向量 $\mathbf{e}_r, \mathbf{e}_\theta, \mathbf{e}_\varphi$ 是互相垂直的,可以被视为在 $P$ 点邻域内形成了一个坐标系的基底。与 $\mathbf{i, j, k}$ 不同的是,它们不是固定不变的,因为当我们从一点移动到另一点时,它们的方向会发生改变。

因此,当处理球面坐标时,我们可以预见到梯度 (gradient)、散度 (divergence)、旋度 (curl) 和拉普拉斯算子 (Laplacian) 的公式会变得更加复杂。

24. Frenet-Serret 公式

欧几里得空间中的三维曲线可以用位置向量终点的轨迹表示,如下所示:

$$\mathbf{r}(t) = x(t)\mathbf{i} + y(t)\mathbf{j} + z(t)\mathbf{k} \quad (93)$$

其中 $t$ 是一个在值集 $t_0 \le t \le t_1$ 范围内变化的参数。我们假设 $x(t), y(t), z(t)$ 具有各阶连续导数,并且可以在曲线任何点的邻域内展开为泰勒级数。

我们在第 2 章第 16 节中已经看到,$\frac{d\mathbf{r}}{ds}$ 是曲线的单位切向量。令 $\mathbf{t} = \frac{d\mathbf{r}}{ds}$。现在 $\mathbf{t}$ 是一个单位向量,因此它的导数垂直于 $\mathbf{t}$。此外,该导数 $\frac{d\mathbf{t}}{ds}$ 告诉我们当我们沿曲线移动时,单位切向量改变方向的速度。曲线的主法线相应地由下式定义:

$$\frac{d\mathbf{t}}{ds} = \kappa \mathbf{n} \quad (94)$$

其中 $\kappa$ 是 $\frac{d\mathbf{t}}{ds}$ 的模,被称为曲率 (curvature)。曲率的倒数 $\rho = 1/\kappa$ 被称为曲率半径 (radius of curvature)。重要的是要注意,方程 (94) 同时定义了 $\kappa$ 和 $\mathbf{n}$,$\kappa$ 是 $\frac{d\mathbf{t}}{ds}$ 的长度,而 $\mathbf{n}$ 是平行于 $\frac{d\mathbf{t}}{ds}$ 的单位向量。在曲线的任意点 $P$,我们现在拥有两个彼此成直角的向量 $\mathbf{t}, \mathbf{n}$(见图 37)。

\begin{figure}[htbp] 
    \centering
    \includegraphics[width=0.8\textwidth]{images/ref/37.png} 
    \caption{\textbf{曲线的主法线}}
\end{figure}

在曲线的任意点 $P$,我们现在拥有两个彼此垂直的向量 $\mathbf{t}, \mathbf{n}$(见图 37)。通过定义第三个与 $\mathbf{t}$ 和 $\mathbf{n}$ 均垂直的向量,我们可以在 $P$ 点建立一个局部坐标系。我们将该向量定义为副法线 (binormal) 向量:

$$\mathbf{b} = \mathbf{t} \times \mathbf{n}$$

所有与曲线在 $P$ 点相关的向量都可以写成这三个基本向量 $\mathbf{t}, \mathbf{n}, \mathbf{b}$ 的线性组合,它们在 $P$ 点构成了一个三面体 (trihedral)。

现在让我们计算 $\frac{d\mathbf{b}}{ds}$ 和 $\frac{d\mathbf{n}}{ds}$。由于 $\mathbf{b}$ 是单位向量,其导数垂直于 $\mathbf{b}$,因此位于 $\mathbf{t}$ 和 $\mathbf{n}$ 构成的平面内。此外,由于 $\mathbf{b} \cdot \mathbf{t} = 0$,通过求导我们得到:

$$\frac{d\mathbf{b}}{ds} \cdot \mathbf{t} + \kappa \mathbf{b} \cdot \mathbf{n} = 0 \text{}$$

即 $\frac{d\mathbf{b}}{ds} \cdot \mathbf{t} = 0$。因此 $\frac{d\mathbf{b}}{ds}$ 也垂直于 $\mathbf{t}$,这意味着 $\frac{d\mathbf{b}}{ds}$ 必须平行于 $\mathbf{n}$。由此可知:

$$\frac{d\mathbf{b}}{ds} = \tau \mathbf{n} \text{}$$

其中 $\tau$ 被定义为 $\frac{d\mathbf{b}}{ds}$ 的模(带有方向符号),被称为曲线的挠率 (torsion)。

主法线导数 $\frac{d\mathbf{n}}{ds}$ 的推导

利用关系式 $\mathbf{n} = \mathbf{b} \times \mathbf{t}$,对其求导:
$$\frac{d\mathbf{n}}{ds} = \mathbf{b} \times \frac{d\mathbf{t}}{ds} + \frac{d\mathbf{b}}{ds} \times \mathbf{t} \text{}$$

代入已知的 $\frac{d\mathbf{t}}{ds} = \kappa \mathbf{n}$ 和 $\frac{d\mathbf{b}}{ds} = \tau \mathbf{n}$:

$$\frac{d\mathbf{n}}{ds} = \mathbf{b} \times (\kappa \mathbf{n}) + (\tau \mathbf{n}) \times \mathbf{t} \text{}$$

根据叉乘性质 $\mathbf{b} \times \mathbf{n} = -\mathbf{t}$ 以及 $\mathbf{n} \times \mathbf{t} = -\mathbf{b}$:

$$\frac{d\mathbf{n}}{ds} = -\kappa \mathbf{t} - \tau \mathbf{b} \text{}$$

总结:著名的 Frenet-Serret 公式组
$$\begin{cases} 
    \frac{d\mathbf{t}}{ds} = \kappa \mathbf{n} \\ \frac{d\mathbf{n}}{ds} = -\kappa \mathbf{t} - \tau \mathbf{b} \\ \frac{d\mathbf{b}}{ds} = \tau \mathbf{n} 
\end{cases} \quad (95) \text{}$$

这些公式表明,曲线在空间中的形态完全由曲率 $\kappa$(反映弯曲程度)和挠率 $\tau$(反映扭曲程度)决定。

例 42:圆柱螺旋线 (Circular Helix) 的代数推导

给定圆柱螺旋线的参数方程为:
$$\mathbf{r}(t) = a \cos t \, \mathbf{i} + a \sin t \, \mathbf{j} + bt \, \mathbf{k} \quad$$

1. 计算弧长变化率 $\frac{ds}{dt}$

首先,我们需要求出速度向量(对参数 $t$ 求导):
$$\mathbf{v} = \frac{d\mathbf{r}}{dt} = -a \sin t \, \mathbf{i} + a \cos t \, \mathbf{j} + b \, \mathbf{k} \quad$$

弧长 $s$ 对时间 $t$ 的变化率为该向量的模:
$$\frac{ds}{dt} = \left| \frac{d\mathbf{r}}{dt} \right| = \sqrt{(-a \sin t)^2 + (a \cos t)^2 + b^2} = \sqrt{a^2(\sin^2 t + \cos^2 t) + b^2} = \sqrt{a^2 + b^2} \quad$$

由此得:$\frac{dt}{ds} = (a^2 + b^2)^{-1/2}$。

2. 求单位切向量 $\mathbf{t}$

根据定义 $\mathbf{t} = \frac{d\mathbf{r}}{ds} = \frac{d\mathbf{r}}{dt} \cdot \frac{dt}{ds}$:

$$\mathbf{t} = (-a \sin t \, \mathbf{i} + a \cos t \, \mathbf{j} + b \, \mathbf{k})(a^2 + b^2)^{-1/2} \quad$$

验证其大小:$\mathbf{t} \cdot \mathbf{t} = (a^2 \sin^2 t + a^2 \cos^2 t + b^2)(a^2 + b^2)^{-1} = 1$。

3. 求曲率 $\kappa$ 和主法向量 $\mathbf{n}$

利用 Frenet 公式 $\frac{d\mathbf{t}}{ds} = \kappa \mathbf{n}$。我们先对 $t$ 求导再转换:

$$\frac{d\mathbf{t}}{ds} = \frac{d\mathbf{t}}{dt} \cdot \frac{dt}{ds} = [(-a \cos t \, \mathbf{i} - a \sin t \, \mathbf{j})(a^2 + b^2)^{-1/2}] \cdot (a^2 + b^2)^{-1/2}$$

$$\frac{d\mathbf{t}}{ds} = (-a \cos t \, \mathbf{i} - a \sin t \, \mathbf{j})(a^2 + b^2)^{-1} \quad$$

由于 $\kappa$ 是该向量的模,且 $\mathbf{n}$ 是单位向量:

曲率:
$\kappa = \left| \frac{d\mathbf{t}}{ds} \right| = \sqrt{a^2 \cos^2 t + a^2 \sin^2 t}(a^2 + b^2)^{-1} = a(a^2 + b^2)^{-1} \quad$

主法向量:$\mathbf{n} = \frac{1}{\kappa} \frac{d\mathbf{t}}{ds} = -\cos t \, \mathbf{i} - \sin t \, \mathbf{j} \quad$

4. 求副法向量 $\mathbf{b}$

通过叉乘计算 $\mathbf{b} = \mathbf{t} \times \mathbf{n}$:

$$\mathbf{b} = \begin{vmatrix} \mathbf{i} & \mathbf{j} & \mathbf{k} \\ -a \sin t & a \cos t & b \\ -\cos t & -\sin t & 0 \end{vmatrix} (a^2 + b^2)^{-1/2}$$

展开行列式:
$$\mathbf{b} = [ (0 - (-b \sin t))\mathbf{i} - (0 - (-b \cos t))\mathbf{j} + (a \sin^2 t - (-a \cos^2 t))\mathbf{k} ] (a^2 + b^2)^{-1/2}$$

$$\mathbf{b} = (b \sin t \, \mathbf{i} - b \cos t \, \mathbf{j} + a\mathbf{k})(a^2 + b^2)^{-1/2} \quad$$

5. 求挠率 $\tau$

利用公式 $\frac{d\mathbf{b}}{ds} = \tau \mathbf{n}$。对 $\mathbf{b}$ 求导:

$$\frac{d\mathbf{b}}{ds} = \frac{d\mathbf{b}}{dt} \cdot \frac{dt}{ds} = [(b \cos t \, \mathbf{i} + b \sin t \, \mathbf{j})(a^2 + b^2)^{-1/2}] \cdot (a^2 + b^2)^{-1/2}$$

$$\frac{d\mathbf{b}}{ds} = (b \cos t \, \mathbf{i} + b \sin t \, \mathbf{j})(a^2 + b^2)^{-1}$$

观察 $\mathbf{n} = -\cos t \, \mathbf{i} - \sin t \, \mathbf{j}$,可以写成:

$$\frac{d\mathbf{b}}{ds} = -b(a^2 + b^2)^{-1} \mathbf{n}$$

因此,挠率为:
$$\tau = -b(a^2 + b^2)^{-1}$$

25. 基本平面 (Fundamental Planes)

在曲线上的 $P$ 点,由 $\mathbf{t, n, b}$ 向量两两确定的三个平面非常重要:

密切平面 (Osculating Plane):

包含切向量 $\mathbf{t}$ 和主法线 $\mathbf{n}$ 的平面。

因为它垂直于副法线 $\mathbf{b}$,若 $\mathbf{s}$ 为平面内任意一点的变化向量,则方程为:
$$(\mathbf{s} - \mathbf{r}) \cdot \mathbf{b} = 0 \quad (96) \text{}$$

法平面 (Normal Plane):

通过 $P$ 点并垂直于切向量 $\mathbf{t}$ 的平面。

方程为:
$$(\mathbf{s} - \mathbf{r}) \cdot \mathbf{t} = 0 \quad (97) \text{}$$

从切平面/展直平面 (Rectifying Plane):

通过 $P$ 点并垂直于主法线 $\mathbf{n}$ 的平面。

方程为:
$$(\mathbf{s} - \mathbf{r}) \cdot \mathbf{n} = 0 \quad (98) \text{}$$
