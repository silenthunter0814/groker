\chapter{向量微分}

\section{微分向量分析}

16. 向量的微分。 让我们考虑如下向量场:
$$\mathbf{u} = \alpha(x, y, z, t)\mathbf{i} + \beta(x, y, z, t)\mathbf{j} + \gamma(x, y, z, t)\mathbf{k} \quad (42)$$

在任意点 $P(x, y, z)$ 及任意时间 $t$,式 (42) 定义了一个向量。如果我们固定点 $P$,由于分量 $\alpha, \beta, \gamma$ 具有时间相关性,向量 $\mathbf{u}$ 仍会发生变化。如果我们固定时间,我们会注意到在点 $P(x, y, z)$ 处的向量通常与在以下点处的向量不同:

$$Q(x + dx, y + dy, z + dz)$$

在微积分中,学生已经学过如何计算 $x, y, z, t$ 的单函数变化。那么在向量的情况下,我们会遇到什么困难吗?实际上完全没有,因为我们很容易注意到,当且仅当向量的分量发生变化时,$\mathbf{u}$ 才会发生变化。因此,$\alpha(x, y, z, t)$ 的变化会产生 $\mathbf{u}$ 在 $x$ 方向上的变化,同理,$\beta$ 和 $\gamma$ 的变化分别产生 $\mathbf{u}$ 在 $y$ 和 $z$ 方向上的变化。由此,我们得出以下定义:

$$\mathbf{du} = d\alpha\,\mathbf{i} + d\beta\,\mathbf{j} + d\gamma\,\mathbf{k} \quad (43)$$

$$\begin{aligned} \mathbf{du} = &\left( \frac{\partial \alpha}{\partial x} dx + \frac{\partial \alpha}{\partial y} dy + \frac{\partial \alpha}{\partial z} dz + \frac{\partial \alpha}{\partial t} dt \right) \mathbf{i} \\ &+ \left( \frac{\partial \beta}{\partial x} dx + \frac{\partial \beta}{\partial y} dy + \frac{\partial \beta}{\partial z} dz + \frac{\partial \beta}{\partial t} dt \right) \mathbf{j} \\ &+ \left( \frac{\partial \gamma}{\partial x} dx + \frac{\partial \gamma}{\partial y} dy + \frac{\partial \gamma}{\partial z} dz + \frac{\partial \gamma}{\partial t} dt \right) \mathbf{k} \end{aligned}$$

例如,设 $\mathbf{r} = x\mathbf{i} + y\mathbf{j} + z\mathbf{k}$ 为三维空间中运动质点 $P(x, y, z)$ 的位置向量。则:
$$\mathbf{dr} = dx\,\mathbf{i} + dy\,\mathbf{j} + dz\,\mathbf{k}$$

以及
$$\mathbf{v} = \frac{d\mathbf{r}}{dt} = \frac{dx}{dt}\mathbf{i} + \frac{dy}{dt}\mathbf{j} + \frac{dz}{dt}\mathbf{k} \quad (44)$$

$$\mathbf{a} = \frac{d^2\mathbf{r}}{dt^2} = \frac{d^2x}{dt^2}\mathbf{i} + \frac{d^2y}{dt^2}\mathbf{j} + \frac{d^2z}{dt^2}\mathbf{k} \quad (45)$$

根据定义,方程 (44) 和 (45) 分别是质点的速度和加速度。我们假设向量 $\mathbf{i, j, k}$ 在空间中保持固定。

如果向量 $\mathbf{u}$ 仅取决于单个变量 $t$,我们可以定义:
$$\frac{d\mathbf{u}}{dt} = \lim_{\Delta t \to 0} \frac{\mathbf{u}(t + \Delta t) - \mathbf{u}(t)}{\Delta t} \quad (46)$$

很容易验证 (46) 与 (43) 是等价的。

例 14。 考虑一个质点 $P$ 在半径为 $r$ 的圆上以恒定角速度 $\omega = \frac{d\theta}{dt}$ 运动(见图 28)。我们注意到:

$$\mathbf{r} = r \cos \theta\,\mathbf{i} + r \sin \theta\,\mathbf{j}$$

因此:
$$\mathbf{v} = \frac{d\mathbf{r}}{dt} = (-r \sin \theta\,\mathbf{i} + r \cos \theta\,\mathbf{j}) \frac{d\theta}{dt}$$

以及:
$$\mathbf{a} = \frac{d\mathbf{v}}{dt} = \frac{d^2\mathbf{r}}{dt^2} = (-r \cos \theta\,\mathbf{i} - r \sin \theta\,\mathbf{j}) \left( \frac{d\theta}{dt} \right)^2$$

所以加速度为:
$$\mathbf{a} = -\omega^2\mathbf{r} \quad (47)$$

点 $P$ 具有指向原点的加速度,其大小为恒定的 $\omega^2r$。这个加速度是由于速度向量以恒定速率改变方向而产生的;它被称为向心加速度。

例 15。 设 $P$ 为空间曲线上的任意一点(见图 29):

$$\begin{aligned} x &= x(s) \\ y &= y(s) \\ z &= z(s) \end{aligned} \text{}$$

其中 $s$ 是从某个固定点 $Q$ 开始测量的弧长。那么:

$$\mathbf{r} = x(s)\mathbf{i} + y(s)\mathbf{j} + z(s)\mathbf{k} \quad (48) \text{}$$

因此:
$$\frac{d\mathbf{r}}{ds} = \frac{dx}{ds}\mathbf{i} + \frac{dy}{ds}\mathbf{j} + \frac{dz}{ds}\mathbf{k} \quad (49) \text{}$$

并且由微积分可知:
$$\begin{aligned} \frac{d\mathbf{r}}{ds} \cdot \frac{d\mathbf{r}}{ds} &= \left( \frac{dx}{ds} \right)^2 + \left( \frac{dy}{ds} \right)^2 + \left( \frac{dz}{ds} \right)^2 \\ &= \frac{dx^2 + dy^2 + dz^2}{ds^2} \equiv 1 \end{aligned} \text{}$$

因此 $\frac{d\mathbf{r}}{ds}$ 是一个单位向量。当 $\Delta s \to 0$ 时,$\frac{\Delta\mathbf{r}}{\Delta s}$ 的位置趋近于点 $P$ 处的切线。因此,(49) 代表了空间曲线 (48) 的单位切向量。

17. 微分规则。 考虑:

$$\begin{aligned} \varphi(t) &= \mathbf{u}(t) \cdot \mathbf{v}(t) \\ \varphi(t + \Delta t) - \varphi(t) &= \mathbf{u}(t + \Delta t) \cdot \mathbf{v}(t + \Delta t) - \mathbf{u}(t) \cdot \mathbf{v}(t) \end{aligned}$$

现在:
$$\begin{aligned} \mathbf{u}(t + \Delta t) &= \mathbf{u}(t) + \Delta\mathbf{u} \\ \mathbf{v}(t + \Delta t) &= \mathbf{v}(t) + \Delta\mathbf{v} \end{aligned}$$

(见图 27),所以:
$$\frac{\varphi(t + \Delta t) - \varphi(t)}{\Delta t} = \mathbf{u} \cdot \frac{\Delta\mathbf{v}}{\Delta t} + \frac{\Delta\mathbf{u}}{\Delta t} \cdot \mathbf{v} + \frac{\Delta\mathbf{u}}{\Delta t} \cdot \Delta\mathbf{v}$$

取极限,我们得到:
$$\frac{d(\mathbf{u} \cdot \mathbf{v})}{dt} = \mathbf{u} \cdot \frac{d\mathbf{v}}{dt} + \frac{d\mathbf{u}}{dt} \cdot \mathbf{v} \quad (50)$$

类似地:
$$\frac{d(\mathbf{u} \times \mathbf{v})}{dt} = \mathbf{u} \times \frac{d\mathbf{v}}{dt} + \frac{d\mathbf{u}}{dt} \times \mathbf{v} \quad (51)$$

$$\frac{d(f\mathbf{u})}{dt} = f\frac{d\mathbf{u}}{dt} + \frac{df}{dt}\mathbf{u} \quad (52)$$

请注意这些公式与微积分规则的一致性。

例 16。 设 $\mathbf{u}(t)$ 为一个模(大小)恒定的向量。因此:

$$\mathbf{u} \cdot \mathbf{u} = u^2 = \text{常数}$$

通过微分,我们得到:

$$\begin{aligned} \mathbf{u} \cdot \frac{d\mathbf{u}}{dt} + \frac{d\mathbf{u}}{dt} \cdot \mathbf{u} &= 0 \\ 
\mathbf{u} \cdot \frac{d\mathbf{u}}{dt} &= 0 \end{aligned}$$

因此,要么 $\frac{d\mathbf{u}}{dt} = 0$,要么 $\frac{d\mathbf{u}}{dt}$ 与 $\mathbf{u}$ 垂直。这是一个重要的结果,学生应当充分理解。读者应给出该定理的几何证明。

例 17。在所有情况下,$\mathbf{u} \cdot \mathbf{u} = u^2$,其中 $u$ 是 $\mathbf{u}$ 的长度。微分得:
$$2\mathbf{u} \cdot \frac{d\mathbf{u}}{dt} = 2u \frac{du}{dt}$$

以及:
$$\mathbf{u} \cdot \frac{d\mathbf{u}}{dt} = u \frac{du}{dt} \quad (53)$$

这一结果并非显而易见,因为 $|d\mathbf{u}| \neq du$。

例 18:平面运动。现在 $\mathbf{r} = r\mathbf{R}$,其中 $\mathbf{R}$ 是一个单位向量(见图 30)。因此:
$$\mathbf{v} = \frac{d\mathbf{r}}{dt} = \frac{dr}{dt}\mathbf{R} + r\frac{d\mathbf{R}}{dt}$$

由于 $\mathbf{R}$ 是单位向量,根据例 16,$\frac{d\mathbf{R}}{dt}$ 与 $\mathbf{R}$ 垂直。同时,通过对 $\mathbf{R} = \cos \theta\,\mathbf{i} + \sin \theta\,\mathbf{j}$ 进行微分,我们可以很容易验证 $|\frac{d\mathbf{R}}{dt}| = \frac{d\theta}{dt}$。因此 $\mathbf{v} = \frac{dr}{dt}\mathbf{R} + r\frac{d\theta}{dt}\mathbf{P}$,其中 $\mathbf{P}$ 是垂直于 $\mathbf{R}$ 的单位向量。再次微分得:
$$\mathbf{a} = \frac{d\mathbf{v}}{dt} = \frac{d^2r}{dt^2}\mathbf{R} + \frac{dr}{dt}\frac{d\mathbf{R}}{dt} + \frac{dr}{dt}\frac{d\theta}{dt}\mathbf{P} + r\frac{d^2\theta}{dt^2}\mathbf{P} + r\frac{d\theta}{dt}\frac{d\mathbf{P}}{dt}$$

或:
$$\mathbf{a} = \left[ \frac{d^2r}{dt^2} - r\left( \frac{d\theta}{dt} \right)^2 \right] \mathbf{R} + \left[ 2\frac{dr}{dt}\frac{d\theta}{dt} + r\frac{d^2\theta}{dt^2} \right] \mathbf{P}$$

因为:
$$\frac{d\mathbf{P}}{dt} = -\frac{d\theta}{dt}\mathbf{R} \quad (54)$$

