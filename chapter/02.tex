\chapter{函数和极限}

函数是数学中一个核心概念,它描述了变量之间的一种对应关系。简单来说,函数将一个输入值(自变量)映射到一个输出值(因变量)。函数的概念起源于17世纪的莱布尼茨和牛顿,用于描述物理现象,如运动和变化。

传统的函数表示是一种泛化的表示,试图包容一切,解决一切问题,自变量和函数值(因变量)之间的关系抽象成规则,试图解决一切假想中的问题;因此在描述定义、定理时需要晦涩的前提限制条件,并形成抽象的形式化定义。本文或格洛克描述的函数只有一种,自变量和函数值之间的关系由自变量表达式确定,保持具体而简单。

极限是微积分的基础概念,它描述了函数在某个点附近的行为,即使函数在该点未定义或不连续。极限的概念由柯西和魏尔斯特拉斯在19世纪正式化,用于处理无穷小和无穷大。

基于格洛克代数空间,定义域的区间和函数的连续性进行了重新表述,极限概念被重新定义,一切变得简单而直观。


\section{理解函数}

函数是数学中描述对应关系的一种基本且非常重要的概念。通常写作 $y = f(x)$。

\medskip

\textbf{函数的关键特点}

唯一性对应 (Uniqueness):对于定义域内的每一个输入值 $x$,函数都只能对应一个唯一的输出值 $f(x)$。

定义域 (Domain):所有允许的输入值 $x$ 的集合。

例如,在函数 $f(x) = \frac{1}{x}$ 中,$x$ 不能为 $0$,所以定义域是除 $0$ 以外的所有实数。

值域 (Range):值域是实际输出值的集合。

\medskip

函数的输入输出关系通过自变量表达式来描述,通常有几何图形来对应表示。例如函数 $f(x) = x^2 + 1$,在几何图形上表现为抛物线。

\medskip

\textbf{定义域区间}

函数的定义域通常是数轴上的一个或多个连续的数值范围,因此我们常用区间符号来表示它。

圆括号 $(a, b)$表示不包含端点(开区间)。表示 $a < x < b$,以及区间边界是 $\infty$ 或 $-\infty$ 的情况。

方括号 $[a, b]$表示包含端点(闭区间)。表示 $a \le x \le b$。

\medskip

\textbf{开区间转换为闭区间}

依据格洛克无穷大和无穷小定义的性质(4),有

$\qquad (a, b) \Rightarrow [a+\epsilon, b-\epsilon]\qquad (a, \infty) \Rightarrow [a+\epsilon, \infty)$

$\infty)$ 既可以理解为开区间也可以理解为闭区间,根据具体的问题进行不同的理解。

\medskip

\textbf{分段函数}

分段函数将定义域分成若干互不相交的子区间,并在每个子区间上定义一个子函数。

分段函数常用大括号表示,例如:

$$f(x) = |x| = \begin{cases} -x, & \text{if } x < 0 \\ x, & \text{if } x \geq 0 \end{cases}$$

$$f(x) = \frac 1 x = \begin{cases} \frac 1 x, & \text{if } x < 0 \\ \frac 1 x, & \text{if } x > 0 \end{cases}$$

$$f(x) = \begin{cases} x^2 + 1, & \text{if } x < 1 \\ 3x - 1, & \text{if } x \geq 1 \end{cases}$$

还有一些常见的特殊分段函数:

阶跃函数 (Step Function): 分段函数由常数函数组成。例如,单位阶跃函数(Unit Step Function)。

分段线性函数 (Piecewise Linear Function): 分段函数由线段(线性函数)组成。

通过以上讨论,我们现在可以明确:\textbf{函数的概念和性质总是作用于单一定义域区间}。

\medskip

\textbf{复合函数}

复合函数指的是将两个或多个函数组合起来,形成一个新的函数。具体来说,如果有函数 $(f)$ 和 $(g)$,则它们的复合函数记作 $f \circ g$,定义为 $(f \circ g)(x) = f(g(x))$
,其中,先计算 $g(x)$,再计算 $f$。

复合函数 $f \circ g$ 的定义域同时受到 $g$ 和 $f$ 定义域的双重约束。

$f \circ g$ 和 $g \circ f$ 通常并不相同,也就是说复合函数的顺序很重要。

\medskip

\begin{example} 假设 $f(x) = x^2,\; g(x) = x + 1$。
    \begin{align*}
        (f \circ g)(x) &= f(g(x)) = f(x+1) = (x+1)^2 = x^2 + 2x + 1 \\
        (g \circ f)(x) &= g(f(x)) = g(x^2) = x^2 + 1
    \end{align*}
\end{example}

\textbf{反函数}

反函数就是“逆转”原函数的操作,将输出变回输入。函数 $f(x)$ 的反函数通常表示为 $f^{-1}(x)$。

\medskip

反函数存在条件

(1)一对一性: 每个 $x$ 对应唯一的 $y$,并且每个 $y$ 对应唯一的 $x$。

(2)覆盖性: 函数的值域覆盖整个定义域。

\medskip

反函数的性质

(1)$f(f^{-1}(x)) = f^{-1}(f(x)) = x$。

(2)原函数和反函数的图形关于直线 $y = x$ 对称。

\medskip

\begin{example} 假设 $f(x) = 2x + 1,\; g(x) = \frac{1}{x}$,求反函数并验证恒等性质。

    \textbf{解:} 设 $y = f(x) = 2x + 1$,那么 $x = \frac{y - 1}{2}$,交换变量 $x,\;y$,得到 $y = \frac{x - 1}{2}$,反函数 $$f^{-1}(x) = \frac{x - 1}{2}$$
    恒等式 
    \begin{align*}
        f(f^{-1}(x)) &= f(\frac{x - 1}{2}) = 2\cdot\frac{x - 1}{2} + 1 = x \\
        f^{-1}(f(x)) &= f^{-1}(2x+1) = \frac{(2x+1) - 1}{2} = x
    \end{align*}
    所以 $f(f^{-1}(x)) = f^{-1}(f(x)) = x$。

    设 $y = g(x) = \frac{1}{x}$,那么 $x = \frac{1}{y}$,交换变量 $x,\;y$,得到 $y = \frac{1}{x}$,反函数 $$f^{-1}(x) = \frac{1}{x}$$
    恒等式 
    \begin{align*}
        f(f^{-1}(x)) &= f(\frac{1}{x}) = \frac{1}{1/x} = x \\
        f^{-1}(f(x)) &= f^{-1}(\frac{1}{x}) = \frac{1}{1/x} = x
    \end{align*}
    所以 $f(f^{-1}(x)) = f^{-1}(f(x)) = x$。
\end{example}

\section{参数方程}
参数方程是一个非常重要的数学概念,尤其在描述运动轨迹和复杂曲线时非常方便。

函数关系是参数方程的一种特殊情况,而参数方程提供了描述更广泛几何图形和运动的更灵活方式。

\medskip

在平面直角坐标系中,我们通常用一个含有 $x$ 和 $y$ 的方程(普通方程)来表示一条曲线。

普通方程(或笛卡尔方程): $y = f(x)$ 或 $F(x, y) = 0$。

参数方程(Parametric Equation)则是引入一个参数(通常用 $t$ 或 $\theta$ 表示),将曲线上点的坐标 $(x, y)$ 分别表示成这个参数的函数:
$$
\begin{cases}
    x = f(t) \\
    y = g(t)    
\end{cases} \quad (t \text{ 为参数})
$$

当参数 $t$ 在一个指定的范围内变化时,对应的 $(x, y)$ 坐标点就描绘出一条曲线。在物理学中,参数 $t$ 常常代表时间。

\medskip

使用参数方程有几个显著的优点:

描述运动轨迹更自然: 如果 $t$ 代表时间,那么 $x=f(t)$ 和 $y=g(t)$ 就直接描述了一个物体在 $t$ 时刻的横坐标和纵坐标,非常适合描述质点运动。

表示复杂曲线更简单: 有些曲线的普通方程非常复杂,甚至无法写出,但它们的参数方程却相对简单。例如摆线 (Cycloid)。

表示非函数关系: 普通方程 $y=f(x)$ 只能表示一个 $x$ 对应一个 $y$ 的函数关系。但对于圆这样的曲线(一个 $x$ 对应两个 $y$),用参数方程表示更为简洁。

\medskip

常见曲线的参数方程

\begin{table}[h]
    \centering
    \label{tab:parametric_equations_minimal}
    \begin{tabular}{llll}
        \textbf{曲线类型} & \textbf{普通方程} & \textbf{参数方程} & \textbf{参数的范围} \\
        \midrule % 仅在表头下方使用这条中等粗细横线
        直线 & $y - y_0 = m(x - x_0)$ & $\begin{cases} x = x_0 + a t \\ y = y_0 + b t \end{cases}$ & $t \in (-\infty, \infty)$ \\
        圆 & $(x-h)^2 + (y-k)^2 = r^2$ & $\begin{cases} x = h + r \cos \theta \\ y = k + r \sin \theta \end{cases}$ & $\theta \in [0, 2\pi)$ \\
        椭圆 & $\frac{x^2}{a^2} + \frac{y^2}{b^2} = 1$ & $\begin{cases} x = a \cos \theta \\ y = b \sin \theta \end{cases}$ & $\theta \in [0, 2\pi)$ \\
        抛物线 & $y^2 = 2px$ & $\begin{cases} x = \frac{p}{2} t^2 \\ y = p t \end{cases}$ & $t \in (-\infty, \infty)$ \\
    \end{tabular}
\end{table}

\medskip

\textbf{普通方程与参数方程的互化}

(1)参数方程化为普通方程:核心思想是消去参数 $t$。

\begin{example} 将 $\begin{cases} x = 2t + 1 \\ y = 4t^2 \end{cases}$ 化为普通方程。

    从第一个方程解出 $t$: $t = \frac{x-1}{2}$
    
    代入第二个方程: $y = 4 \left( \frac{x-1}{2} \right)^2$
    
    化简: $y = 4 \cdot \frac{(x-1)^2}{4} = (x-1)^2$
    
    结果: 普通方程为 $y = (x-1)^2$ (开口向上的抛物线)。
\end{example}

\begin{example} 将 $\begin{cases} x = 3 \cos \theta \\ y = 3 \sin \theta \end{cases}$ 化为普通方程。

    利用三角恒等式: $\cos^2 \theta + \sin^2 \theta = 1$
    
    解出 $\cos \theta$ 和 $\sin \theta$: $\cos \theta = x/3$, $\sin \theta = y/3$
    
    代入恒等式: $(x/3)^2 + (y/3)^2 = 1$
    
    化简: $x^2 + y^2 = 9$
    
    结果: 普通方程为 $x^2 + y^2 = 9$ (圆心在原点,半径为 3 的圆)。
\end{example}

\medskip

(2)普通方程化为参数方程:核心思想是选择一个合适的参数 $t$。

选择参数没有固定方法,但通常选择 $x$ 或 $y$ 或它们的组合,或选择一个有几何意义的角。

\medskip

\begin{example} 将直线 $y = 3x - 5$ 化为参数方程。

    设 $x = t$:
    
    代入原方程: $y = 3t - 5$
    
    结果: $\begin{cases} x = t \\ y = 3t - 5 \end{cases}$
\end{example}

\begin{example} 将椭圆 $\frac{x^2}{16} + \frac{y^2}{9} = 1$ 化为参数方程。

    改写方程: $(\frac{x}{4})^2 + (\frac{y}{3})^2 = 1$
    
    利用三角恒等式,设 $\frac{x}{4} = \cos \theta$ 和 $\frac{y}{3} = \sin \theta$:
    
    结果: $\begin{cases} x = 4 \cos \theta \\ y = 3 \sin \theta \end{cases}$ ($\theta$ 为参数)
\end{example}

参数方程是用一个独立的变量 $t$ 来间接描述 $x$ 和 $y$ 的关系。


\section{函数的极限}
函数极限是微积分中的一个基本概念,它描述了一个函数在自变量(输入值)接近某一给定值时,它的函数值(输出值)所表现出的趋势。

\medskip

\textbf{函数极限的直观理解}

(1)函数 $f(x)$ 在 $x$ 趋近于 $a$ 时的极限是 $L$,记作 $\lim_{x \to a} f(x) = L$。

(2)当 $x$ 无限地接近 $a$(但不等于 $a$)时,函数值 $f(x)$ 会无限地接近一个确定的值 $L$。

(3)$f(x)$ 在 $x=a$ 处是否有定义,以及 $f(a)$ 的值是多少,并不影响 $\lim_{x \to a} f(x)$ 的结果。极限关注的是趋近过程,而非终点的值。

从上面直觉出发,逐渐演变完善了极限的 $\epsilon-\delta$ 定义。 $\epsilon-\delta$ 定义为极限提供了逻辑严密性,但代价是牺牲了直觉性、增加了学习难度和操作复杂性。

\medskip

\textbf{格洛克极限定义}

格洛克采用逆向思维,并以可视化的格洛克代数空间和无穷大无穷小的定义作为逻辑支撑,给出极限的定义。

\begin{definition}[函数的极限]
    对于函数 $f(x),\, x \to a \;\iff\; x = a\pm\epsilon$,函数 $f(x)$ 在 $x$ 趋近于 $a$ 时的极限

    $\qquad (1)\; \lim_{x\to a} f(x) = f(a \pm \epsilon) = f(a)$,如果 $a$ 位于定义域区间内。

    $\qquad (2)\; \lim_{x\to a^+} f(x) = f(a + \epsilon)$,如果 $a$ 位于定义域区间左边界。

    $\qquad (3)\; \lim_{x\to a^-} f(x) = f(a - \epsilon)$,如果 $a$ 位于定义域区间右边界。
\end{definition}

对于(2)和(3),如果 $a$ 是闭区间边界,极限值应用 $f(a)$ 仍然成立。

\medskip

格洛克将传统极限的动态趋近描述转换成了静态描述,因为在格洛克微空间“看到了”最接近 $a$ 的极限位置。由此不再需要晦涩难解的逻辑推理描述,简单直接地进行极限计算。

\medskip

\begin{example} 计算极限

    $(1)\; \lim_{x \to 1}(2x + 1) \quad (2)\; \lim_{x \to \infty} \frac{3x^3 - x + 5}{2x^3 + 4x^2 - 1} \quad (3)\; \lim_{x \to 2} \frac{x^2 - 4}{x - 2} \quad (4)\; \lim_{x \to 0} \frac{\sqrt{1 + x} - 1}{x}$
    
\medskip

    \textbf{解:}
    \begin{flalign*}
        &(1)\; \lim_{x \to 1}(2x + 1) = 2\cdot (1) + 1 = 3 & \\
        &(2)\; \lim_{x \to \infty} \frac{3x^3 - x + 5}{2x^3 + 4x^2 - 1} = \frac{3\infty^3 - \infty + 5}{2\infty^3 + 4\infty^2 - 1} = \frac{3\infty^3}{2\infty^3} = \frac 3 2 & \\
        &(3)\; \lim_{x \to 2} \frac{x^2 - 4}{x - 2} = \frac{(2+\epsilon)^2 - 4}{(2+\epsilon) - 2} = \frac{4\epsilon + \epsilon^2}{\epsilon} = 4 + \epsilon = 4 & \\
        &(4)\; \lim_{x \to 0} \frac{\sqrt{1 + x} - 1}{x} = \frac{\sqrt{1 + \epsilon} - 1}{\epsilon} &\\
        &\qquad = \frac{(\sqrt{1 + \epsilon} - 1)(\sqrt{1 + \epsilon} + 1)}{\epsilon(\sqrt{1 + \epsilon} + 1)} &\\
        &\qquad = \frac{1}{\sqrt{1 + \epsilon} + 1} = \frac 1 2 &
    \end{flalign*}    
\end{example}

上例(3)和(4)略去了计算左极限,因为左右极限相等。

\medskip

\textbf{函数的连续性}

(1)在定义域区间内函数总是连续的,在闭区间左边界右连续,在闭区间右边界左连续。

(2)如果分段函数的分段点有定义,分段子函数分别为 $f_1, f_2$,并且有 $f_1(a-\epsilon) = f_2(a+\epsilon) = f(a)$,那么分段函数在分段点处连续。

\medskip

函数的连续性是显而易见的,换句话说,函数 $f(x)$ 的值随着自变量 $x$ 的连续变化而连续变化。函数图形可视化的证明了函数的连续性。事实上,证明函数的连续性很容易进入逻辑上的循环证明。

讨论函数的连续性只有在函数的两个相邻区间的分段点处才有意义,格洛克采用分段函数的方式描述了这种情况,分段子函数 $f_1, f_2$ 可以相同,也可以不同。

具有苛刻定义的非常规函数因失去一般性不在这里进行讨论。

\medskip

\begin{example} 计算极限

    $(1)\; \lim_{x \to 0} \frac 1 x \qquad (2)\; \lim_{x \to 0} |x|$
    
\medskip

    \textbf{解:}
    \begin{flalign*}
        &(1)\; \lim_{x \to 0^-} \frac 1 x = \frac{1}{-\epsilon} = -\infty \qquad\; \lim_{x \to 0^+} \frac 1 x = \frac{1}{\epsilon} = \infty& \\
        &\text{左右极限不相等,极限不存在。}& \\
        &(2)\; \lim_{x \to 0^-} |x| = |-\epsilon| = \epsilon = 0 \qquad \lim_{x \to 0^+} |x| = |\epsilon| = 0& \\
        &\text{分段点 0 左右极限相等,极限存在,极限值为 0。}& \\
        &\text{分段点 0 有定义且函数值为 0,所以函数在分段点 0 处连续。}&
    \end{flalign*}
        
\end{example}

\medskip

\begin{example} 计算极限

    $$\lim_{x \to 0}\frac{\sin x}{x}$$
    
\medskip

    \textbf{解:} 这是一个著名的极限问题,传统上使用夹逼定理进行求解。

    当光滑曲线上的两点 $A, B$ 越来越接近时,曲线弧 $AB$ 与割线 $AB$ 逐渐重合,换句话说,只要 $A, B$ 两点的距离足够小,$\widehat{AB} = |AB|$,如图2.1 所示。

    \begin{figure}[htbp] 
        \centering
        \includegraphics[width=0.4\textwidth]{images/02/2.1.png} 
        \caption{\textbf{单位圆上的弧}}
    \end{figure}

    我们取弧长 $\widehat{AB} = \epsilon$,得到如下关系:
    $$|AB| = \epsilon,\quad \angle{AOB} = \epsilon,\quad |AC| = \angle{AOC} = \frac{\epsilon}{2}$$

    根据三角公式
    \begin{align*}
        sin\frac{\epsilon}{2} &= \frac{AC}{OA} = AC = \frac{\epsilon}{2} \\
        \cos\frac{\epsilon}{2} &= \frac{OC}{OA} = \sqrt{1-\left(\frac\epsilon 2 \right)^2}
    \end{align*}
    
    现在我们可以计算极限
    \begin{align*}
        \lim_{x \to 0}\frac{\sin x}{x} &= \frac{\sin \epsilon}{\epsilon} = \frac{2\sin\frac{\epsilon}{2}\cos\frac{\epsilon}{2}}{\epsilon} \\
        &= \frac{2\cdot \frac\epsilon 2 \cdot \sqrt{1-\left(\frac\epsilon 2 \right)^2}}{\epsilon} \\
        &= \sqrt{1-\left(\frac\epsilon 2 \right)^2} \\
        &= 1
    \end{align*}

\end{example}

\medskip

\begin{example} 计算极限 $$\lim_{x \to 0}\frac{1 - \cos x}{x}$$
    \textbf{解:} 利用半角公式有 $\cos x = 1 - 2\sin^2\left(\frac x 2 \right)$,所以
    \begin{align*}
        \lim_{x \to 0}\frac{1 - \cos x}{x} &= \frac{1 - \cos \epsilon}{\epsilon} \\
        &= \frac{1 - 1 + 2\sin^2\left(\frac \epsilon 2 \right)}{\epsilon} \\
        &= \frac{2\cdot \left(\frac \epsilon 2 \right)^2}{\epsilon} \\
        &= \frac 1 2 \epsilon = 0
    \end{align*}        
\end{example}

\medskip

\begin{info}{极限的 $\epsilon-\delta$ 定义}
    设函数 $f(x)$ 在点 $x_0$ 的某个去心邻域内有定义。如果存在一个实数 $L$,使得:

    对于任意给定的正数 $\epsilon$ (无论它多么小),总存在一个正数 $\delta$ (它通常依赖于 $\epsilon$),使得当 $x$ 满足 $0 < |x - x_0| < \delta$ 时,都有 $|f(x) - L| < \epsilon$ 成立。

则称 $L$ 是函数 $f(x)$ 当 $x$ 趋近于 $x_0$ 时的极限,记作:
$$\lim_{x \to x_0} f(x) = L$$
\end{info}

\section{自然常数和自然对数}
自然常数 $e$ 是数学中一个重要的无理数,常作为自然对数的底数出现。它大约等于 2.71828,并在许多自然现象、增长模型(如复利)和极限计算中扮演关键角色。

$e$ 最早由瑞士数学家雅各布·伯努利(Jacob Bernoulli)在研究复利问题时发现,后来由莱昂哈德·欧拉(Leonhard Euler)在 18 世纪正式引入符号 "e"(源自 "exponential",指数的)。

\medskip

$e$ 可以定义为以下极限:
$$e = \lim_{x \to \infty}\left(1 + \frac 1 x \right)^x = \left(1 + \frac 1 \infty \right)^\infty = (1 + \epsilon)^\infty$$
从格洛克代数空间的角度看,自然常数是一个未解析的无穷大无穷小计算表达式。使用无穷大等效重定义 $\infty \gets f(\infty)$,格洛克给出自然常数的定义。

\begin{definition}{自然常数}

    设 $f(\infty)$ 是关于 $\infty$ 的表达式,即 $f(\infty)$ 位于格洛克宏空间;$f(\epsilon)$ 是关于 $\epsilon$ 的表达式,即 $f(\epsilon)$ 位于格洛克微空间。自然常数 $e$ 定义为
    \begin{align}
        e &= \left(1 + \frac 1 {f(\infty)} \right)^{f(\infty)} \\
        e &= (1 + f(\epsilon))^{1 / f(\epsilon)}
    \end{align}
\end{definition}

需要明确的是,即使 $f(\infty), f(\epsilon)$ 是负值时,自然常数的定义仍然有效。
\begin{align*}
    \left(1-\frac 1 \infty \right)^{-\infty} &= \left(\frac{\infty-1}{\infty} \right)^{-\infty} = \left(\frac{\infty}{\infty-1}\right)^\infty \\
    &= \left(1 + \frac{1}{\infty-1} \right)^{\infty-1} \cdot \left(1 + \frac{1}{\infty-1} \right) \\
    &= e \cdot (1+\epsilon) \\
    &= e
\end{align*}
自然对数,记作 $\ln(x)$,是以 $e$ 为底的对数函数,它是 $\log_e(x)$ 的简写。
\begin{theorem} 自然恒等式
    \begin{align}
        (1 + a \cdot \epsilon)^b &= e^{ab \cdot \epsilon} = 1 + ab \cdot \epsilon \\
        a^\epsilon &= e^{\ln a \cdot \epsilon} = 1 + \ln a \cdot \epsilon \\
        \ln(1 + a \cdot \epsilon) &= a\cdot \epsilon
    \end{align}
\end{theorem}

\textbf{证明}
\begin{align*}
    (1 + a \cdot \epsilon)^b &= (1 + a \cdot \epsilon)^{\frac{1}{a\epsilon} \cdot a\epsilon \cdot b} \\
    &= e^{ab \cdot \epsilon} \\
    &= (1 + ab \cdot \epsilon)^{\frac{1}{ab \cdot \epsilon} \cdot (ab \cdot \epsilon)} \\
    &= 1 + ab \cdot \epsilon \\
    \\
    a^\epsilon &= e^{\ln a ^ \epsilon} = e^{\epsilon \cdot \ln a} \\
    &= (1 + \epsilon \cdot \ln a)^{\frac{1}{\epsilon \cdot \ln a} \cdot (\epsilon \cdot \ln a)} \\
    &= 1 + \ln a \cdot \epsilon \\
    \\
    \ln(1 + a \cdot \epsilon) &= \ln e^{a \cdot \epsilon} \\
    &= a \cdot \epsilon \cdot \ln e \\
    &= a \cdot \epsilon
\end{align*}

\begin{example} 计算极限
    \begin{align*}
        \lim_{x \to \infty} \left(1 + \frac{2}{x}\right)^x &= \left(1 + \frac{2}{\infty}\right)^\infty \\
        &= \left(1 + \frac{2}{\infty}\right)^{\frac \infty 2 \cdot 2} \\
        &= e^2 \\
        \\
        \lim_{x \to 0}\frac{\ln(1+3x)}{x} &= \frac{\ln(1+3\epsilon)}{\epsilon} \\
        &= \frac{\ln e^{3\epsilon}}{\epsilon} = \frac{3\epsilon}{\epsilon} \\
        &= 3 \\
        \\
        \lim_{x \to 0}\frac{e^x - 1}{x} &= \frac{e^\epsilon - 1}{\epsilon} \\
        &= \left[(1 + \epsilon)^{\infty \cdot \epsilon} - 1 \right] \cdot \frac 1 \epsilon \\
        &= \frac{(1 + \epsilon) - 1}{\epsilon} = \frac \epsilon \epsilon \\
        &= 1  \\
        \\
        \lim_{x \to 0}\frac{a^x - 1}{x} &= \frac{a^\epsilon - 1}{\epsilon} \\
        &= \frac{(1 + \ln a \cdot \epsilon) - 1}{\epsilon} \\
        &= \frac{\ln a \cdot \epsilon}{\epsilon} \\
        &= \ln a              
    \end{align*}
\end{example}

\begin{example} 计算极限 $\lim_{x \to 1}\frac{\ln x}{1-x},\; \lim_{x \to \infty}\frac{\ln x}{\sqrt[n]{x}}$
    \begin{align*}
        \lim_{x \to 1}\frac{\ln x}{1-x} &= \frac{\ln(1 + \epsilon)}{1 - (1+\epsilon)} \\
        &= \frac{\ln e^\epsilon}{-\epsilon} \\
        &= \frac{\epsilon}{-\epsilon} = -1 \\
        \\
        \lim_{x \to \infty}\frac{\ln x}{\sqrt[n]{x}} &= \frac{\ln \infty}{\sqrt[n]{\infty}} \\
        &= \frac{\ln \infty^n}{\sqrt[n]{\infty^n}} \qquad (\infty \gets \infty^n) \\
        &= \frac{n\ln \infty}{\infty} = 0
    \end{align*}
\end{example}

\medskip

\begin{warning}{自然恒等式是极限运算}
    显然,定理2.1 的恒等结果是在舍弃更高阶无穷小后的极限运算结果,例如极限计算
    $$\frac{e^{\epsilon^2} - 1 - \epsilon^2}{\epsilon^4}$$
    我们不能简单的应用定理2.1,那样分子将为0,此时正确的方法是使用“平滑”的泰勒展开来得到更高阶无穷小,从而计算出正确的结果。

    定理2.1 的意义在于简化指数函数导数公式的推导,免去复杂的逻辑推理。
\end{warning}


\section{泰勒展开在极限计算中的应用}
泰勒展开(Taylor Series Expansion)是计算涉及 $e$, $\ln(x)$, $\sin(x)$, $\cos(x)$ 等基本函数的极限时,一个非常强大且高效的工具。

它主要利用函数在某一点(通常是 $x=0$,即麦克劳林展开)附近的多项式近似来简化分子和分母。

常用函数的麦克劳林展开式 ($x \to 0$)
\begin{align}
    \sin x &= x - \frac{x^3}{3!} + \frac{x^5}{5!} - \cdots \\
    \cos x &= 1 - \frac{x^2}{2!} + \frac{x^4}{4!} - \cdots \\
    \tan x &= x + \frac{x^3}{3} + \frac{2x^5}{15} + \cdots \\
    e^x &= 1 + x +\frac{x^2}{2!} + \frac{x^3}{3!} + \cdots \\
    \ln(1+x) &= x - \frac{x^2}{2} + \frac{x^3}{3} - \cdots \\
    (1+x)^a &= 1 + ax + \frac{a(a-1)}{2!}x^2 + \frac{a(a-1)(a-2)}{3!}x^3 + \cdots
\end{align}

\medskip

\begin{example} 计算极限
    \begin{align*}
        \lim_{x \to 0} \frac{\ln(1+x)}{x} &= \frac{\ln(1+\epsilon)}{\epsilon} = \frac{\epsilon}{\epsilon} = 1 \\
        \lim_{x \to 0} \frac{\ln(\cos x)}{x^2} &= \frac{\ln(\cos \epsilon)}{\epsilon^2} = \frac{\ln\left(1 - \frac{1}{2}\epsilon^2 \right)}{\epsilon^2} \\
        &= \frac{-\frac{1}{2}\epsilon^2}{\epsilon^2} = -\frac 1 2 \\
        \lim_{x \to 0} \frac{e^{x^2} - 1 - x^2}{x^4} &= \frac{e^{\epsilon^2} - 1 - \epsilon^2}{\epsilon^4} \\
        &= \frac{\left(1+\epsilon^2+\frac{1}{2}\epsilon^4 \right) - 1 - \epsilon^2}{\epsilon^4} \\
        &= \frac{\frac{1}{2}\epsilon^4}{\epsilon^4} = \frac 1 2
    \end{align*}
\end{example}

\medskip

\begin{example} 计算极限 
    \begin{align*}
        \lim_{x \to 0}\frac{\sin x}{x} &= \frac{\sin \epsilon}{\epsilon} = \frac \epsilon \epsilon = 1 \\
        \lim_{x \to 0}\frac{1 - \cos x}{x} &= \frac{1 - \cos \epsilon}{\epsilon} = \frac{1 - (1 - \frac 1 2 \epsilon^2)}{\epsilon} = \frac 1 2 \epsilon = 0 \\
    \end{align*}
\end{example}

使用泰勒展开计算极限的要点:

(1)确定阶数: 根据分母的无穷小阶数来确定分子和分母需要展开到的阶数。目标是让分子中最低次项(非零项)的阶数与分母的阶数相等。

(2)抵消低阶项: 展开后,分子中的低阶项(如常数项、一次项等)通常会相互抵消,只剩下与分母同阶或更高阶的无穷小项。

(3)提取主部: 保留分子分母中最低次项的系数,即可求出极限。

我们在学习泰勒级数之前,提前利用泰勒级数的展开式来简化复杂极限的计算,避免进行不必要的复杂逻辑分析。

\medskip

\begin{warning}{循环论证}
    复杂多项式的泰勒级数展开在计算极限时很强大和直观,但是注意不能用它来证明极限和导数的推导,因为泰勒级数是在极限和导数的理论框架下推导出来的。否则可能会出现循环论证的逻辑错误。
\end{warning}