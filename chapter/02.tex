\chapter{函数和极限}
\section{理解函数}

函数是数学中描述对应关系的一种基本且非常重要的概念。通常写作 $y = f(x)$。

\medskip

\textbf{函数的关键特点}

唯一性对应 (Uniqueness):对于定义域内的每一个输入值 $x$,函数都只能对应一个唯一的输出值 $f(x)$。

定义域 (Domain):所有允许的输入值 $x$ 的集合。

例如,在函数 $f(x) = \frac{1}{x}$ 中,$x$ 不能为 $0$,所以定义域是除 $0$ 以外的所有实数。

值域 (Range):值域是实际输出值的集合。

\medskip


\textbf{常见表示方法}

函数通常可以通过以下几种方式来表示:

方式描述示例代数式用一个数学公式或方程来表示规则。$f(x) = x^2 + 1$图象在坐标系中用一条曲线或图形来表示。抛物线(对于 $f(x)=x^2$)表格列出有限个输入值和它们对应的输出值。$x=1$ 时 $f(x)=2$, $x=2$ 时 $f(x)=5$文字用语言描述对应的规则。

函数 $f(x)$ 在点 $x_0$ 处的导数定义为:
$$ f'(x_0) = \lim_{\Delta x \to 0} \frac{f(x_0 + \Delta x) - f(x_0)}{\Delta x} $$
导数表示函数在某点变化率的精确度量。