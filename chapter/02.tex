\chapter{函数和极限}

函数是数学中一个核心概念,它描述了变量之间的一种对应关系。简单来说,函数将一个输入值(自变量)映射到一个输出值(因变量)。函数的概念起源于17世纪的莱布尼茨和牛顿,用于描述物理现象,如运动和变化。

传统的函数表示是一种泛化的表示,试图包容一切,解决一切问题,自变量和函数值(因变量)之间的关系抽象成规则,试图解决一切假想中的问题;因此在描述定义、定理时需要晦涩的前提限制条件,并形成抽象的形式化定义。本文或格洛克描述的函数只有一种,自变量和函数值之间的关系由自变量表达式确定,保持具体而简单。

极限是微积分的基础概念,它描述了函数在某个点附近的行为,即使函数在该点未定义或不连续。极限的概念由柯西和魏尔斯特拉斯在19世纪正式化,用于处理无穷小和无穷大。

基于格洛克代数空间,定义域的区间和函数的连续性进行了重新表述,极限概念被重新定义,一切变得简单而直观。


\section{理解函数}

函数是数学中描述对应关系的一种基本且非常重要的概念。通常写作 $y = f(x)$。

\medskip

\textbf{函数的关键特点}

唯一性对应 (Uniqueness):对于定义域内的每一个输入值 $x$,函数都只能对应一个唯一的输出值 $f(x)$。

定义域 (Domain):所有允许的输入值 $x$ 的集合。

例如,在函数 $f(x) = \frac{1}{x}$ 中,$x$ 不能为 $0$,所以定义域是除 $0$ 以外的所有实数。

值域 (Range):值域是实际输出值的集合。

\medskip

函数的输入输出关系通过自变量表达式来描述,通常有几何图形来对应表示。例如函数 $f(x) = x^2 + 1$,在几何图形上表现为抛物线。

\medskip

\textbf{定义域区间}

函数的定义域通常是数轴上的一个或多个连续的数值范围,因此我们常用区间符号来表示它。

圆括号 $(a, b)$表示不包含端点(开区间)。表示 $a < x < b$,以及区间边界是 $\infty$ 或 $-\infty$ 的情况。

方括号 $[a, b]$表示包含端点(闭区间)。表示 $a \le x \le b$。

\medskip

\textbf{开区间转换为闭区间}

依据格洛克无穷大和无穷小定义的性质(4),有

$\qquad (a, b) \Rightarrow [a+\varepsilon, b-\varepsilon]\qquad (a, \infty) \Rightarrow [a+\varepsilon, \infty)$

$\infty)$ 既可以理解为开区间也可以理解为闭区间,根据具体的问题进行不同的理解。

\medskip

\textbf{分段函数}

分段函数将定义域分成若干互不相交的子区间,并在每个子区间上定义一个子函数。

分段函数常用大括号表示,例如:

$$f(x) = |x| = \begin{cases} -x, & \text{if } x < 0 \\ x, & \text{if } x \geq 0 \end{cases}$$

$$f(x) = \frac 1 x = \begin{cases} \frac 1 x, & \text{if } x < 0 \\ \frac 1 x, & \text{if } x > 0 \end{cases}$$

$$f(x) = \begin{cases} x^2 + 1, & \text{if } x < 1 \\ 3x - 1, & \text{if } x \geq 1 \end{cases}$$

还有一些常见的特殊分段函数:

阶跃函数 (Step Function): 分段函数由常数函数组成。例如,单位阶跃函数(Unit Step Function)。

分段线性函数 (Piecewise Linear Function): 分段函数由线段(线性函数)组成。

通过以上讨论,我们现在可以明确:\textbf{函数的概念和性质总是作用于单一定义域区间}。

\medskip

\begin{warning}{复合函数}
    复合函数本质上是自变量的函数(表达式)变换并受到变换函数的额外约束。例如 复合函数 $f(g(x))$,我们可以理解为函数 $f(y)$,并通过 $y = g(x)$ 变换得到的。事实上,函数变换更常见,因此不作为一个单独的函数分类提出。
\end{warning}

\section{函数的极限}
函数极限是微积分中的一个基本概念,它描述了一个函数在自变量(输入值)接近某一给定值时,它的函数值(输出值)所表现出的趋势。

\medskip

\textbf{函数极限的直观理解}

(1)函数 $f(x)$ 在 $x$ 趋近于 $a$ 时的极限是 $L$,记作 $\lim_{x \to a} f(x) = L$。

(2)当 $x$ 无限地接近 $a$(但不等于 $a$)时,函数值 $f(x)$ 会无限地接近一个确定的值 $L$。

(3)$f(x)$ 在 $x=a$ 处是否有定义,以及 $f(a)$ 的值是多少,并不影响 $\lim_{x \to a} f(x)$ 的结果。极限关注的是趋近过程,而非终点的值。

从上面直觉出发,逐渐演变完善了极限的 $\epsilon-\delta$ 定义。 $\epsilon-\delta$ 定义为极限提供了逻辑严密性,但代价是牺牲了直觉性、增加了学习难度和操作复杂性。

\medskip

\textbf{格洛克极限定义}

格洛克采用逆向思维,并以可视化的格洛克代数空间和无穷大无穷小的定义作为逻辑支撑,给出极限的定义。

\begin{definition}[函数的极限]
    对于函数 $f(x),\, x \to a \;\iff\; x = a\pm\varepsilon$,函数 $f(x)$ 在 $x$ 趋近于 $a$ 时的极限

    $\qquad (1)\; \lim_{x\to a} f(x) = f(a \pm \varepsilon) = f(a)$,如果 $a$ 位于定义域区间内。

    $\qquad (2)\; \lim_{x\to a^+} f(x) = f(a + \varepsilon)$,如果 $a$ 位于定义域区间左边界。

    $\qquad (3)\; \lim_{x\to a^-} f(x) = f(a - \varepsilon)$,如果 $a$ 位于定义域区间右边界。
\end{definition}

对于(2)和(3),如果 $a$ 是闭区间边界,极限值应用 $f(a)$ 仍然成立。

\medskip

格洛克将传统极限的动态趋近描述转换成了静态描述,因为在格洛克微空间“看到了”最接近 $a$ 的极限位置。由此不再需要晦涩难解的逻辑推理描述,简单直接地进行极限计算。

\medskip

\begin{example} 计算极限

    $(1)\; \lim_{x \to 1}(2x + 1) \quad (2)\; \lim_{x \to \infty} \frac{3x^3 - x + 5}{2x^3 + 4x^2 - 1} \quad (3)\; \lim_{x \to 2} \frac{x^2 - 4}{x - 2} \quad (4)\; \lim_{x \to 0} \frac{\sqrt{1 + x} - 1}{x}$
    
\medskip

    \textbf{解:}
    \begin{flalign*}
        &(1)\; \lim_{x \to 1}(2x + 1) = 2\cdot (1) + 1 = 3 & \\
        &(2)\; \lim_{x \to \infty} \frac{3x^3 - x + 5}{2x^3 + 4x^2 - 1} = \frac{3\infty^3 - \infty + 5}{2\infty^3 + 4\infty^2 - 1} = \frac{3\infty^3}{2\infty^3} = \frac 3 2 & \\
        &(3)\; \lim_{x \to 2} \frac{x^2 - 4}{x - 2} = \frac{(2+\varepsilon)^2 - 4}{(2+\varepsilon) - 2} = \frac{4\varepsilon + \varepsilon^2}{\varepsilon} = 4 + \varepsilon = 4 & \\
        &(4)\; \lim_{x \to 0} \frac{\sqrt{1 + x} - 1}{x} = \frac{\sqrt{1 + \varepsilon} - 1}{\varepsilon} &\\
        &\qquad = \frac{(\sqrt{1 + \varepsilon} - 1)(\sqrt{1 + \varepsilon} + 1)}{\varepsilon(\sqrt{1 + \varepsilon} + 1)} &\\
        &\qquad = \frac{1}{\sqrt{1 + \varepsilon} + 1} = \frac 1 2 &
    \end{flalign*}    
\end{example}

上例(3)和(4)略去了计算左极限,因为左右极限相等。

\medskip

\textbf{函数的连续性}

(1)在定义域区间内函数总是连续的,在闭区间左边界右连续,在闭区间右边界左连续。

(2)如果分段函数的分段点有定义,分段子函数分别为 $f_1, f_2$,并且有 $f_1(a-\varepsilon) = f_2(a+\varepsilon) = f(a)$,那么分段函数在分段点处连续。

\medskip

函数的连续性是显而易见的,换句话说,函数 $f(x)$ 的值随着自变量 $x$ 的连续变化而连续变化。函数图形可视化的证明了函数的连续性。事实上,证明函数的连续性很容易进入逻辑上的循环证明。

讨论函数的连续性只有在函数的两个相邻区间的分段点处才有意义,格洛克采用分段函数的方式描述了这种情况,分段子函数 $f_1, f_2$ 可以相同,也可以不同。

具有苛刻定义的非常规函数因失去一般性不在这里进行讨论。

\medskip

\begin{example} 计算极限

    $(1)\; \lim_{x \to 0} \frac 1 x \qquad (2)\; \lim_{x \to 0} |x|$
    
\medskip

    \textbf{解:}
    \begin{flalign*}
        &(1)\; \lim_{x \to 0^-} \frac 1 x = \frac{1}{-\varepsilon} = -\infty \qquad\; \lim_{x \to 0^+} \frac 1 x = \frac{1}{\varepsilon} = \infty& \\
        &\text{左右极限不相等,极限不存在。}& \\
        &(2)\; \lim_{x \to 0^-} |x| = |-\varepsilon| = \varepsilon = 0 \qquad \lim_{x \to 0^+} |x| = |\varepsilon| = 0& \\
        &\text{分段点 0 左右极限相等,极限存在,极限值为 0。}& \\
        &\text{分段点 0 有定义且函数值为 0,所以函数在分段点 0 处连续。}&
    \end{flalign*}
        
\end{example}

\medskip

\begin{example} 计算极限

    $$\lim_{x \to 0}\frac{\sin x}{x}$$
    
\medskip

    \textbf{解:} 这是一个著名的极限问题,传统上使用夹逼定理进行求解。

    当光滑曲线上的两点 $A, B$ 越来越接近时,曲线弧 $AB$ 与割线 $AB$ 逐渐重合,换句话说,只要 $A, B$ 两点的距离足够小,$\widehat{AB} = |AB|$,如图2.1 所示。

    \begin{figure}[htbp] 
        \centering
        \includegraphics[width=0.4\textwidth]{images/02/2.1.png} 
        \caption{\textbf{单位圆上的弧}}
    \end{figure}

    我们取弧长 $\widehat{AB} = \varepsilon$,得到如下关系:
    $$|AB| = \varepsilon,\quad \angle{AOB} = \varepsilon,\quad |AC| = \angle{AOC} = \frac{\varepsilon}{2}$$

    根据三角公式
    \begin{align*}
        sin\frac{\varepsilon}{2} &= \frac{AC}{OA} = AC = \frac{\varepsilon}{2} \\
        \cos\frac{\varepsilon}{2} &= \frac{OC}{OA} = \sqrt{1-\left(\frac\varepsilon 2 \right)^2}
    \end{align*}
    
    现在我们可以计算极限
    \begin{align*}
        \lim_{x \to 0}\frac{\sin x}{x} &= \frac{\sin \varepsilon}{\varepsilon} = \frac{2\sin\frac{\varepsilon}{2}\cos\frac{\varepsilon}{2}}{\varepsilon} \\
        &= \frac{2\cdot \frac\varepsilon 2 \cdot \sqrt{1-\left(\frac\varepsilon 2 \right)^2}}{\varepsilon} \\
        &= \sqrt{1-\left(\frac\varepsilon 2 \right)^2} \\
        &= 1
    \end{align*}

\end{example}

\medskip

\begin{example} 计算极限

    $$\lim_{x \to 0}\frac{1 - \cos x}{x}$$
    
\medskip

    \textbf{解:} 利用半角公式有 $\cos x = 1 - 2\sin^2\left(\frac x 2 \right)$,所以

    \begin{align*}
        \lim_{x \to 0}\frac{1 - \cos x}{x} &= \frac{1 - \cos \varepsilon}{\varepsilon} \\
        &= \frac{1 - 1 + 2\sin^2\left(\frac \varepsilon 2 \right)}{\varepsilon} \\
        &= \frac{2\cdot \left(\frac \varepsilon 2 \right)^2}{\varepsilon} \\
        &= \frac 1 2 \varepsilon = 0
    \end{align*}

        
\end{example}

\begin{info}{极限的 $\epsilon-\delta$ 定义}
    设函数 $f(x)$ 在点 $x_0$ 的某个去心邻域内有定义。如果存在一个实数 $L$,使得:

    对于任意给定的正数 $\epsilon$ (无论它多么小),总存在一个正数 $\delta$ (它通常依赖于 $\epsilon$),使得当 $x$ 满足 $0 < |x - x_0| < \delta$ 时,都有 $|f(x) - L| < \epsilon$ 成立。

则称 $L$ 是函数 $f(x)$ 当 $x$ 趋近于 $x_0$ 时的极限,记作:
$$\lim_{x \to x_0} f(x) = L$$
\end{info}


函数 $f(x)$ 在点 $x_0$ 处的导数定义为:
$$ f'(x_0) = \lim_{\Delta x \to 0} \frac{f(x_0 + \Delta x) - f(x_0)}{\Delta x} $$
导数表示函数在某点变化率的精确度量。