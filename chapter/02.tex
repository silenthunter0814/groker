\chapter{函数和极限}
\section{理解函数}

函数是数学中描述对应关系的一种基本且非常重要的概念。通常写作 $y = f(x)$。

\medskip

\textbf{函数的关键特点}

唯一性对应 (Uniqueness):对于定义域内的每一个输入值 $x$,函数都只能对应一个唯一的输出值 $f(x)$。

定义域 (Domain):所有允许的输入值 $x$ 的集合。

例如,在函数 $f(x) = \frac{1}{x}$ 中,$x$ 不能为 $0$,所以定义域是除 $0$ 以外的所有实数。

值域 (Range):值域是实际输出值的集合。

\medskip

函数的输入输出关系通过自变量表达式来描述,通常有几何图形来对应表示。例如函数 $f(x) = x^2 + 1$,在几何图形上表现为抛物线。

\medskip

\begin{warning}{函数的表示}
    传统的函数表示是一种泛化的表示,试图包容一切,解决一切问题,自变量和函数值(因变量)之间的关系抽象成规则;因此在描述定义、定理时需要晦涩的前提限制条件。
    
    本文或格洛克描述的函数只有一种,自变量和函数值之间的关系由自变量表达式确定,保持具体而简单。
\end{warning}


函数 $f(x)$ 在点 $x_0$ 处的导数定义为:
$$ f'(x_0) = \lim_{\Delta x \to 0} \frac{f(x_0 + \Delta x) - f(x_0)}{\Delta x} $$
导数表示函数在某点变化率的精确度量。