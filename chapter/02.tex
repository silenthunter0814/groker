\chapter{函数和极限}

函数是数学中一个核心概念,它描述了变量之间的一种对应关系。简单来说,函数将一个输入值(自变量)映射到一个输出值(因变量)。函数的概念起源于17世纪的莱布尼茨和牛顿,用于描述物理现象,如运动和变化。

传统的函数表示是一种泛化的表示,试图包容一切,解决一切问题,自变量和函数值(因变量)之间的关系抽象成规则,试图解决一切假想中的问题;因此在描述定义、定理时需要晦涩的前提限制条件,并形成抽象的形式化定义。本文或格洛克描述的函数只有一种,自变量和函数值之间的关系由自变量表达式确定,保持具体而简单。

极限是微积分的基础概念,它描述了函数在某个点附近的行为,即使函数在该点未定义或不连续。极限的概念由柯西和魏尔斯特拉斯在19世纪正式化,用于处理无穷小和无穷大。

基于格洛克代数空间,定义域的区间和函数的连续性进行了重新表述,极限概念被重新定义,一切变得简单而直观。


\section{理解函数}

函数是数学中描述对应关系的一种基本且非常重要的概念。通常写作 $y = f(x)$。

\medskip

\textbf{函数的关键特点}

唯一性对应 (Uniqueness):对于定义域内的每一个输入值 $x$,函数都只能对应一个唯一的输出值 $f(x)$。

定义域 (Domain):所有允许的输入值 $x$ 的集合。

例如,在函数 $f(x) = \frac{1}{x}$ 中,$x$ 不能为 $0$,所以定义域是除 $0$ 以外的所有实数。

值域 (Range):值域是实际输出值的集合。

\medskip

函数的输入输出关系通过自变量表达式来描述,通常有几何图形来对应表示。例如函数 $f(x) = x^2 + 1$,在几何图形上表现为抛物线。

\medskip

\textbf{定义域区间}

函数的定义域通常是数轴上的一个或多个连续的数值范围,因此我们常用区间符号来表示它。

圆括号 $(a, b)$表示不包含端点(开区间)。表示 $a < x < b$,以及区间边界是 $\infty$ 或 $-\infty$ 的情况。

方括号 $[a, b]$表示包含端点(闭区间)。表示 $a \le x \le b$。

\medskip

\textbf{开区间转换为闭区间}

依据格洛克无穷大和无穷小定义的性质(4),有

$\qquad (a, b) \Rightarrow [a+\varepsilon, b-\varepsilon]\qquad (a, \infty) \Rightarrow [a+\varepsilon, \infty)$

$\infty)$ 既可以理解为开区间也可以理解为闭区间,根据具体的问题进行不同的理解。

\medskip

\textbf{分段函数}

分段函数将定义域分成若干互不相交的子区间,并在每个子区间上定义一个子函数。

分段函数常用大括号表示,例如:

$$f(x) = |x| = \begin{cases} -x, & \text{if } x < 0 \\ x, & \text{if } x \geq 0 \end{cases}$$

$$f(x) = \frac 1 x = \begin{cases} \frac 1 x, & \text{if } x < 0 \\ \frac 1 x, & \text{if } x > 0 \end{cases}$$

$$f(x) = \begin{cases} x^2 + 1, & \text{if } x < 1 \\ 3x - 1, & \text{if } x \geq 1 \end{cases}$$

还有一些常见的特殊分段函数:

阶跃函数 (Step Function): 分段函数由常数函数组成。例如,单位阶跃函数(Unit Step Function)。

分段线性函数 (Piecewise Linear Function): 分段函数由线段(线性函数)组成。

通过以上讨论,我们现在可以明确:\textbf{函数的概念和性质总是作用于单一定义域区间}。

\medskip

\begin{warning}{复合函数}
    复合函数本质上是自变量的函数(表达式)变换并受到变换函数的额外约束。例如 复合函数 $f(g(x))$,我们可以理解为函数 $f(y)$,并通过 $y = g(x)$ 变换得到的。事实上,函数变换更常见,因此不作为一个单独的函数分类提出。
\end{warning}

\section{函数的极限}
函数极限是微积分中的一个基本概念,它描述了一个函数在自变量(输入值)接近某一给定值时,它的函数值(输出值)所表现出的趋势。

\medskip

\textbf{函数极限的直观理解}

(1)函数 $f(x)$ 在 $x$ 趋近于 $a$ 时的极限是 $L$,记作 $\lim_{x \to a} f(x) = L$。

(2)当 $x$ 无限地接近 $a$(但不等于 $a$)时,函数值 $f(x)$ 会无限地接近一个确定的值 $L$。

(3)$f(x)$ 在 $x=a$ 处是否有定义,以及 $f(a)$ 的值是多少,并不影响 $\lim_{x \to a} f(x)$ 的结果。极限关注的是趋近过程,而非终点的值。



函数 $f(x)$ 在点 $x_0$ 处的导数定义为:
$$ f'(x_0) = \lim_{\Delta x \to 0} \frac{f(x_0 + \Delta x) - f(x_0)}{\Delta x} $$
导数表示函数在某点变化率的精确度量。