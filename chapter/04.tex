\chapter{导数和微分的应用}
本章我们利用导数和微分来解决一些数学上的问题,如函数的单调性、渐近线和极值等。

\section{函数单调性}
单调函数是指在给定区间上是增函数或减函数的函数。具有单调性的区间 $I$ 称为单调区间。

\medskip

\begin{definition}{单调函数}

    设函数 $y=f(x)$ 的单一区间定义域为 $D$,区间 $I \subseteq D$。对于区间 $I$ 上自变量的任意值 $x$,我们有以下分类:

    (1)单调递增,如果微分 $df(x) > 0$。

    (2)单调递减,如果微分 $df(x) < 0$。

    $df(x) = f(x+\epsilon)-f(x) = f(x)-f(x-\epsilon)$,在区间 $I$ 的左边界右微分,在区间 $I$ 的右边界左微分。
\end{definition}

\medskip

单调函数的主要特点:

(1)最值:单调函数在闭区间上的最大值和最小值一定在区间的端点处取到。

(2)零点:如果单调函数在其区间端点处函数值的符号相反,则该函数在区间内有且仅有一个零点。

(3)可逆性:严格单调函数存在反函数。

\medskip

\begin{example} 函数 $f(x) = x^3$ 在区间 $I = (-\infty, +\infty)$ 上的单调性。
    \begin{align*}
        df(x) &= (x+\epsilon)^3 - x^3 = 3x^2\epsilon + 3x\epsilon^2 + \epsilon^3 \\
        &= 3x^2\epsilon + \epsilon^3 > 0
    \end{align*}
    因为 $x^2 \ge 0$,可以看出,对于任意 $x\in I,\;df(x) > 0$ 恒成立,所以函数 $f(x) = x^3$ 单调递增。
\end{example}

\medskip

在一般情况下,我们可以根据单调性的定义直接进行计算来判断函数的单调性。因为 $df(x) = f'(x)\,dx$,$dx = \epsilon > 0$ 不会改变正负符号,因此当函数相对复杂时,可以考虑利用导函数 $f'(x)$ 的符号来判断:

(1)若在区间 $I$ 上 $f'(x) > 0$,则函数 $f(x)$ 在 $I$ 上单调递增。

(2)若在区间 $I$ 上 $f'(x) < 0$,则函数 $f(x)$ 在 $I$ 上单调递减。

(3)若在区间 $I$ 上 $f'(x) \ge 0$,且仅在有限个点上 $f'(x) = 0$,则函数 $f(x)$ 在 $I$ 上单调递增。

(4)若在区间 $I$ 上 $f'(x) \le 0$,且仅在有限个点上 $f'(x) = 0$,则函数 $f(x)$ 在 $I$ 上单调递减。

对于(3)和(4)的理解要点:

$f'(x) \ge 0$ (非负):这意味着函数 $f(x)$ 在区间 $I$ 上是单调不减的(即增函数)。函数值 $f(x)$ 永远不会随着 $x$ 的增大而减小。

$f'(x) = 0$ (有限点):这意味着函数曲线只有在孤立的几个点上有水平切线(即瞬时增长率为零)。

\medskip

\begin{example} 判定函数 $f(x) = x e^{-x}$ 的单调性。

    (3)求导数
    
    首先,使用乘积法则 $(uv)' = u'v + uv'$ 对函数 $f(x)$ 求导:
    
    设 $u = x$ 和 $v = e^{-x}$。则 $u' = 1$ 和 $v' = -e^{-x}$。
    \begin{align*}
        f'(x) &= (1) \cdot e^{-x} + x \cdot (-e^{-x}) \\
        &= e^{-x} - x e^{-x} = e^{-x} (1 - x)
    \end{align*}
    (2)分析导数的符号
    
    由于 $e^{-x}$ 对于所有的实数 $x$ 恒为正 ($e^{-x} > 0$),因此导数 $f'(x)$ 的符号完全由因子 $(1 - x)$ 决定。

    令 $f'(x) = 0$ 找到临界点:$$e^{-x} (1 - x) = 0$$
    因为 $e^{-x} \ne 0$,所以我们解 $1 - x = 0$,得到:$$x = 1$$

    (3)确定单调区间
    
    当 $x < 1$ 时单调递增:取 $x=0$ 检验。$1 - x = 1 - 0 = 1 > 0$。因此,$f'(x) = e^{-x} (1 - x) > 0$。函数 $f(x)$ 在区间 $(-\infty, 1)$ 上单调递增。
    
    当 $x > 1$ 时单调递减:取 $x=2$ 检验。$1 - x = 1 - 2 = -1 < 0$。因此,$f'(x) = e^{-x} (1 - x) < 0$。函数 $f(x)$ 在区间 $(1, +\infty)$ 上单调递减。

    函数在 $x=1$ 处达到局部最大值,$f(1) = 1 \cdot e^{-1} = \frac{1}{e}$。
\end{example}

\section{函数的极值}
函数极值(Extremum)是局部极大值(Local Maximum)和局部极小值(Local Minimum)的统称,反映了函数在其定义域的局部范围内的最大或最小性。

\medskip

\begin{definition}{函数极值}
    对于函数 $f(x)$ 区间内某一点 $x = x_0$:

    (1)$f(x_0)$ 是局部极大值,如果:$f(x_0) > f(x_0-\epsilon)$ 并且 $f(x_0) > f(x_0+\epsilon)$

    (2)$f(x_0)$ 是局部极小值,如果:$f(x_0) < f(x_0-\epsilon)$ 并且 $f(x_0) < f(x_0+\epsilon)$
\end{definition}

从图像上看,局部极大值点就像是函数图形上的山顶,局部极小值点就像是函数图形上的山谷底部。

\begin{example} 判断 $x = 0$ 是否是函数的极值点,$f(x) = x^2,\;g(x) = x^3$。

    对于 $f(x) = x^2$:
    \begin{align*}
        &f(0) - f(0-\epsilon) = (0)^2 - (-\epsilon)^2 = -\epsilon^2 < 0 \\
        &f(0) - f(0+\epsilon) = (0)^2 - (\epsilon)^2 = -\epsilon^2 < 0
    \end{align*}
    所以 $f(0) < f(0-\epsilon)$ 并且 $f(0) < f(0+\epsilon)$,$f(0)$ 是函数 $f(x) = x^2$ 的局部极小值。

    对于 $g(x) = x^3$:
    \begin{align*}
        &g(0) - g(0-\epsilon) = (0)^3 - (-\epsilon)^3 = \epsilon^3 > 0 \\
        &g(0) - g(0+\epsilon) = (0)^3 - (\epsilon)^3 = -\epsilon^3 < 0
    \end{align*}
    所以 $x = 0$ 不是函数 $g(x) = x^3$ 的极值点。
\end{example}


