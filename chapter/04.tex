\chapter{导数和微分的应用}
本章我们利用导数和微分来解决一些数学上的问题,如函数的单调性、渐近线和极值等。

\section{极限计算}

\textbf{1. $\frac{0}{0}$ 型不定式的极限}

由微分的基本定义 $df(x) = f(x+\epsilon) - f(x)$ 得到$$f(x+\epsilon) = df(x) + f(x)$$

如果 $f(x)$ 在 $x = c$ 处为 $0$,即 $f(c) = 0$,则
$$f(c + \epsilon) = df(c) + f(c) = df(c) \qquad \text{if } f(c) = 0$$

同时,基于 $df(x) = f'(x) \,dx \iff df(c) = f'(c) \cdot \epsilon$。

结合以上两点,我们考虑当 $f(c) = g(c) = 0,\; x \to c$ 时函数 $\frac{f(x)}{g(x)}$ 的极限:
\begin{align*}
    \lim_{x \to c}\frac{f(x)}{g(x)} &= \frac{f(c+\epsilon)}{g(c+\epsilon)} \\
    &= \frac{df(c) + f(c)}{dg(c) + g(c)} \\
    &= \frac{df(c)}{dg(c)} \qquad (\text{if } f(c) = g(c) = 0) \\
    &= \frac{f'(c) \cdot \epsilon}{g'(c) \cdot \epsilon} \\
    &= \frac{f'(c)}{g'(c)}
\end{align*}

将上面的内容进行总结,我们得到:
 
对于比值函数 $\frac{f(x)}{g(x)}$,如果 $f(c) = g(c) = 0$,那么 $x \to c$ 时函数的极限
\begin{align}
    \lim_{x \to c}\frac{f(x)}{g(x)} = \frac{f(c+\epsilon)}{g(c+\epsilon)} = \frac{f'(c)}{g'(c)} \qquad (\text{if } f(c) = g(c) = 0)
\end{align}

\medskip

\begin{example} 求 $\lim_{x\to 0} \frac{\sin x}{x}$。
    
    当 $x \to 0$ 时,分子 $f(0) = \sin 0 = 0$,分母 $g(0) = (0) = 0$。
    \begin{align*}
        f'(x) = (\sin x)' = \cos x \qquad g'(x) = (x)' = 1 \\
        f'(0) = \cos 0 = 1 \qquad g'(0) = 1 \\
        \lim_{x\to 0} \frac{\sin x}{x} = \frac{f'(0)}{g'(0)} = \frac{1}{1} = 1
    \end{align*}
\end{example}

\begin{example} 求  $\lim_{x\to 0} \frac{e^x - 1 - x}{x^2}$。
    
    当 $x \to 0$ 时,分子 $f(0) = e^0 - 1 - (0) = 0$,分母 $g(0) = (0)^2 = 0$。
    \begin{align*}
        \lim_{x\to 0} \frac{e^x - 1 - x}{x^2} = \lim_{x\to 0} \frac{(e^x - 1 - x)'}{(x^2)'} = \lim_{x\to 0} \frac{e^x - 1}{2x}
    \end{align*}

    当 $x \to 0$ 时,分子 $f(0) = e^0 - 1 = 0$,分母 $g(0) = 2(0) = 0$。
    \begin{align*}
        \lim_{x\to 0} \frac{e^x - 1}{2x} &= \lim_{x\to 0} \frac{(e^x - 1)'}{(2x)'} \\
        &= \lim_{x\to 0} \frac{e^x}{2} = \frac{e^0}{2} = \frac{1}{2}
    \end{align*}
\end{example}

\textbf{2.  $\frac{\infty}{\infty}$ 型不定式的极限}

当求取极限的表达式难解时,很多时候进行格洛克无穷大等效重定义能够简化极限问题,方便进一步的极限计算。

\medskip

\begin{example} 求 $\lim_{x\to 0^+} x \ln x$。
    \begin{align*}
        \lim_{x\to 0^+} x \ln x &= \epsilon \cdot \ln \epsilon \\
        &= e^{-\infty} \cdot \ln{e^{-\infty}} \qquad (\epsilon \gets e^{-\infty}) \\
        &= e^{-\infty} \cdot (-\infty) = -\frac{\infty}{e^\infty} = 0
    \end{align*}
\end{example}

\begin{example} 求 $\lim_{x\to \infty} \frac{\ln x}{x}$。
    \begin{align*}
        \lim_{x\to \infty} \frac{\ln x}{x} &= \frac{\ln \infty}{\infty} \\
        &= \frac{\ln e^\infty}{e^\infty} \qquad (\infty \gets e^\infty) \\
        &= \frac{\infty}{e^\infty} = 0
    \end{align*}
\end{example}


\section{函数单调性}
单调函数是指在给定区间上是增函数或减函数的函数。具有单调性的区间 $I$ 称为单调区间。

\medskip

\begin{definition}{单调函数}

    设函数 $y=f(x)$ 的单一区间定义域为 $D$,区间 $I \subseteq D$。对于区间 $I$ 上自变量的任意值 $x$,我们有以下分类:

    (1)单调递增,如果微分 $df(x) > 0$。

    (2)单调递减,如果微分 $df(x) < 0$。

    $df(x) = f(x+\epsilon)-f(x) = f(x)-f(x-\epsilon)$,在区间 $I$ 的左边界右微分,在区间 $I$ 的右边界左微分。
\end{definition}

\medskip

单调函数的主要特点:

(1)最值:单调函数在闭区间上的最大值和最小值一定在区间的端点处取到。

(2)零点:如果单调函数在其区间端点处函数值的符号相反,则该函数在区间内有且仅有一个零点。

(3)可逆性:严格单调函数存在反函数。

\medskip

\begin{example} 函数 $f(x) = x^3$ 在区间 $I = (-\infty, +\infty)$ 上的单调性。
    \begin{align*}
        df(x) &= (x+\epsilon)^3 - x^3 = 3x^2\epsilon + 3x\epsilon^2 + \epsilon^3 \\
        &= 3x^2\epsilon + \epsilon^3 > 0
    \end{align*}
    因为 $x^2 \ge 0$,可以看出,对于任意 $x\in I,\;df(x) > 0$ 恒成立,所以函数 $f(x) = x^3$ 单调递增。
\end{example}

\medskip

在一般情况下,我们可以根据单调性的定义直接进行计算来判断函数的单调性。因为 $df(x) = f'(x)\,dx$,$dx = \epsilon > 0$ 不会改变正负符号,因此当函数相对复杂时,可以考虑利用导函数 $f'(x)$ 的符号来判断:

(1)若在区间 $I$ 上 $f'(x) > 0$,则函数 $f(x)$ 在 $I$ 上单调递增。

(2)若在区间 $I$ 上 $f'(x) < 0$,则函数 $f(x)$ 在 $I$ 上单调递减。

(3)若在区间 $I$ 上 $f'(x) \ge 0$,且仅在有限个点上 $f'(x) = 0$,则函数 $f(x)$ 在 $I$ 上单调递增。

(4)若在区间 $I$ 上 $f'(x) \le 0$,且仅在有限个点上 $f'(x) = 0$,则函数 $f(x)$ 在 $I$ 上单调递减。

对于(3)和(4)的理解要点:

$f'(x) \ge 0$ (非负):这意味着函数 $f(x)$ 在区间 $I$ 上是单调不减的(即增函数)。函数值 $f(x)$ 永远不会随着 $x$ 的增大而减小。

$f'(x) = 0$ (有限点):这意味着函数曲线只有在孤立的几个点上有水平切线(即瞬时增长率为零)。

\medskip

\begin{example} 判定函数 $f(x) = x e^{-x}$ 的单调性。

    (3)求导数
    
    首先,使用乘积法则 $(uv)' = u'v + uv'$ 对函数 $f(x)$ 求导:
    
    设 $u = x$ 和 $v = e^{-x}$。则 $u' = 1$ 和 $v' = -e^{-x}$。
    \begin{align*}
        f'(x) &= (1) \cdot e^{-x} + x \cdot (-e^{-x}) \\
        &= e^{-x} - x e^{-x} = e^{-x} (1 - x)
    \end{align*}
    (2)分析导数的符号
    
    由于 $e^{-x}$ 对于所有的实数 $x$ 恒为正 ($e^{-x} > 0$),因此导数 $f'(x)$ 的符号完全由因子 $(1 - x)$ 决定。

    令 $f'(x) = 0$ 找到临界点:$$e^{-x} (1 - x) = 0$$
    因为 $e^{-x} \ne 0$,所以我们解 $1 - x = 0$,得到:$$x = 1$$

    (3)确定单调区间
    
    当 $x < 1$ 时单调递增:取 $x=0$ 检验。$1 - x = 1 - 0 = 1 > 0$。因此,$f'(x) = e^{-x} (1 - x) > 0$。函数 $f(x)$ 在区间 $(-\infty, 1)$ 上单调递增。
    
    当 $x > 1$ 时单调递减:取 $x=2$ 检验。$1 - x = 1 - 2 = -1 < 0$。因此,$f'(x) = e^{-x} (1 - x) < 0$。函数 $f(x)$ 在区间 $(1, +\infty)$ 上单调递减。

    函数在 $x=1$ 处达到局部最大值,$f(1) = 1 \cdot e^{-1} = \frac{1}{e}$。
\end{example}

\section{函数的极值}
函数极值(Extremum)是局部极大值(Local Maximum)和局部极小值(Local Minimum)的统称,反映了函数在其定义域的局部范围内的最大或最小性。

\medskip

\begin{definition}{函数极值}

    对于函数 $f(x)$ 区间内某一点 $x = c$:

    (1)$f(c)$ 是局部极大值,如果:$f(c) > f(c-\epsilon)$ 并且 $f(c) > f(c+\epsilon)$

    (2)$f(c)$ 是局部极小值,如果:$f(c) < f(c-\epsilon)$ 并且 $f(c) < f(c+\epsilon)$
\end{definition}

我们在 $x = c$ 的最小微区间 $[c-\epsilon,\;c+\epsilon]$ 进行极值定义和判断,彻底消除了传统极值定义的模糊化表述。

从图像上看,局部极大值点就像是函数图形上的山顶,局部极小值点就像是函数图形上的山谷底部。

\medskip

\begin{example} 判断 $x = 0$ 是否是函数的极值点,$f(x) = x^2,\;g(x) = x^3$。

    对于 $f(x) = x^2$:
    \begin{align*}
        &f(0) - f(0-\epsilon) = (0)^2 - (-\epsilon)^2 = -\epsilon^2 < 0 \\
        &f(0) - f(0+\epsilon) = (0)^2 - (\epsilon)^2 = -\epsilon^2 < 0
    \end{align*}
    所以 $f(0) < f(0-\epsilon)$ 并且 $f(0) < f(0+\epsilon)$,$f(0)$ 是函数 $f(x) = x^2$ 的局部极小值。

    对于 $g(x) = x^3$:
    \begin{align*}
        &g(0) - g(0-\epsilon) = (0)^3 - (-\epsilon)^3 = \epsilon^3 > 0 \\
        &g(0) - g(0+\epsilon) = (0)^3 - (\epsilon)^3 = -\epsilon^3 < 0
    \end{align*}
    所以 $x = 0$ 不是函数 $g(x) = x^3$ 的极值点。
\end{example}

\medskip

\begin{theorem}{极值判定}

    对于函数 $f(x)$ 区间内某一点 $x = c$,如果 $f'(c) = 0$ 并且 $f'(c-\epsilon)\cdot f'(c+\epsilon) < 0$,那么 $c$ 是函数 $f(x)$ 的极值点。并且

    (1)$f(c)$ 是局部极大值,如果:$f'(c-\epsilon) > 0$ 或者 $f'(c+\epsilon) < 0$

    (2)$f(c)$ 是局部极小值,如果:$f'(c-\epsilon) < 0$ 或者 $f'(c+\epsilon) > 0$

\end{theorem}

如果:$f'(c-\epsilon) > 0$,说明 $c$ 点左侧单调递增,如果 $f'(c+\epsilon) < 0$,说明 $c$ 点右侧单调递减,因此 $f(c)$ 是极大值。极小值的情况正好相反。

\medskip

\begin{example} 求函数 $f(x) = x^3 - 3x + 2$ 的所有极值点及极值。

    求导数:$f'(x) = (x^3 - 3x + 2)' = 3x^2 - 3$

    求驻点(令 $f'(x) = 0$):
    \begin{align*}
        &3x^2 - 3 = 0 \\
        &3(x^2 - 1) = 0 \\
        &x^2 = 1
    \end{align*}
    解得驻点:$x_1 = -1$,$x_2 = 1$。

    判断极值:
    \begin{align*}
        &f'(-1-\epsilon) = 3(-1-\epsilon)^2 - 3 = 6\epsilon + 3\epsilon^2 > 0 \\
        &f'(-1+\epsilon) = 3(-1+\epsilon)^2 - 3 = -6\epsilon + 3\epsilon^2 < 0 \\
        &f'(1-\epsilon) = 3(1-\epsilon)^2 - 3 = -6\epsilon + 3\epsilon^2 < 0 \\
        &f'(1+\epsilon) = 3(1+\epsilon)^2 - 3 = 6\epsilon + 3\epsilon^2 > 0 \\        
    \end{align*}
    $x_1 = -1$ 是极大值点,$f(-1) = (-1)^3 -3(-1) + 2 = 4$

    $x_2 = 1$ 是极小值点,$f(1) = (1)^3 -3(1) + 2 = 0$
\end{example}

\medskip

\begin{example} 求函数 $f(x) = x \ln x$ 的极值。

    由于函数定义域要求 $x > 0$,因此我们只在开区间 $(0, \infty)$ 上求解。

    求导数:使用乘积法则 $(uv)' = u'v + uv'$,其中 $u=x$, $v=\ln x$。
    \begin{align*}
        f'(x) &= (x)' \cdot \ln x + x \cdot (\ln x)' \\
        &= 1 \cdot \ln x + x \cdot \frac{1}{x} = \ln x + 1
    \end{align*}
    求驻点(令 $f'(x) = 0$):$\ln x + 1 = 0 \Rightarrow \ln x = -1$

    解得驻点:$x = e^{-1} = \frac{1}{e}$。
    
    判断极值:
    \begin{align*}
        f'\left(\frac{1}{e}-\epsilon\right) &= \ln\left(\frac{1}{e}-\epsilon\right) + 1 = \ln\left(\frac{1 - e\epsilon}{e}\right) + 1 \\
        &= \ln(1 - e\epsilon) - \ln e + 1 \\
        &= -e\epsilon < 0 \\
        \\
        f'\left(\frac{1}{e}+\epsilon\right) &= \ln\left(\frac{1}{e}+\epsilon\right) + 1 = \ln\left(\frac{1 + e\epsilon}{e}\right) + 1 \\
        &= \ln(1 + e\epsilon) - \ln e + 1 \\
        &= e\epsilon > 0        
    \end{align*}
    $x = e^{-1} = \frac{1}{e}$ 是极小值点,$f\left(\frac{1}{e}\right) = \frac{1}{e} \cdot \ln\left(\frac{1}{e}\right) = -\frac{1}{e}$
\end{example}

\medskip

\textbf{极值与最值(全局极值)的区别}

极值是一个局部性的概念,而最值是全局性的概念。

可以有多个局部极值,最多只有一个最大值和一个最小值。

局部极大值不一定大于局部极小值,最大值必然是所有函数值的最大。

最值需要评估所有极值和区间边界的值,并选取最大值或最小值。

\medskip

求解闭区间最值需要比较函数在所有驻点和闭区间端点处的函数值。

\begin{example} 求函数 $f(x) = x - 2 \sin x$ 在闭区间 $\left[0, \frac{\pi}{2}\right]$ 上的最大值和最小值。

    求一阶导数:$f'(x) = (x - 2 \sin x)' = 1 - 2 \cos x$

    求驻点(令 $f'(x) = 0$):$1 - 2 \cos x = 0 \Rightarrow \cos x = \frac{1}{2}$

    在区间 $\left[0, \frac{\pi}{2}\right]$ 内,满足 $\cos x = 1/2$ 的点为:$$x = \frac{\pi}{3}$$
    计算函数值: 比较区间端点和驻点处的函数值。
    \begin{align*}
        f(0) &= 0 - 2 \sin(0) = 0 - 0 = 0 \\
        f\left(\frac{\pi}{2}\right) &= \frac{\pi}{2} - 2 \sin\left(\frac{\pi}{2}\right) = \frac{\pi}{2} - 2(1) \\
        &\approx 1.57 - 2 = -0.43 \\
        f\left(\frac{\pi}{3}\right) &= \frac{\pi}{3} - 2 \sin\left(\frac{\pi}{3}\right) = \frac{\pi}{3} - 2 \left(\frac{\sqrt{3}}{2}\right) = \frac{\pi}{3} - \sqrt{3} \\
        &\approx 1.047 - 1.732 = -0.685
    \end{align*}
    比较结果:最大值:$f(0) = 0$,最小值:$f\left(\frac{\pi}{3}\right) \approx -0.685$
\end{example}

\section{函数的渐近线}
渐近线(Asymptote)是一条直线,当函数曲线上的点无限远离原点(即 $x$ 或 $y$ 坐标趋于无穷大)时,该曲线与这条直线之间的距离趋近于零。

简单来说,渐近线就是函数图形在无穷远处无限靠近却永远不会相交的直线。

渐近线主要分为三类:垂直渐近线,水平渐近线和斜渐近线。

\medskip

\textbf{垂直渐近线 (Vertical Asymptote)}

垂直渐近线是形如 $x = a$ 的竖直线。

\medskip

\begin{definition}{垂直渐近线}

    如果当 $x$ 从左侧或右侧趋近于某个有限值 $a$ 时,函数 $f(x)$ 的值趋向于正无穷 ($\infty$) 或负无穷 ($-\infty$),即:
    \begin{align}
        \lim_{x \to a^-} f(x) &= f(a-\epsilon) = \pm \infty \\
        \lim_{x \to a^+} f(x) &= f(a+\epsilon) = \pm \infty
    \end{align}
    那么直线 $x = a$ 就是函数 $f(x)$ 的垂直渐近线。
\end{definition}
对于有理函数(即两个多项式的比 $f(x) = \frac{P(x)}{Q(x)}$),垂直渐近线通常出现在使分母 $Q(x)$ 为零,但分子 $P(x)$ 不为零的点 $x=a$ 处。

\medskip

\begin{example} 求函数 $f(x) = \frac{1}{x-2}$ 的垂直渐近线。

    分母 $x-2 = 0$ 时,$x = 2$,此时分子 $1 \neq 0$。

    检查极限:
    \begin{align*}
        \lim_{x \to 2^-} f(x) &= f(2-\epsilon) = \frac{1}{(2-\epsilon)-2} \\
        &= \frac{1}{-\epsilon} = -\infty \\
        \\
        \lim_{x \to 2^+} f(x) &= f(2+\epsilon) = \frac{1}{(2+\epsilon)-2} \\
        &= \frac{1}{\epsilon} = \infty
    \end{align*}
    所以,直线 $x = 2$ 是垂直渐近线。
\end{example}

\medskip

\textbf{水平渐近线 (Horizontal Asymptote)}

水平渐近线是形如 $y = b$ 的水平线。

\medskip

\begin{definition}{水平渐近线}

    如果当 $x$ 趋向于正无穷 ($\infty$) 或负无穷 ($-\infty$) 时,函数 $f(x)$ 的值趋近于某个有限值 $b$,即:
    \begin{align}
        \lim_{x \to \infty} f(x) &= f(\infty) = b \\
        \lim_{x \to -\infty} f(x) &= f(-\infty) = b
    \end{align}
    那么直线 $y = b$ 就是函数 $f(x)$ 的水平渐近线。
\end{definition}

\medskip

\begin{example} 求函数 $f(x) = \frac{3x^2 - 5}{2x^2 + x}$ 的水平渐近线。

    检查极限:
    \begin{align*}
        \lim_{x \to \infty} f(x) &= f(\infty) = \frac{3\infty^2 - 5}{2\infty^2 + x} \\
        &= \frac{3\infty^2}{2\infty^2} = \frac{3}{2} 
    \end{align*}
    所以水平渐近线为: $y = \frac{3}{2}$。
\end{example}

\medskip

\textbf{斜渐近线 (Slant/Oblique Asymptote))}

斜渐近线是形如 $y = kx + b$ 的斜直线(其中 $k \neq 0$)。

\medskip

\begin{definition}{斜渐近线}
    
    如果函数 $f(x)$ 满足:
    \begin{align}
        k &= \lim_{x \to \pm \infty} \frac{f(x)}{x} = \frac{f(\pm\infty)}{\pm\infty} \\
        b &= \lim_{x \to \pm \infty} [f(x) - kx] = f(\pm\infty) - k(\pm\infty)
    \end{align}
    且 $k$ 和 $b$ 都是有限的实数,那么直线 $y = kx + b$ 就是函数 $f(x)$ 的斜渐近线。
\end{definition}

对于有理函数 $f(x) = \frac{P(x)}{Q(x)}$,只有当分子的次数恰好比分母的次数高 1 ($n = m + 1$) 时,函数才可能存在斜渐近线。

\medskip

\begin{example} 求函数 $f(x) = \frac{x^2 + 1}{x - 1}$ 的斜渐近线。

    
    分子次数 $n=2$,分母次数 $m=1$,$n=m+1$,存在斜渐近线。
    \begin{align*}
        k &= \frac{f(\infty)}{\infty} = \frac{\infty^2 + 1}{(\infty - 1)\cdot \infty} \\
        &= \frac{\infty^2 + 1}{\infty^2 - \infty} = \frac{\infty^2 }{\infty^2 } \\
        &= 1 \\
        b &= f(\infty) - k(\infty) = \frac{\infty^2 + 1}{\infty - 1} - \infty \\
        &= \frac{\infty^2 + 1}{\infty - 1} - \frac{\infty^2 - \infty}{\infty - 1} \\
        &= \frac{\infty + 1}{\infty - 1} \\
        &= 1
    \end{align*}
    因此,直线 $y = x + 1$ 是斜渐近线。
\end{example}

\section{牛顿法解方程}
牛顿法是一种高效的数值计算方法,广泛应用于各种需要求解非线性方程的工程和科学计算领域。

牛顿法的核心思想是用切线来逐步逼近函数 $f(x)$ 的根(即函数图形与 $x$ 轴的交点):

(1)初始点: 首先,选择一个接近真实解的初始猜测值 $x_0$。

(2)切线逼近: 在函数曲线 $y=f(x)$ 上,找到点 $(x_n, f(x_n))$。

(3)局部线性化: 用该点的切线来代替曲线在 $x_n$ 附近的走势(这是对函数进行局部线性化的操作)。

(4)求新近似解: 找到这条切线与 $x$ 轴的交点 $x_{n+1}$。这个 $x_{n+1}$ 通常会比 $x_n$ 更接近方程的真实解。

(5)迭代: 重复以上步骤,用 $x_{n+1}$ 作为新的猜测值,不断迭代,直到相邻两次近似值的差小于预设的精度要求,或者函数值 $|f(x_{n+1})|$ 足够接近零。

牛顿法背后的几何结构如图所示。
\begin{figure}[htbp] 
    \centering
    \includegraphics[width=0.6\textwidth]{images/04/4.1.png} 
    \caption{\textbf{牛顿法解方程}}
\end{figure}

\textbf{迭代公式代数推导}

从代数的角度,牛顿法是基于函数的泰勒级数展开。

我们假设 $x^*$ 是方程 $f(x)=0$ 的精确解,即 $f(x^*)=0$,而 $x_n$ 是当前的一个近似解。令 $\Delta x = x^* - x_n$,则 $x^* = x_n + \Delta x$。

将 $f(x)$ 在 $x_n$ 处进行泰勒级数展开:
$$f(x) = f(x_n) + f'(x_n)(x - x_n) + \frac{f''(x_n)}{2!}(x - x_n)^2 + \cdots$$
由于我们寻找的是根 $x^*$,所以 $f(x^*)=0$。代入 $x=x^*$:
$$0 = f(x_n) + f'(x_n)(x^* - x_n) + \frac{f''(x_n)}{2!}(x^* - x_n)^2 + \cdots$$
牛顿法的关键在于截断:我们只保留泰勒级数的前两项(即用切线进行线性近似),忽略高阶项(二阶及更高阶的项):
$$0 \approx f(x_n) + f'(x_n)(x^* - x_n)$$
将 $x^*$ 用下一个近似值 $x_{n+1}$ 代替,并解出 $x_{n+1}$:
\begin{align*}
    &0 = f(x_n) + f'(x_n)(x_{n+1} - x_n) \\
    &f'(x_n)(x_{n+1} - x_n) = -f(x_n) \\
    &x_{n+1} - x_n = -\frac{f(x_n)}{f'(x_n)} \\
    &x_{n+1} = x_n - \frac{f(x_n)}{f'(x_n)}
\end{align*}
最终得到牛顿法的迭代公式:$$x_{n+1} = x_n - \frac{f(x_n)}{f'(x_n)}$$
其中,$f'(x_n)$ 是函数 $f(x)$ 在 $x_n$ 处的导数(切线的斜率)。

\medskip

\textbf{牛顿法优点与限制}

收敛速度快:在解的附近,通常具有二阶收敛的特性,这意味着每次迭代正确的小数位数大致会翻倍,收敛速度非常快。

需要导数:要求函数 $f(x)$ 可导,并且每次迭代都需要计算导数值 $f'(x_n)$。

对初值敏感:如果初始值 $x_0$ 离真实根太远,或者在迭代过程中 $f'(x_n)$ 接近于零(即切线接近水平),方法可能不收敛或收敛到错误的根。

\medskip

\begin{example} 求解 $e^{-x} = x$ 的根

    这等价于求解函数 $f(x) = e^{-x} - x = 0$ 的根。

    原函数: $f(x) = e^{-x} - x$

    导数: $f'(x) = (e^{-x} - x)' = -e^{-x} - 1$

    确定迭代公式

    将 $f(x)$ 和 $f'(x)$ 代入牛顿法的迭代公式 $x_{n+1} = x_n - \frac{f(x_n)}{f'(x_n)}$:
    $$x_{n+1} = x_n - \frac{e^{-x_n} - x_n}{-e^{-x_n} - 1} = x_n + \frac{e^{-x_n} - x_n}{e^{-x_n} + 1}$$
    开始迭代
    
    首先,通过观察或作图,我们可以找到一个合理的初始猜测。
    
    当 $x=0$ 时,$f(0) = e^0 - 0 = 1$;当 $x=1$ 时,$f(1) = e^{-1} - 1 \approx 0.368 - 1 = -0.632$。
    
    由于函数值在 0 和 1 之间变号,所以根在 $[0, 1]$ 区间内。我们取初始值 $x_0 = 0.5$:
    \begin{align*}
        &x_1 = x_0 - \frac{f(x_0)}{f'(x_0)} = 0.5 - \frac{f(0.5)}{f'(0.5)} \approx 0.566311003197 \\
        &x_2 = x_1 - \frac{f(x_1)}{f'(x_1)} \approx 0.567143165035 \\
        &x_3 = x_2 - \frac{f(x_2)}{f'(x_2)} \approx 0.56714329041
    \end{align*}
    经过 3 次迭代,我们得到的近似解 $x_3 \approx 0.56714329041$ 已经达到了极高的精度。
\end{example}

这个例子强调了牛顿法的优势:对于具有良好导数性质的函数,它能以二次收敛速度迅速找到方程的根。
