\chapter{点积分和积分方法}



\section{导数的定义与几何意义}
函数 $f(x)$ 在点 $x_0$ 处的导数定义为:
$$ f'(x_0) = \lim_{\Delta x \to 0} \frac{f(x_0 + \Delta x) - f(x_0)}{\Delta x} $$
导数表示函数在某点变化率的精确度量。


💡 是的,**积分**(特指**不定积分**)是**微分**的**逆运算**(或反运算)。

---

## 🔁 积分与微分的互逆关系

这种互逆关系是微积分学的核心,由**微积分基本定理**(Fundamental Theorem of Calculus)所确立。

### 1. 不定积分(求原函数)

*   **运算过程:**
    *   **微分:** 给定一个函数 $F(x)$,求其导数 $F'(x) = f(x)$。
    *   **不定积分:** 给定一个函数 $f(x)$,求其**原函数**(或反导数)$F(x)$,即 $\int f(x)\,dx = F(x) + C$。
*   **逆运算体现:** 不定积分(求原函数)正是**微分**(求导数)的**逆过程**。
    *   如果你对一个函数**先求导,再求不定积分**,你会回到原来的函数(相差一个常数 $C$)。
    $$\int \left(\frac{d}{dx} F(x)\right)\,dx = \int f(x)\,dx = F(x) + C$$
    *   如果你对一个函数**先求不定积分,再求导**,你会回到原来的函数。
    $$\frac{d}{dx} \left(\int f(x)\,dx\right) = \frac{d}{dx} (F(x) + C) = f(x)$$

### 2. 定积分(求面积)

定积分 $\int_a^b f(x)\,dx$ 本身是求面积(或积累量)的操作,其结果是一个**数值**,但它与微分的关系是通过**微积分基本定理的第二部分**联系起来的:

$$ \int_a^b f(x)\,dx = F(b) - F(a) $$

其中 $F(x)$ 是 $f(x)$ 的任意一个原函数(不定积分)。这表明,虽然定积分的定义是基于黎曼和的极限,但它的**计算**必须依赖于**不定积分**(即微分的逆运算)。

## 📝 总结

| 运算 | 目的 | 逆运算 |
| :--- | :--- | :--- |
| **微分** | 描述**变化率**(切线斜率) | **不定积分** |
| **不定积分** | 寻找**原函数**(反导数) | **微分** |
| **定积分** | 求**积累量**(曲线下面积) | 通过不定积分与微分间接关联 |

因此,**不定积分**与**微分**是互为逆运算的,这是整个微积分体系能够成立和方便计算的基石。

您想知道如何计算一些常见函数的微分和积分吗?