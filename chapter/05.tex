\chapter{点积分和积分方法}



\section{导数的定义与几何意义}
函数 $f(x)$ 在点 $x_0$ 处的导数定义为:
$$ f'(x_0) = \lim_{\Delta x \to 0} \frac{f(x_0 + \Delta x) - f(x_0)}{\Delta x} $$
导数表示函数在某点变化率的精确度量。


\textbf{积分与微分的互逆关系}

不定积分(求原函数)

运算过程:

微分: 给定一个函数 $F(x)$,求其导数 $F'(x) = f(x)$。

不定积分: 给定一个函数 $f(x)$,求其原函数 $F(x)$,即 $\int f(x)\,dx = F(x) + C$。

逆运算体现: 不定积分正是微分(求导数)的逆过程。如果你对一个函数先求导,再求不定积分,你会回到原来的函数(相差一个常数 $C$)。
$$\int \frac{dF(x)}{dx}\,dx = \int f(x)\,dx = F(x) + C$$
如果对一个函数先求不定积分,再求导,就会回到原来的函数。
$$\frac{d}{dx} \left(\int f(x)\,dx\right) = \frac{d}{dx} (F(x) + C) = f(x)$$

\begin{center}
\begin{tikzpicture}[
    % 样式定义
    function/.style={font=\color{blue!70!black}\large}, 
    op_arrow/.style={very thick, draw=black, >=stealth}
]

% 1. 定义节点 (Functions)
\node[function] (Fx) at (0, 0) {$F(x)$};
\node[function, right=3.5cm of Fx] (dFx) {$dF(x)$};
\node[function, right=3.5cm of dFx] (fx) {$f(x)$};


% --- 上半部分: 微分标签 (方向: F(x) -> dF(x) -> f(x)) ---
% 路径 1: F(x) -> dF(x)
\draw[->, op_arrow] (Fx.north) to[out=60, in=120] node[midway, above, label] {微分} (dFx.north);
% 路径 2: dF(x) -> f(x)
\draw[->, op_arrow] (dFx.north) to[out=60, in=120] node[midway, above, label] {$\frac{dF(x)}{dx}$} (fx.north);


% --- 下半部分: 积分标签 (方向: f(x) -> dF(x) -> F(x)) ---
% 路径 3: f(x) -> dF(x)
\draw[->, op_arrow] (fx.south) to[out=-120, in=-60] node[midway, below, label] {$f(x) \cdot dx$} (dFx.south);
% 路径 4: dF(x) -> F(x)
\draw[->, op_arrow] (dFx.south) to[out=-120, in=-60] node[midway, below, label] {积分} (Fx.south);


\end{tikzpicture}
\end{center}
