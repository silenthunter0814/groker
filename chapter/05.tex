\chapter{不定积分和积分方法}
不定积分的概念起源于17世纪,由艾萨克·牛顿和戈特弗里德·莱布尼兹独立发展,作为微积分学的基础。

不定积分也称为反导函数或原函数,它本质上是求导数的逆运算,用于找到一个函数的“原始形式”。

\section{不定积分的基本概念}
考虑微分表达式 $dF(x) = F(x + \epsilon) - F(x)$,我们希望用数学的方法由微分 $dF(x)$ 返回原函数 $F(x)$,这时我们可以定义一个专用符号如 $\int$ 来表示这种运算操作,即
\begin{align*}
    \int\,dF(x) = F(x)
\end{align*}
由此我们可以进行一些基本运算:
\begin{align*}
    \int\,dx = x \qquad \int\,d(\ln x + \sin x) = \ln x + \sin x \\
    F(x) = x^2 + 2x + 3 \Rigntarrow \int\,dF(x) = x^2 + 2x + 3
\end{align*}

接下来考虑微分的导数展开形式:
\begin{align}
    dF(x) = f(x)\,dx
\end{align}
如果等式(5.1.1)从右向左计算:
\begin{align}
    f(x)\,dx = d(F(x) + C) = [F(x+\epsilon) + C] - [F(x) + C] = F(x + \epsilon) - F(x)
\end{align}
其中 $C$ 为任意常数。也就是说导函数 $f(x)$ 对应的不是唯一原函数 $F(x)$,而是一个原函数集合 $F(x) + C$。所以我们有
begin{align}
    \int f(x)\,dx = \int\,d(F(x) + C) = F(x) + C
\end{align}
如果我们将等式(5.1.2)看作众所周知的常识,为了避免计算中间过程的繁琐性,作为一种约定,我们省略微分表达式中的常数项 $C$,而只是在最终结果中列出。这样等式(5.1.3)就变成:
begin{align}
    \int f(x)\,dx = \int\,dF(x) = F(x) + C
\end{align}
这样,我们就完成了由导函数推导计算原函数的过程,并可据此总结出不定积分的定义。

\begin{definition}{不定积分}

    不定积分是微分的逆运算,作用于微分表达式并解析出原函数。用符号 $\int$ 表示不定积分,我们有
    \begin{align}
        \int\,dF(x) = F(x)
    \end{align}
    对于以导函数表示的微分展开形式,不定积分的运算结果是原函数(相对于导函数)的集合,即
    \begin{align}
        \int f(x)\,dx = \int\,dF(x) = F(x) + C
    \end{align}
\end{definition}

\medskip

\textbf{不定积分与微分的互逆关系}

微分: 给定一个函数 $F(x)$,求其导数 $\frac{dF(x)}{dx} = f(x)$。

不定积分: 给定一个函数 $f(x)$,求其原函数 $F(x)$,即 $\int f(x)\,dx = F(x) + C$。

逆运算体现: 不定积分正是微分(求导数)的逆过程。如果对一个函数先求导,再求不定积分,会回到原来的函数(相差一个常数 $C$)。
$$\int \frac{dF(x)}{dx}\,dx = \int f(x)\,dx = F(x) + C$$
如果对一个函数先求不定积分,再求导,就会回到原来的函数。
$$\frac{d}{dx} \left(\int f(x)\,dx\right) = \frac{d}{dx} (F(x) + C) = f(x)$$

\begin{center}
\begin{tikzpicture}[
    % 样式定义
    function/.style={font=\color{blue!70!black}\large}, 
    op_arrow/.style={very thick, draw=black, >=stealth}
]

% 1. 定义节点 (Functions)
\node[function] (Fx) at (0, 0) {$F(x)$};
\node[function, right=3.5cm of Fx] (dFx) {$dF(x)$};
\node[function, right=3.5cm of dFx] (fx) {$f(x)$};


% --- 上半部分: 微分标签 (方向: F(x) -> dF(x) -> f(x)) ---
% 路径 1: F(x) -> dF(x)
\draw[->, op_arrow] (Fx.north) to[out=60, in=120] node[midway, above, label] {微分} (dFx.north);
% 路径 2: dF(x) -> f(x)
\draw[->, op_arrow] (dFx.north) to[out=60, in=120] node[midway, above, label] {$\frac{dF(x)}{dx}$} (fx.north);


% --- 下半部分: 积分标签 (方向: f(x) -> dF(x) -> F(x)) ---
% 路径 3: f(x) -> dF(x)
\draw[->, op_arrow] (fx.south) to[out=-120, in=-60] node[midway, below, label] {$f(x) \cdot dx$} (dFx.south);
% 路径 4: dF(x) -> F(x)
\draw[->, op_arrow] (dFx.south) to[out=-120, in=-60] node[midway, below, label] {积分} (Fx.south);

\end{tikzpicture}
\end{center}




\section{导数的定义与几何意义}
函数 $f(x)$ 在点 $x_0$ 处的导数定义为:
$$ f'(x_0) = \lim_{\Delta x \to 0} \frac{f(x_0 + \Delta x) - f(x_0)}{\Delta x} $$
导数表示函数在某点变化率的精确度量。


