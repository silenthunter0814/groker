\chapter{格洛克代数空间}
本章首先从常识入手理解无穷大和无穷小,并得出基本的运算法则。在此基础上实现格洛克代数空间。最后给出无穷大和无穷小的正式定义和运算(规则)列表。

\section{初步理解无穷大和无穷小}
自微积分诞生以来,如何阐释、理解无穷大和无穷小以及由此引出的函数的极限一直是问题的核心,所以让我们以此为切入点,开始问题的探索。

\medskip

{\textbf{格洛克:无穷大和无穷小的直观理解}}

无穷大和无穷小是具有特殊作用的两个数,分别用 $\infty, \epsilon$ 表示,对于 $\infty$,所有实数都是它的无穷小;对于 $\epsilon$,所有实数都是它的无穷大。

由此产生如下运算法则:对于任意实数 C
\begin{align*}\infty + C = \infty \\\epsilon + C = C \end{align*}

这是基于 $\infty, \epsilon$ 本身的直观语义得出的。更进一步,基于数的性质,我们得到:
\begin{align*}
    C_1\infty + C_2 = C_1\left(\infty + \frac{C_2}{C_1}\right) = C_1\infty \\
    C_1\epsilon + C_2 = C_1\left(\epsilon + \frac{C_2}{C_1}\right) = C_2 \\
    C_1\infty^2 + C_2\infty = \infty(C_1\infty + C_2) = C_1\infty^2 \\
    C_1\epsilon^2 + C_2\epsilon = \epsilon(C_1\epsilon + C_2) = C_2\epsilon
\end{align*}

这和 200 多年来的传统解释完全不同,甚至相反。传统解释中并不认为无穷大和无穷小是数,因此不能直接进行数学运算,强调它是一个无限趋近的“动态"过程,因此理解起来要困难复杂得多。

格洛克的直观解释如果可以成立,会使微积分学变得简单而清晰,这就需要建立一个直观而简单的可视化代数模型来阐释其合理性。

\section{格洛克代数空间}

\medskip

\textbf{格洛克代数环轴}

和传统的实数数轴不同,格洛克数轴是一个圆环,圆环长度固定为 $\infty$ ,并将其分割为相等的 $\infty$ 段,每段长度为 1, 将数轴逆时针依次标注 0, 1, 2, ...,顺时针标注为 -1, -2, ...,这样我们就得到了一个覆盖 $\left(-\infty, \infty\right)$ 数字空间的环形数轴。

\begin{figure}[htbp] 
    \centering
    \includegraphics[width=0.4\textwidth]{images/01/1.1.png} 
    \caption{\textbf{格洛克代数环轴}}
\end{figure}

下图是格洛克环轴展开后的样子,在数轴的右端标注的是单位刻度的长度量级 $\infty^0 = 1$,因此称为格洛克 0 阶数轴。
\begin{figure}[htbp] 
    \centering
    \includegraphics[width=0.6\textwidth]{images/01/1.2.png} 
    \caption{\textbf{格洛克 0 阶数轴}}
\end{figure}

需要注意的是,$\infty, -\infty$ 和原点 0 重合,这意味着数值大小达到 $\infty$ 时需要进位,以表示 $\infty$ 量级或更大的数。基于同样的原理,我们还需要量级更小的单位,即 $\epsilon$,用来进位到格洛克 0 阶数轴。
\begin{figure}[htbp] 
    \centering
    \includegraphics[width=0.6\textwidth]{images/01/1.3.png} 
    \caption{\textbf{宏空间和微空间}}
\end{figure}

$\infty$ 以及更大的量级称为\textbf{格洛克宏空间},对应的,$\epsilon$ 以及更小的量级称为\textbf{格洛克微空间},图1.3 表示了格洛克宏空间 1 轴,格洛克 0 轴和格洛克微空间 1 轴。

宏空间 1 轴覆盖区间 $[0, \infty^2)$,单位刻度大小为 $1\infty$,将 1 单位刻度放大 $\infty$ 倍,得到格洛克 0 轴。

格洛克 0 轴覆盖区间 $[0, \infty)$,单位刻度大小为 1,将 1 单位刻度放大 $\infty$ 倍,得到微空间 1 轴。

微空间 1 轴覆盖区间 $[0, 1)$,单位刻度大小为 $\frac 1 \infty$,即 $1\epsilon$。

将宏空间和微空间以同样的方式分别向上和向下延展,形成完整的格洛克代数空间。

\medskip

\begin{info}{宏空间和微空间}
    这是一个相对的概念,如 $\infty^{n+1}$ 是 $\infty^n$ 的宏空间,$\infty^n$ 是 $\infty^{n+1}$ 的微空间;$\epsilon^n$ 是 $\epsilon^{n+1}$ 的宏空间,$\epsilon^{n+1}$ 是 $\epsilon^n$ 的微空间。

    宏空间是微空间的无穷大,微空间是宏空间的无穷小。
\end{info}

\medskip

\textbf{格洛克代数空间}

完整格洛克代数空间如图1.4 所示。
\begin{figure}[htbp] 
    \centering
    \includegraphics[width=0.8\textwidth]{images/01/1.4.png} 
    \caption{\textbf{格洛克代数空间}}
\end{figure}

由于指数进位回 0,$\infty^\infty$ 和 $\epsilon^\infty$ 回归原点 0,图中未予显示。

格洛克代数空间最大覆盖范围 $[0,\infty^\infty)$,最小单位刻度 $1\epsilon^{\infty-1}$,可以满足任何数值的可视化标注。

需要注意的是,由于所有可描述的实数都是 $\infty$ 的无穷小,正数 C 只能标注在任一轴的最左侧的相对无穷小区间;同样的,负数 -C 只能标注在最右侧的相对无穷小区间。

\begin{example}
    在格洛克代数空间上标注:
    
    $(1)\: a = 0.5\infty + 1,\quad (2)\: b = 2\infty - 2,\quad (3)\: c = a + b$。
    
    \medskip

    \textbf{解:}
    \begin{align*}
    c &= a + b \\
      &= (0.5\infty + 1) + (2\infty - 2) \\
      &= (0.5\infty + 2\infty) + (1 - 2) \\
      &= 2.5\infty - 1
    \end{align*}

    a,b,c 如图1.5 所示。
\end{example}
\begin{figure}[htbp] 
    \centering
    \includegraphics[width=0.8\textwidth]{images/01/1.5.png} 
    \caption{\textbf{数值标注}}
\end{figure}

\section{无穷大和无穷小的定义}
经过前面两节关于无穷大和无穷小的讨论和格洛克代数空间的形成,下面给出无穷大和无穷小的正式定义。

\begin{definition}{无穷大和无穷小}

    设 $\infty, \epsilon$ 是未解析数(unsolved number), $\infty$ 是正整数, 表示无穷大, $\epsilon$ 表示无穷小。

    \medskip

    \textbf{性质列表}

    (1)$\infty, \epsilon$ 互为倒数,即 $\infty\cdot\epsilon = 1$ 。

    (2)对于任意实数 C,整数 n,$\infty^{n+1} + C\infty^n = \infty^{n+1},\; \epsilon^n + C\epsilon^{n+1} = \epsilon^n$。

    (3) $\epsilon$ 是最小的正数,$-\epsilon$ 是最大的负数,$0^+ = \epsilon,\; 0^- = -\epsilon$。

    (4)对于任意实数 C,$C^+ = C + \epsilon,\; C^- = C - \epsilon$。

    (5)对于 $n > 1$ 有,$\log_n\infty < \sqrt[n]{\infty} < \infty < \infty^n < n^\infty < \infty!$。

    (6)等效重定义。如果关于 $\infty$ 的表达式 $g(\infty)$ 的值仍是无穷大,那么 $\infty$ 可以重定义为 $g(\infty)$,记作:$\infty \gets g(\infty), \epsilon \gets \frac{1}{g(\infty)}$。
\end{definition}

\medskip

性质(1)对无穷大和无穷小的关系进行了标准化,使得数学属性更加清晰,高阶无穷大和无穷小的比较和转换变成了数学计算,而不是逻辑分析。

\medskip

对于性质(2),当 $n = 0$ 时有
\begin{align}
    \infty + C = \infty \\
    \quad 1 + C\epsilon = 1
\end{align}

将 (1.1)两边同时减去 $\infty$,将(1.2)两边同时减去 1,得到
\begin{align}
    C = 0 \\
    C\epsilon = 0
\end{align}

(1.3)说明在格洛克宏空间1轴(1阶无穷大)的视角来看,无论 C 多么大,都是 $\infty$ 的无穷小,格洛克宏空间1轴上的值都为 0;(1.4)说明了在格洛克 0 轴(实数轴)的视角来看,无论 n 多么大,$n\epsilon$ 都是无穷小,在格洛克 0 轴上的值都为 0。这也是后续章节进行极限计算时的正确结果。

性质(2)本质上是将本章第一节关于无穷大和无穷小的运算规则进行了合并。

\medskip

性质(3)可以合并到性质(4),分开是为了突出 0 的特殊性。$C^+$ 表示实数 C 右侧最靠近它的数,$C^-$ 表示 C 左侧最靠近它的数。这可以用格洛克微空间展开的术语进行理解。

\medskip

\textbf{格洛克微空间展开}

对于展开点 C,取单位长度区间 $[C, C+1]$ 并放大 $\infty$ 倍数,当我们将 C 点与微空间 1 轴的 0 对齐,将会“看到”最靠近 C 右侧的坐标刻度是 $1\epsilon$。类似地,取单位长度区间 $[C-1, C]$ 并放大 $\infty$ 倍数,将 C-1 点与微空间 1 轴的 0 对齐,将会“看到”最靠近 C 左侧的坐标刻度是 $-1\epsilon$。如图1.6 所示。
\begin{figure}[htbp] 
    \centering
    \includegraphics[width=0.8\textwidth]{images/01/1.6.png} 
    \caption{\textbf{格洛克微空间展开}}
\end{figure}

事实上,我们也可以指定 $C^+ = C + n\epsilon,\; C^- = C - n\epsilon$,结果也是正确的。从格洛克 0 轴(或实数轴)的角度来看,无论我们指定靠近 C 的距离是多么小,无论是百万分之一还是千万分之一,都是 $n\epsilon$ 的无穷大。

\medskip

性质(5)明确了不等式的各项位于不同阶的格洛克宏空间,不等式的左侧都是右侧的无穷小,体现了如下运算规则:
\begin{align*}
    \log_n\infty + \sqrt[n]{\infty} + \infty + \infty^n + n^\infty + \infty! \\
    = \sqrt[n]{\infty} + \infty + \infty^n + n^\infty + \infty! \\
    = \infty + \infty^n + n^\infty + \infty! \\
    = \infty^n + n^\infty + \infty! \\
    = n^\infty + \infty! \\
    = \infty!
\end{align*}

虽然不等式的每一项都是无穷大,但它们的增长速度比值都达到了无穷大的量级,这可以由函数图形进行直观的观察,这里略去繁琐的证明过程。

\medskip

关于性质(6),传统的无穷大和无穷小并没有明确的等阶划分,事实上覆盖了整个格洛克代数空间。通过等效重定义,我们可以得到一个明确的计算结果。如
\begin{align*}
    f(\infty) &= \log_n{\infty} \\
    &= \log_n{n^\infty} \qquad (\infty \gets n^\infty) \\
    &= \infty
\end{align*}

观察对数函数的图形,它是一个增长异常缓慢的函数,因此 $\log_n{\infty}$ 是一个非常低阶的无穷大,通过等效重定义,我们得到了一个明确的无穷大结果,本质上我们改变了无穷大在宏空间的位置,但没有改变传统无穷大的结果,这在极限计算时很有用。

\medskip

\begin{example}
    数列用符号 $\{f(n)\}$ 表示,其中 $f(n)$ 是通项公式。如果 $n \to \infty,\; f(n) = C$ 那么数列是收敛的;如果 $n \to \infty,\; f(n) = \infty$ 那么数列是发散的;否则数列既不收敛也不发散。

    判断下列数列的收敛性:

    $(1)\,\left\{(-1)^n \frac 1 n\right\} \quad (2)\,\left\{(-1)^n \frac {n+1}{n}\right\} \quad (3)\,\left\{n - \frac {1}{n}\right\}$
    
    \medskip

    \textbf{解:}

    (1)
    \begin{align*}
        f(n) &= (-1)^n \frac 1 n \\
        f(\infty) &= (-1)^\infty \frac {1}{\infty} \\
        f(\infty) &= (-1)^\infty \epsilon 
    \end{align*}

    $f(\infty) = \epsilon = 0$,如果 $\infty$ 为偶数;$f(\infty) = -\epsilon = 0$,如果 $\infty$ 为奇数。 
    
    综合奇偶两种情况,$f(\infty) = 0$,数列收敛。

    (2)
    \begin{align*}
        f(n) &= (-1)^n \frac {n+1}{n} \\
        f(\infty) &= (-1)^\infty \frac {\infty+1}{\infty} \\
        f(\infty) &= (-1)^\infty \frac {\infty}{\infty} \\
        f(\infty) &= (-1)^\infty
    \end{align*}
    数列在 1 和 -1 之间来回振荡,既不收敛也不发散。
    
    (3)
    \begin{align*}
        f(n) &= n - \frac {1}{n} \\
        f(\infty) &= \infty - \frac {1}{\infty} \\
        f(\infty) &= \infty - \epsilon \\
        f(\infty) &= \infty
    \end{align*}
    数列发散。    
\end{example}

通过上面的例子可以看出,关于无穷大和无穷小的计算完全符合数学运算法则,比传统的基于数学分析的形式化方法简单而清晰。

\medskip

\begin{example}
    \begin{align*}
        f(\infty) &= \frac{\sqrt{\infty + 3}}{\infty} \\
        &= \frac{\sqrt{(\infty^2 -3) + 3}}{\infty^2 -3} \qquad (\infty \gets \infty^2 -3) \\
        &= \frac \infty {\infty^2} \\
        &= \frac 1 \infty \\
        &= \epsilon \\
        &= 0
    \end{align*}
\end{example}

通过等效重定义,我们精确计算出了表达式的结果,无需进行繁琐的形式化数学分析。

\medskip

\begin{info}{格洛克观点}
    格洛克代数空间的建立可以使我们抛弃传统的基于语言的数学逻辑,转而专注于简单精确的数学计算。后续章节我们将会看到微积分学是多么的简单,大量晦涩的定义、定理将被抛弃,一切变得如此不可思议。
\end{info}
