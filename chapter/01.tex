\chapter{格洛克代数空间}
本章首先从常识入手理解无穷大和无穷小,并得出基本的运算法则。在此基础上实现格洛克代数空间。最后给出无穷大和无穷小的正式定义和运算(规则)列表。

\section{初步理解无穷大和无穷小}
自微积分诞生以来,如何阐释、理解无穷大和无穷小以及由此引出的函数的极限一直是问题的核心,所以让我们以此为切入点,开始问题的探索。

\medskip

{\textbf{格洛克:无穷大和无穷小的直观理解}}

无穷大和无穷小是具有特殊作用的两个数,分别用 $\infty, \varepsilon$ 表示,对于 $\infty$,所有实数都是它的无穷小;对于 $\varepsilon$,所有实数都是它的无穷大。

由此产生如下运算法则:对于任意实数 C
\begin{align*}\infty + C = \infty \\\varepsilon + C = C \end{align*}

这是基于 $\infty, \varepsilon$ 本身的直观语义得出的。更进一步,基于数的性质,我们得到:
\begin{align*}
    C_1\infty + C_2 = C_1\left(\infty + \frac{C_2}{C_1}\right) = C_1\infty \\
    C_1\varepsilon + C_2 = C_1\left(\varepsilon + \frac{C_2}{C_1}\right) = C_2 \\
    C_1\infty^2 + C_2\infty = \infty(C_1\infty + C_2) = C_1\infty^2 \\
    C_1\varepsilon^2 + C_2\varepsilon = \varepsilon(C_1\varepsilon + C_2) = C_2\varepsilon
\end{align*}

这和 200 多年来的传统解释完全不同,甚至相反。传统解释中并不认为无穷大和无穷小是数,因此不能直接进行数学运算,强调它是一个无限趋近的“动态"过程,因此理解起来要困难复杂得多。

格洛克的直观解释如果可以成立,会使微积分学变得简单而清晰,这就需要建立一个直观而简单的可视化代数模型来阐释其合理性。

\section{格洛克代数空间}

\medskip

\textbf{格洛克代数环轴}

和传统的实数数轴不同,格洛克数轴是一个圆环,圆环长度固定为 $\infty$ ,并将其分割为相等的 $\infty$ 段,每段长度为 1, 将数轴逆时针依次标注 0, 1, 2, ...,顺时针标注为 -1, -2, ...,这样我们就得到了一个覆盖 $\left(-\infty, \infty\right)$ 数字空间的环形数轴。

\begin{figure}[htbp] 
    \centering
    \includegraphics[width=0.4\textwidth]{images/01/1.1.png} 
    \caption{\textbf{格洛克代数环轴}}
\end{figure}

下图是格洛克环轴展开后的样子,在数轴的右端标注的是单位刻度的长度量级 $\infty^0 = 1$,因此称为格洛克 0 阶数轴。
\begin{figure}[htbp] 
    \centering
    \includegraphics[width=0.6\textwidth]{images/01/1.2.png} 
    \caption{\textbf{格洛克 0 阶数轴}}
\end{figure}

需要注意的是,$\infty, -\infty$ 和原点 0 重合,这意味着数值大小达到 $\infty$ 时需要进位,以表示 $\infty$ 量级或更大的数。基于同样的原理,我们还需要量级更小的单位,即 $\varepsilon$,用来进位到格洛克 0 阶数轴。
\begin{figure}[htbp] 
    \centering
    \includegraphics[width=0.6\textwidth]{images/01/1.3.png} 
    \caption{\textbf{宏空间和微空间}}
\end{figure}

$\infty$ 以及更大的量级称为\textbf{格洛克宏空间},对应的,$\varepsilon$ 以及更小的量级称为\textbf{格洛克微空间},图1.3 表示了格洛克宏空间 1 轴,格洛克 0 轴和格洛克微空间 1 轴。

宏空间 1 轴覆盖区间 $[0, \infty^2)$,单位刻度大小为 $1\infty$,将 1 单位刻度放大 $\infty$ 倍,得到格洛克 0 轴。

格洛克 0 轴覆盖区间 $[0, \infty)$,单位刻度大小为 1,将 1 单位刻度放大 $\infty$ 倍,得到微空间 1 轴。

微空间 1 轴覆盖区间 $[0, 1)$,单位刻度大小为 $\frac 1 \infty$,即 $1\varepsilon$。

将宏空间和微空间以同样的方式分别向上和向下延展,形成完整的格洛克代数空间。

\medskip

\begin{info}{宏空间和微空间}
    这是一个相对的概念,如 $\infty^{n+1}$ 是 $\infty^n$ 的宏空间,$\infty^n$ 是 $\infty^{n+1}$ 的微空间;$\varepsilon^n$ 是 $\varepsilon^{n+1}$ 的宏空间,$\varepsilon^{n+1}$ 是 $\varepsilon^n$ 的微空间。

    宏空间是微空间的无穷大,微空间是宏空间的无穷小。
\end{info}

\medskip

\textbf{格洛克代数空间}

完整格洛克代数空间如图1.4 所示。
\begin{figure}[htbp] 
    \centering
    \includegraphics[width=0.8\textwidth]{images/01/1.4.png} 
    \caption{\textbf{格洛克代数空间}}
\end{figure}

由于指数进位回 0,$\infty^\infty$ 和 $\varepsilon^\infty$ 回归原点 0,图中未予显示。

格洛克代数空间最大覆盖范围 $[0,\infty^\infty)$,最小单位刻度 $1\varepsilon^{\infty-1}$,可以满足任何数值的可视化标注。

\begin{example}
    在格洛克代数空间上标注:
    
    $(1)\: a = 0.5\infty + 1,\quad (2)\: b = 2\infty - 2,\quad (3)\: c = a + b$。
    
    \medskip

    \textbf{解:}
    \begin{align*}
    c &= a + b \\
      &= (0.5\infty + 1) + (2\infty - 2) \\
      &= (0.5\infty + 2\infty) + (1 - 2) \\
      &= 2.5\infty - 1
    \end{align*}

    a,b,c 如图1.5 所示。
\end{example}
\begin{figure}[htbp] 
    \centering
    \includegraphics[width=0.8\textwidth]{images/01/1.5.png} 
    \caption{\textbf{数值标注}}
\end{figure}

\section{无穷大和无穷小的定义}
经过前面两节关于无穷大和无穷小的讨论和格洛克代数空间的形成,下面给出无穷大和无穷小的正式定义。

\begin{definition}{无穷大和无穷小}

    设 $\infty, \varepsilon$ 是未解析数(unsolved number), $\infty$ 是正整数, 表示无穷大, $\varepsilon$ 表示无穷小。

    \medskip

    \textbf{性质列表}

    (1)$\infty, \varepsilon$ 互为倒数,即 $\infty\cdot\varepsilon = 1$ 。

    (2)对于任意实数 C,整数 n,$\infty^{n+1} + C\infty^n = \infty^{n+1},\; \varepsilon^n + C\varepsilon^{n+1} = \varepsilon^n$。

    (3) $\varepsilon$ 是最小的正数,$-\varepsilon$ 是最大的负数,$0^+ = \varepsilon,\; 0^- = -\varepsilon$。

    (4)对于任意实数 C,$C^+ = C + \varepsilon,\; C^- = C - \varepsilon$。

    (5)对于 $n > 1$ 有,$\log_n\infty < \sqrt[n]{\infty} < \infty < \infty^n < n^\infty < \infty!$。
\end{definition}



\begin{example}
    求函数 $f(x) = \frac{\sqrt{x-4}}{x-5}$ 的定义域。
\end{example}

\section{极限的概念}
\begin{definition}[极限的 $\epsilon-\delta$ 定义]
\label{def:limit}
对于函数 $f(x)$,如果存在实数 $L$,使得对任意给定的 $\epsilon > 0$,都存在 $\delta > 0$,当 $0 < |x - a| < \delta$ 时,有 $|f(x) - L| < \epsilon$,则称 $L$ 为函数 $f(x)$ 在点 $a$ 处的极限,记为 $\lim_{x \to a} f(x) = L$。

设函数 $f(x)$ 在点 $x_0$ 的某个去心邻域内有定义。如果存在一个实数 $L$,使得:

    对于任意给定的正数 $\epsilon$ (无论它多么小),总存在一个正数 $\delta$ (它通常依赖于 $\epsilon$),使得当 $x$ 满足 $0 < |x - x_0| < \delta$ 时,都有 $|f(x) - L| < \epsilon$ 成立。

则称 $L$ 是函数 $f(x)$ 当 $x$ 趋近于 $x_0$ 时的极限,记作:
$$\lim_{x \to x_0} f(x) = L$$
\end{definition}

\begin{theorem}[极限的四则运算]
设 $\lim_{x \to a} f(x)$ 和 $\lim_{x \to a} g(x)$ 都存在,则有:
$$ \lim_{x \to a} [f(x) \pm g(x)] = \lim_{x \to a} f(x) \pm \lim_{x \to a} g(x) $$
$$ \lim_{x \to a} [f(x) g(x)] = \lim_{x \to a} f(x) \cdot \lim_{x \to a} g(x) $$
$$ \lim_{x \to a} \frac{f(x)}{g(x)} = \frac{\lim_{x \to a} f(x)}{\lim_{x \to a} g(x)}, \quad \text{其中 } \lim_{x \to a} g(x) \neq 0 $$
\end{theorem}