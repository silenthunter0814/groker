\chapter{格洛克代数空间}
自微积分诞生以来,如何阐释、理解无穷大和无穷小以及由此引出的函数的极限一直是问题的核心,所以让我们以此为切入点,开始问题的探索。

\medskip

{\heiti 格洛克:无穷大和无穷小的直观理解}

无穷大和无穷小是具有特殊作用的两个数,分别用 $\infty, \varepsilon$ 表示,对于 $\infty$,所有实数都是它的无穷小;对于 $\varepsilon$,所有实数都是它的无穷大。

由此产生如下运算法则:对于任意实数 C
\begin{align}\infty + C = \infty \\\varepsilon + C = C \end{align}

这是基于 $\infty, \varepsilon$ 本身的直观语义得出的。更进一步,基于数的性质,我们得到:
\begin{align}
    C_1\infty + C_2 = C_1\left(\infty + \frac{C_2}{C_1}\right) = C_1\infty \\
    C_1\varepsilon + C_2 = C_1\left(\varepsilon + \frac{C_2}{C_1}\right) = C_2 \\
    C_1\infty^2 + C_2\infty = \infty(C_1\infty + C_2) = C_1\infty^2 \\
    C_1\varepsilon^2 + C_2\varepsilon = \varepsilon(C_1\varepsilon + C_2) = C_2\varepsilon
\end{align}

这和 200 多年来的传统解释完全不同,甚至相反。传统解释中并不认为无穷大和无穷小是数,因此不能直接进行数学运算,强调它是一个无限趋近的“动态"过程,因此理解起来要困难复杂得多。

格洛克的直观解释如果可以成立,会使微积分学变得简单而清晰,这就需要建立一个直观而简单的可视化代数模型来阐释其合理性。

\section{格洛克代数环轴}
和传统的实数数轴不同,格洛克数轴是一个圆环,圆环长度固定为 $\infty$ ,并将其分割为相等的 $\infty$ 段,每段长度为 1, 将数轴逆时针依次标注 0, 1, 2, ...,顺时针标注为 -1, -2, ...,这样我们就得到了一个覆盖 $\left(-\infty, \infty\right)$ 数字空间的环形数轴。

\begin{figure}[htp] 
    \centering
    \includegraphics[width=0.5\textwidth]{images/01/1.1.png} 
    \caption{\heiti 格洛克代数环轴}
\end{figure}

\begin{example}
求函数 $f(x) = \frac{\sqrt{x-4}}{x-5}$ 的定义域。
\end{example}

\section{极限的概念}
\begin{definition}[极限的 $\epsilon-\delta$ 定义]
\label{def:limit}
对于函数 $f(x)$,如果存在实数 $L$,使得对任意给定的 $\epsilon > 0$,都存在 $\delta > 0$,当 $0 < |x - a| < \delta$ 时,有 $|f(x) - L| < \epsilon$,则称 $L$ 为函数 $f(x)$ 在点 $a$ 处的极限,记为 $\lim_{x \to a} f(x) = L$。
\end{definition}

\begin{theorem}[极限的四则运算]
设 $\lim_{x \to a} f(x)$ 和 $\lim_{x \to a} g(x)$ 都存在,则有:
$$ \lim_{x \to a} [f(x) \pm g(x)] = \lim_{x \to a} f(x) \pm \lim_{x \to a} g(x) $$
$$ \lim_{x \to a} [f(x) g(x)] = \lim_{x \to a} f(x) \cdot \lim_{x \to a} g(x) $$
$$ \lim_{x \to a} \frac{f(x)}{g(x)} = \frac{\lim_{x \to a} f(x)}{\lim_{x \to a} g(x)}, \quad \text{其中 } \lim_{x \to a} g(x) \neq 0 $$
\end{theorem}