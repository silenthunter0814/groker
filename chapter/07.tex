\chapter{定积分}
定积分的起源可以追溯到古希腊的“穷竭法”,它主要用来解决求不规则图形面积的问题。

现代定积分的系统理论是由牛顿(Isaac Newton)和莱布尼茨(Gottfried Wilhelm Leibniz)在17世纪独立发展的微积分学奠定的基础。19世纪,黎曼(Bernhard Riemann)提出了黎曼积分的定义,使其更严谨。

格洛克提出区间微分的概念,并据此进行区间积分,计算在格洛克微空间进行,极限计算得出微积分的基本定理,在理解上更加简单直接。

\section{区间微分和区间积分}
考虑函数 $F(x)$ 的点微分:$dF(x) = f(x) \,dx$,在几何上,左侧是线性的极限变化量,右侧是一个高度为 $f(x)$,宽度为无穷小的微矩形的面积,如图7.1 所示。
\begin{figure}[htbp] 
    \centering
    \includegraphics[width=0.5\textwidth]{images/07/7.1.png} 
    \caption{\textbf{微分的几何表示}}
\end{figure}

需要注意的是,微矩形的面积和对应点的 $F(x)$ 线性极限变化量精确相等。

接下来我们考虑图中函数 $f(x)$ 在区间 $[x, x+\delta]$ 投影到 $x$ 轴(曲边梯形)的面积 $S$:
\begin{align*}
    S &= f(x) \,dx + \frac{1}{2}[f(x+\delta)-f(x)]\cdot dx \\
    &= f(x) \,dx + \frac{1}{2} \,df(x)\cdot dx \\
    &= f(x) \,dx + \frac{1}{2}f'(x) \,d^2x
\end{align*}

很明显,在格洛克微空间,曲边和红色三角形斜边重合,如图7.1 所示。$d^2x = \delta^2$,微三角形的面积是一个二阶无穷小,因此可以忽略,所以 $$S = f(x) \,dx = dF(x) = F(x+\delta) - F(x) $$

沿着这个思路,我们着手解决求不规则图形面积的问题。

考虑一个函数 $f(x)$,我们想要求曲线 $y=f(x)$、直线 $x=a$、$x=b$ 以及 $x$ 轴所围成的曲边梯形的面积 $A$, 如下图。
\begin{figure}[htbp] 
    \centering
    \includegraphics[width=0.5\textwidth]{images/07/7.2.png} 
    \caption{\textbf{区间微分}}
\end{figure}


计算这个面积的基本思路是:

\medskip

(1)区间微分

分割:将区间 $[a, b]$ 分割成 $\infty$ 个小区间,区间宽度 $\delta = \frac{b-a}{\infty} = (b-a)\epsilon$。

右微分:对于区间 $[a, b-\delta]$ 内任意分割点 $x$,微分形式:$dx = (x + \delta) - x = \delta$。

左微分:对于区间 $[a+\delta, b]$ 内任意分割点 $x$,微分形式:$dx = x - (x - \delta) = \delta$。

\medskip

(2)面积计算

右微分 $x$ 取值:$$x = a, a+\delta, a+2\delta, \dots, a+(\infty-2)\delta, a+(\infty-1)\delta$$
注意:$a+(\infty-1)\delta = b-\delta, a+(\infty-2)\delta = b-2\delta, \dots$

左微分 $x$ 取值:$$x = a+\delta, a+2\delta, \dots, a+(\infty-2)\delta, a+(\infty-1)\delta, b$$

对应关系及面积计算如下表所示:
$$
\begin{array}{llll}
    n \text{值} &\quad x \text{值} &\quad \text{面积} &\quad F(x) \text{线性变化量} \\
    \hline
    0 &\quad a+0\delta &\quad f(a+0\delta) \,dx &\quad F(a+1\delta)-F(a+0\delta) \\
    1 &\quad a+1\delta &\quad f(a+1\delta) \,dx &\quad F(a+2\delta)-F(a+1\delta) \\
    2 &\quad a+2\delta &\quad f(a+2\delta) \,dx &\quad F(a+3\delta)-F(a+2\delta) \\
    \vdots  &\quad\vdots &\quad\vdots &\quad\vdots \\
    \infty-2 &\quad a+(\infty-2)\delta &\quad f(a+(\infty-2)\delta) \,dx &\quad F(b-\delta)-F(b-2\delta) \\
    \infty-1 &\quad a+(\infty-1)\delta &\quad f(a+(\infty-1)\delta) \,dx &\quad F(b)-F(b-\delta) \\
    \hline
\end{array}
$$

表中面积列的每行和对应的 $F(x)$ 线性变化量列的每行都精确相等,两列分别相加,有
\begin{align*}
    &f(a) \,dx + f(a+\delta) \,dx + \dots + f(a+(\infty-2)\delta) \,dx + f(a+(\infty-1)\delta) \,dx \\
    &= F(a+\delta)-F(a) + F(a+2\delta)-F(a+\delta) + \cdots + F(b)-F(b-\delta) \\
    &= F(b) - F(a) \\
    \\
    &\sum_{n=0}^{\infty - 1}f(a+n\delta) \,dx = F(b) - F(a)
\end{align*}

曲边梯形的面积 $A$ 等于面积列的总和,所以
\begin{align}
    \delta &= \frac{b-a}{\infty} = (b-a)\epsilon \\
    dx &= (x + \delta) - x = x - (x - \delta) = \delta \\
    A &= \sum_{n=0}^{\infty - 1}f(a+n\delta) \,dx = F(b) - F(a) \qquad \text{(右微分)} \\
    A &= \sum_{n=1}^{\infty}f(a+n\delta) \,dx = F(b) - F(a) \qquad \text{(左微分)}
\end{align}

\medskip

(3)积分表示

应用积分符号,并限定积分区间,我们得到定积分计算曲边梯形面积的表示形式:
\begin{align}
    \int_a^b f(x) \,dx = \sum_{n=1}^{\infty}f(a+n\delta) \,dx = F(b) - F(a) = F(x) \Bigg|_{x=a}^{x=b}
\end{align}

$a$ 和 $b$ 分别是积分下限和积分上限,定义了积分的区间。

定积分 $\int_{a}^{b} f(x) \, dx$ 在几何上表示函数 $y=f(x)$ 的图像、直线 $x=a$、$x=b$ 和 $x$ 轴所围成的有向面积。

定积分的结果是一个确定的数值。

在 $x$ 轴上方的面积取正值。

在 $x$ 轴下方的面积取负值。

定积分的无限和形式(及变种)主要用于近似计算,数学精确计算采用不定积分计算原函数再应用积分区间,如公式右侧的形式。

\medskip

传统上定义的定积分也可称作区间积分。


\section{定积分的主要性质}

\textbf{1. 微积分基本定理}
$$\int_a^b f(x) \, dx = F(b) - F(a)$$
其中 $F(x)$ 是 $f(x)$ 的任意一个原函数(即 $F'(x) = f(x)$)。

这是连接不定积分(原函数)与定积分的最重要定理。

定积分与积分变量无关: 定积分的结果是一个数值,与积分变量的符号无关。
$$\int_a^b f(x) \, dx = \int_a^b f(t) \, dt = \int_a^b f(u) \, du$$

\medskip

\textbf{2. 线性性质}

齐次性(常数乘法): 被积函数乘以一个常数 $k$ 的定积分,等于这个常数乘以原函数的定积分。
$$\int_a^b k \cdot f(x) \, dx = k \int_a^b f(x) \, dx \quad \text{($k$ 是常数)}$$

加减性: 两个函数的和(或差)的定积分,等于它们各自定积分的和(或差)。
$$\int_a^b [f(x) \pm g(x)] \, dx = \int_a^b f(x) \, dx \pm \int_a^b g(x) \, dx$$

\medskip

\textbf{3. 区间性质}

定积分上下限的对调: 对调定积分的上下限,积分值会改变符号。
$$\int_a^b f(x) \, dx = - \int_b^a f(x) \, dx$$

零长度区间: 如果积分区间的上下限相同,则定积分的值为零。
$$\int_a^a f(x) \, dx = 0$$

区间可加性(对积分区间的拆分与合并): 如果 $c$ 是 $[a, b]$ 区间上的任意一点(不一定非要在 $a$ 和 $b$ 之间),定积分可以在该点处进行拆分或合并。
$$\int_a^c f(x) \, dx + \int_c^b f(x) \, dx = \int_a^b f(x) \, dx$$

对称区间:

偶函数在对称区间上的定积分等于它在 $[0, a]$ 区间上定积分的两倍。
$$\int_{-a}^a f(x) \, dx = 2 \int_0^a f(x) \, dx \quad (\text{当 } f(x) \text{ 为偶函数时})$$
奇函数在对称区间上的定积分恒等于零。$$\int_{-a}^a f(x) \, dx = 0 \quad (\text{当 } f(x) \text{ 为奇函数时})$$

\medskip

\begin{example} 计算定积分。
    \begin{align*}
        \int_{-1}^1 (x^2 + 1) \, dx &= 2 \int_0^1 (x^2 + 1) \, dx \\
        &= 2 \left[ \frac{x^3}{3} + x \right]_0^1 = 2 \left[ \left(\frac{1^3}{3} + 1\right) - 0 \right] \\
        &= 2 \left( \frac{1}{3} + 1 \right) = 2 \cdot \frac{4}{3} = \frac{8}{3} \\
        \\
        \int_{-2}^2 x^3 \, dx &= \left[ \frac{x^4}{4} \right]_{-2}^2 \\
        &= \frac{2^4}{4} - \frac{(-2)^4}{4} = \frac{16}{4} - \frac{16}{4} = 0
    \end{align*}
\end{example}

在遇到复杂的定积分时,如果积分区间是对称的,可以将函数拆分为奇函数部分和偶函数部分,然后分别应用上述性质进行简化:
\begin{align*}
    \int_{-a}^a [f_{\text{even}}(x) + f_{\text{odd}}(x)] \, dx &= \int_{-a}^a f_{\text{even}}(x) \, dx + \int_{-a}^a f_{\text{odd}}(x) \, dx \\
    &= 2 \int_0^a f_{\text{even}}(x) \, dx + 0
\end{align*}

\medskip

\begin{example} 计算定积分 $\int_{- \pi}^{\pi} (x^5 + \cos x) \, dx$。

    $x^5$ 是奇函数,其积分在 $[-\pi, \pi]$ 上为 0。
    
    $\cos x$ 是偶函数。因此
    \begin{align*}
        \int_{- \pi}^{\pi} (x^5 + \cos x) \, dx &= 0 + \int_{- \pi}^{\pi} \cos x \, dx \\
        &= 2 \int_0^{\pi} \cos x \, dx = 2 [\sin x]_0^{\pi} \\
        &= 2 (\sin \pi - \sin 0) = 2 (0 - 0) = 0
    \end{align*}
\end{example}

\medskip

\textbf{4. 比较性质}

这些性质允许我们比较不同函数的定积分大小,通常假设 $a \le b$。

非负性: 如果在积分区间 $[a, b]$ 上,被积函数 $f(x)$ 恒大于等于零 ($f(x) \ge 0$),那么它的定积分也大于等于零。
$$\text{若 } f(x) \ge 0 \text{ 且 } a \le b, \text{ 则 } \int_a^b f(x) \, dx \ge 0$$

保序性: 如果在积分区间 $[a, b]$ 上,一个函数 $f(x)$ 始终小于等于另一个函数 $g(x)$ ($f(x) \le g(x)$),那么 $f(x)$ 的定积分也小于等于 $g(x)$ 的定积分。
$$\text{若 } f(x) \le g(x) \text{ 且 } a \le b, \text{ 则 } \int_a^b f(x) \, dx \le \int_a^b g(x) \, dx$$

有界性: 如果函数 $f(x)$ 在区间 $[a, b]$ 上的最大值为 $M$,最小值为 $m$,那么定积分的值介于 $m(b-a)$ 和 $M(b-a)$ 之间。
$$m(b-a) \le \int_a^b f(x) \, dx \le M(b-a)$$

\medskip

\textbf{5. 变限积分函数}

如果定积分的上限(或下限)是变量,这个定积分就变成了一个以该变量为自变量的函数。我们称这种函数为变限积分函数。
$$F(x) = \int_a^x f(t) \, dt$$
其中 $a$ 是常数下限,$x$ 是变量上限,$t$ 是积分变量(哑变量)。

变限积分函数 $F(x)$ 实际上是 $f(x)$ 的一个原函数。

变限积分函数最重要的性质是它的导数:
\begin{align}
    F'(x) = \frac{d}{dx} \int_a^x f(t) \, dt = f(x)
\end{align}

链式法则应用:如果上限不是 $x$,而是关于 $x$ 的函数 $u(x)$
\begin{align}
    \frac{d}{dx} \int_a^{u(x)} f(t) \, dt = f(u(x)) \cdot u'(x)
\end{align}

根据上面的求导法则,$F'(x) = f(x)$,这证明了任何连续函数都存在原函数。

变限积分函数是许多求解微分方程的方法的基础。

\medskip

\begin{example} 给定变限积分函数 $F(x)$:$$F(x) = \int_1^x (t^3 + 2t) \, dt$$
    求 $F(x)$ 的解析表达式和导函数 $F'(x)$。

    \begin{align*}
        \int_1^x (t^3 + 2t) \, dt &=  \left[\frac{t^4}{4} + t^2 + C\right]_1^x \\
        &= \left( \frac{x^4}{4} + x^2 \right) - \left( \frac{1^4}{4} + 1^2 \right) \\
        &= \frac{x^4}{4} + x^2 - \frac{5}{4}
    \end{align*}

    $F(x)$ 的解析表达式:$$F(x) = \frac{x^4}{4} + x^2 - \frac{5}{4}$$

    导函数 $F'(x)$:
    \begin{align*}
        F'(x) &= \frac{d}{dx} \left( \frac{1}{4}x^4 + x^2 - \frac{5}{4} \right) \\
        &= \frac{1}{4} \cdot 4x^3 + 2x - 0 \\
        &= x^3 + 2x
    \end{align*}
\end{example}

如果例子中上限是 $u(x) = \sin x$,则
\begin{align*}
    H(x) &= \int_1^{\sin x} (t^3 + 2t) \, dt = \frac{1}{4}\sin^4 x + \sin^2 x - \frac{5}{4} \\
    H'(x) &= \frac{d}{dx}\int_1^{\sin x} (t^3 + 2t) \, dt \\
    &= \left(\sin^3 x + 2\sin x \right) \cdot \frac{d\sin x}{dx} \\
    &= \left(\sin^3 x + 2\sin x \right)\cos x
\end{align*}

\medskip

\textbf{6. 广义积分}

定积分的上限(或下限)是开区间或无穷大时,这被称为广义积分。

广义积分(Improper Integral),也常被称为瑕积分或反常积分。

在格洛克代数空间,开区间可以转换为闭区间,因此广义积分和普通积分没有什么不同,除了最后需要进行极限计算。
\begin{align*}
    \int_{a}^{\infty} f(x) \, dx &= F(\infty) - F(a) \\
    \int_{-\infty}^{b} f(x) \, dx &= F(b) - F(-\infty) \\
    \int_{-\infty}^{\infty} f(x) \, dx &= \int_{-\infty}^{C} f(x) \, dx + \int_{C}^{\infty} f(x) \, dx = F(\infty) - F(-\infty) \\
    \\
    \int_{a}^{b-\epsilon} f(x) \,dx &= F(b-\epsilon) - F(a) \\
    \int_{a+\epsilon}^{b} f(x) \,dx &= F(b) - F(a+\epsilon)
\end{align*}

\medskip

\begin{example} 计算广义积分 $\int_1^{\infty} \frac{1}{x^2} \, dx$。
    \begin{align*}
        \int_1^{\infty} \frac{1}{x^2} \, dx &= \int_1^\infty x^{-2} \, dx \\
        &= \left[ \frac{x^{-1}}{-1} \right]_1^\infty = \left[ -\frac{1}{x} \right]_1^\infty \\
        &= \left( -\frac{1}{\infty} - \left(-\frac{1}{1}\right) \right) \\
        &= -\epsilon + 1 = 1
    \end{align*}
    极限存在且为有限值 $1$,因此该广义积分收敛,其值为 $1$。
\end{example}

\begin{example} 计算广义积分 $\int_0^1 \frac{1}{\sqrt{x}} \, dx$。
    
    被积函数 $f(x) = \frac{1}{\sqrt{x}}$ 在积分区间的下限 $x=0$ 处没有定义,且 $\lim_{x \to 0^+} \frac{1}{\sqrt{x}} = +\infty$。因此 $x=0$ 是一个瑕点。

    \begin{align*}
        \int_0^1 \frac{1}{\sqrt{x}} \, dx &= \int_{\epsilon}^1 x^{-1/2} \, dx \\
        &= \left[ \frac{x^{1/2}}{1/2} \right]_{\epsilon}^1 = 2\sqrt{x} \Biggr |_{\epsilon}^1 \\
        &= 2\sqrt{1} - 2\sqrt{\epsilon} = 2
    \end{align*}

    极限存在且为有限值 $2$,因此该广义积分收敛,其值为 $2$。
\end{example}


\section{曲线的弧长}
曲线的弧长,简单来说,就是曲线上两点之间那段路径的精确长度。在微积分中,我们使用定积分来计算这个长度

弧长公式的推导基于毕达哥拉斯定理和极限的概念,将曲线分成无数段极小的直线段($ds$),然后将这些极小段的长度求和(即积分)。

