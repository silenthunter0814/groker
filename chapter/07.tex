\chapter{定积分}
定积分的起源可以追溯到古希腊的“穷竭法”,它主要用来解决求不规则图形面积的问题。

现代定积分的系统理论是由牛顿(Isaac Newton)和莱布尼茨(Gottfried Wilhelm Leibniz)在17世纪独立发展的微积分学奠定的基础。19世纪,黎曼(Bernhard Riemann)提出了黎曼积分的定义,使其更严谨。

格洛克提出区间微分的概念,并据此进行区间积分,计算在格洛克微空间进行,极限计算得出微积分的基本定理,在理解上更加简单直接。

\section{区间微分和区间积分}
考虑函数 $F(x)$ 的点微分:$dF(x) = f(x) \,dx$,在几何上,左侧是线性的极限变化量,右侧是一个高度为 $f(x)$,宽度为无穷小的微矩形的面积,如图7.1 所示。
\begin{figure}[htbp] 
    \centering
    \includegraphics[width=0.5\textwidth]{images/07/7.1.png} 
    \caption{\textbf{微分的几何表示}}
\end{figure}

需要注意的是,微矩形的面积和对应点的 $F(x)$ 线性极限变化量精确相等。

接下来我们考虑图中函数 $f(x)$ 在区间 $[x, x+\delta]$ 投影到 $x$ 轴(曲边梯形)的面积 $S$:
\begin{align*}
    S &= f(x) \,dx + \frac{1}{2}[f(x+\delta)-f(x)]\cdot dx \\
    &= f(x) \,dx + \frac{1}{2} \,df(x)\cdot dx \\
    &= f(x) \,dx + \frac{1}{2}f'(x) \,d^2x
\end{align*}

很明显,在格洛克微空间,曲边和红色三角形斜边重合,如图7.1 所示。$d^2x = \delta^2$,微三角形的面积是一个二阶无穷小,因此可以忽略,所以 $$S = f(x) \,dx = dF(x) = F(x+\delta) - F(x) $$

沿着这个思路,我们着手解决求不规则图形面积的问题。

考虑一个函数 $f(x)$,我们想要求曲线 $y=f(x)$、直线 $x=a$、$x=b$ 以及 $x$ 轴所围成的曲边梯形的面积 $A$, 如下图。
\begin{figure}[htbp] 
    \centering
    \includegraphics[width=0.5\textwidth]{images/07/7.2.png} 
    \caption{\textbf{区间微分}}
\end{figure}


计算这个面积的基本思路是:

\medskip

(1)区间微分

分割:将区间 $[a, b]$ 分割成 $\infty$ 个小区间,区间宽度 $\delta = \frac{b-a}{\infty} = (b-a)\epsilon$。

右微分:对于区间 $[a, b-\delta]$ 内任意分割点 $x$,微分形式:$dx = (x + \delta) - x = \delta$。

左微分:对于区间 $[a+\delta, b]$ 内任意分割点 $x$,微分形式:$dx = x - (x - \delta) = \delta$。

\medskip

(2)面积计算

右微分 $x$ 取值:$$x = a, a+\delta, a+2\delta, \dots, a+(\infty-2)\delta, a+(\infty-1)\delta$$
注意:$a+(\infty-1)\delta = b-\delta, a+(\infty-2)\delta = b-2\delta, \dots$

左微分 $x$ 取值:$$x = a+\delta, a+2\delta, \dots, a+(\infty-2)\delta, a+(\infty-1)\delta, b$$

对应关系及面积计算如下表所示:
$$
\begin{array}{llll}
    n \text{值} &\quad x \text{值} &\quad \text{面积} &\quad F(x) \text{线性变化量} \\
    \hline
    0 &\quad a+0\delta &\quad f(a+0\delta) \,dx &\quad F(a+1\delta)-F(a+0\delta) \\
    1 &\quad a+1\delta &\quad f(a+1\delta) \,dx &\quad F(a+2\delta)-F(a+1\delta) \\
    2 &\quad a+2\delta &\quad f(a+2\delta) \,dx &\quad F(a+3\delta)-F(a+2\delta) \\
    \vdots  &\quad\vdots &\quad\vdots &\quad\vdots \\
    \infty-2 &\quad a+(\infty-2)\delta &\quad f(a+(\infty-2)\delta) \,dx &\quad F(b-\delta)-F(b-2\delta) \\
    \infty-1 &\quad a+(\infty-1)\delta &\quad f(a+(\infty-1)\delta) \,dx &\quad F(b)-F(b-\delta) \\
    \hline
\end{array}
$$

表中面积列的每行和对应的 $F(x)$ 线性变化量列的每行都精确相等,两列分别相加,有曲边梯形的面积 $A$ 等于面积列的总和,所以
\begin{align*}
    &f(a+\delta) \,dx + f(a+1\delta) \,dx + \dots + f(a+(\infty-2)\delta) \,dx + f(a+(\infty-1)\delta) \,dx \\
    &= F(a+\delta)-F(a) + F(a+2\delta)-F(a+\delta) + \cdots + F(b)-F(b-\delta) \\
    &= F(b) - F(a) \\
    \\
    &\sum_{n=0}^{\infty - 1}f(a+n\delta) \,dx = F(b) - F(a)
\end{align*}

曲边梯形的面积 $A$ 等于面积列的总和,所以
\begin{align}
    A = \sum_{n=0}^{\infty - 1}f(a+n\delta) \,dx = F(b) - F(a) \qquad \text{(右微分)} \\
    A = \sum_{n=1}^{\infty}f(a+n\delta) \,dx = F(b) - F(a) \qquad \text{(左微分)}
\end{align}

\medskip

(3)积分表示

应用积分符号,并限定积分区间,我们得到定积分计算曲边梯形面积的表示形式:
\begin{align}
    \int_a^b f(x) \,dx = \sum_{n=1}^{\infty}f(a+n\delta) \,dx = F(b) - F(a) = F(x) \Bigg|_{x=a}^{x=b}
\end{align}

$a$ 和 $b$ 分别是积分下限和积分上限,定义了积分的区间。

定积分 $\int_{a}^{b} f(x) \, dx$ 在几何上表示函数 $y=f(x)$ 的图像、直线 $x=a$、$x=b$ 和 $x$ 轴所围成的有向面积。

定积分的结果是一个确定的数值。

在 $x$ 轴上方的面积取正值。

在 $x$ 轴下方的面积取负值。

定积分的无限和形式(及变种)主要用于近似计算,数学精确计算采用不定积分计算原函数再应用积分区间,如公式右侧的形式。

\medskip

传统上定义的定积分也可称作区间积分。


\section{定积分的主要性质}
