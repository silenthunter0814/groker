\chapter{点微分和导数}
传统上,微分是描述函数在某一点附近变化的线性近似。这也是微积分学混乱而难学的根源之一。

格洛克利用全新定义的无穷小概念,对微分进行了重新定义:微分是函数在某一点的极限变化量。并和导数进行了统一,形成微分表达式。

导数是描述函数在某一点变化率的精确数值。传统的极限理解方式和计算方法让初学者对导数的学习曲线非常陡峭,莱布尼兹的导数定义符号 $\frac{d}{dx}F(x)$ 大大增加了导数的神秘性。

格洛克弃用莱布尼兹导数表示符号,回归常识,重新描述导数为微分表达式的变换,即导数是函数微分和自变量微分的商。与传统导数概念描述不同的是,导数本质上是一个二级定义,它不是原子定义,而是建立在微分定义之上。

\section{极限变化量与点微分}
通过前面二章的学习,我们具备了解决瞬时速度和切线问题的理论框架,以此为切入点,引出点微分的概念。

\medskip

\textbf{瞬时速度问题}

我们首先考察平均速度的问题,如果位移函数 $S = F(x)$,其中自变量 $x$ 表示时间,那么在 $t_1, t_2$时间间隔内的平均速度表示为
$$\overline{v} = \frac{F(t_2) - F(t_1)}{t_2 - t_1} $$
如果我们用 $\Delta t$ 表示时间间隔,即 $\Delta t = t_2 - t_1$,那么平均速度又可以表示为
$$\overline{v} = \frac{F(t + \Delta t) - F(t)}{\Delta t} $$
可以看出,计算平均速度的公式很简单,计算也很容易。

我们现在的问题是如何计算 $x = t$ 时刻的瞬时速度?问题的难点在于速度的计算要求时间间隔必须存在,唯一的办法是不断缩小时间间隔 $\Delta t$,以此来获取一个近似的 $t$ 时刻的瞬时速度。$\Delta t$ 越小,得到的平均速度就越接近瞬时速度。

基于格洛克代数空间,我们在数值 $t$ 处进行格洛克微空间展开,获取最小的时间间隔 $\epsilon$,得到 $t$ 时刻的瞬时速度
$$v = \frac{F(t+\epsilon) - F(t)}{\epsilon}$$
对于任意时刻 $x$,我们得到数学描述的速度函数
\begin{align}
    v(x) = \frac{F(x+\epsilon) - F(x)}{\epsilon}
\end{align}
这是一个极限运算,可以用我们在第二章的极限计算方法计算出 $v(x)$ 函数。

\begin{example}
    假设一个物体沿直线运动,其位移 $S$(单位:米,m)随时间 $t$(单位:秒,s)变化的函数关系为:$$S(t) = 3t^2 + 5t - 2$$计算该物体在 $t = 2$ 秒时的瞬时速度。

    \textbf{解:}

    首先求解瞬时速度函数 $v(t)$:
    \begin{align*}
        v(t) &= \frac{S(t+\epsilon) - S(t)}{(t+\epsilon) - t} \\
        &=\frac{[3(t+\epsilon)^2 + 5(t+\epsilon) - 2] - (3t^2 + 5t - 2)}{\epsilon}\\
        &= \frac{(3t^2+6t\cdot\epsilon+3\epsilon^2+5t+5\epsilon-2) - (3t^2 + 5t - 2)}{\epsilon} \\
        &= \frac{6t\cdot\epsilon + 5\epsilon + 3\epsilon^2}{\epsilon} \\
        &= 6t + 5 + 3\epsilon \\
        &= 6t + 5 
    \end{align*}
    有了速度函数,现在我们可以计算物体在 $t = 2$ 秒时的瞬时速度 $$v(2) = 6\cdot (2) + 5 = 17$$
\end{example}

\medskip

\textbf{切线斜率问题}

几何上,直线的斜率可通过已知的两点坐标计算 $$m = \frac{y_2 - y_1}{x_2 - x_1}$$
切线是一条“刚好触碰”曲线上某一点的直线。对于曲线 $y = F(x)$ 上的某一点 $P$,如果我们要计算经过 $P$ 点切线的斜率,会遇到和上面计算瞬时速度相同的问题,我们还需要曲线上额外的一点 $Q$,即计算割线 $PQ$ 的斜率来近似 $P$ 点切线的斜率。$Q$ 越接近 $P$,计算的割线斜率就越接近 $P$ 点切线的斜率。
$$m_{PQ} = \frac{y_Q - y_P}{x_Q - x_P}$$

令 $\Delta x = x_Q - x_P$,并使 $\Delta x$ 足够小,曲线 $y = F(x)$ 上经过 $P$ 点切线的斜率可近似表示为
$$m_P \approx \frac{F(x_P + \Delta x) - F(x_P)}{\Delta x}$$
我们在数值 $x_P$ 处进行格洛克微空间展开,获取最小的 $PQ$ 间隔 $\epsilon$,得到 $P$ 点切线的斜率
$$m_P = \frac{F(x_P + \epsilon) - F(x_P)}{\epsilon}$$
对于曲线上的任意点 $(x, F(x))$,切线的斜率函数表示为
\begin{align}
    m(x) = \frac{F(x + \epsilon) - F(x)}{\epsilon}
\end{align}

\begin{example}
    计算抛物线 $y = x^2$ 在点 $(2, 4)$ 处的切线斜率,并求出通过该点的切线方程。

    \textbf{解:}

    首先求解抛物线的切线斜率函数 $m(x)$:
    \begin{align*}
        m(x) &= \frac{F(x + \epsilon) - F(x)}{\epsilon} \\
        &=\frac{(x+\epsilon)^2 - x^2}{\epsilon}\\
        &= \frac{(x^2 + 2x\cdot \epsilon + \epsilon^2) - x^2}{\epsilon} \\
        &= \frac{2x\cdot \epsilon + \epsilon^2}{\epsilon} \\
        &= 2x + \epsilon \\
        &= 2x 
    \end{align*}
    有了斜率函数,现在我们可以计算抛物线 $y = x^2$ 在点 $(2, 4)$ 处的切线斜率 $$m(2) = 2\cdot (2) = 4$$
    切线方程 (点斜式): 
    \begin{align*}
        y - y_0 &= m (x - x_0) \\
        y - 4 &= 4 (x - 2) \\
        y &= 4x - 8 + 4 \\
        y &= 4x - 4
    \end{align*}
\end{example}

从上面两个例子可以看出,速度问题和切线斜率问题虽然完全不同,但在解决问题的数学方法上却完全相同:我们需要找到自变量的极限变化量和对应函数的极限变化量,通过除法得到要解决问题的极限变化率。由此引出和传统定义完全不同的微分概念和定义:

\begin{definition}{微分}

    对于函数 $F(x)$,自变量 $x$ 通过格洛克微空间展开的方式获得 $(x, F(x)$ 极限变化量的过程称为点微分,简称微分。微分用符号 $d$ 表示,并且

    (1)$dx = (x+\epsilon) - x$,称为微分自变量。
    
    (2)$dF(x) = F(x+\epsilon) - F(x)$,称为微分表达式或微分函数。
\end{definition}

对于(1)包含两层含义,首先定义了微分点 $x$,然后是自变量极限变化量的大小 $\epsilon$,这是最简单的表示,也可以是无穷小的表达式。

对于(2)首先是一个微分表达式而不是一个直接计算结果的值,因为进行极限计算 $$F(x+\epsilon) - F(x) = F(x) - F(x) = 0$$ 其次 $dF(x)$ 是与 $x$相对应的极限变化量,也就是随着 $x$ 的变化,$dF(x)$ 也随之变化。最后,定义中给出的是 $x$ 右侧的极限变化量,取左侧或同时取两侧也是正确的,定义中未明确是为了保持简单。因此 $dF(x)$ 可以表示为 $$dF(x) = F(x+\epsilon) - F(x) = F(x) - F(x - \epsilon) = F(x+\epsilon/2) - F(x-\epsilon/2)$$
对于定义域的闭区间边界只能取右侧或左侧极限变化量。

因此,(3.1)和(3.2)可以统一表示为 $$f(x) = \frac{dF(x)}{dx}$$ 微分表达式与微分自变量的比值称为微分变化率,极限变化率或瞬时变化率,传统上统一定义为导函数,简称导数。

\section{导数和导数公式}

有了微分的定义,我们在此基础上定义导函数。导函数简称导数。

\begin{definition}{导函数}
    微分函数与微分自变量的比值函数称为导函数,简称导数。如果原函数用 $f(x)$ 表示,那么导函数用 $f'(x)$ 表示。$$f'(x) = \frac{df(x)}{dx}$$    
    导数用来描述函数相对其自变量的微分变化率,极限变化率或瞬时变化率,或者几何上的曲线切线的斜率。
    如果 $F(x)$ 表示原函数,那么通常用 $f(x)$ 表示导函数,有时也用 $F'(x)$ 表示导函数。
\end{definition}

\begin{example}
    已知函数 $f(x) = \frac 1 x$,求函数在点 $(1, f(1))$ 处的切线斜率,并求出通过该点的切线方程。

    \textbf{解:}

    首先求导函数 $f'(x)$:
    \begin{align*}
        f'(x) &= \frac{df(x)}{dx} \\
        &= \frac{f(x + \epsilon) - f(x)}{(x+\epsilon) - \epsilon} \\
        &=\frac{1/(x+\epsilon) - 1/x}{\epsilon} \\
        &= \frac{x - (x+\epsilon)}{x(x+\epsilon)}\cdot \frac 1 {\epsilon}\\
        &= \frac{-\epsilon}{x(x+\epsilon)}\cdot \frac 1 {\epsilon}\\
        &= -\frac{1}{x^2 + x\cdot\epsilon} \\
        &= -\frac{1}{x^2}
    \end{align*}
    有了导函数,现在我们可以计算函数在点 $(1, f(1)) = (1, 1)$ 处的切线斜率 $$f'(1) = -\frac{1}{(1)^2} = -1$$
    切线方程 (点斜式): 
    \begin{align*}
        y - y_0 &= m (x - x_0) \\
        y - 1 &= (-1)\cdot (x - 1) \\
        y &= -x + 1 + 1 \\
        y &= -x + 2
    \end{align*}
\end{example}

\textbf{常用导函数公式}

求解导函数,需要同时进行代数运算和极限运算,从示例可以看出,即使是简单的原函数,求解导函数也相当繁琐。因此有必要推导一些常用函数的导函数形成导函数公式,避免重复进行极限的繁琐计算。

(1)常数函数 $f(x) = C$
$$\frac{dC}{dx} = \frac{f(x + \epsilon) - f(x)}{\epsilon} = \frac{C - C}{\epsilon} = 0$$
所以,常数函数的导函数为 0,即
\begin{align}
    f'(x) = \frac{dC}{dx} = 0
\end{align}

(2)幂函数 $f(x) = x^n$
当 $n$ 为整数时,考虑二项式定理 $$(x+\epsilon)^n = \sum_{k=0}^{n} \binom{n}{k} x^{n-k} \epsilon^k = x^n + \binom{n}{1} x^{n-1}\epsilon + \binom{n}{2} x^{n-2}\epsilon^2 + \cdots + \epsilon^n$$
\begin{align*}
    \frac{dx^n}{dx} &= \frac{(x+\epsilon)^n - x^n}{\epsilon} \\
    &= \frac{\left[x^n + nx^{n-1}\epsilon + \binom{n}{2} x^{n-2}\epsilon^2 + \cdots + \epsilon^n\right]-x^n}{\epsilon} \\
    &= nx^{n-1} + \binom{n}{2} x^{n-2}\epsilon + \cdots + \epsilon^{n-1} \\
    &= nx^{n-1}
\end{align*}
当 $n$ 为一般实数时上面的公式也成立,通常需要依赖于指数和对数函数的导数公式,这里略去推导过程。

所以,幂函数的导数公式
\begin{align}
    f'(x) = \frac{dx^n}{dx}} = nx^{n-1}
\end{align}

(3)指数函数 $f(x) = e^x,\; f(x) = a^x$

使用上一章的自然恒等式 $e^\epsilon = 1 + \epsilon,\; a^\epsilon = 1 + \ln a \cdot \epsilon$
\begin{align*}
    \frac{de^x}{dx} &= \frac{e^{x+\epsilon} - e^x}{\epsilon} \\
    &= \frac{e^x \cdot (e^\epsilon -1)}{\epsilon} \\
    &= e^x \cdot \frac{e^\epsilon -1}{\epsilon} \\
    &= e^x \cdot \frac{(1 + \epsilon) -1}{\epsilon} \\
    &= e^x \cdot 1 = e^x
\end{align*}

\begin{align*}
    \frac{da^x}{dx} &= \frac{a^{x+\epsilon} - a^x}{\epsilon} \\
    &= \frac{a^x \cdot (a^\epsilon -1)}{\epsilon} \\
    &= a^x \cdot \frac{a^\epsilon -1}{\epsilon} \\
    &= a^x \cdot \frac{(1 + \ln a \cdot \epsilon) -1}{\epsilon} \\
    &= a^x \ln a
\end{align*}
所以,指数函数的导数公式
\begin{align}
    f'(x) = \frac{e^x}{dx} &= e^x \\
    f'(x) = \frac{a^x}{dx} &= a^x \ln a
\end{align}

(4)对数函数 $f(x) = \ln x,\; f(x) = log_a x$

使用上一章的自然恒等式 $ln(1 + a \epsilon) = a \epsilon$
\begin{align*}
    \frac{d\ln x}{dx} &= \frac{\ln(x+\epsilon) - \ln x}{\epsilon} \\
    &= \ln \frac{x+\epsilon}{x} \cdot 1 \epsilon \\
    &= \ln (1 + \frac{1}{x} \cdot \epsilon) \cdot 1 \epsilon \\
    &= \frac{1}{x} \cdot \epsilon) \cdot 1 \epsilon \\
    &= \frac 1 x
\end{align*}

\begin{align*}
    \frac{d\log_a x}{dx} &= \frac{\log_a(x+\epsilon) - \log_a x}{\epsilon} \\
    &= \log_a \frac{x + \epsilon}{x} \cdot \frac{1}{\epsilon}\\
    &= \log_a(1 + \frac{\epsilon}{x}) \cdot \frac{1}{\epsilon}\\
    &=\frac{\ln(1 + \frac{\epsilon}{x})}{\ln a} \cdot \frac{1}{\epsilon}\\
    &= \frac{1}{\ln a}\cdot\frac{\epsilon}{x} \cdot \frac{1}{\epsilon}\\
    &= \frac{1}{x\ln a}
\end{align*}
所以,对数函数的导数公式
\begin{align}
    f'(x) = \frac{d\ln x}{dx} &= \frac 1 x \\
    f'(x) = \frac{d\log_a x}{dx} &= \frac{1}{x\ln a}
\end{align}

(5)三角函数 $f(x) = \sin x,\; f(x) = \cos x$

对于 $f(x) = \sin x$,考虑使用和角公式 $\sin(x+\epsilon) = \sin x\cdot\cos\epsilon + \cos x\cdot\sin\epsilon$。
\begin{align*}
    \frac{d\sin x}{dx} &= \frac{\sin(x+\epsilon) - \sin x}{\epsilon} \\
    &= \frac{(\sin x\cdot\cos\epsilon + \cos x\cdot\sin\epsilon) - \sin x}{\epsilon} \\
    &= \frac{\sin x(\cos\epsilon - 1) + \cos x\cdot\sin\epsilon}{\epsilon} \\
    &= \sin x\cdot\frac{\cos\epsilon-1}{\epsilon} + \cos x\cdot\frac{\sin\epsilon}{\epsilon} \\
    &= \sin x\cdot 0 + \cos x\cdot 1 \\
    &= \cos x
\end{align*}

对于 $f(x) = \cos x$,考虑使用和角公式 $\cos(x+\epsilon) = \cos x\cos\epsilon - \sin x\sin\epsilon$。
\begin{align*}
    \frac{d\cos x}{dx} &= \frac{\cos(x+\epsilon) - \cos x}{\epsilon} \\
    &= \frac{\cos x\cos\epsilon - \sin x\sin\epsilon - \cos x}{\epsilon} \\
    &= \frac{\cos x(\cos\epsilon -1) - \sin x\sin\epsilon}{\epsilon} \\
    &= \cos\cdot\frac{\cos\epsilon - 1}{\epsilon} - \sin x\cdot\frac{\sin\epsilon}{\epsilon} \\
    &= \cos\cdot 0 - \sin x\cdot 1 \\
    &= -\sin x
\end{align*}
所以,三角函数的导数公式
\begin{align}
    f'(x) = \frac{d\sin x}{dx} &= \cos x \\
    f'(x) = \frac{d\cos x}{dx} &= -\sin x
\end{align}

这些是常用的基本函数导数公式,一些不太常用到的函数,如反函数等我们不再一一推导。

\textbf{基本函数导数公式}
$$\begin{array}{ll}
    \text{函数 } f(x) & \text{导数 } f'(x) \\
    \hline
    C \text{ (常数)} & 0 \\
    x^n & nx^{n-1} \\
    e^x & e^x \\
    a^x & a^x \ln a \\
    \ln x & \frac{1}{x} \\
    \log_a x & \frac{1}{x \ln a} \\
    \sin x & \cos x \\
    \cos x & -\sin x \\
    \tan x & \sec^2 x \\
    \cot x & -\csc^2 x \\
    \sec x & \sec x \tan x \\
    \csc x & -\csc x \cot x \\
    \arcsin x & \frac{1}{\sqrt{1-x^2}} \\
    \arccos x & -\frac{1}{\sqrt{1-x^2}} \\
    \arctan x & \frac{1}{1+x^2} \\
    \text{arccot } x & -\frac{1}{1+x^2} \\
    \end{array}
$$


\section{导数的运算法则}

假设 $u = u(x)$ 和 $v = v(x)$ 都是可导函数,$C$ 是常数。


常数与函数积的导数$$(Cu)' = C u'$$

和或差的导数$$(u \pm v)' = u' \pm v'$$

积的导数(乘法定则)$$(uv)' = u'v + uv'$$

商的导数(除法定则)$$\left(\frac{u}{v}\right)' = \frac{u'v - uv'}{v^2} \quad (v \neq 0)$

$复合函数的导数(链式法则)如果 $y = f(u)$ 且 $u = g(x)$,那么 $y$ 对 $x$ 的导数是:$$\frac{dy}{dx} = \frac{dy}{du} \cdot \frac{du}{dx} \quad \text{或} \quad [f(g(x))]' = f'(g(x)) \cdot g'(x)$$


例子 1: 求 $f(x) = x^5 + \ln x$ 的导数。$$f'(x) = (x^5)' + (\ln x)' = 5x^{5-1} + \frac{1}{x} = 5x^4 + \frac{1}{x}$$

例子 2: 求 $g(x) = x \cos x$ 的导数(使用积的导数公式)。$$g'(x) = (x)' \cos x + x (\cos x)' = 1 \cdot \cos x + x (-\sin x) = \cos x - x \sin x$$

例子 3: 求 $h(x) = e^{2x}$ 的导数(使用链式法则)。令 $u = 2x$,则 $h(x) = e^u$。$$\frac{dh}{dx} = \frac{d(e^u)}{du} \cdot \frac{d(2x)}{dx} = e^u \cdot 2 = 2e^{2x}$$

函数 $f(x)$ 在点 $x_0$ 处的导数定义为:
$$ f'(x_0) = \lim_{\Delta x \to 0} \frac{f(x_0 + \Delta x) - f(x_0)}{\Delta x} $$
导数表示函数在某点变化率的精确度量。


