\chapter{点微分和导数}
传统上,微分是描述函数在某一点附近变化的线性近似。这也是微积分学混乱而难学的根源之一。

格洛克利用全新定义的无穷小概念,对微分进行了重新定义:微分是函数在某一点的极限变化量。并和导数进行了统一,形成微分表达式。

导数是描述函数在某一点变化率的精确数值。传统的极限理解方式和计算方法让初学者对导数的学习曲线非常陡峭,莱布尼兹的导数定义符号 $\frac{d}{dx}F(x)$ 大大增加了导数的神秘性。

格洛克弃用莱布尼兹导数表示符号,回归常识,重新描述导数为微分表达式的变换,即导数是函数微分和自变量微分的商。与传统导数概念描述不同的是,导数本质上是一个二级定义,它不是原子定义,而是建立在微分定义之上。

\section{极限变化量与点微分}
通过前面三章的学习,我们具备了解决瞬时速度和切线问题的理论框架,以此为切入点,引出点微分的概念。

\medskip

\textbf{瞬时速度问题}

我们首先考察平均速度的问题,如果位移函数 $S = F(x)$,其中自变量 $x$ 表示时间,那么在 $t_1, t_2$时间间隔内的平均速度表示为
$$\overline{v} = \frac{F(t_2) - F(t_1)}{t_2 - t_1} $$
如果我们用 $\Delta t$ 表示时间间隔,即 $\Delta t = t_2 - t_1$,那么平均速度又可以表示为
$$\overline{v} = \frac{F(t + \Delta t) - F(t)}{\Delta t} $$
可以看出,计算平均速度的公式很简单,计算也很容易。

我们现在的问题是如何计算 $x = t$ 时刻的瞬时速度?问题的难点在于速度的计算要求时间间隔必须存在,唯一的办法是不断缩小时间间隔 $\Delta t$,以此来获取一个近似的 $t$ 时刻的瞬时速度。$\Delta t$ 越小,得到的平均速度就越接近瞬时速度。

基于格洛克代数空间,我们在数值 $t$ 处进行格洛克微空间展开,获取最小的时间间隔 $\epsilon$,得到 $t$ 时刻的瞬时速度
$$v = \frac{F(t+\epsilon) - F(t)}{\epsilon}$$
对于任意时刻 $x$,我们得到数学描述的速度函数
\begin{align}
    v(x) = \frac{F(x+\epsilon) - F(x)}{\epsilon}
\end{align}

\begin{example}
    假设一个物体沿直线运动,其位移 $S$(单位:米,m)随时间 $t$(单位:秒,s)变化的函数关系为:$$S(t) = 3t^2 + 5t - 2$$计算该物体在 $t = 2$ 秒时的瞬时速度。

    \textbf{解:}

    首先求解瞬时速度函数 $v(t)$:
    \begin{align*}
        v(t) &= \frac{S(t+\epsilon) - S(t)}{(t+\epsilon) - t} \\
        &=\frac{[3(t+\epsilon)^2 + 5(t+\epsilon) - 2] - (3t^2 + 5t - 2)}{\epsilon}\\
        &= \frac{(3t^2+6t\cdot\epsilon+3\epsilon^2+5t+5\epsilon-2) - (3t^2 + 5t - 2)}{\epsilon} \\
        &= \frac{6t\cdot\epsilon + 5\epsilon + 3\epsilon^2}{\epsilon} \\
        &= 6t + 5 + 3\epsilon \\
        &= 6t + 5 
    \end{align*}
    有了速度函数,现在我们可以计算物体在 $t = 2$ 秒时的瞬时速度 $$v(2) = 6\cdot (2) + 5 = 17$$
\end{example}

\medskip

\textbf{切线斜率问题}

几何上,直线的斜率可通过已知的两点坐标计算 $$m = \frac{y_2 - y_1}{x_2 - x_1}$$
切线是一条“刚好触碰”曲线上某一点的直线。对于曲线 $y = F(x)$ 上的某一点 $P$,如果我们要计算经过 $P$ 点切线的斜率,会遇到和上面计算瞬时速度相同的问题,我们还需要曲线上额外的一点 $Q$,即计算割线 $PQ$ 的斜率来近似 $P$ 点切线的斜率。$Q$ 越接近 $P$,计算的割线斜率就越接近 $P$ 点切线的斜率。
$$m_{PQ} = \frac{y_Q - y_P}{x_Q - x_P}$$

令 $\Delta x = x_Q - x_P$,并使 $\Delta x$ 足够小 曲线 $y = F(x)$ 上经过 $P$ 点切线的斜率可近似表示为
$$m_P \approx \frac{F(x_P + \Delta x) - F(x_P)}{\Delta x}$$
我们在数值 $x_P$ 处进行格洛克微空间展开,获取最小的 $PQ$ 间隔 $\epsilon$,得到 $P$ 点切线的斜率
$$m_P \approx \frac{F(x_P + \epsilon) - F(x_P)}{\epsilon}$$
对于曲线上的任意点 $(x, F(x))$,切线的斜率函数表示为
\begin{align}
    m(x) = \frac{F(x + \epsilon) - F(x)}{\epsilon}
\end{align}

\begin{example}
    计算抛物线 $y = x^2$ 在点 $(2, 4)$ 处的切线斜率,并求出通过该点的切线方程。

    \textbf{解:}

    首先求解抛物线的切线斜率函数 $m(x)$:
    \begin{align*}
        m(x) &= \frac{F(x + \epsilon) - F(x)}{\epsilon} \\
        &=\frac{(x+\epsilon)^2 - x^2}{\epsilon}\\
        &= \frac{(x^2 + 2x\cdot \epsilon + \epsilon^2) - x^2}{\epsilon} \\
        &= \frac{2x\cdot \epsilon + \epsilon^2}{\epsilon} \\
        &= 2x + \epsilon \\
        &= 2x 
    \end{align*}
    有了斜率函数,现在我们可以计算抛物线 $y = x^2$ 在点 $(2, 4)$ 处的切线斜率 $$m(2) = 2\cdot (2) = 4$$
    切线方程 (点斜式): 
    \begin{align*}
        y - y_0 &= m (x - x_0) \\
        y - 4 &= 4 (x - 2) \\
        y &= 4x - 8 + 4 \\
        y = 4x - 4
    \end{align*}
\end{example}

函数 $f(x)$ 在点 $x_0$ 处的导数定义为:
$$ f'(x_0) = \lim_{\Delta x \to 0} \frac{f(x_0 + \Delta x) - f(x_0)}{\Delta x} $$
导数表示函数在某点变化率的精确度量。


