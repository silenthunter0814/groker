\chapter{点微分和导数}
传统上,微分是描述函数在某一点附近变化的线性近似。这也是微积分学混乱而难学的根源之一。

格洛克利用全新定义的无穷小概念,对微分进行了重新定义:微分是函数在某一点的极限变化量。并和导数进行了统一,形成微分表达式。

导数是描述函数在某一点变化率的精确数值。传统的极限理解方式和计算方法让初学者对导数的学习曲线非常陡峭,莱布尼兹的导数定义符号 $\frac{d}{dx}F(x)$ 大大增加了导数的神秘性。

格洛克弃用莱布尼兹导数表示符号,回归常识,重新描述导数为微分表达式的变换,即导数是函数微分和自变量微分的商。与传统导数概念描述不同的是,导数本质上是一个二级定义,它不是原子定义,而是建立在微分定义之上。

\section{极限变化量与点微分}
函数 $f(x)$ 在点 $x_0$ 处的导数定义为:
$$ f'(x_0) = \lim_{\Delta x \to 0} \frac{f(x_0 + \Delta x) - f(x_0)}{\Delta x} $$
导数表示函数在某点变化率的精确度量。


