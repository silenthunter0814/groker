\chapter{空间曲线}

\section{向量微分和曲线弧长}



\section{切向量 $\mathbf{T}$、法向量 $\mathbf{N}$ 和副法向量 $\mathbf{B}$}

在微分几何中,Frenet-Serret 标架($\mathbf{T}$, $\mathbf{N}$, $\mathbf{B}$)是描述曲线局部特征的核心。

当曲线按弧长参数化时,切向量的模长恒为 1,这简化了所有的导数关系。

1. 切向量 $\mathbf{T}$ 的推导

定义曲线为 $\mathbf{r}(s)$。切向量定义为位置对弧长的变化率:
$$\mathbf{T} = \frac{d\mathbf{r}}{ds}$$

由于 $s$ 是弧长,根据定义 $\|\mathbf{T}\| = 1$。

2. 主法向量 $\mathbf{N}$ 与曲率 $\kappa$

由于 $\mathbf{T}$ 是单位向量,其模长平方为常数:$\mathbf{T} \cdot \mathbf{T} = 1$。两边对 $s$ 求导:
$$\frac{d\mathbf{T}}{ds} \cdot \mathbf{T} + \mathbf{T} \cdot \frac{d\mathbf{T}}{ds} = 0 \implies 2\mathbf{T} \cdot \frac{d\mathbf{T}}{ds} = 0$$

这证明了导向量 $d\mathbf{T}/ds$ 始终垂直于 $\mathbf{T}$。

曲率 $\kappa$:定义为切线方向随弧长变化的速率,即 $\kappa = \|\frac{d\mathbf{T}}{ds}\|$。

主法向量 $\mathbf{N}$:定义为 $d\mathbf{T}/ds$ 方向上的单位向量:
$$\frac{d\mathbf{T}}{ds} = \kappa \mathbf{N}$$

3. 副法向量 $\mathbf{B}$ 与挠率 $\tau$

为了构成右手正交标架,我们定义副法向量为:
$$\mathbf{B} = \mathbf{T} \times \mathbf{N}$$

挠率 $\tau$ 的得出:

我们要考察 $\mathbf{B}$ 随弧长的变化。对 $\mathbf{B}$ 求导:
$$\frac{d\mathbf{B}}{ds} = \frac{d\mathbf{T}}{ds} \times \mathbf{N} + \mathbf{T} \times \frac{d\mathbf{N}}{ds}$$

由于 $d\mathbf{T}/ds = \kappa \mathbf{N}$,而 $\mathbf{N} \times \mathbf{N} = 0$,第一项消失:
$$\frac{d\mathbf{B}}{ds} = \mathbf{T} \times \frac{d\mathbf{N}}{ds}$$

这意味着 $d\mathbf{B}/ds$ 垂直于 $\mathbf{T}$。同时,由于 $\mathbf{B}$ 是单位向量,$d\mathbf{B}/ds$ 也垂直于 $\mathbf{B}$。既然它同时垂直于 $\mathbf{T}$ 和 $\mathbf{B}$,它必须在 $\mathbf{N}$ 的方向上。

因此,我们定义:
$$\frac{d\mathbf{B}}{ds} = -\tau \mathbf{N}$$

这里的 $\tau$ 称为挠率(Torsion),负号是几何学上的约定,表示当 $\tau > 0$ 时,曲线随 $s$ 增加向副法向量定义的右手螺旋方向扭转。

总结

$\mathbf{T}$ 描述前进方向。

$\mathbf{N}$ 描述向哪弯曲,曲率 $\kappa$ 是弯曲程度(偏离直线的程度)。

$\mathbf{B}$ 描述运动平面的法线,挠率 $\tau$ 是扭曲程度(偏离平面的程度)。对于平面曲线(如 $y=x^2$),$\mathbf{r}'''$ 依然在 $xy$ 平面,外积后与 $z$ 轴垂直,因此 $\tau$ 恒等于 0。

曲率 $\kappa$ 的求解公式

曲率 $\kappa$ 的基本定义是切向量对弧长的变化率的模:
$$\kappa = \left\| \frac{d\mathbf{T}}{ds} \right\|$$

由于我们通常使用的是参数 $t$(如时间),根据链式法则:
$$\frac{d\mathbf{T}}{dt} = \frac{d\mathbf{T}}{ds} \cdot \frac{ds}{dt} = \frac{d\mathbf{T}}{ds} \cdot \|\mathbf{r}'(t)\|$$

因此有:
$$\kappa = \frac{\|\mathbf{T}'(t)\|}{\|\mathbf{r}'(t)\|}$$

我们知道切向量 $\mathbf{T} = \frac{\mathbf{r}'}{\|\mathbf{r}'\|}$,所以速度向量可以写为:

$$\mathbf{r}' = \|\mathbf{r}'\| \mathbf{T}$$

对上式两边关于 $t$ 求导(使用乘法法则):
$$\mathbf{r}'' = (\|\mathbf{r}'\|)' \mathbf{T} + \|\mathbf{r}'\| \mathbf{T}'$$

利用外积(叉乘)消项

现在我们将 $\mathbf{r}'$ 和 $\mathbf{r}''$ 做外积:
$$\mathbf{r}' \times \mathbf{r}'' = (\|\mathbf{r}'\| \mathbf{T}) \times ((\|\mathbf{r}'\|)' \mathbf{T} + \|\mathbf{r}'\| \mathbf{T}')$$

利用外积的分配律:
$$\mathbf{r}' \times \mathbf{r}'' = \|\mathbf{r}'\| (\|\mathbf{r}'\|)' (\mathbf{T} \times \mathbf{T}) + \|\mathbf{r}'\|^2 (\mathbf{T} \times \mathbf{T}')$$

因为任何向量与自身的外积为零 ($\mathbf{T} \times \mathbf{T} = 0$),所以:
$$\mathbf{r}' \times \mathbf{r}'' = \|\mathbf{r}'\|^2 (\mathbf{T} \times \mathbf{T}')$$

取两边的模长:
$$\|\mathbf{r}' \times \mathbf{r}''\| = \|\mathbf{r}'\|^2 \cdot \|\mathbf{T} \times \mathbf{T}'\|$$

因为 $\mathbf{T}$ 是单位向量,且我们已知 $\mathbf{T}'$ 垂直于 $\mathbf{T}$(见前文推导),所以 $\|\mathbf{T} \times \mathbf{T}'\| = \|\mathbf{T}\| \|\mathbf{T}'\| \sin(90^\circ) = \|\mathbf{T}'\|$。代入上式:

$$\|\mathbf{r}' \times \mathbf{r}''\| = \|\mathbf{r}'\|^2 \|\mathbf{T}'\|$$

解出 $\|\mathbf{T}'\|$:
$$\|\mathbf{T}'\| = \frac{\|\mathbf{r}' \times \mathbf{r}''\|}{\|\mathbf{r}'\|^2}$$

最后,将这个结果代回最初的曲率定义:
$$\kappa = \frac{\|\mathbf{T}'\|}{\|\mathbf{r}'\|} = \frac{\|\mathbf{r}' \times \mathbf{r}''\|}{\|\mathbf{r}'\|^2 \cdot \|\mathbf{r}'\|} = \frac{\|\mathbf{r}' \times \mathbf{r}''\|}{\|\mathbf{r}'\|^3}$$

这个推导巧妙地利用了 $\mathbf{r}''$ 在 $\mathbf{T}$ 方向(切向加速度)和 $\mathbf{N}$ 方向(法向加速度)的分解。外积操作自动过滤掉了不改变方向的切向部分,只留下了反映“弯曲”的法向部分,从而直接提取出了曲率。




$\mathbf{T}$, $\mathbf{N}$, $\mathbf{B}$ 求解步骤:

第一步:切向量 $\mathbf{T}$ (Tangent)

对曲线 $\mathbf{r}(t)$ 求导并单位化:
$$\mathbf{T} = \frac{\mathbf{r}'(t)}{\|\mathbf{r}'(t)\|}$$

第二步:副法向量 $\mathbf{B}$ (Binormal)

在 3D 中,通常先求 $\mathbf{B}$ 反而更容易。利用加速度(二阶导)和速度的外积:
$$\mathbf{B} = \frac{\mathbf{r}'(t) \times \mathbf{r}''(t)}{\|\mathbf{r}'(t) \times \mathbf{r}''(t)\|}$$

原理: 速度和加速度构成的平面被称为密切平面,$\mathbf{B}$ 就是该平面的法向量。

第三步:主法向量 $\mathbf{N}$ (Normal)

有了 $\mathbf{T}$ 和 $\mathbf{B}$,$\mathbf{N}$ 可以通过叉乘直接得出,确保右手系:$$\mathbf{N} = \mathbf{B} \times \mathbf{T}$$

\begin{example} 抛物线 $y = x^2$ 的切向量 $\mathbf{T}$、法向量 $\mathbf{N}$ 和副法向量 $\mathbf{B}$。

    参数化方程为 $\mathbf{r}(x) = (x, x^2)$。
    
    为了方便计算空间中的 $B$ 向量,我们可以将其视为在 $z=0$ 平面上的三维曲线,即 $\mathbf{r}(x) = (x, x^2, 0)$。
    
    首先计算位置向量的一阶和二阶导数:
    
    $\mathbf{r}'(x) = (1, 2x, 0)$
    
    $\mathbf{r}''(x) = (0, 2, 0)$
    
    一阶导的模长:$\|\mathbf{r}'(x)\| = \sqrt{1 + (2x)^2} = \sqrt{1 + 4x^2}$

    切向量 $\mathbf{T}$ (Tangent)
    
    直接对一阶导进行单位化:
    $$\mathbf{T} = \frac{\mathbf{r}'(x)}{\|\mathbf{r}'(x)\|} = \left( \frac{1}{\sqrt{1+4x^2}}, \frac{2x}{\sqrt{1+4x^2}}, 0 \right)$$

    副法向量 $\mathbf{B}$ (Binormal)
    
    对于平面曲线,副法向量总是垂直于平面。我们通过叉乘计算:
    $$\mathbf{r}' \times \mathbf{r}'' = \begin{vmatrix} \mathbf{i} & \mathbf{j} & \mathbf{k} \\ 1 & 2x & 0 \\ 0 & 2 & 0 \end{vmatrix} = (0, 0, 2)$$

    单位化后得到:
    $$\mathbf{B} = (0, 0, 1)$$
    
    (这说明抛物线始终在 $xy$ 平面内弯曲)

    主法向量 $\mathbf{N}$ (Normal)
    
    利用 $\mathbf{N} = \mathbf{B} \times \mathbf{T}$(满足右手系且指向凹侧):
    
    $$\mathbf{N} = \begin{vmatrix} \mathbf{i} & \mathbf{j} & \mathbf{k} \\ 0 & 0 & 1 \\ \frac{1}{\sqrt{1+4x^2}} & \frac{2x}{\sqrt{1+4x^2}} & 0 \end{vmatrix} = \left( -\frac{2x}{\sqrt{1+4x^2}}, \frac{1}{\sqrt{1+4x^2}}, 0 \right)$$

    计算曲率 $\kappa$ (Curvature)
    
    曲率公式为:$$\kappa = \frac{\|\mathbf{r}' \times \mathbf{r}''\|}{\|\mathbf{r}'\|^3}$$
    
    代入已知量:
    
    $\|\mathbf{r}' \times \mathbf{r}''\| = 2$
    
    $\|\mathbf{r}'\| = (1+4x^2)^{1/2}$

    得到:
    $$\kappa(x) = \frac{2}{(1 + 4x^2)^{3/2}}$$

    在顶点 $(0,0)$ 处: $\kappa = 2$。此时曲率最大,密切圆半径 $R = 1/\kappa = 0.5$。
    
    当 $x \to \infty$ 时: $\kappa \to 0$。这符合直觉,因为抛物线远端越来越趋于直线。

\end{example}

\section{通量和散度}

通量的物理直觉是:衡量矢量场 $\mathbf{F}$ 在单位时间内穿过某个微小表面的“净流量”。

在直角坐标系下,我们推导矢量场 $\mathbf{F} = (F_1, F_2, F_3)$ 流出一个微元体(体积 $dV = dx dy dz$)的净通量。

1. 建立微元模型

想象一个中心位于 $(x, y, z)$ 的微小长方体,六个面分别垂直于坐标轴。我们以穿过垂直于 $x$ 轴的两个表面的通量为例进行推导。

右侧面(位于 $x + \frac{dx}{2}$):法向量指向 $+x$ 方向。该面中心处的场分量 $F_1$ 约为:
$$F_1(x + \frac{dx}{2}, y, z) \approx F_1 + \frac{\partial F_1}{\partial x} \frac{dx}{2}$$

穿出通量:$\Phi_{right} = (F_1 + \frac{\partial F_1}{\partial x} \frac{dx}{2}) \cdot dy dz$

左侧面(位于 $x - \frac{dx}{2}$):法向量指向 $-x$ 方向(流出体积的方向)。该面中心处的场分量 $F_1$ 约为:
$$F_1(x - \frac{dx}{2}, y, z) \approx F_1 - \frac{\partial F_1}{\partial x} \frac{dx}{2}$$

左侧面(位于 $x - \frac{dx}{2}$):法向量指向 $-x$ 方向(流出体积的方向)。该面中心处的场分量 $F_1$ 约为:
$$F_1(x - \frac{dx}{2}, y, z) \approx F_1 - \frac{\partial F_1}{\partial x} \frac{dx}{2}$$

穿出通量(由于法向量向左,需乘面积向量 $-dy dz \mathbf{i}$):$\Phi_{left} = -(F_1 - \frac{\partial F_1}{\partial x} \frac{dx}{2}) \cdot dy dz$

$x$ 方向的净通量(两面之和):

$$d\Phi_x = \Phi_{right} + \Phi_{left} = \left( \frac{\partial F_1}{\partial x} dx \right) dy dz = \frac{\partial F_1}{\partial x} dV$$

3. 汇总三个方向

同理,我们可以得到穿过另外两对面的净通量:

$y$ 方向净通量: $d\Phi_y = \frac{\partial F_2}{\partial y} dy (dx dz) = \frac{\partial F_2}{\partial y} dV$

$z$ 方向净通量: $d\Phi_z = \frac{\partial F_3}{\partial z} dz (dx dy) = \frac{\partial F_3}{\partial z} dV$

总净通量(流出该微元体的总和):
$$d\Phi = d\Phi_x + d\Phi_y + d\Phi_z = \left( \frac{\partial F_1}{\partial x} + \frac{\partial F_2}{\partial y} + \frac{\partial F_3}{\partial z} \right) dx dy dz$$

4. 结论:从通量到散度

通过上面的推导,我们发现了一个极其重要的物理量:

特征量:括号内的 $\left( \frac{\partial F_1}{\partial x} + \frac{\partial F_2}{\partial y} + \frac{\partial F_3}{\partial z} \right)$ 描述了该点处通量的源性。

散度的诞生:如果我们定义单位体积的净通量为该点的“散度”,那么:

$$\text{div } \mathbf{F} = \frac{d\Phi}{dV} = \frac{\partial F_1}{\partial x} + \frac{\partial F_2}{\partial y} + \frac{\partial F_3}{\partial z}$$

算子表示:这恰好对应 $\nabla$ 算子与 $\mathbf{F}$ 的点乘:$\nabla \cdot \mathbf{F}$。



\section{环量和旋度}

1. 物理直觉:环量密度

如果一个场在某处有旋转趋势,那么你沿着一个极小的闭合回路 $L$ 走一圈,矢量场沿路径的累积效应(环量) $\oint_L \mathbf{F} \cdot d\mathbf{l}$ 就不应该为零。

我们要寻找一个量,用来描述 $\mathbf{F}$ 在某点附近的旋转强度。这个量不应该依赖于我们取的回路大小,因此我们要计算的是单位面积的环量极限。

1. 考察 $xy$ 平面内的旋转 (绕 $z$ 轴)

我们在 $xy$ 平面上取一个微小矩形,中心为 $(x, y, z)$,边长为 $dx, dy$。我们计算沿边界逆时针走一圈的环量 $\Gamma_z = \oint \mathbf{F} \cdot d\mathbf{l}$。

底边 (1): 路径 $dx$,位于 $y - \frac{dy}{2}$。$F_1$ 的贡献:
$$\left( F_1(x, y - \frac{dy}{2}, z) \right) dx \approx (F_1 - \frac{\partial F_1}{\partial y}\frac{dy}{2}) dx$$

顶边 (2): 路径 $-dx$ (向左),位于 $y + \frac{dy}{2}$。$F_1$ 的贡献:
$$-\left( F_1(x, y + \frac{dy}{2}, z) \right) dx \approx -(F_1 + \frac{\partial F_1}{\partial y}\frac{dy}{2}) dx$$

右边 (3): 路径 $dy$ (向上),位于 $x + \frac{dx}{2}$。$F_2$ 的贡献:
$$\left( F_2(x + \frac{dx}{2}, y, z) \right) dy \approx (F_2 + \frac{\partial F_2}{\partial x}\frac{dx}{2}) dy$$

左边 (4): 路径 $-dy$ (向下),位于 $x - \frac{dx}{2}$。$F_2$ 的贡献:
$$-\left( F_2(x - \frac{dx}{2}, y, z) \right) dy \approx -(F_2 - \frac{\partial F_2}{\partial x}\frac{dx}{2}) dy$$

累加总环量 $\Gamma_z$:
$$\Gamma_z = \underbrace{\left[ (F_2 + \frac{\partial F_2}{\partial x}\frac{dx}{2}) - (F_2 - \frac{\partial F_2}{\partial x}\frac{dx}{2}) \right] dy}_{\text{垂直边贡献}} + \underbrace{\left[ (F_1 - \frac{\partial F_1}{\partial y}\frac{dy}{2}) - (F_1 + \frac{\partial F_1}{\partial y}\frac{dy}{2}) \right] dx}_{\text{水平边贡献}}$$

简化后得到:
$$\Gamma_z = \left( \frac{\partial F_2}{\partial x} - \frac{\partial F_1}{\partial y} \right) dx dy$$

由此定义绕 $z$ 轴的旋转强度(单位面积环量):
$$\omega_z = \frac{\Gamma_z}{dx dy} = \frac{\partial F_2}{\partial x} - \frac{\partial F_1}{\partial y}$$

2. 推广到其他两个平面

利用同样的逻辑(轮换对称性),我们可以求出场在其他两个正交平面上的旋转强度:

在 $yz$ 平面内 (绕 $x$ 轴):考察 $F_2$ 和 $F_3$ 随 $y$ 和 $z$ 的变化:
$$\omega_x = \frac{\partial F_3}{\partial y} - \frac{\partial F_2}{\partial z}$$

在 $zx$ 平面内 (绕 $y$ 轴):考察 $F_3$ 和 $F_1$ 随 $z$ 和 $x$ 的变化:
$$\omega_y = \frac{\partial F_1}{\partial z} - \frac{\partial F_3}{\partial x}$$

3. 合成旋度矢量

至此,我们发现这三个量 $(\omega_x, \omega_y, \omega_z)$ 完整地描述了矢量场在这一点三个维度的旋转特性。我们将这三个分量组合成一个新的矢量,并命名为 $\text{curl } \mathbf{F}$:

$$\text{curl } \mathbf{F} = \left( \frac{\partial F_3}{\partial y} - \frac{\partial F_2}{\partial z} \right) \mathbf{i} + \left( \frac{\partial F_1}{\partial z} - \frac{\partial F_3}{\partial x} \right) \mathbf{j} + \left( \frac{\partial F_2}{\partial x} - \frac{\partial F_1}{\partial y} \right) \mathbf{k}$$

4. 符号化的最终形式

为了方便记忆这种复杂的偏微分组合,我们引入算子 2$\nabla = (\frac{\partial}{\partial x}, \frac{\partial}{\partial y}, \frac{\partial}{\partial z})$。观察3发现,上述结果恰好等于 $\nabla$ 与 $\mathbf{F}$ 的叉乘结果:

$$\nabla \times \mathbf{F} = \begin{vmatrix} \mathbf{i} & \mathbf{j} & \mathbf{k} \\ \frac{\partial}{\partial x} & \frac{\partial}{\partial y} & \frac{\partial}{\partial z} \\ F_1 & F_2 & F_3 \end{vmatrix}$$

总结比较

概念	微元基础	物理意义	最终形成的算子

环量微元

闭合线积分 $\oint \mathbf{F} \cdot d\mathbf{l}$描述局部的旋转旋度 $\nabla \times \mathbf{F}$

通量微元

闭合面积分 $\oiint \mathbf{F} \cdot d\mathbf{S}$描述局部的发散/汇聚散度 $\nabla \cdot \mathbf{F}$


\section{格林定理}

格林定理(Green's Theorem)是向量分析中一个非常优美的结论,它将闭合曲线上的线积分与该曲线所围区域上的面积分联系了起来。

实际上,格林定理可以看作是旋度微元原理在二维平面上的宏观累加。

1. 物理直觉:小漩涡合成大漩涡

想象一个平面区域 $D$,被一条闭合曲线 $C$ 包围。我们将区域 $D$ 划分为无数个极其微小的矩形元(即我们之前推导过的“环量微元”)。

内部抵消:对于两个相邻的微元,它们相交的边会被经过两次,但方向相反。因此,沿这些内部公共边的线积分会全部相互抵消。

边界残留:唯一没有被抵消掉的,只有最外层紧贴曲线 $C$ 的那些微元边。

结论:所有微元环量的总和,等于绕最外圈大边界的环量。

2. 数学推导过程

设平面矢量场为 $\mathbf{F} = P(x, y)\mathbf{i} + Q(x, y)\mathbf{j}$。我们要证明:

$$\oint_C (P dx + Q dy) = \iint_D \left( \frac{\partial Q}{\partial x} - \frac{\partial P}{\partial y} \right) dA$$

第一步:处理 $Q$ 分量的部分

我们先考察面积分中的 $\iint_D \frac{\partial Q}{\partial x} dA$。假设区域 $D$ 是一个简单的 $x$-型区域(左边界为 $x=g_1(y)$,右边界为 $x=g_2(y)$):

$$\iint_D \frac{\partial Q}{\partial x} dx dy = \int_{c}^{d} \left[ \int_{g_1(y)}^{g_2(y)} \frac{\partial Q}{\partial x} dx \right] dy$$

据微积分基本定理,内部积分结果为:
$$\int_{g_1(y)}^{g_2(y)} \frac{\partial Q}{\partial x} dx = Q(g_2(y), y) - Q(g_1(y), y)$$

代回原式:
$$\iint_D \frac{\partial Q}{\partial x} dA = \int_{c}^{d} Q(g_2(y), y) dy - \int_{c}^{d} Q(g_1(y), y) dy$$

这两个积分正好对应曲线 $C$ 的右半部分和左半部分对 $dy$ 的积分。合并后即为:
$$\iint_D \frac{\partial Q}{\partial x} dA = \oint_C Q dy$$

第二步:处理 $P$ 分量的部分

同理,假设区域 $D$ 是一个 $y$-型区域(下边界 $y=f_1(x)$,上边界 $y=f_2(x)$),计算 $\iint_D \frac{\partial P}{\partial y} dA$:

$$\iint_D \frac{\partial P}{\partial y} dy dx = \int_{a}^{b} [P(x, f_2(x)) - P(x, f_1(x))] dx$$

注意,当我们沿着曲线 $C$ 逆时针走时,上边界是从右向左走的($dx$ 为负),下边界是从左向右走的($dx$ 为正)。因此:

$$\oint_C P dx = \int_{a}^{b} P(x, f_1(x)) dx + \int_{b}^{a} P(x, f_2(x)) dx = -\iint_D \frac{\partial P}{\partial y} dA$$

第三步:组合结果

将上述两部分相加:
$$\oint_C P dx + \oint_C Q dy = \iint_D \frac{\partial Q}{\partial x} dA - \iint_D \frac{\partial P}{\partial y} dA$$

合并后得到格林定理的标准形式:
$$\oint_C (P dx + Q dy) = \iint_D \left( \frac{\partial Q}{\partial x} - \frac{\partial P}{\partial y} \right) dx dy$$

右边的被积函数:$\left( \frac{\partial Q}{\partial x} - \frac{\partial P}{\partial y} \right)$。这正是我们之前推导的旋度在 $z$ 方向的分量(平面旋转强度)吗。

所以格林定理的本质就是:“区域内所有微小旋转的总和 = 边界上的总环流”。



