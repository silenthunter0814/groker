\chapter{初等向量分析}


\section{向量的微分与积分}

曲线的曲率与挠率

55. 本章的主要目的是说明向量关于标量变量(Scalar variable)的微分与积分。

在本章中,我们仅讨论单一自变量的情况,偏微分(Partial differentiation)的内容将留在第二卷中进行探讨。由于篇幅仅限一章,我们无法详尽地考究有关连续性(Continuity)和极限存在性(Existence of a limit)的所有细节。因此,我们并不强求达到无穷小分析(Infinitesimal calculus)专著中所要求的那种严密性。

我们的目标在于阐明概念,而非进行严谨的数学证明。

向量的导数 (Derivative of a Vector)

假设向量 $\mathbf{r}$ 是标量变量 $t$ 的连续且单值的函数。这意味着对于 $t$ 的每一个值,都有唯一的向量 $\mathbf{r}$ 与之对应;且当 $t$ 连续变化时,$\mathbf{r}$ 也随之连续变化。

相对于固定原点 $O$,设 $P$ 为位置向量等于 $\mathbf{r}$ 的点。若 $t$ 连续变化,则点 $P$ 在空间中沿一条连续曲线运动。

\begin{figure}[htbp] 
    \centering
    \includegraphics[width=0.6\textwidth]{images/ref/fig46.png} 
    \caption{\textbf{向量的导数}}
\end{figure}

设 $\overrightarrow{OP}$ 为对应于标量变量 $t$ 值的向量 $\mathbf{r}$。在该标量变量上增加一个增量 $\delta t$,会使前者产生一个增量 $\delta \mathbf{r}$。因此,标量的值 $t + \delta t$ 对应于向量值 $\mathbf{r} + \delta \mathbf{r}$,而这正是曲线上另一个点 $P'$ 的位置向量。增量 $\delta \mathbf{r}$ 等于向量 $\overrightarrow{PP'}$。

向量 $\delta \mathbf{r}$ 除以标量 $\delta t$ 所得的商 $\frac{\delta \mathbf{r}}{\delta t}$ 本身也是一个向量。

若此时 $\delta t$ 是一个微小增量,通常情况下 $\delta \mathbf{r}$ 也会很小。当 $\delta t$ 趋于零时,点 $P'$ 会移动到与点 $P$ 重合,且弦 $PP'$ 会趋于与曲线在 $P$ 点处的切线重合。

当 $\delta t$ 趋于零时,商 $\frac{\delta \mathbf{r}}{\delta t}$ 的极限值是一个向量,其方向是 $\delta \mathbf{r}$ 的极限方向,即 $P$ 点处的切线方向。该商的极限值(如果存在)被称为 $\mathbf{r}$ 关于 $t$ 的导数(Derivative)或微分系数(Differential coefficient),记作 $\frac{d\mathbf{r}}{dt}$。即:

$$\frac{d\mathbf{r}}{dt} = \lim_{\delta t \to 0} \frac{\delta \mathbf{r}}{\delta t}$$

确定导数的过程被称为微分(differentiation)。

向量 $\mathbf{r}$ 的导数通常本身也是关于 $t$ 的函数,因此它也拥有导数,这被称为 $\mathbf{r}$ 的二阶导数,记作 $\frac{d^2\mathbf{r}}{dt^2}$。同理,二阶导数的导数被称为 $\mathbf{r}$ 的三阶导数,记作 $\frac{d^3\mathbf{r}}{dt^3}$。

一种特别重要的情况是:当 $t$ 代表时间变量,而 $\mathbf{r}$ 代表运动质点 $P$ 相对于原点 $O$ 的位置向量时。在这种情况下,$\delta\mathbf{r}$ 表示质点在时间间隔 $\delta t$ 内的位移,因此 $\frac{\delta\mathbf{r}}{\delta t}$ 代表该时段内的平均速度。当 $\delta t$ 趋于零时,该平均速度的极限值即为质点的瞬时速度。因此,代表 $P$ 的瞬时速度的向量 $\mathbf{v}$ 为:

$$\mathbf{v} = \frac{d\mathbf{r}}{dt}$$

该向量的方向自然是沿着质点运动轨迹的切线方向。

与之类似,如果 $\delta\mathbf{v}$ 是速度向量 $\mathbf{v}$ 在时间间隔 $\delta t$ 内的增量,则比值 $\frac{\delta\mathbf{v}}{\delta t}$ 代表该时段内的平均加速度。质点的瞬时加速度是该平均加速度在 $\delta t$ 趋于零时的极限值。因此,向量

$$\mathbf{a} = \frac{d\mathbf{v}}{dt} = \frac{d^2\mathbf{r}}{dt^2}$$

代表了运动质点的瞬时加速度。

常向量的导数:任何常向量 $\mathbf{c}$ 的导数均为零;因为增量 $\delta t$ 不会对 $\mathbf{c}$ 产生任何改变。

和的导数:两个向量 $\mathbf{r}$ 与 $\mathbf{s}$(均为 $t$ 的函数)之和 $\mathbf{r} + \mathbf{s}$ 的导数,等于它们各自导数的和。

如果 $\delta \mathbf{r}$ 和 $\delta \mathbf{s}$ 是由增量 $\delta t$ 引起的向量增量,则有:$\delta(\mathbf{r} + \mathbf{s}) = (\mathbf{r} + \delta \mathbf{r} + \mathbf{s} + \delta \mathbf{s}) - (\mathbf{r} + \mathbf{s}) = \delta \mathbf{r} + \delta \mathbf{s}$。

因此其商为:$\frac{\delta(\mathbf{r} + \mathbf{s})}{\delta t} = \frac{\delta \mathbf{r}}{\delta t} + \frac{\delta \mathbf{s}}{\delta t}$。

当 $\delta t$ 趋于零时,取两侧的极限值,可得:

$$\frac{d}{dt}(\mathbf{r} + \mathbf{s}) = \frac{d\mathbf{r}}{dt} + \frac{d\mathbf{s}}{dt}$$

多向量求和:上述论证显然也适用于任意数量向量之和。

复合函数求导(链式法则)

假设 $\mathbf{r}$ 是标量变量 $s$ 的连续函数,而 $s$ 是另一个变量 $t$ 的连续函数。那么后者产生的增量 $\delta t$ 会导致其他变量产生增量 $\delta \mathbf{r}$ 和 $\delta s$,且它们都随 $\delta t$ 趋于零而趋于零。

关系式 $\frac{\delta \mathbf{r}}{\delta t} = \frac{\delta \mathbf{r}}{\delta s} \frac{\delta s}{\delta t}$ 是一个代数恒等式,其中置于向量之后的数值与置于其前的意义相同。

当 $\delta t$ 趋于零时,取两侧极限可得公式:

$$\frac{d\mathbf{r}}{dt} = \frac{d\mathbf{r}}{ds} \frac{ds}{dt}$$

这与代数微积分中的形式一致。

56. 乘积的导数

任意向量乘积的导数求法与代数乘积相同,即等于依次对其中一个因子求导而保持其他因子不变所得到的各项之和。

以标量 $u$ 和向量 $\mathbf{r}$ 的乘积 $u\mathbf{r}$ 为例(两者均为变量 $t$ 的函数):

若 $\delta u$ 和 $\delta \mathbf{r}$ 是由增量 $\delta t$ 引起的增量,则乘积的增量为:$\delta(u\mathbf{r}) = (u + \delta u)(\mathbf{r} + \delta \mathbf{r}) - u\mathbf{r}$$= \delta u \mathbf{r} + u \delta \mathbf{r} + \delta u \delta \mathbf{r}$。

对 $\frac{\delta(u\mathbf{r})}{\delta t} = \frac{\delta u}{\delta t} \mathbf{r} + u \frac{\delta \mathbf{r}}{\delta t} + \frac{\delta u}{\delta t} \delta \mathbf{r}$ 取极限可得:

$$\frac{d}{dt}(u\mathbf{r}) = \frac{du}{dt} \mathbf{r} + u \frac{d\mathbf{r}}{dt} \quad \dots (1)$$

直角坐标分量形式:若 $\mathbf{r} = x\mathbf{i} + y\mathbf{j} + z\mathbf{k}$,由于 $\mathbf{i, j, k}$ 是常向量,其导数为:

$$\frac{d\mathbf{r}}{dt} = \frac{dx}{dt}\mathbf{i} + \frac{dy}{dt}\mathbf{j} + \frac{dz}{dt}\mathbf{k}$$

点积(标量积)与叉积(向量积)

点积导数:$\frac{d}{dt}(\mathbf{r} \cdot \mathbf{s}) = \frac{d\mathbf{r}}{dt} \cdot \mathbf{s} + \mathbf{r} \cdot \frac{d\mathbf{s}}{dt} \quad \dots (2)$

叉积导数:$\frac{d}{dt}(\mathbf{r} \times \mathbf{s}) = \frac{d\mathbf{r}}{dt} \times \mathbf{s} + \mathbf{r} \times \frac{d\mathbf{s}}{dt} \quad \dots (3)$

注意:在叉积公式中,各项因子的顺序不能改变,除非同时改变符号。

特殊情形:模与方向

若在公式 (2) 中令 $\mathbf{s} = \mathbf{r}$,可得 $\frac{d}{dt}(\mathbf{r} \cdot \mathbf{r}) = 2\mathbf{r} \cdot \frac{d\mathbf{r}}{dt}$。

若 $r$ 是向量 $\mathbf{r}$ 的模,则 $\mathbf{r} \cdot \mathbf{r} = r^2$。由于 $r^2$ 的导数是 $2r \frac{dr}{dt}$,故有:

$$\mathbf{r} \cdot \frac{d\mathbf{r}}{dt} = r \frac{dr}{dt}$$

恒定长度向量:若向量 $\mathbf{a}$ 的长度恒定,则 $\mathbf{a}^2 = a^2 = \text{常数}$。