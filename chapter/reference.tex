\chapter{空间曲线}


\section{切向量 $\mathbf{T}$、法向量 $\mathbf{N}$ 和副法向量 $\mathbf{B}$}

在微分几何中,Frenet-Serret 标架($\mathbf{T}$, $\mathbf{N}$, $\mathbf{B}$)是描述曲线局部特征的核心。

求解逻辑:

第一步:切向量 $\mathbf{T}$ (Tangent)

对曲线 $\mathbf{r}(t)$ 求导并单位化:
$$\mathbf{T} = \frac{\mathbf{r}'(t)}{\|\mathbf{r}'(t)\|}$$

第二步:副法向量 $\mathbf{B}$ (Binormal)

在 3D 中,通常先求 $\mathbf{B}$ 反而更容易。利用加速度(二阶导)和速度的外积:
$$\mathbf{B} = \frac{\mathbf{r}'(t) \times \mathbf{r}''(t)}{\|\mathbf{r}'(t) \times \mathbf{r}''(t)\|}$$

原理: 速度和加速度构成的平面被称为密切平面,$\mathbf{B}$ 就是该平面的法向量。

第三步:主法向量 $\mathbf{N}$ (Normal)

有了 $\mathbf{T}$ 和 $\mathbf{B}$,$\mathbf{N}$ 可以通过叉乘直接得出,确保右手系:$$\mathbf{N} = \mathbf{B} \times \mathbf{T}$$

\begin{example} 抛物线 $y = x^2$ 的切向量 $\mathbf{T}$、法向量 $\mathbf{N}$ 和副法向量 $\mathbf{B}$。

    参数化方程为 $\mathbf{r}(x) = (x, x^2)$。
    
    为了方便计算空间中的 $B$ 向量,我们可以将其视为在 $z=0$ 平面上的三维曲线,即 $\mathbf{r}(x) = (x, x^2, 0)$。
    
    首先计算位置向量的一阶和二阶导数:
    
    $\mathbf{r}'(x) = (1, 2x, 0)$
    
    $\mathbf{r}''(x) = (0, 2, 0)$
    
    一阶导的模长:$\|\mathbf{r}'(x)\| = \sqrt{1 + (2x)^2} = \sqrt{1 + 4x^2}$

    切向量 $\mathbf{T}$ (Tangent)
    
    直接对一阶导进行单位化:
    $$\mathbf{T} = \frac{\mathbf{r}'(x)}{\|\mathbf{r}'(x)\|} = \left( \frac{1}{\sqrt{1+4x^2}}, \frac{2x}{\sqrt{1+4x^2}}, 0 \right)$$

    副法向量 $\mathbf{B}$ (Binormal)
    
    对于平面曲线,副法向量总是垂直于平面。我们通过叉乘计算:
    $$\mathbf{r}' \times \mathbf{r}'' = \begin{vmatrix} \mathbf{i} & \mathbf{j} & \mathbf{k} \\ 1 & 2x & 0 \\ 0 & 2 & 0 \end{vmatrix} = (0, 0, 2)$$

    单位化后得到:
    $$\mathbf{B} = (0, 0, 1)$$
    
    (这说明抛物线始终在 $xy$ 平面内弯曲)

    主法向量 $\mathbf{N}$ (Normal)
    
    利用 $\mathbf{N} = \mathbf{B} \times \mathbf{T}$(满足右手系且指向凹侧):
    
    $$\mathbf{N} = \begin{vmatrix} \mathbf{i} & \mathbf{j} & \mathbf{k} \\ 0 & 0 & 1 \\ \frac{1}{\sqrt{1+4x^2}} & \frac{2x}{\sqrt{1+4x^2}} & 0 \end{vmatrix} = \left( -\frac{2x}{\sqrt{1+4x^2}}, \frac{1}{\sqrt{1+4x^2}}, 0 \right)$$

    计算曲率 $\kappa$ (Curvature)
    
    曲率公式为:$$\kappa = \frac{\|\mathbf{r}' \times \mathbf{r}''\|}{\|\mathbf{r}'\|^3}$$
    
    代入已知量:
    
    $\|\mathbf{r}' \times \mathbf{r}''\| = 2$
    
    $\|\mathbf{r}'\| = (1+4x^2)^{1/2}$

    得到:
    $$\kappa(x) = \frac{2}{(1 + 4x^2)^{3/2}}$$

    在顶点 $(0,0)$ 处: $\kappa = 2$。此时曲率最大,密切圆半径 $R = 1/\kappa = 0.5$。
    
    当 $x \to \infty$ 时: $\kappa \to 0$。这符合直觉,因为抛物线远端越来越趋于直线。

\end{example}