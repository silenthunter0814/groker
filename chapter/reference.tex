\chapter{矢量与张量分析}


\section{矢量代数}

1.1 矢量的定义 (Definition of a Vector)
在物理学和工程学中,我们经常遇到两类量:标量 (Scalars) 和 矢量 (Vectors)。

标量: 仅由其大小(带正号或负号)确定的量。例如:质量、时间、温度、密度、功和能量。标量遵循普通代数的运算法则。

矢量: 既有大小又有方向的量。例如:位移、速度、加速度、力和电场强度。

在几何上,我们用一个带箭头的线段来表示矢量。线段的长度代表矢量的大小,箭头的指向代表矢量的方向。

1.2 矢量的表示法 (Notation)

在本书中,我们将用粗体字母(如 $\mathbf{A}, \mathbf{B}, \mathbf{a}, \mathbf{b}$)表示矢量。标量则用斜体字母表示。矢量 $\mathbf{A}$ 的大小(或长度)记作 $|\mathbf{A}|$ 或简单的 $A$。

单位矢量 (Unit Vector): 大小为 1 的矢量称为单位矢量。

零矢量 (Zero Vector): 大小为 0 的矢量称为零矢量,记作 $\mathbf{0}$。它的方向是不确定的。

1.3 矢量的加法 (Addition of Vectors)

两个矢量 $\mathbf{A}$ 和 $\mathbf{B}$ 的和 $\mathbf{C} = \mathbf{A} + \mathbf{B}$ 可以通过平行四边形法则或三角形法则来定义。

三角形法则: 将矢量 $\mathbf{B}$ 的起点放在矢量 $\mathbf{A}$ 的终点,那么从 $\mathbf{A}$ 的起点指向 $\mathbf{B}$ 的终点的矢量就是 $\mathbf{A} + \mathbf{B}$。

法则属性:

交换律: $\mathbf{A} + \mathbf{B} = \mathbf{B} + \mathbf{A}$

结合律: $(\mathbf{A} + \mathbf{B}) + \mathbf{C} = \mathbf{A} + (\mathbf{B} + \mathbf{C})$

1.4 矢量的减法 (Subtraction of Vectors)

如果 $\mathbf{B}$ 是一个矢量,那么 $-\mathbf{B}$ 是一个与 $\mathbf{B}$ 大小相等但方向相反的矢量。两个矢量的差定义为:
$$\mathbf{A} - \mathbf{B} = \mathbf{A} + (-\mathbf{B})$$

在几何上,如果 $\mathbf{A}$ 和 $\mathbf{B}$ 从同一点出发,那么 $\mathbf{A} - \mathbf{B}$ 是从 $\mathbf{B}$ 的终点指向 $\mathbf{A}$ 的终点的矢量。

1.5 标量与矢量的乘法 (Multiplication of a Vector by a Scalar)

标量 $m$ 与矢量 $\mathbf{A}$ 的乘积记作 $m\mathbf{A}$。

如果 $m > 0$,则 $m\mathbf{A}$ 的方向与 $\mathbf{A}$ 相同,大小为 $m|\mathbf{A}|$。

如果 $m < 0$,则 $m\mathbf{A}$ 的方向与 $\mathbf{A}$ 相反,大小为 $|m||\mathbf{A}|$。

如果 $m = 0$,则结果为零矢量 $\mathbf{0}$。

该运算满足分配律:
$$m(\mathbf{A} + \mathbf{B}) = m\mathbf{A} + m\mathbf{B}$$
$$(m + n)\mathbf{A} = m\mathbf{A} + n\mathbf{A}$$

1.6 共线与共面矢量 (Collinear and Coplanar Vectors)

共线: 如果两个矢量平行于同一条直线,则称它们为共线矢量。如果 $\mathbf{A}$ 与 $\mathbf{B}$ 共线,则存在标量 $k$ 使得 $\mathbf{A} = k\mathbf{B}$。

共面: 如果三个或多个矢量平行于同一个平面,则称它们为共面矢量。

第 6 页:矢量的正交分解 (Orthogonal Components of a Vector)

设 $\mathbf{i, j, k}$ 为沿直角笛卡尔坐标系 $x, y, z$ 轴正方向的一组单位矢量。任何矢量 $\mathbf{A}$ 都可以表示为这些单位矢量的线性组合。
$$\mathbf{A} = A_x \mathbf{i} + A_y \mathbf{j} + A_z \mathbf{k}$$

标量 $A_x, A_y, A_z$ 称为 $\mathbf{A}$ 的分量。$\mathbf{A}$ 的模(大小)由下式给出:

$$A = |\mathbf{A}| = \sqrt{A_x^2 + A_y^2 + A_z^2}$$。

第 7 页:方向余弦 (Direction Cosines)

若 $\alpha, \beta, \gamma$ 分别是 $\mathbf{A}$ 与 $x, y, z$ 轴正方向的夹角,则有:

$$A_x = A \cos \alpha, \quad A_y = A \cos \beta, \quad A_z = A \cos \gamma$$。

这些被称为 $\mathbf{A}$ 的方向余弦。由此可知:
$$\cos^2 \alpha + \cos^2 \beta + \cos^2 \gamma = 1$$。

第 8 页:点积/数量积 (The Dot or Scalar Product)

两个矢量 $\mathbf{A}$ 与 $\mathbf{B}$ 的点积(或称数量积),记作 $\mathbf{A \cdot B}$,定义为它们的模与它们之间夹角 $\theta$ 的余弦之积。

$$\mathbf{A \cdot B} = AB \cos \theta, \quad (0 \le \theta \le \pi)$$

根据该定义,显而易见:

$\mathbf{A \cdot B} = \mathbf{B \cdot A}$ (交换律)

$\mathbf{A \cdot A} = A^2$

若 $\mathbf{A \cdot B} = 0$ 且 $\mathbf{A, B} \neq 0$,则 $\mathbf{A}$ 垂直于 $\mathbf{B}$。

第 9 页:分量形式的点积与分配律

点积满足分配律 $\mathbf{A \cdot (B + C)} = \mathbf{A \cdot B} + \mathbf{A \cdot C}$。这可以通过观察发现:$\mathbf{A \cdot B}$ 等于 $A$ 与 $\mathbf{B}$ 在 $\mathbf{A}$ 方向上的投影的乘积。

用分量表示时,由于 $\mathbf{i \cdot i} = \mathbf{j \cdot j} = \mathbf{k \cdot k} = 1$ 且 $\mathbf{i \cdot j} = \mathbf{j \cdot k} = \mathbf{k \cdot i} = 0$,我们得到:$$\mathbf{A \cdot B} = (A_x \mathbf{i} + A_y \mathbf{j} + A_z \mathbf{k}) \cdot (B_x \mathbf{i} + B_y \mathbf{j} + B_z \mathbf{k}) = A_x B_x + A_y B_y + A_z B_z$$

第 10 页:点积的应用:余弦定理与投影

利用点积可以轻松推导出三角形的余弦定理。设 $\mathbf{C = A - B}$,则:

$$\mathbf{C \cdot C} = (\mathbf{A - B}) \cdot (\mathbf{A - B}) = \mathbf{A \cdot A} + \mathbf{B \cdot B} - 2\mathbf{A \cdot B}$$
$$C^2 = A^2 + B^2 - 2AB \cos \theta$$

第 11 页:矢量积或叉积 (The Vector or Cross Product)

两个矢量 $\mathbf{A}$ 与 $\mathbf{B}$ 的矢量积(又称叉积),记作 $\mathbf{A} \times \mathbf{B}$,其结果是一个矢量 $\mathbf{C}$。其模长定义为 $C = AB \sin \theta$,其中 $\theta$ 是 $\mathbf{A}$ 与 $\mathbf{B}$ 之间的夹角($0 \le \theta \le \pi$)。

矢量 $\mathbf{C} = \mathbf{A} \times \mathbf{B}$ 的方向垂直于 $\mathbf{A}$ 和 $\mathbf{B}$ 所确定的平面,且 $\mathbf{A, B, C}$ 构成一个右手系。这意味着,如果右手四指从 $\mathbf{A}$ 经较小夹角 $\theta$ 弯向 $\mathbf{B}$,则大拇指所指的方向即为 $\mathbf{C}$ 的方向。

第 12 页:叉积的几何性质 (Geometric Properties of the Cross Product)

根据定义可知 $\mathbf{A} \times \mathbf{B} = -(\mathbf{B} \times \mathbf{A})$。因此,交换律不适用于矢量积。

若 $\mathbf{A} \times \mathbf{B} = \mathbf{0}$ 且 $\mathbf{A, B} \neq \mathbf{0}$,则 $\sin \theta = 0$,这意味着 $\mathbf{A}$ 与 $\mathbf{B}$ 平行或共线。

模长 $AB \sin \theta$ 在几何上表示以 $\mathbf{A}$ 和 $\mathbf{B}$ 为邻边的平行四边形的面积。

第 13 页:单位矢量的叉积 (Cross Product of Unit Vectors)

对于右手笛卡尔坐标系中的基本单位矢量 $\mathbf{i, j, k}$,我们有:

$\mathbf{i} \times \mathbf{i} = \mathbf{j} \times \mathbf{j} = \mathbf{k} \times \mathbf{k} = \mathbf{0}$


此外:

$\mathbf{i} \times \mathbf{j} = \mathbf{k}, \quad \mathbf{j} \times \mathbf{k} = \mathbf{i}, \quad \mathbf{k} \times \mathbf{i} = \mathbf{j}$。
$\mathbf{j} \times \mathbf{i} = -\mathbf{k}, \quad \mathbf{k} \times \mathbf{j} = -\mathbf{i}, \quad \mathbf{i} \times \mathbf{k} = -\mathbf{j}$。

第 14 页:叉积的分量形式 (Cross Product in Component Form)

可以证明分配律 $\mathbf{A} \times (\mathbf{B} + \mathbf{C}) = \mathbf{A} \times \mathbf{B} + \mathbf{A} \times \mathbf{C}$ 是成立的。利用该定律,我们可以用分量表示 $\mathbf{A} \times \mathbf{B}$:
$$\mathbf{A} \times \mathbf{B} = (A_y B_z - A_z B_y)\mathbf{i} + (A_z B_x - A_x B_z)\mathbf{j} + (A_x B_y - A_y B_x)\mathbf{k}$$

这一结果最容易通过行列式的形式来记忆:
$$\mathbf{A} \times \mathbf{B} = \begin{vmatrix} \mathbf{i} & \mathbf{j} & \mathbf{k} \\ A_x & A_y & A_z \\ B_x & B_y & B_z \end{vmatrix}$$

第 15 页:标量三重积 (The Scalar Triple Product)

乘积 $\mathbf{A} \cdot (\mathbf{B} \times \mathbf{C})$ 被称为标量三重积(又称混合积)。其结果是一个标量。

在几何上,$\mathbf{A} \cdot (\mathbf{B} \times \mathbf{C})$ 的绝对值表示以 $\mathbf{A, B, C}$ 为共点棱的平行六面体的体积。

用行列式形式表示为:
$$\mathbf{A} \cdot (\mathbf{B} \times \mathbf{C}) = \begin{vmatrix} A_x & A_y & A_z \\ B_x & B_y & B_z \\ C_x & C_y & C_z \end{vmatrix}$$

第 16 页:矢量三重积 (The Vector Triple Product)

乘积 $\mathbf{A} \times (\mathbf{B} \times \mathbf{C})$ 被称为矢量三重积。与标量三重积不同,其结果是一个矢量。由于 $\mathbf{B} \times \mathbf{C}$ 垂直于 $\mathbf{B}$ 和 $\mathbf{C}$ 构成的平面,因此矢量 $\mathbf{A} \times (\mathbf{B} \times \mathbf{C})$ 必然位于 $\mathbf{B}$ 和 $\mathbf{C}$ 所在的平面内。

下述重要的展开公式成立:
$$\mathbf{A} \times (\mathbf{B} \times \mathbf{C}) = (\mathbf{A} \cdot \mathbf{C})\mathbf{B} - (\mathbf{A} \cdot \mathbf{B})\mathbf{C}$$

这通常被称为“BAC-CAB”法则。请注意,一般情况下,$\mathbf{A} \times (\mathbf{B} \times \mathbf{C}) \neq (\mathbf{A} \times \mathbf{B}) \times \mathbf{C}$。

第 17 页:标量三重积的循环特性 (Cyclic Permutations)

对于标量三重积 $\mathbf{A} \cdot (\mathbf{B} \times \mathbf{C})$,点乘和叉乘符号可以互换而不改变其值:

$\mathbf{A} \cdot (\mathbf{B} \times \mathbf{C}) = (\mathbf{A} \times \mathbf{B}) \cdot \mathbf{C}$

此外,在矢量的循环轮换下,其值保持不变:

$[\mathbf{A, B, C}] = \mathbf{A} \cdot (\mathbf{B} \times \mathbf{C}) = \mathbf{B} \cdot (\mathbf{C} \times \mathbf{A}) = \mathbf{C} \cdot (\mathbf{A} \times \mathbf{B})$

如果顺序非循环,则符号改变:

$\mathbf{A} \cdot (\mathbf{B} \times \mathbf{C}) = -\mathbf{A} \cdot (\mathbf{C} \times \mathbf{B})$

第 18 页:四重积恒等式 (Quadruple Products)

利用前面的公式,我们可以推导出涉及四个矢量的恒等式。四个矢量的数量积:

$(\mathbf{A} \times \mathbf{B}) \cdot (\mathbf{C} \times \mathbf{D}) = (\mathbf{A} \cdot \mathbf{C})(\mathbf{B} \cdot \mathbf{D}) - (\mathbf{A} \cdot \mathbf{D})(\mathbf{B} \cdot \mathbf{C})$

四个矢量的向量积:

$(\mathbf{A} \times \mathbf{B}) \times (\mathbf{C} \times \mathbf{D}) = [\mathbf{A, C, D}]\mathbf{B} - [\mathbf{B, C, D}]\mathbf{A}$

这表明所得矢量既位于 $\mathbf{A}$ 和 $\mathbf{B}$ 确定的平面内,也位于 $\mathbf{C}$ 和 $\mathbf{D}$ 确定的平面内。

第 19 页:矢量方程 (Vector Equations)

考虑方程 $\mathbf{A} \cdot \mathbf{X} = p$,其中 $\mathbf{A}$ 和 $p$ 为已知。该方程不能唯一确定 $\mathbf{X}$。在几何上,它表示一个垂直于 $\mathbf{A}$ 的平面,其到原点的距离为 $p/|\mathbf{A}|$。

现在考虑 $\mathbf{A} \times \mathbf{X} = \mathbf{B}$。为了使解存在,$\mathbf{A}$ 必须垂直于 $\mathbf{B}$(即 $\mathbf{A} \cdot \mathbf{B} = 0$)。其通解为:
$$\mathbf{X} = \frac{\mathbf{B} \times \mathbf{A}}{A^2} + \lambda \mathbf{A}$$
其中 $\lambda$ 是任意标量。

第 20 页:倒易矢量系 (Reciprocal System of Vectors)

如果两组矢量 $\mathbf{a, b, c}$ 和 $\mathbf{a', b', c'}$ 满足以下条件,则称它们为倒易系:
$\mathbf{a} \cdot \mathbf{a'} = \mathbf{b} \cdot \mathbf{b'} = \mathbf{c} \cdot \mathbf{c'} = 1$
$\mathbf{a} \cdot \mathbf{b'} = \mathbf{a} \cdot \mathbf{c'} = \dots = 0$

倒易矢量可以按如下方式构造:
$$\mathbf{a'} = \frac{\mathbf{b} \times \mathbf{c}}{[\mathbf{a, b, c}]}, \quad \mathbf{b'} = \frac{\mathbf{c} \times \mathbf{a}}{[\mathbf{a, b, c}]}, \quad \mathbf{c'} = \frac{\mathbf{a} \times \mathbf{b}}{[\mathbf{a, b, c}]}$$

前提是 $[\mathbf{a, b, c}] \neq 0$。

第 21 页:直线的矢量方程

直线的表示:空间中的一条直线可以通过一个已知点 $A$(位置矢量为 $\mathbf{a}$)以及直线所平行的方向矢量 $\mathbf{b}$ 来唯一确定。设 $P$ 是直线上任意一点,其位置矢量为 $\mathbf{r}$。

则矢量 $\vec{AP} = \mathbf{r} - \mathbf{a}$ 必定与 $\mathbf{b}$ 平行。因此,存在一个标量参数 $t$,使得:
$$\mathbf{r} - \mathbf{a} = t\mathbf{b}$$

或者写作:
$$\mathbf{r} = \mathbf{a} + t\mathbf{b}$$

这就是直线的参数矢量方程。

另一种形式:由于 $\mathbf{r} - \mathbf{a}$ 与 $\mathbf{b}$ 平行,它们的叉积必须为零:
$$(\mathbf{r} - \mathbf{a}) \times \mathbf{b} = \mathbf{0}$$

第 22 页:两点确定的直线与距离公式

两点式方程:若直线通过两个已知点 $A(\mathbf{a})$ 和 $B(\mathbf{b})$,则直线的方向矢量可以取为 $\mathbf{b} - \mathbf{a}$。此时方程变为:
$$\mathbf{r} = \mathbf{a} + t(\mathbf{b} - \mathbf{a})$$

或
$$\mathbf{r} = (1 - t)\mathbf{a} + t\mathbf{b}$$

点到直线的距离:

设已知点 $Q$ 的位置矢量为 $\mathbf{q}$,直线方程为 $\mathbf{r} = \mathbf{a} + t\mathbf{b}$。点 $Q$ 到该直线的垂直距离 $d$ 由下式给出:

$$d = \frac{|(\mathbf{q} - \mathbf{a}) \times \mathbf{b}|}{|\mathbf{b}|}$$

这里 $(\mathbf{q} - \mathbf{a}) \times \mathbf{b}$ 的模表示以 $\mathbf{q} - \mathbf{a}$ 和 $\mathbf{b}$ 为邻边的平行四边形的面积,除以底边长 $|\mathbf{b}|$ 即得高 $d$。

第 23 页:平面的矢量方程

点法式方程:一个平面可以通过平面内的一点 $A(\mathbf{a})$ 和一个垂直于平面的法矢量 $\mathbf{n}$ 来确定。若 $P(\mathbf{r})$ 是平面上的任意一点,则矢量 $\mathbf{r} - \mathbf{a}$ 必然位于平面内,因此与 $\mathbf{n}$ 垂直。其点积必为零:

$$(\mathbf{r} - \mathbf{a}) \cdot \mathbf{n} = 0$$

或者
$$\mathbf{r} \cdot \mathbf{n} = \mathbf{a} \cdot \mathbf{n} = p$$

其中 $p$ 是一个常数。

三点定平面:

若平面通过不共线的三点 $A(\mathbf{a}), B(\mathbf{b}), C(\mathbf{c})$,则平面上任意一点 $P(\mathbf{r})$ 满足矢量 $\mathbf{r} - \mathbf{a}, \mathbf{b} - \mathbf{a}, \mathbf{c} - \mathbf{a}$ 共面。因此它们的标量三重积为零:

$$(\mathbf{r} - \mathbf{a}) \cdot [(\mathbf{b} - \mathbf{a}) \times (\mathbf{c} - \mathbf{a})] = 0$$

第 24 页:点到平面的距离与两平面夹角

点到平面的距离:点 $Q(\mathbf{q})$ 到平面 $\mathbf{r} \cdot \mathbf{n} = p$ 的垂直距离 $D$ 是 $\mathbf{q} - \mathbf{r}$ 在法方向上的投影长度。如果 $\mathbf{n}$ 是单位法矢量,则:

$$D = | \mathbf{q} \cdot \mathbf{n} - p |$$

若 $\mathbf{n}$ 不是单位矢量,则需除以其模长。

两平面的夹角:两个平面之间的夹角定义为它们法矢量 $\mathbf{n}_1$ 和 $\mathbf{n}_2$ 之间的夹角 $\theta$:
$$\cos \theta = \frac{\mathbf{n}_1 \cdot \mathbf{n}_2}{|\mathbf{n}_1| |\mathbf{n}_2|}$$

两平面的交线:

两个平面的交线方向与两个法矢量的叉积 $\mathbf{n}_1 \times \mathbf{n}_2$ 平行。

第 25 页:球体与圆的矢量表示

球体方程:

设球心位置矢量为 $\mathbf{c}$,半径为 $a$。球面上任意一点 $P(\mathbf{r})$ 到球心的距离恒为 $a$:

$$|\mathbf{r} - \mathbf{c}| = a$$平方可得:$$(\mathbf{r} - \mathbf{c}) \cdot (\mathbf{r} - \mathbf{c}) = a^2$$

或者
$$r^2 - 2\mathbf{r} \cdot \mathbf{c} + c^2 = a^2$$

切平面方程:

球面上一点 $P_0(\mathbf{r}_0)$ 处的切平面垂直于半径矢量 $\mathbf{r}_0 - \mathbf{c}$。因此,切平面的方程为:
$$(\mathbf{r} - \mathbf{r}_0) \cdot (\mathbf{r}_0 - \mathbf{c}) = 0$$


\section{矢量微积分 —— 矢量函数}

矢量函数的定义:

如果对于标量变量 $t$ 在某个区间内的每一个值,都有一个唯一的矢量 $\mathbf{u}$ 与之对应,则称 $\mathbf{u}$ 是 $t$ 的矢量函数,记作 $\mathbf{u} = \mathbf{f}(t)$。在直角坐标系中,这等价于其三个分量都是 $t$ 的标量函数:
$$\mathbf{u}(t) = u_x(t)\mathbf{i} + u_y(t)\mathbf{j} + u_z(t)\mathbf{k}$$

极限与连续性:

若当 $t \to t_0$ 时,矢量 $\mathbf{u}(t)$ 的模与某一固定矢量 $\mathbf{L}$ 的差趋于零,即 $\lim_{t \to t_0} |\mathbf{u}(t) - \mathbf{L}| = 0$,则称 $\mathbf{L}$ 为 $\mathbf{u}(t)$ 的极限。如果 $\lim_{t \to t_0} \mathbf{u}(t) = \mathbf{u}(t_0)$,则称该矢量函数在 $t_0$ 处连续。

第 27 页:矢量的导数

导数的定义:设 $\mathbf{r}(t)$ 是随标量 $t$ 变化的矢量。其关于 $t$ 的导数定义为:

$$\frac{d\mathbf{r}}{dt} = \lim_{\Delta t \to 0} \frac{\mathbf{r}(t + \Delta t) - \mathbf{r}(t)}{\Delta t}$$

如果 $\mathbf{r}$ 表示质点的位置矢量,且 $t$ 表示时间,那么 $d\mathbf{r}/dt$ 就是质点的瞬时速度矢量 $\mathbf{v}$。

几何意义:

从几何上看,$\Delta \mathbf{r} = \mathbf{r}(t + \Delta t) - \mathbf{r}(t)$ 是曲线上的弦矢量。随着 $\Delta t \to 0$,该矢量的方向趋于曲线在该点处的切线方向。因此,$d\mathbf{r}/dt$ 是一个沿切线方向的矢量。

第 28 页:微分运算法则

矢量导数的运算法则与标量微积分极其相似,但必须注意叉积的顺序。设 $\mathbf{u}, \mathbf{v}$ 为可微矢量函数,$\phi$ 为可微标量函数:

加法法则:
$$\frac{d}{dt}(\mathbf{u} + \mathbf{v}) = \frac{d\mathbf{u}}{dt} + \frac{d\mathbf{v}}{dt}$$

标量乘积法则:
$$\frac{d}{dt}(\phi \mathbf{u}) = \frac{d\phi}{dt}\mathbf{u} + \phi \frac{d\mathbf{u}}{dt}$$

点积法则:
$$\frac{d}{dt}(\mathbf{u} \cdot \mathbf{v}) = \frac{d\mathbf{u}}{dt} \cdot \mathbf{v} + \mathbf{u} \cdot \frac{d\mathbf{v}}{dt}$$

叉积法则:
$$\frac{d}{dt}(\mathbf{u} \times \mathbf{v}) = \frac{d\mathbf{u}}{dt} \times \mathbf{v} + \mathbf{u} \times \frac{d\mathbf{v}}{dt}$$

注意:叉积项的顺序必须保持不变。

第 29 页:常量模矢量的性质

重要定理:如果一个矢量 $\mathbf{a}(t)$ 的模长是常数(即 $|\mathbf{a}| = c$),则该矢量与其导数矢量互相垂直。

证明:因为 $|\mathbf{a}|^2 = \mathbf{a} \cdot \mathbf{a} = c^2$。对等式两边关于 $t$ 求导:
$$\frac{d}{dt}(\mathbf{a} \cdot \mathbf{a}) = 0$$
$$\mathbf{a} \cdot \frac{d\mathbf{a}}{dt} + \frac{d\mathbf{a}}{dt} \cdot \mathbf{a} = 0$$
$$2\mathbf{a} \cdot \frac{d\mathbf{a}}{dt} = 0$$

由此得 $\mathbf{a} \cdot \frac{d\mathbf{a}}{dt} = 0$,即 $\mathbf{a} \perp \frac{d\mathbf{a}}{dt}$。
这个结论在研究圆周运动(半径矢量模长不变)或单位矢量场时非常有用。

第 30 页:偏导数与复合函数求导

偏导数:

如果矢量 $\mathbf{A}$ 是多个标量变量(如 $x, y, z$)的函数,我们可以定义其偏导数。例如,关于 $x$ 的偏导数为:

$$\frac{\partial \mathbf{A}}{\partial x} = \lim_{\Delta x \to 0} \frac{\mathbf{A}(x + \Delta x, y, z) - \mathbf{A}(x, y, z)}{\Delta x}$$

这可以通过对 $\mathbf{A}$ 的各个分量分别求偏导来实现。

全微分:

若 $\mathbf{A} = \mathbf{A}(x, y, z)$,则其全微分为:

$$d\mathbf{A} = \frac{\partial \mathbf{A}}{\partial x}dx + \frac{\partial \mathbf{A}}{\partial y}dy + \frac{\partial \mathbf{A}}{\partial z}dz$$

链式法则:

若 $\mathbf{A}$ 是 $s$ 的函数,而 $s$ 又是 $t$ 的函数,则:
$$\frac{d\mathbf{A}}{dt} = \frac{d\mathbf{A}}{ds} \frac{ds}{dt}$$


第 31 页:空间曲线与弧长

曲线的参数化:

考虑由方程 $\mathbf{r} = \mathbf{r}(s)$ 定义的空间曲线,其中 $s$ 是沿曲线测量的弧长。使用弧长作为参数具有特殊的数学意义,因为当参数变化 $\Delta s$ 时,点在空间移动的距离(弦长)在极限情况下等于弧长。

单位切矢量:

导数 $\mathbf{T} = d\mathbf{r}/ds$ 是一个矢量,其方向沿曲线的切线方向。由于 $|\Delta \mathbf{r}| / \Delta s \to 1$(当 $\Delta s \to 0$ 时),因此 $\mathbf{T}$ 是一个单位矢量,即 $|\mathbf{T}| = 1$。
我们称 $\mathbf{T}$ 为曲线在某点处的单位切矢量。

第 32 页:曲率与主法矢量

曲率的定义:由于 $\mathbf{T}$ 是单位矢量,根据前一页的定理,其导数 $d\mathbf{T}/ds$ 必然与 $\mathbf{T}$ 垂直。我们定义:

其中:
$$\frac{d\mathbf{T}}{ds} = \kappa \mathbf{N}$$

$\kappa$(Kappa)称为曲线在该点处的曲率(Curvature),它衡量了切线方向随弧长变化的速率。

$\mathbf{N}$ 是与 $\mathbf{T}$ 垂直的单位矢量,称为主法矢量(Principal Normal)。

曲率半径:

曲率的倒数 $\rho = 1/\kappa$ 称为曲率半径。$\kappa = 0$ 意味着曲线在这一点是直线。

第 33 页:副法矢量与密切平面

副法矢量的定义:

我们引入第三个单位矢量 $\mathbf{B}$,定义为切矢量和主法矢量的叉积:$$\mathbf{B} = \mathbf{T} \times \mathbf{N}$$

矢量 $\mathbf{B}$ 称为副法矢量(Binormal)。由定义可知,$\mathbf{T, N, B}$ 构成一个彼此垂直的右手正交标架(称为 Frenet 标架)。

相关的平面:

密切平面(Osculating Plane):由 $\mathbf{T}$ 和 $\mathbf{N}$ 确定的平面(法矢量为 $\mathbf{B}$)。曲线在该平面内有最大的弯曲趋势。

法平面(Normal Plane):由 $\mathbf{N}$ 和 $\mathbf{B}$ 确定的平面(法矢量为 $\mathbf{T}$)。

从切平面(Rectifying Plane):由 $\mathbf{T}$ 和 $\mathbf{B}$ 确定的平面(法矢量为 $\mathbf{N}$)。

第 34 页:挠率与 Serret-Frenet 公式 (1)

挠率的定义:现在考虑副法矢量 $\mathbf{B}$ 随弧长的变化率 $d\mathbf{B}/ds$。可以证明它一定平行于 $\mathbf{N}$。我们定义:

$$\frac{d\mathbf{B}}{ds} = -\tau \mathbf{N}$$

其中标量 $\tau$(Tau)称为曲线的挠率(Torsion)。挠率衡量了曲线脱离其密切平面的程度(即曲线在空间中“扭曲”的程度)。

若 $\tau = 0$,则曲线始终位于同一个平面内(平面曲线)。

Serret-Frenet 公式的前两个:

$d\mathbf{T}/ds = \kappa \mathbf{N}$

$d\mathbf{B}/ds = -\tau \mathbf{N}$

第 35 页:Serret-Frenet 公式 (2)

推导 $d\mathbf{N}/ds$:利用 $\mathbf{N} = \mathbf{B} \times \mathbf{T}$,我们可以通过对乘积求导来推导主法矢量的变化率:

$$\frac{d\mathbf{N}}{ds} = \frac{d\mathbf{B}}{ds} \times \mathbf{T} + \mathbf{B} \times \frac{d\mathbf{T}}{ds}$$

代入已知的关系式:
$$\frac{d\mathbf{N}}{ds} = (-\tau \mathbf{N}) \times \mathbf{T} + \mathbf{B} \times (\kappa \mathbf{N})$$

利用右手定则,最终得到:
$$\frac{d\mathbf{N}}{ds} = \tau \mathbf{B} - \kappa \mathbf{T}$$

总结(Serret-Frenet 公式组):这是一组描述空间曲线几何特性的基本方程:

$$\begin{cases} \frac{d\mathbf{T}}{ds} = \kappa \mathbf{N} \\ \frac{d\mathbf{N}}{ds} = \tau \mathbf{B} - \kappa \mathbf{T} \\ \frac{d\mathbf{B}}{ds} = -\tau \mathbf{N} \end{cases}$$

第 36 页:运动学中的应用:速度与加速度

速度矢量:设一个质点沿曲线运动,其位置矢量为 $\mathbf{r}(t)$,其中 $t$ 代表时间。速度矢量定义为:

$$\mathbf{v} = \frac{d\mathbf{r}}{dt} = \frac{d\mathbf{r}}{ds} \frac{ds}{dt} = v\mathbf{T}$$

这里 $v = ds/dt$ 是质点运动的速率,而 $\mathbf{T}$ 是单位切矢量。这表明速度矢量的方向始终沿曲线的切线方向。

加速度矢量:对速度矢量关于时间 $t$ 再次求导,得到加速度矢量 $\mathbf{a}$:
$$\mathbf{a} = \frac{d\mathbf{v}}{dt} = \frac{d}{dt}(v\mathbf{T}) = \frac{dv}{dt}\mathbf{T} + v\frac{d\mathbf{T}}{dt}$$

利用链式法则 $d\mathbf{T}/dt = (d\mathbf{T}/ds)(ds/dt) = \kappa \mathbf{N} v$,代入上式得:
$$\mathbf{a} = \frac{dv}{dt}\mathbf{T} + \kappa v^2 \mathbf{N}$$

这说明加速度有两个分量:

切向加速度:$a_t = dv/dt$,反映速率的变化。

法向加速度:$a_n = \kappa v^2 = v^2/\rho$,反映运动方向的变化。

第 37 页:标量场与等值面

标量场的定义:

如果在空间区域内的每一个点 $(x, y, z)$,都有一个标量 $\phi(x, y, z)$ 与之对应,则称在该区域内定义了一个标量场。例如:空间中的温度分布、大气压分布或电势。

等值面:

方程 $\phi(x, y, z) = C$(其中 $C$ 为常数)定义了一系列曲面,称为标量场的等值面(如等温面、等势面)。在同一等值面上,标量函数的值保持不变。

第 38 页:梯度算子 (The Gradient)

梯度的定义:

对于标量场 $\phi(x, y, z)$,我们定义其梯度为一个矢量场,记作 $\text{grad} \phi$ 或 $\nabla \phi$(读作 del phi)。在直角坐标系中,其形式为:

$$\nabla \phi = \frac{\partial \phi}{\partial x}\mathbf{i} + \frac{\partial \phi}{\partial y}\mathbf{j} + \frac{\partial \phi}{\partial z}\mathbf{k}$$

算子 $\nabla$:

符号 $\nabla$ 称为哈密顿算子(Hamiltonian Operator)或倒三角算子:

$$\nabla = \mathbf{i}\frac{\partial}{\partial x} + \mathbf{j}\frac{\partial}{\partial y} + \mathbf{k}\frac{\partial}{\partial z}$$

这是一个矢量微分算子,它作用于标量场产生矢量场。

第 39 页:梯度的几何意义 (1)

梯度与等值面的关系:考虑通过点 $P$ 的等值面 $\phi(x, y, z) = C$。设 $\mathbf{r}$ 是该面上一点的位置矢量,则面上的微小位移 $d\mathbf{r}$ 满足全微分方程:

$$d\phi = \frac{\partial \phi}{\partial x}dx + \frac{\partial \phi}{\partial y}dy + \frac{\partial \phi}{\partial z}dz = 0$$

利用点积的形式,这可以写成:
$$(\nabla \phi) \cdot d\mathbf{r} = 0$$

由于 $d\mathbf{r}$ 位于等值面的切平面内,而点积为零意味着垂直。因此:在空间任意一点,梯度矢量 $\nabla \phi$ 始终垂直于过该点的等值面。

第 40 页:梯度的几何意义 (2) 与方向导数

方向导数:

如果我们想知道 $\phi$ 沿任意方向 $\mathbf{u}$(单位矢量)的变化率,这个变化率称为 $\phi$ 沿方向 $\mathbf{u}$ 的方向导数,记作 $d\phi/ds$。根据复合函数求导法则:

$$\frac{d\phi}{ds} = \nabla \phi \cdot \mathbf{u} = |\nabla \phi| \cos \theta$$

其中 $\theta$ 是 $\nabla \phi$ 与 $\mathbf{u}$ 之间的夹角。

最大变化率:

当 $\theta = 0$ 时,即沿梯度的方向,$d\phi/ds$ 取得最大值,其值为 $|\nabla \phi|$。

结论:梯度矢量的方向是标量场增加最快的方向,其模长等于该最大增加率。

第 41 页:梯度的代数性质与恒等式

基本运算法则:设 $\phi$ 和 $\psi$ 是可微的标量场,$c$ 为常数,则梯度运算满足以下代数性质:

线性性质:$\nabla(\phi + \psi) = \nabla\phi + \nabla\psi$ 以及 $\nabla(c\phi) = c\nabla\phi$。

乘积法则:$\nabla(\phi\psi) = \phi\nabla\psi + \psi\nabla\phi$。

商法则:$\nabla\left(\frac{\phi}{\psi}\right) = \frac{\psi\nabla\phi - \phi\nabla\psi}{\psi^2}$ (在 $\psi \neq 0$ 处)。

复合函数梯度:若 $f$ 是标量 $u$ 的函数,而 $u$ 又是坐标的函数 $u(x, y, z)$,则:

$$\nabla f(u) = f'(u) \nabla u$$

例如,若 $r = \sqrt{x^2+y^2+z^2}$ 是到原点的距离,则 $\nabla r = \frac{\mathbf{r}}{r}$(单位径向矢量)。

第 42 页:矢量场的散度 (The Divergence)

定义:设 $\mathbf{V}(x, y, z) = V_x\mathbf{i} + V_y\mathbf{j} + V_z\mathbf{k}$ 是一个可微的矢量场。$\mathbf{V}$ 的散度定义为一个标量场,记作 $\text{div} \mathbf{V}$ 或 $\nabla \cdot \mathbf{V}$:

$$\nabla \cdot \mathbf{V} = \left(\mathbf{i}\frac{\partial}{\partial x} + \mathbf{j}\frac{\partial}{\partial y} + \mathbf{k}\frac{\partial}{\partial z}\right) \cdot (V_x\mathbf{i} + V_y\mathbf{j} + V_z\mathbf{k})$$

展开得:
$$\nabla \cdot \mathbf{V} = \frac{\partial V_x}{\partial x} + \frac{\partial V_y}{\partial y} + \frac{\partial V_z}{\partial z}$$

物理意义:在流体动力学中,若 $\mathbf{V}$ 代表流体的速度,那么 $\nabla \cdot \mathbf{V}$ 表示单位时间内从单位体积元中流出的流体净通量。

若 $\nabla \cdot \mathbf{V} > 0$,该点存在“源”(Source)。

若 $\nabla \cdot \mathbf{V} < 0$,该点存在“汇”(Sink)。

若 $\nabla \cdot \mathbf{V} = 0$,则称该场为无散场或螺线场(Solenoidal Field)。

第 43 页:矢量场的旋度 (The Curl)

定义:矢量场 $\mathbf{V}$ 的旋度定义为一个新的矢量场,记作 $\text{curl} \mathbf{V}$ 或 $\nabla \times \mathbf{V}$。它可以通过算子 $\nabla$ 与 $\mathbf{V}$ 的叉积得到:

$$\nabla \times \mathbf{V} = \begin{vmatrix} \mathbf{i} & \mathbf{j} & \mathbf{k} \\ \frac{\partial}{\partial x} & \frac{\partial}{\partial y} & \frac{\partial}{\partial z} \\ V_x & V_y & V_z \end{vmatrix}$$

展开分量形式为:
$$\nabla \times \mathbf{V} = \left( \frac{\partial V_z}{\partial y} - \frac{\partial V_y}{\partial z} \right)\mathbf{i} + \left( \frac{\partial V_x}{\partial z} - \frac{\partial V_z}{\partial x} \right)\mathbf{j} + \left( \frac{\partial V_y}{\partial x} - \frac{\partial V_x}{\partial y} \right)\mathbf{k}$$

物理意义:

旋度描述了矢量场在某点附近的旋转强度和方向。例如,在流体中,旋度代表了流体微团的局部角速度。若 $\nabla \times \mathbf{V} = \mathbf{0}$,则称该场为无旋场(Irrotational Field)。

第 44 页:包含 $\nabla$ 的重要组合恒等式

当 $\nabla$ 算子与多个场结合时,会产生一些极其重要的二阶恒等式。设 $\phi$ 为标量场,$\mathbf{V}$ 为矢量场:

标量场梯度的旋度恒为零:
$$\nabla \times (\nabla \phi) = \mathbf{0}$$

(这意味着任何梯度场都是无旋场。)

矢量场旋度的散度恒为零:
$$\nabla \cdot (\nabla \times \mathbf{V}) = 0$$

(这意味着任何旋度场都是无散场。)

乘积的散度:
$$\nabla \cdot (\phi \mathbf{V}) = \phi (\nabla \cdot \mathbf{V}) + (\nabla \phi) \cdot \mathbf{V}$$

第 45 页:拉普拉斯算子 (The Laplacian)

定义:标量场 $\phi$ 梯度的散度称为 $\phi$ 的拉普拉斯运算,记作 $\nabla^2 \phi$ 或 $\Delta \phi$:

$$\nabla^2 \phi = \nabla \cdot (\nabla \phi) = \frac{\partial^2 \phi}{\partial x^2} + \frac{\partial^2 \phi}{\partial y^2} + \frac{\partial^2 \phi}{\partial z^2}$$

这是一个二阶偏微分算子,广泛应用于波动方程、热传导方程以及静电势分析(泊松方程和拉普拉斯方程)。

矢量拉普拉斯算子:对于矢量场 $\mathbf{V}$,其拉普拉斯运算定义为对其每个分量分别进行拉普拉斯运算:
$$\nabla^2 \mathbf{V} = (\nabla^2 V_x)\mathbf{i} + (\nabla^2 V_y)\mathbf{j} + (\nabla^2 V_z)\mathbf{k}$$

此外,它与梯度、散度、旋度之间存在著名的恒等式:
$$\nabla \times (\nabla \times \mathbf{V}) = \nabla (\nabla \cdot \mathbf{V}) - \nabla^2 \mathbf{V}$$

第 46 页:正交曲线坐标系 (Orthogonal Curvilinear Coordinates)

变换定义:在直角坐标系 $(x, y, z)$ 之外,我们可以引入新的坐标变量 $(u_1, u_2, u_3)$,它们与直角坐标的关系由下列方程确定:

$$x = x(u_1, u_2, u_3), \quad y = y(u_1, u_2, u_3), \quad z = z(u_1, u_2, u_3)$$

若三组坐标面(如 $u_1 = \text{常数}$ 等)在每一点都彼此垂直,则称该系统为正交曲线坐标系。

比例因子 (Scale Factors):位置矢量 $\mathbf{r}$ 的微分可以表示为:

$$d\mathbf{r} = \frac{\partial \mathbf{r}}{\partial u_1}du_1 + \frac{\partial \mathbf{r}}{\partial u_2}du_2 + \frac{\partial \mathbf{r}}{\partial u_3}du_3$$

我们定义比例因子 $h_i$ 为:
$$h_i = \left| \frac{\partial \mathbf{r}}{\partial u_i} \right|$$

于是,弧长的平方(第一基本形式)表示为:
$$ds^2 = d\mathbf{r} \cdot d\mathbf{r} = h_1^2 du_1^2 + h_2^2 du_2^2 + h_3^2 du_3^2$$

第 47 页:曲线坐标系中的单位矢量

局部正交基:

在正交曲线坐标系中的每一点,我们定义一组单位矢量 $\mathbf{e}_1, \mathbf{e}_2, \mathbf{e}_3$:

$$\mathbf{e}_1 = \frac{1}{h_1}\frac{\partial \mathbf{r}}{\partial u_1}, \quad \mathbf{e}_2 = \frac{1}{h_2}\frac{\partial \mathbf{r}}{\partial u_2}, \quad \mathbf{e}_3 = \frac{1}{h_3}\frac{\partial \mathbf{r}}{\partial u_3}$$

这些单位矢量彼此正交,并随点的位置改变而改变方向。任何矢量 $\mathbf{A}$ 都可以表示为:
$$\mathbf{A} = A_1\mathbf{e}_1 + A_2\mathbf{e}_2 + A_3\mathbf{e}_3$$

常见的比例因子:

柱坐标 $(r, \theta, z)$:$h_r = 1, h_\theta = r, h_z = 1$。

球坐标 $(r, \theta, \phi)$:$h_r = 1, h_\theta = r, h_\phi = r \sin \theta$。

第 48 页:曲线坐标系中的梯度 (Gradient)

梯度的一般形式:标量场 $\phi(u_1, u_2, u_3)$ 的全微分可以写成:
$$d\phi = \frac{\partial \phi}{\partial u_1}du_1 + \frac{\partial \phi}{\partial u_2}du_2 + \frac{\partial \phi}{\partial u_3}du_3$$

同时根据梯度的定义 $d\phi = \nabla \phi \cdot d\mathbf{r}$,通过对比可以得到梯度在一般正交曲线坐标系下的表达式:
$$\nabla \phi = \frac{1}{h_1}\frac{\partial \phi}{\partial u_1}\mathbf{e}_1 + \frac{1}{h_2}\frac{\partial \phi}{\partial u_2}\mathbf{e}_2 + \frac{1}{h_3}\frac{\partial \phi}{\partial u_3}\mathbf{e}_3$$

示例(柱坐标):
$$\nabla \phi = \frac{\partial \phi}{\partial r}\mathbf{e}_r + \frac{1}{r}\frac{\partial \phi}{\partial \theta}\mathbf{e}_\theta + \frac{\partial \phi}{\partial z}\mathbf{e}_z$$

第 49 页:曲线坐标系中的散度 (Divergence)

散度的一般形式:利用体积元在变换下的性质,可以推导出矢量场 $\mathbf{A} = A_1\mathbf{e}_1 + A_2\mathbf{e}_2 + A_3\mathbf{e}_3$ 的散度公式:

$$\nabla \cdot \mathbf{A} = \frac{1}{h_1 h_2 h_3} \left[ \frac{\partial}{\partial u_1}(h_2 h_3 A_1) + \frac{\partial}{\partial u_2}(h_3 h_1 A_2) + \frac{\partial}{\partial u_3}(h_1 h_2 A_3) \right]$$

物理含义:由于比例因子的存在,散度不仅包含了分量本身的变化率,还包含了由于坐标线“发散”或“汇聚”而导致的额外项。例如在球坐标中,随着半径 $r$ 增大,单位 $dr$ 所对应的截面积也在增大。

第 50 页:曲线坐标系中的旋度与拉普拉斯算子

旋度的一般形式:旋度 $\nabla \times \mathbf{A}$ 的行列式表示形式在一般正交坐标系下修正为:

$$\nabla \times \mathbf{A} = \frac{1}{h_1 h_2 h_3} \begin{vmatrix} h_1\mathbf{e}_1 & h_2\mathbf{e}_2 & h_3\mathbf{e}_3 \\ \frac{\partial}{\partial u_1} & \frac{\partial}{\partial u_2} & \frac{\partial}{\partial u_3} \\ h_1 A_1 & h_2 A_2 & h_3 A_3 \end{vmatrix}$$

拉普拉斯算子的一般形式:结合梯度和散度的公式,$\nabla^2 \phi = \nabla \cdot (\nabla \phi)$ 得到:
$$\nabla^2 \phi = \frac{1}{h_1 h_2 h_3} \left[ \frac{\partial}{\partial u_1}\left( \frac{h_2 h_3}{h_1} \frac{\partial \phi}{\partial u_1} \right) + \frac{\partial}{\partial u_2}\left( \frac{h_3 h_1}{h_2} \frac{\partial \phi}{\partial u_2} \right) + \frac{\partial}{\partial u_3}\left( \frac{h_1 h_2}{h_3} \frac{\partial \phi}{\partial u_3} \right) \right]$$

