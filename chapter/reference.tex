\chapter{矢量与张量分析}


\section{矢量代数}

1.1 矢量的定义 (Definition of a Vector)
在物理学和工程学中,我们经常遇到两类量:标量 (Scalars) 和 矢量 (Vectors)。

标量: 仅由其大小(带正号或负号)确定的量。例如:质量、时间、温度、密度、功和能量。标量遵循普通代数的运算法则。

矢量: 既有大小又有方向的量。例如:位移、速度、加速度、力和电场强度。

在几何上,我们用一个带箭头的线段来表示矢量。线段的长度代表矢量的大小,箭头的指向代表矢量的方向。

1.2 矢量的表示法 (Notation)

在本书中,我们将用粗体字母(如 $\mathbf{A}, \mathbf{B}, \mathbf{a}, \mathbf{b}$)表示矢量。标量则用斜体字母表示。矢量 $\mathbf{A}$ 的大小(或长度)记作 $|\mathbf{A}|$ 或简单的 $A$。

单位矢量 (Unit Vector): 大小为 1 的矢量称为单位矢量。

零矢量 (Zero Vector): 大小为 0 的矢量称为零矢量,记作 $\mathbf{0}$。它的方向是不确定的。

1.3 矢量的加法 (Addition of Vectors)

两个矢量 $\mathbf{A}$ 和 $\mathbf{B}$ 的和 $\mathbf{C} = \mathbf{A} + \mathbf{B}$ 可以通过平行四边形法则或三角形法则来定义。

三角形法则: 将矢量 $\mathbf{B}$ 的起点放在矢量 $\mathbf{A}$ 的终点,那么从 $\mathbf{A}$ 的起点指向 $\mathbf{B}$ 的终点的矢量就是 $\mathbf{A} + \mathbf{B}$。

法则属性:

交换律: $\mathbf{A} + \mathbf{B} = \mathbf{B} + \mathbf{A}$

结合律: $(\mathbf{A} + \mathbf{B}) + \mathbf{C} = \mathbf{A} + (\mathbf{B} + \mathbf{C})$

1.4 矢量的减法 (Subtraction of Vectors)

如果 $\mathbf{B}$ 是一个矢量,那么 $-\mathbf{B}$ 是一个与 $\mathbf{B}$ 大小相等但方向相反的矢量。两个矢量的差定义为:
$$\mathbf{A} - \mathbf{B} = \mathbf{A} + (-\mathbf{B})$$

在几何上,如果 $\mathbf{A}$ 和 $\mathbf{B}$ 从同一点出发,那么 $\mathbf{A} - \mathbf{B}$ 是从 $\mathbf{B}$ 的终点指向 $\mathbf{A}$ 的终点的矢量。

1.5 标量与矢量的乘法 (Multiplication of a Vector by a Scalar)

标量 $m$ 与矢量 $\mathbf{A}$ 的乘积记作 $m\mathbf{A}$。

如果 $m > 0$,则 $m\mathbf{A}$ 的方向与 $\mathbf{A}$ 相同,大小为 $m|\mathbf{A}|$。

如果 $m < 0$,则 $m\mathbf{A}$ 的方向与 $\mathbf{A}$ 相反,大小为 $|m||\mathbf{A}|$。

如果 $m = 0$,则结果为零矢量 $\mathbf{0}$。

该运算满足分配律:
$$m(\mathbf{A} + \mathbf{B}) = m\mathbf{A} + m\mathbf{B}$$
$$(m + n)\mathbf{A} = m\mathbf{A} + n\mathbf{A}$$

1.6 共线与共面矢量 (Collinear and Coplanar Vectors)

共线: 如果两个矢量平行于同一条直线,则称它们为共线矢量。如果 $\mathbf{A}$ 与 $\mathbf{B}$ 共线,则存在标量 $k$ 使得 $\mathbf{A} = k\mathbf{B}$。

共面: 如果三个或多个矢量平行于同一个平面,则称它们为共面矢量。

第 6 页:矢量的正交分解 (Orthogonal Components of a Vector)

设 $\mathbf{i, j, k}$ 为沿直角笛卡尔坐标系 $x, y, z$ 轴正方向的一组单位矢量。任何矢量 $\mathbf{A}$ 都可以表示为这些单位矢量的线性组合。
$$\mathbf{A} = A_x \mathbf{i} + A_y \mathbf{j} + A_z \mathbf{k}$$

标量 $A_x, A_y, A_z$ 称为 $\mathbf{A}$ 的分量。$\mathbf{A}$ 的模(大小)由下式给出:

$$A = |\mathbf{A}| = \sqrt{A_x^2 + A_y^2 + A_z^2}$$。

第 7 页:方向余弦 (Direction Cosines)

若 $\alpha, \beta, \gamma$ 分别是 $\mathbf{A}$ 与 $x, y, z$ 轴正方向的夹角,则有:

$$A_x = A \cos \alpha, \quad A_y = A \cos \beta, \quad A_z = A \cos \gamma$$。

这些被称为 $\mathbf{A}$ 的方向余弦。由此可知:
$$\cos^2 \alpha + \cos^2 \beta + \cos^2 \gamma = 1$$。

第 8 页:点积/数量积 (The Dot or Scalar Product)

两个矢量 $\mathbf{A}$ 与 $\mathbf{B}$ 的点积(或称数量积),记作 $\mathbf{A \cdot B}$,定义为它们的模与它们之间夹角 $\theta$ 的余弦之积。

$$\mathbf{A \cdot B} = AB \cos \theta, \quad (0 \le \theta \le \pi)$$

根据该定义,显而易见:

$\mathbf{A \cdot B} = \mathbf{B \cdot A}$ (交换律)

$\mathbf{A \cdot A} = A^2$

若 $\mathbf{A \cdot B} = 0$ 且 $\mathbf{A, B} \neq 0$,则 $\mathbf{A}$ 垂直于 $\mathbf{B}$。

第 9 页:分量形式的点积与分配律

点积满足分配律 $\mathbf{A \cdot (B + C)} = \mathbf{A \cdot B} + \mathbf{A \cdot C}$。这可以通过观察发现:$\mathbf{A \cdot B}$ 等于 $A$ 与 $\mathbf{B}$ 在 $\mathbf{A}$ 方向上的投影的乘积。

用分量表示时,由于 $\mathbf{i \cdot i} = \mathbf{j \cdot j} = \mathbf{k \cdot k} = 1$ 且 $\mathbf{i \cdot j} = \mathbf{j \cdot k} = \mathbf{k \cdot i} = 0$,我们得到:$$\mathbf{A \cdot B} = (A_x \mathbf{i} + A_y \mathbf{j} + A_z \mathbf{k}) \cdot (B_x \mathbf{i} + B_y \mathbf{j} + B_z \mathbf{k}) = A_x B_x + A_y B_y + A_z B_z$$

第 10 页:点积的应用:余弦定理与投影

利用点积可以轻松推导出三角形的余弦定理。设 $\mathbf{C = A - B}$,则:

$$\mathbf{C \cdot C} = (\mathbf{A - B}) \cdot (\mathbf{A - B}) = \mathbf{A \cdot A} + \mathbf{B \cdot B} - 2\mathbf{A \cdot B}$$
$$C^2 = A^2 + B^2 - 2AB \cos \theta$$

第 11 页:矢量积或叉积 (The Vector or Cross Product)

两个矢量 $\mathbf{A}$ 与 $\mathbf{B}$ 的矢量积(又称叉积),记作 $\mathbf{A} \times \mathbf{B}$,其结果是一个矢量 $\mathbf{C}$。其模长定义为 $C = AB \sin \theta$,其中 $\theta$ 是 $\mathbf{A}$ 与 $\mathbf{B}$ 之间的夹角($0 \le \theta \le \pi$)。

矢量 $\mathbf{C} = \mathbf{A} \times \mathbf{B}$ 的方向垂直于 $\mathbf{A}$ 和 $\mathbf{B}$ 所确定的平面,且 $\mathbf{A, B, C}$ 构成一个右手系。这意味着,如果右手四指从 $\mathbf{A}$ 经较小夹角 $\theta$ 弯向 $\mathbf{B}$,则大拇指所指的方向即为 $\mathbf{C}$ 的方向。

第 12 页:叉积的几何性质 (Geometric Properties of the Cross Product)

根据定义可知 $\mathbf{A} \times \mathbf{B} = -(\mathbf{B} \times \mathbf{A})$。因此,交换律不适用于矢量积。

若 $\mathbf{A} \times \mathbf{B} = \mathbf{0}$ 且 $\mathbf{A, B} \neq \mathbf{0}$,则 $\sin \theta = 0$,这意味着 $\mathbf{A}$ 与 $\mathbf{B}$ 平行或共线。

模长 $AB \sin \theta$ 在几何上表示以 $\mathbf{A}$ 和 $\mathbf{B}$ 为邻边的平行四边形的面积。

第 13 页:单位矢量的叉积 (Cross Product of Unit Vectors)

对于右手笛卡尔坐标系中的基本单位矢量 $\mathbf{i, j, k}$,我们有:

$\mathbf{i} \times \mathbf{i} = \mathbf{j} \times \mathbf{j} = \mathbf{k} \times \mathbf{k} = \mathbf{0}$


此外:

$\mathbf{i} \times \mathbf{j} = \mathbf{k}, \quad \mathbf{j} \times \mathbf{k} = \mathbf{i}, \quad \mathbf{k} \times \mathbf{i} = \mathbf{j}$。
$\mathbf{j} \times \mathbf{i} = -\mathbf{k}, \quad \mathbf{k} \times \mathbf{j} = -\mathbf{i}, \quad \mathbf{i} \times \mathbf{k} = -\mathbf{j}$。

第 14 页:叉积的分量形式 (Cross Product in Component Form)

可以证明分配律 $\mathbf{A} \times (\mathbf{B} + \mathbf{C}) = \mathbf{A} \times \mathbf{B} + \mathbf{A} \times \mathbf{C}$ 是成立的。利用该定律,我们可以用分量表示 $\mathbf{A} \times \mathbf{B}$:
$$\mathbf{A} \times \mathbf{B} = (A_y B_z - A_z B_y)\mathbf{i} + (A_z B_x - A_x B_z)\mathbf{j} + (A_x B_y - A_y B_x)\mathbf{k}$$

这一结果最容易通过行列式的形式来记忆:
$$\mathbf{A} \times \mathbf{B} = \begin{vmatrix} \mathbf{i} & \mathbf{j} & \mathbf{k} \\ A_x & A_y & A_z \\ B_x & B_y & B_z \end{vmatrix}$$

第 15 页:标量三重积 (The Scalar Triple Product)

乘积 $\mathbf{A} \cdot (\mathbf{B} \times \mathbf{C})$ 被称为标量三重积(又称混合积)。其结果是一个标量。

在几何上,$\mathbf{A} \cdot (\mathbf{B} \times \mathbf{C})$ 的绝对值表示以 $\mathbf{A, B, C}$ 为共点棱的平行六面体的体积。

用行列式形式表示为:
$$\mathbf{A} \cdot (\mathbf{B} \times \mathbf{C}) = \begin{vmatrix} A_x & A_y & A_z \\ B_x & B_y & B_z \\ C_x & C_y & C_z \end{vmatrix}$$

第 16 页:矢量三重积 (The Vector Triple Product)

乘积 $\mathbf{A} \times (\mathbf{B} \times \mathbf{C})$ 被称为矢量三重积。与标量三重积不同,其结果是一个矢量。由于 $\mathbf{B} \times \mathbf{C}$ 垂直于 $\mathbf{B}$ 和 $\mathbf{C}$ 构成的平面,因此矢量 $\mathbf{A} \times (\mathbf{B} \times \mathbf{C})$ 必然位于 $\mathbf{B}$ 和 $\mathbf{C}$ 所在的平面内。

下述重要的展开公式成立:
$$\mathbf{A} \times (\mathbf{B} \times \mathbf{C}) = (\mathbf{A} \cdot \mathbf{C})\mathbf{B} - (\mathbf{A} \cdot \mathbf{B})\mathbf{C}$$

这通常被称为“BAC-CAB”法则。请注意,一般情况下,$\mathbf{A} \times (\mathbf{B} \times \mathbf{C}) \neq (\mathbf{A} \times \mathbf{B}) \times \mathbf{C}$。

第 17 页:标量三重积的循环特性 (Cyclic Permutations)

对于标量三重积 $\mathbf{A} \cdot (\mathbf{B} \times \mathbf{C})$,点乘和叉乘符号可以互换而不改变其值:

$\mathbf{A} \cdot (\mathbf{B} \times \mathbf{C}) = (\mathbf{A} \times \mathbf{B}) \cdot \mathbf{C}$

此外,在矢量的循环轮换下,其值保持不变:

$[\mathbf{A, B, C}] = \mathbf{A} \cdot (\mathbf{B} \times \mathbf{C}) = \mathbf{B} \cdot (\mathbf{C} \times \mathbf{A}) = \mathbf{C} \cdot (\mathbf{A} \times \mathbf{B})$

如果顺序非循环,则符号改变:

$\mathbf{A} \cdot (\mathbf{B} \times \mathbf{C}) = -\mathbf{A} \cdot (\mathbf{C} \times \mathbf{B})$

第 18 页:四重积恒等式 (Quadruple Products)

利用前面的公式,我们可以推导出涉及四个矢量的恒等式。四个矢量的数量积:

$(\mathbf{A} \times \mathbf{B}) \cdot (\mathbf{C} \times \mathbf{D}) = (\mathbf{A} \cdot \mathbf{C})(\mathbf{B} \cdot \mathbf{D}) - (\mathbf{A} \cdot \mathbf{D})(\mathbf{B} \cdot \mathbf{C})$

四个矢量的向量积:

$(\mathbf{A} \times \mathbf{B}) \times (\mathbf{C} \times \mathbf{D}) = [\mathbf{A, C, D}]\mathbf{B} - [\mathbf{B, C, D}]\mathbf{A}$

这表明所得矢量既位于 $\mathbf{A}$ 和 $\mathbf{B}$ 确定的平面内,也位于 $\mathbf{C}$ 和 $\mathbf{D}$ 确定的平面内。

第 19 页:矢量方程 (Vector Equations)

考虑方程 $\mathbf{A} \cdot \mathbf{X} = p$,其中 $\mathbf{A}$ 和 $p$ 为已知。该方程不能唯一确定 $\mathbf{X}$。在几何上,它表示一个垂直于 $\mathbf{A}$ 的平面,其到原点的距离为 $p/|\mathbf{A}|$。

现在考虑 $\mathbf{A} \times \mathbf{X} = \mathbf{B}$。为了使解存在,$\mathbf{A}$ 必须垂直于 $\mathbf{B}$(即 $\mathbf{A} \cdot \mathbf{B} = 0$)。其通解为:
$$\mathbf{X} = \frac{\mathbf{B} \times \mathbf{A}}{A^2} + \lambda \mathbf{A}$$
其中 $\lambda$ 是任意标量。

第 20 页:倒易矢量系 (Reciprocal System of Vectors)

如果两组矢量 $\mathbf{a, b, c}$ 和 $\mathbf{a', b', c'}$ 满足以下条件,则称它们为倒易系:
$\mathbf{a} \cdot \mathbf{a'} = \mathbf{b} \cdot \mathbf{b'} = \mathbf{c} \cdot \mathbf{c'} = 1$
$\mathbf{a} \cdot \mathbf{b'} = \mathbf{a} \cdot \mathbf{c'} = \dots = 0$

倒易矢量可以按如下方式构造:
$$\mathbf{a'} = \frac{\mathbf{b} \times \mathbf{c}}{[\mathbf{a, b, c}]}, \quad \mathbf{b'} = \frac{\mathbf{c} \times \mathbf{a}}{[\mathbf{a, b, c}]}, \quad \mathbf{c'} = \frac{\mathbf{a} \times \mathbf{b}}{[\mathbf{a, b, c}]}$$

前提是 $[\mathbf{a, b, c}] \neq 0$。

第 21 页:直线的矢量方程

直线的表示:空间中的一条直线可以通过一个已知点 $A$(位置矢量为 $\mathbf{a}$)以及直线所平行的方向矢量 $\mathbf{b}$ 来唯一确定。设 $P$ 是直线上任意一点,其位置矢量为 $\mathbf{r}$。

则矢量 $\vec{AP} = \mathbf{r} - \mathbf{a}$ 必定与 $\mathbf{b}$ 平行。因此,存在一个标量参数 $t$,使得:
$$\mathbf{r} - \mathbf{a} = t\mathbf{b}$$

或者写作:
$$\mathbf{r} = \mathbf{a} + t\mathbf{b}$$

这就是直线的参数矢量方程。

另一种形式:由于 $\mathbf{r} - \mathbf{a}$ 与 $\mathbf{b}$ 平行,它们的叉积必须为零:
$$(\mathbf{r} - \mathbf{a}) \times \mathbf{b} = \mathbf{0}$$

第 22 页:两点确定的直线与距离公式

两点式方程:若直线通过两个已知点 $A(\mathbf{a})$ 和 $B(\mathbf{b})$,则直线的方向矢量可以取为 $\mathbf{b} - \mathbf{a}$。此时方程变为:
$$\mathbf{r} = \mathbf{a} + t(\mathbf{b} - \mathbf{a})$$

或
$$\mathbf{r} = (1 - t)\mathbf{a} + t\mathbf{b}$$

点到直线的距离:

设已知点 $Q$ 的位置矢量为 $\mathbf{q}$,直线方程为 $\mathbf{r} = \mathbf{a} + t\mathbf{b}$。点 $Q$ 到该直线的垂直距离 $d$ 由下式给出:

$$d = \frac{|(\mathbf{q} - \mathbf{a}) \times \mathbf{b}|}{|\mathbf{b}|}$$

这里 $(\mathbf{q} - \mathbf{a}) \times \mathbf{b}$ 的模表示以 $\mathbf{q} - \mathbf{a}$ 和 $\mathbf{b}$ 为邻边的平行四边形的面积,除以底边长 $|\mathbf{b}|$ 即得高 $d$。

第 23 页:平面的矢量方程

点法式方程:一个平面可以通过平面内的一点 $A(\mathbf{a})$ 和一个垂直于平面的法矢量 $\mathbf{n}$ 来确定。若 $P(\mathbf{r})$ 是平面上的任意一点,则矢量 $\mathbf{r} - \mathbf{a}$ 必然位于平面内,因此与 $\mathbf{n}$ 垂直。其点积必为零:

$$(\mathbf{r} - \mathbf{a}) \cdot \mathbf{n} = 0$$

或者
$$\mathbf{r} \cdot \mathbf{n} = \mathbf{a} \cdot \mathbf{n} = p$$

其中 $p$ 是一个常数。

三点定平面:

若平面通过不共线的三点 $A(\mathbf{a}), B(\mathbf{b}), C(\mathbf{c})$,则平面上任意一点 $P(\mathbf{r})$ 满足矢量 $\mathbf{r} - \mathbf{a}, \mathbf{b} - \mathbf{a}, \mathbf{c} - \mathbf{a}$ 共面。因此它们的标量三重积为零:

$$(\mathbf{r} - \mathbf{a}) \cdot [(\mathbf{b} - \mathbf{a}) \times (\mathbf{c} - \mathbf{a})] = 0$$

第 24 页:点到平面的距离与两平面夹角

点到平面的距离:点 $Q(\mathbf{q})$ 到平面 $\mathbf{r} \cdot \mathbf{n} = p$ 的垂直距离 $D$ 是 $\mathbf{q} - \mathbf{r}$ 在法方向上的投影长度。如果 $\mathbf{n}$ 是单位法矢量,则:

$$D = | \mathbf{q} \cdot \mathbf{n} - p |$$

若 $\mathbf{n}$ 不是单位矢量,则需除以其模长。

两平面的夹角:两个平面之间的夹角定义为它们法矢量 $\mathbf{n}_1$ 和 $\mathbf{n}_2$ 之间的夹角 $\theta$:
$$\cos \theta = \frac{\mathbf{n}_1 \cdot \mathbf{n}_2}{|\mathbf{n}_1| |\mathbf{n}_2|}$$

两平面的交线:

两个平面的交线方向与两个法矢量的叉积 $\mathbf{n}_1 \times \mathbf{n}_2$ 平行。

第 25 页:球体与圆的矢量表示

球体方程:

设球心位置矢量为 $\mathbf{c}$,半径为 $a$。球面上任意一点 $P(\mathbf{r})$ 到球心的距离恒为 $a$:

$$|\mathbf{r} - \mathbf{c}| = a$$平方可得:$$(\mathbf{r} - \mathbf{c}) \cdot (\mathbf{r} - \mathbf{c}) = a^2$$

或者
$$r^2 - 2\mathbf{r} \cdot \mathbf{c} + c^2 = a^2$$

切平面方程:

球面上一点 $P_0(\mathbf{r}_0)$ 处的切平面垂直于半径矢量 $\mathbf{r}_0 - \mathbf{c}$。因此,切平面的方程为:
$$(\mathbf{r} - \mathbf{r}_0) \cdot (\mathbf{r}_0 - \mathbf{c}) = 0$$


