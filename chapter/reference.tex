\chapter{空间曲线}


\section{切向量 $\mathbf{T}$、法向量 $\mathbf{N}$ 和副法向量 $\mathbf{B}$}

在微分几何中,Frenet-Serret 标架($\mathbf{T}$, $\mathbf{N}$, $\mathbf{B}$)是描述曲线局部特征的核心。

当曲线按弧长参数化时,切向量的模长恒为 1,这简化了所有的导数关系。

1. 切向量 $\mathbf{T}$ 的推导

定义曲线为 $\mathbf{r}(s)$。切向量定义为位置对弧长的变化率:
$$\mathbf{T} = \frac{d\mathbf{r}}{ds}$$

由于 $s$ 是弧长,根据定义 $\|\mathbf{T}\| = 1$。

2. 主法向量 $\mathbf{N}$ 与曲率 $\kappa$

由于 $\mathbf{T}$ 是单位向量,其模长平方为常数:$\mathbf{T} \cdot \mathbf{T} = 1$。两边对 $s$ 求导:
$$\frac{d\mathbf{T}}{ds} \cdot \mathbf{T} + \mathbf{T} \cdot \frac{d\mathbf{T}}{ds} = 0 \implies 2\mathbf{T} \cdot \frac{d\mathbf{T}}{ds} = 0$$

这证明了导向量 $d\mathbf{T}/ds$ 始终垂直于 $\mathbf{T}$。

曲率 $\kappa$:定义为切线方向随弧长变化的速率,即 $\kappa = \|\frac{d\mathbf{T}}{ds}\|$。

主法向量 $\mathbf{N}$:定义为 $d\mathbf{T}/ds$ 方向上的单位向量:
$$\frac{d\mathbf{T}}{ds} = \kappa \mathbf{N}$$

3. 副法向量 $\mathbf{B}$ 与挠率 $\tau$

为了构成右手正交标架,我们定义副法向量为:
$$\mathbf{B} = \mathbf{T} \times \mathbf{N}$$

挠率 $\tau$ 的得出:

我们要考察 $\mathbf{B}$ 随弧长的变化。对 $\mathbf{B}$ 求导:
$$\frac{d\mathbf{B}}{ds} = \frac{d\mathbf{T}}{ds} \times \mathbf{N} + \mathbf{T} \times \frac{d\mathbf{N}}{ds}$$

由于 $d\mathbf{T}/ds = \kappa \mathbf{N}$,而 $\mathbf{N} \times \mathbf{N} = 0$,第一项消失:
$$\frac{d\mathbf{B}}{ds} = \mathbf{T} \times \frac{d\mathbf{N}}{ds}$$

这意味着 $d\mathbf{B}/ds$ 垂直于 $\mathbf{T}$。同时,由于 $\mathbf{B}$ 是单位向量,$d\mathbf{B}/ds$ 也垂直于 $\mathbf{B}$。既然它同时垂直于 $\mathbf{T}$ 和 $\mathbf{B}$,它必须在 $\mathbf{N}$ 的方向上。

因此,我们定义:
$$\frac{d\mathbf{B}}{ds} = -\tau \mathbf{N}$$

这里的 $\tau$ 称为挠率(Torsion),负号是几何学上的约定,表示当 $\tau > 0$ 时,曲线随 $s$ 增加向副法向量定义的右手螺旋方向扭转。

总结

$\mathbf{T}$ 描述前进方向。

$\mathbf{N}$ 描述向哪弯曲,曲率 $\kappa$ 是弯曲程度(偏离直线的程度)。

$\mathbf{B}$ 描述运动平面的法线,挠率 $\tau$ 是扭曲程度(偏离平面的程度)。对于平面曲线(如 $y=x^2$),$\mathbf{r}'''$ 依然在 $xy$ 平面,外积后与 $z$ 轴垂直,因此 $\tau$ 恒等于 0。

曲率 $\kappa$ 的求解公式

曲率 $\kappa$ 的基本定义是切向量对弧长的变化率的模:
$$\kappa = \left\| \frac{d\mathbf{T}}{ds} \right\|$$

由于我们通常使用的是参数 $t$(如时间),根据链式法则:
$$\frac{d\mathbf{T}}{dt} = \frac{d\mathbf{T}}{ds} \cdot \frac{ds}{dt} = \frac{d\mathbf{T}}{ds} \cdot \|\mathbf{r}'(t)\|$$

因此有:
$$\kappa = \frac{\|\mathbf{T}'(t)\|}{\|\mathbf{r}'(t)\|}$$

我们知道切向量 $\mathbf{T} = \frac{\mathbf{r}'}{\|\mathbf{r}'\|}$,所以速度向量可以写为:

$$\mathbf{r}' = \|\mathbf{r}'\| \mathbf{T}$$

对上式两边关于 $t$ 求导(使用乘法法则):
$$\mathbf{r}'' = (\|\mathbf{r}'\|)' \mathbf{T} + \|\mathbf{r}'\| \mathbf{T}'$$

利用外积(叉乘)消项

现在我们将 $\mathbf{r}'$ 和 $\mathbf{r}''$ 做外积:
$$\mathbf{r}' \times \mathbf{r}'' = (\|\mathbf{r}'\| \mathbf{T}) \times ((\|\mathbf{r}'\|)' \mathbf{T} + \|\mathbf{r}'\| \mathbf{T}')$$

利用外积的分配律:
$$\mathbf{r}' \times \mathbf{r}'' = \|\mathbf{r}'\| (\|\mathbf{r}'\|)' (\mathbf{T} \times \mathbf{T}) + \|\mathbf{r}'\|^2 (\mathbf{T} \times \mathbf{T}')$$

因为任何向量与自身的外积为零 ($\mathbf{T} \times \mathbf{T} = 0$),所以:
$$\mathbf{r}' \times \mathbf{r}'' = \|\mathbf{r}'\|^2 (\mathbf{T} \times \mathbf{T}')$$

取两边的模长:
$$\|\mathbf{r}' \times \mathbf{r}''\| = \|\mathbf{r}'\|^2 \cdot \|\mathbf{T} \times \mathbf{T}'\|$$

因为 $\mathbf{T}$ 是单位向量,且我们已知 $\mathbf{T}'$ 垂直于 $\mathbf{T}$(见前文推导),所以 $\|\mathbf{T} \times \mathbf{T}'\| = \|\mathbf{T}\| \|\mathbf{T}'\| \sin(90^\circ) = \|\mathbf{T}'\|$。代入上式:

$$\|\mathbf{r}' \times \mathbf{r}''\| = \|\mathbf{r}'\|^2 \|\mathbf{T}'\|$$

解出 $\|\mathbf{T}'\|$:
$$\|\mathbf{T}'\| = \frac{\|\mathbf{r}' \times \mathbf{r}''\|}{\|\mathbf{r}'\|^2}$$

最后,将这个结果代回最初的曲率定义:
$$\kappa = \frac{\|\mathbf{T}'\|}{\|\mathbf{r}'\|} = \frac{\|\mathbf{r}' \times \mathbf{r}''\|}{\|\mathbf{r}'\|^2 \cdot \|\mathbf{r}'\|} = \frac{\|\mathbf{r}' \times \mathbf{r}''\|}{\|\mathbf{r}'\|^3}$$

这个推导巧妙地利用了 $\mathbf{r}''$ 在 $\mathbf{T}$ 方向(切向加速度)和 $\mathbf{N}$ 方向(法向加速度)的分解。外积操作自动过滤掉了不改变方向的切向部分,只留下了反映“弯曲”的法向部分,从而直接提取出了曲率。




$\mathbf{T}$, $\mathbf{N}$, $\mathbf{B}$ 求解步骤:

第一步:切向量 $\mathbf{T}$ (Tangent)

对曲线 $\mathbf{r}(t)$ 求导并单位化:
$$\mathbf{T} = \frac{\mathbf{r}'(t)}{\|\mathbf{r}'(t)\|}$$

第二步:副法向量 $\mathbf{B}$ (Binormal)

在 3D 中,通常先求 $\mathbf{B}$ 反而更容易。利用加速度(二阶导)和速度的外积:
$$\mathbf{B} = \frac{\mathbf{r}'(t) \times \mathbf{r}''(t)}{\|\mathbf{r}'(t) \times \mathbf{r}''(t)\|}$$

原理: 速度和加速度构成的平面被称为密切平面,$\mathbf{B}$ 就是该平面的法向量。

第三步:主法向量 $\mathbf{N}$ (Normal)

有了 $\mathbf{T}$ 和 $\mathbf{B}$,$\mathbf{N}$ 可以通过叉乘直接得出,确保右手系:$$\mathbf{N} = \mathbf{B} \times \mathbf{T}$$

\begin{example} 抛物线 $y = x^2$ 的切向量 $\mathbf{T}$、法向量 $\mathbf{N}$ 和副法向量 $\mathbf{B}$。

    参数化方程为 $\mathbf{r}(x) = (x, x^2)$。
    
    为了方便计算空间中的 $B$ 向量,我们可以将其视为在 $z=0$ 平面上的三维曲线,即 $\mathbf{r}(x) = (x, x^2, 0)$。
    
    首先计算位置向量的一阶和二阶导数:
    
    $\mathbf{r}'(x) = (1, 2x, 0)$
    
    $\mathbf{r}''(x) = (0, 2, 0)$
    
    一阶导的模长:$\|\mathbf{r}'(x)\| = \sqrt{1 + (2x)^2} = \sqrt{1 + 4x^2}$

    切向量 $\mathbf{T}$ (Tangent)
    
    直接对一阶导进行单位化:
    $$\mathbf{T} = \frac{\mathbf{r}'(x)}{\|\mathbf{r}'(x)\|} = \left( \frac{1}{\sqrt{1+4x^2}}, \frac{2x}{\sqrt{1+4x^2}}, 0 \right)$$

    副法向量 $\mathbf{B}$ (Binormal)
    
    对于平面曲线,副法向量总是垂直于平面。我们通过叉乘计算:
    $$\mathbf{r}' \times \mathbf{r}'' = \begin{vmatrix} \mathbf{i} & \mathbf{j} & \mathbf{k} \\ 1 & 2x & 0 \\ 0 & 2 & 0 \end{vmatrix} = (0, 0, 2)$$

    单位化后得到:
    $$\mathbf{B} = (0, 0, 1)$$
    
    (这说明抛物线始终在 $xy$ 平面内弯曲)

    主法向量 $\mathbf{N}$ (Normal)
    
    利用 $\mathbf{N} = \mathbf{B} \times \mathbf{T}$(满足右手系且指向凹侧):
    
    $$\mathbf{N} = \begin{vmatrix} \mathbf{i} & \mathbf{j} & \mathbf{k} \\ 0 & 0 & 1 \\ \frac{1}{\sqrt{1+4x^2}} & \frac{2x}{\sqrt{1+4x^2}} & 0 \end{vmatrix} = \left( -\frac{2x}{\sqrt{1+4x^2}}, \frac{1}{\sqrt{1+4x^2}}, 0 \right)$$

    计算曲率 $\kappa$ (Curvature)
    
    曲率公式为:$$\kappa = \frac{\|\mathbf{r}' \times \mathbf{r}''\|}{\|\mathbf{r}'\|^3}$$
    
    代入已知量:
    
    $\|\mathbf{r}' \times \mathbf{r}''\| = 2$
    
    $\|\mathbf{r}'\| = (1+4x^2)^{1/2}$

    得到:
    $$\kappa(x) = \frac{2}{(1 + 4x^2)^{3/2}}$$

    在顶点 $(0,0)$ 处: $\kappa = 2$。此时曲率最大,密切圆半径 $R = 1/\kappa = 0.5$。
    
    当 $x \to \infty$ 时: $\kappa \to 0$。这符合直觉,因为抛物线远端越来越趋于直线。

\end{example}