\chapter{常微分方程}
常微分方程(Ordinary Differential Equations, ODE)是数学中描述变化率的重要工具,广泛应用于物理、工程、生物等领域。

常微分方程涉及未知函数及其导数的方程,其中未知函数只有一个自变量(通常是时间或空间变量)。例如,$\frac{dy}{dx} = f(x, y)$ 就是一个一阶常微分方程。

阶数:方程中最高导数的阶数。一阶涉及一阶导数,二阶涉及二阶导数,以此类推。

线性 vs. 非线性:线性ODE的未知函数及其导数以线性形式出现(无乘积或幂次),如 $y'' + p(x)y' + q(x)y = r(x)$。

齐次 vs. 非齐次:齐次方程的右边为 $0$,非齐次有非零项。

通解 vs. 特解:通解包含任意常数,特解通过初始条件确定。



\section{一阶常微分方程}
一阶ODE的形式为 $\frac{dy}{dx} = f(x, y)$。常见类型包括可分离、线性、齐次等。

\medskip

\textbf{可分离方程}

如果方程可写成 $\frac{dy}{dx} = g(x) h(y)$,则可分离。

\medskip

求解步骤:

(1)分离变量

假设 $h(y) \neq 0$,将 $h(y)$ 除到等号左边,并将 $dx$ 乘到等号右边,得到微分形式:$$\frac{1}{h(y)} dy = g(x) dx$$
为了简化表示,我们通常用 $p(y) = \frac{1}{h(y)}$ 来表示左边只含 $y$ 的部分。方程变为:$$p(y) dy = g(x) dx$$

(2)积分

对等式的两边分别进行不定积分:$$\int p(y) dy = \int g(x) dx + C$$
其中 $C$ 是任意常数。

(3)得到通解

执行积分后,通常可以得到一个包含 $x, y$ 和常数 $C$ 的隐式解,即方程的通解。如果可能,也可以进一步解出 $y$ 关于 $x$ 的显式解。

\medskip

\begin{example} 求解 $\frac{dy}{dx} = \frac{y}{x}$(假设 $x \neq 0,\; y \neq 0$)。
    
    分离变量得 $\frac{dy}{y} = \frac{dx}{x}$。

    两边积分:
    \begin{align*}
        &\int \frac{dy}{y} = \int \frac{dx}{x} \\
        &\ln |y| = \ln |x| + C \\
        &e^{\ln |y|} = e^{\ln |x| + C} \\
        &|y| = e^C \cdot |x| \\
        &y = A x \qquad (A = \pm e^C,\quad \text{其中$A$ 为任意非零常数})
    \end{align*}
    
    方程通解:$y(x) = A x$。
\end{example}

这是一个比例增长模型,如人口增长率与人口成正比。

\begin{example} 求解微分方程:$\frac{dy}{dx} = x^2 y$。

    分离变量:假设 $y \neq 0$,将 $y$ 除到左边, $dx$ 乘到右边:$$\frac{1}{y} dy = x^2 dx$$

    积分:对两边分别积分:
    \begin{align*}
        &\int \frac{1}{y} \,dy = \int x^2 \,dx + C \\
        &\ln|y| = \frac{1}{3} x^3 + C
    \end{align*}

    得到通解(显式解):将通解转化为显式形式(解出 $y$):
    $$|y| = e^{\frac{1}{3} x^3 + C} = e^C e^{\frac{1}{3} x^3}$$

    令 $A = \pm e^C$($A$ 为任意非零常数),则:$$y = A e^{\frac{1}{3} x^3}$$

    如果考虑到 $y=0$ 也是原方程的一个解(即 $y \equiv 0$),并且 $y \equiv 0$ 可以包含在上述通解中(令 $A=0$),所以最终的通解为:
    $$y = A e^{\frac{1}{3} x^3} \quad (\text{其中 } A \text{ 是任意常数})$$
\end{example}

\medskip

\textbf{线性方程}

一阶线性微分方程的标准形式可以表示为:$$\frac{dy}{dx} + P(x) y = Q(x)$$
其中:

$y$ 是未知函数(因变量),它是 $x$ 的函数。

$\frac{dy}{dx}$ 是 $y$ 对 $x$ 的一阶导数。

$P(x)$ 和 $Q(x)$ 是只与自变量 $x$ 有关的已知函数(或常数)。

\medskip

求解步骤:

求解一阶线性微分方程的标准方法是利用积分因子法。

(1)确定积分因子 $I(x)$

积分因子 $I(x)$ 定义为:
\begin{align*}
    &I(x) = e^{\int P(x) \,dx} \\
    &I'(x) = e^{\int P(x) \,dx} \cdot \left(\int P(x) \,dx \right)' = I(x)P(x)
\end{align*}
注意: 在计算 $\int P(x) \,dx$ 时,可以不加任意常数 $C$,因为 $C$ 最终会被消去。

(2)求解过程

将标准形式的微分方程 $\frac{dy}{dx} + P(x) y = Q(x)$ 两边同乘以积分因子 $I(x)$:
$$I(x) \frac{dy}{dx} + I(x) P(x) y = I(x) Q(x)$$
关键在于,等式的左边正好是函数乘积 $I(x) y$ 的导数:
$$\frac{d}{dx} [I(x) y] = I(x) \frac{dy}{dx} + I(x) P(x) y$$
所以原方程变为:$$\frac{d}{dx} [I(x) y] = I(x) Q(x)$$

(3)积分并得到通解

对上式两边关于 $x$ 进行积分:$$I(x) y = \int I(x) Q(x) \,dx + C$$

其中 $C$ 是积分常数。

最后,解出 $y$,得到方程的通解:
$$y = \frac{1}{I(x)} \left[ \int I(x) Q(x) \,dx + C \right]$$

\medskip

\begin{example} 求解微分方程:$$\frac{dy}{dx} + \frac{2}{x} y = x$$

    确定 $P(x)$ 和 $Q(x)$:$P(x) = \frac{2}{x},\;Q(x) = x$

    计算积分因子 $I(x)$:
    \begin{align*}
        &\int P(x) \,dx = \int \frac{2}{x} \,dx = 2 \ln|x| = \ln(x^2) \\
        &I(x) = e^{\int P(x) dx} = e^{\ln(x^2)} = x^2
    \end{align*}

    应用公式或直接求解:将原方程乘以 $I(x) = x^2$:$$x^2 \frac{dy}{dx} + 2xy = x^3$$
    左边可以写成 $\frac{d}{dx} (x^2 y)$:$$\frac{d}{dx} (x^2 y) = x^3$$

    积分:
    \begin{align*}
        \int \frac{d}{dx} (x^2 y) \,dx &= \int x^3 \,dx \\
        x^2 y &= \frac{1}{4} x^4 + C
    \end{align*}

    得到通解:
    \begin{align*}
        &y = \frac{1}{x^2} \left( \frac{1}{4} x^4 + C \right) \\
        &y = \frac{1}{4} x^2 + \frac{C}{x^2}
    \end{align*}
\end{example}

一阶线性方程的求解公式是:
$$y = e^{-\int P(x) \,dx} \left[ \int Q(x) e^{\int P(x) \,dx} \,dx + C \right]$$
这个公式是通用且可靠的。

\medskip

\textbf{齐次方程}

一个一阶常微分方程可以写成:$$\frac{dy}{dx} = f(x, y)$$
如果函数 $f(x, y)$ 是一个零次齐次函数,它总是可以写成只依赖于 $\frac{y}{x}$ 的某个函数 $g\left(\frac{y}{x}\right)$。

因此,一阶齐次微分方程的标准形式可以写成:$$\frac{dy}{dx} = g\left(\frac{y}{x}\right)$$

一阶齐次微分方程通过代换 $u = \frac{y}{x}$ 转化为可分离变量方程来求解。

\medskip

求解步骤:

齐次微分方程的求解方法是利用一个巧妙的变量代换,将其转化为可分离变量方程来求解。

(1)变量代换

令一个新的变量 $u$ 为 $\frac{y}{x}$:$$u = \frac{y}{x} \quad \implies \quad y = ux$$

(2)转换微分

对 $y = ux$ 两边关于 $x$ 求导(使用乘积求导法则):
$$\frac{dy}{dx} = \frac{d}{dx} (ux) = \frac{du}{dx} x + u \frac{dx}{dx} = x \frac{du}{dx} + u$$

(3)代入原方程

将 $\frac{dy}{dx}$ 和 $\frac{y}{x}$ 代入原方程 $\frac{dy}{dx} = g\left(\frac{y}{x}\right)$:
$$x \frac{du}{dx} + u = g(u)$$

(4)转化为可分离变量方程

将 $u$ 移到等式右边:$$x \frac{du}{dx} = g(u) - u$$
假设 $g(u) - u \neq 0$,现在方程变成了关于 $x$ 和 $u$ 的可分离变量方程:
$$\frac{1}{g(u) - u} du = \frac{1}{x} dx$$

(5)积分求解

对两边积分:$$\int \frac{1}{g(u) - u} du = \int \frac{1}{x} dx + C$$
解出关于 $u$ 和 $x$ 的关系。

(6)还原变量

用 $u = \frac{y}{x}$ 替换回 $u$,得到原方程关于 $y$ 和 $x$ 的通解。

\medskip

\begin{example} 求解微分方程:$$\frac{dy}{dx} = \frac{y^2 + x^2}{xy}$$

    判断齐次性:
    $$\frac{dy}{dx} = \frac{y^2}{xy} + \frac{x^2}{xy} = \frac{y}{x} + \frac{x}{y}$$
    由于右边可以表示成 $\frac{y}{x}$ 的函数 $$g\left(\frac{y}{x}\right) = \frac{y}{x} + \frac{1}{\frac{y}{x}}$$
    所以是齐次方程。

    变量代换:令 $u = \frac{y}{x}$,则 $\frac{dy}{dx} = x \frac{du}{dx} + u$。

    代入方程:$$x \frac{du}{dx} + u = u + \frac{1}{u}$$

    分离变量:$u$ 被抵消:
    \begin{align*}
        x \frac{du}{dx} &= \frac{1}{u} \\
        u du &= \frac{1}{x} \,dx
    \end{align*}

    积分求解:
    \begin{align*}
        \int u \,du &= \int \frac{1}{x} \,dx + C \\
        \frac{1}{2} u^2 &= \ln|x| + C
    \end{align*}    
    
    还原变量:用 $u = \frac{y}{x}$ 替换:
    \begin{align*}
        \frac{1}{2} \left(\frac{y}{x}\right)^2 = \ln|x| + C \\
        \frac{y^2}{2x^2} = \ln|x| + C \\
        y^2 = 2x^2 (\ln|x| + C)
    \end{align*}
\end{example}


\section{二阶线性常微分方程}
一个二阶线性常微分方程的一般形式可以表示为:$$y'' + P(x)y' + Q(x)y = R(x)$$

其中:

$x$ 是自变量(通常是时间 $t$ 或空间坐标)。

$y$ 是因变量,它是 $x$ 的函数,即 $y=y(x)$。

$y'$ 和 $y''$ 分别是 $y$ 对 $x$ 的一阶和二阶导数。

“二阶”指的是方程中出现的最高阶导数是二阶导数 $y''$。

齐次 (Homogeneous): 如果 $R(x) = 0$,方程为齐次方程。

非齐次 (Non-homogeneous): 如果 $R(x) \neq 0$,方程为非齐次方程。

\medskip

\textbf{齐次线性常系数方程}

方程的一般形式为:$$y'' + Py' + Qy = 0$$
其中 $P$ 和 $Q$ 都是常数。

求解这类方程的关键步骤是引入一个特征方程。我们假设方程有一个形式为 $y = e^{rx}$ 的解(其中 $r$ 是待定常数),代入原方程:
$$y' = re^{rx} \qquad y'' = r^2e^{rx}$$

代入原方程:$$r^2 e^{rx} + P (r e^{rx}) + Q (e^{rx}) = 0$$

因为 $e^{rx} \neq 0$,两边除以 $e^{rx}$,得到特征方程:$$r^2 + P r + Q = 0$$

解出这个一元二次代数方程的两个根 $r_1$ 和 $r_2$,它们被称为特征根。根据 $\Delta = P^2 - 4Q$ 的符号,特征根有三种情况,从而决定了微分方程的通解形式。

\medskip

三种通解情况

微分方程的通解 $y(x)$ 是由两个线性无关的基本解 $y_1(x)$ 和 $y_2(x)$ 的线性组合构成:$y(x) = C_1 y_1(x) + C_2 y_2(x)$(其中 $C_1, C_2$ 为任意常数)。

(1)两不相等实根 ($\Delta > 0$)

特征根: $r_1$ 和 $r_2$ 是两个不相等的实数。

基本解: $y_1(x) = e^{r_1 x}$ 和 $y_2(x) = e^{r_2 x}$

通解:$$y(x) = C_1 e^{r_1 x} + C_2 e^{r_2 x}$$

物理意义 (例如阻尼振动): 通常对应于过阻尼情况,系统衰减到平衡位置,不发生振荡。

(2)两相等实根(重根) ($\Delta = 0$)

特征根: $r_1 = r_2 = r$(二重实根)。

基本解: $y_1(x) = e^{r x}$ 和 $y_2(x) = x e^{r x}$

通解:$$y(x) = (C_1 + C_2 x) e^{r x}$$

物理意义 (例如阻尼振动): 对应于临界阻尼情况,系统以最快速度衰减到平衡位置,不发生振荡。

(3)共轭复根 ($\Delta < 0$)

特征根: $r_1, r_2 = \alpha \pm i\beta$ (其中 $\alpha = -P/2, \beta = \sqrt{4Q - P^2}/2, i^2=-1$)。

基本解: $y_1(x) = e^{\alpha x} \cos(\beta x)$ 和 $y_2(x) = e^{\alpha x} \sin(\beta x)$

通解:$$y(x) = e^{\alpha x} (C_1 \cos(\beta x) + C_2 \sin(\beta x))$$

物理意义 (例如阻尼振动): 对应于欠阻尼情况,系统在衰减过程中会发生振荡。

\medskip
\begin{example} 求解微分方程:$y'' + 4y' - 12y = 0$。

    首先,根据微分方程写出特征方程 $r^2 + P r + Q = 0$。
    
    这里 $P=4$,$Q=-12$,所以特征方程是:$$r^2 + 4r - 12 = 0$$

    使用因式分解或求根公式来求解 $r$:$$(r + 6)(r - 2) = 0$$
    
    得到两个特征根:$$r_1 = -6 \quad \text{和} \quad r_2 = 2$$

    由于我们得到了两不相等实根(即 情况 1),通解的形式为 $y(x) = C_1 e^{r_1 x} + C_2 e^{r_2 x}$。
    
    最终通解为:$$y(x) = C_1 e^{-6x} + C_2 e^{2x}$$

\end{example}

\begin{example} 求解微分方程:$y'' - 6y' + 9y = 0$。

    特征方程: $r^2 - 6r + 9 = 0$
    
    特征根: $(r - 3)^2 = 0 \implies r_1 = r_2 = 3$ (重根)
    
    通解 (情况 2):$$y(x) = (C_1 + C_2 x) e^{3x}$$
\end{example}

\begin{example} 求解微分方程:$y'' + 2y' + 5y = 0$。

    特征方程: $r^2 + 2r + 5 = 0$
    
    特征根 (使用求根公式 $r = \frac{-P \pm \sqrt{P^2 - 4Q}}{2}$):
    \begin{align*}
        r &= \frac{-2 \pm \sqrt{2^2 - 4(1)(5)}}{2} = \frac{-2 \pm \sqrt{4 - 20}}{2} \\
        &= \frac{-2 \pm \sqrt{-16}}{2} = \frac{-2 \pm 4i}{2} \\
        &= -1 \pm 2i
    \end{align*}
    即 $\alpha = -1$ 和 $\beta = 2$。

    通解 (情况 3):
    \begin{align*}
        y(x) &= e^{\alpha x} (C_1 \cos(\beta x) + C_2 \sin(\beta x)) \\
        y(x) &= e^{-x} (C_1 \cos(2x) + C_2 \sin(2x))
    \end{align*}
\end{example}

如果需要确定通解中的常数 $C_1$ 和 $C_2$,这通常需要初始条件或边界条件才能确定。

\medskip

\textbf{非齐次线性常系数方程}

非齐次二阶常微分方程的求解是建立在齐次方程求解基础之上的,因为它引入了外部驱动项(非齐次项)。

非齐次线性常微分方程的一般形式为:$$y'' + p(x)y' + q(x)y = r(x)$$

其中 $r(x)$ 是非齐次项(或驱动项),且 $r(x) \neq 0$。

\medskip

求解结构:通解的叠加原理

非齐次方程的通解 $y(x)$ 总是由两部分组成,即叠加原理:$$y(x) = y_h(x) + y_p(x)$$

(1)齐次通解 ($y_h$): 对应于 $r(x) = 0$ 时(即齐次方程)的通解。$$y'' + p(x)y' + q(x)y = 0$$
这一部分包含了两个任意常数 $C_1$ 和 $C_2$,我们已在之前的讨论中解决了常系数齐次方程的求解。

(2)特解 ($y_p$): 满足原非齐次方程的任意一个特定解。$$y_p'' + p(x)y_p' + q(x)y_p = r(x)$$
这一部分不包含任意常数。

求解非齐次方程的关键和挑战在于找到这个特解 $y_p(x)$。

\medskip

待定系数法求解特解 $y_p$

这种方法适用于非齐次项 $r(x)$ 具有特定形式(多项式、指数函数、正弦/余弦函数或它们的乘积)的情况。

核心思想: 猜测 $y_p(x)$ 的形式应与 $r(x)$ 的形式相似。

$$
\begin{array}{cc}
\text{\textbf{非齐次项 } r(x) \text{\textbf{ 的形式}} & y_p(x) \text{ \textbf{的初始猜测形式}} \\
\hline
A e^{ax} & K e^{ax} \\
A x^n & K_n x^n + K_{n-1} x^{n-1} + \dots + K_0 \\
A \cos(\omega x) \text{ 或 } B \sin(\omega x) & K_1 \cos(\omega x) + K_2 \sin(\omega x) \\
\text{乘积:例如 } x e^{ax} & \text{乘积:例如 } (K_1 x + K_0) e^{ax} \\
\end{array}
$$

修正规则 (当猜测形式与 $y_h$ 重复时):

如果 $y_p$ 的猜测形式的任何一项与 $y_h$ 中的某一项相同(即是齐次方程的解),则必须将猜测形式乘以 $x^s$,其中 $s$ 是使新的 $y_p$ 猜测形式中所有项都不是齐次方程解的最小非负整数(通常 $s=1$ 或 $s=2$)。

\medskip

\begin{example} 求解微分方程:$y'' - 3y' - 4y = 2\sin x$。

    第 1 步:求解齐次通解 ($y_h$)
    
    齐次方程: $y'' - 3y' - 4y = 0$
    
    特征方程: $r^2 - 3r - 4 = 0$
    
    特征根: $(r - 4)(r + 1) = 0 \implies r_1 = 4, r_2 = -1$
    
    齐次通解:$$y_h(x) = C_1 e^{4x} + C_2 e^{-x}$$

    第 2 步:求解特解 ($y_p$)
    
    非齐次项 $r(x) = 2 \sin(x)$。 根据表格,猜测特解形式为:$$y_p = A \cos(x) + B \sin(x)$$
    (注意:由于 $y_h$ 中不包含 $\cos(x)$ 或 $\sin(x)$ 项,不需要进行修正。)
    
    求导:
    \begin{align*}
        y_p' &= -A \sin(x) + B \cos(x) \\
        y_p'' = -A \cos(x) - B \sin(x)
    \end{align*}
    代入原方程 $y'' - 3y' - 4y = 2 \sin(x)$:
    \begin{align*}
        &(-A \cos x - B \sin x) - 3(-A \sin x + B \cos x) - 4(A \cos x + B \sin x) = 2 \sin x \\
        &(-A - 3B -4A)\cos x +(-B + 3A - 4B)\sin x = 2 \sin x \\
        &(-5A - 3B)\cos x + (3A - 5B)\sin x = 2 \sin x
    \end{align*}
    系数对比:
    
    $\cos(x)$ 系数: $-5A - 3B = 0$
    
    $\sin(x)$ 系数: $3A - 5B = 2$

    解线性方程组:$A = \frac{3}{17},\; B = -\frac{5}{17}$

    特解:$$y_p(x) = \frac{3}{17} \cos(x) - \frac{5}{17} \sin(x)$$

    将 $y_h$ 和 $y_p$ 相加写出通解:
    $$y(x) = C_1 e^{4x} + C_2 e^{-x} + \frac{3}{17} \cos(x) - \frac{5}{17} \sin(x)$$
\end{example}

当非齐次项 $r(x)$ 的形式与齐次通解 $y_h$ 中的某一项线性相关时,直接代入 $r(x)$ 的形式作为特解猜测会导致矛盾(即代入后左侧为零),因此必须应用修正规则。

\medskip

\begin{example} 求解微分方程:$y'' - 2y' - 3y = 5e^{3x}$。

    第 1 步:求解齐次通解 ($y_h$)
    
    齐次方程: $y'' - 2y' - 3y = 0$
    
    特征方程: $r^2 - 2r - 3 = 0$
    
    特征根: $(r - 3)(r + 1) = 0 \implies r_1 = 3, r_2 = -1$
    
    齐次通解:$$y_h(x) = C_1 e^{3x} + C_2 e^{-x}$$

    第 2 步:求解特解 ($y_p$)
    
    非齐次项 $r(x) = 5 e^{3x}$。
    
    初始猜测: 如果没有重合,我们会猜测 $y_p = A e^{3x}$。
    
    检查重合: 但是,我们的 $y_h$ 中包含 $C_1 e^{3x}$ 这一项。这意味着 $e^{3x}$ 是齐次方程的解,代入原方程左侧将得到 $0$,无法等于 $5 e^{3x}$。
    
    应用修正规则:由于 $e^{3x}$ 是齐次解,我们需要将初始猜测乘以 $x$,即:$$y_p = A x e^{3x}$$
    (因为 $x e^{3x}$ 不是齐次方程的解。)

    求导并代入:现在我们必须计算 $y_p'$ 和 $y_p''$:
    \begin{align*}
        y_p' &= A (1 \cdot e^{3x} + x \cdot 3 e^{3x}) = A e^{3x} (1 + 3x) \\
        y_p'' &= A [3 e^{3x} (1 + 3x) + e^{3x} (3)] = A e^{3x} (3 + 9x + 3) \\
        &= A e^{3x} (6 + 9x)
    \end{align*}

    将 $y_p, y_p', y_p''$ 代入原方程 $y'' - 2y' - 3y = 5 e^{3x}$:
    \begin{align*}
        &A e^{3x} (6 + 9x) - 2\cdot A e^{3x} (1 + 3x) - 3\cdot A x e^{3x} = 5 e^{3x} \\
        &A e^{3x} [ (6 + 9x) - 2(1 + 3x) - 3x ] = 5 e^{3x} \\
        &A e^{3x} [ 6 + 9x - 2 - 6x - 3x ] = 5 e^{3x} \\
        &4A e^{3x} = 5 e^{3x} \\
        &A = \frac{5}{4} 
    \end{align*}

    特解:$$y_p(x) = \frac{5}{4} x e^{3x}$$

    将 $y_h$ 和 $y_p$ 相加写出通解:
    $$y(x) = y_h(x) + y_p(x) = C_1 e^{3x} + C_2 e^{-x} + \frac{5}{4} x e^{3x}$$
\end{example}

单重根重合: $r(x) = 5 e^{3x}$ 与 $y_h$ 中的 $C_1 e^{3x}$ 重合,则 $s=1$,特解形式为 $A x e^{3x}$。

双重根(重根)重合: 如果齐次方程的特征根是重根 $r=3$,即 $y_h = C_1 e^{3x} + C_2 x e^{3x}$,而 $r(x) = 5 e^{3x}$。此时 $e^{3x}$ 和 $x e^{3x}$ 都是齐次解,则需乘以 $x^2$,特解形式为 $A x^2 e^{3x}$。