\chapter{常微分方程}
常微分方程(Ordinary Differential Equations, ODE)是数学中描述变化率的重要工具,广泛应用于物理、工程、生物等领域。

常微分方程涉及未知函数及其导数的方程,其中未知函数只有一个自变量(通常是时间或空间变量)。例如,$\frac{dy}{dx} = f(x, y)$ 就是一个一阶常微分方程。

阶数:方程中最高导数的阶数。一阶涉及一阶导数,二阶涉及二阶导数,以此类推。

线性 vs. 非线性:线性ODE的未知函数及其导数以线性形式出现(无乘积或幂次),如 $y'' + p(x)y' + q(x)y = r(x)$。

齐次 vs. 非齐次:齐次方程的右边为 $0$,非齐次有非零项。

通解 vs. 特解:通解包含任意常数,特解通过初始条件确定。



\section{一阶常微分方程}
一阶ODE的形式为 $\frac{dy}{dx} = f(x, y)$。常见类型包括可分离、线性、齐次等。

\medskip

\textbf{可分离方程}

如果方程可写成 $\frac{dy}{dx} = g(x) h(y)$,则可分离。

\medskip

求解步骤:

(1)分离变量

假设 $h(y) \neq 0$,将 $h(y)$ 除到等号左边,并将 $dx$ 乘到等号右边,得到微分形式:$$\frac{1}{h(y)} dy = g(x) dx$$
为了简化表示,我们通常用 $p(y) = \frac{1}{h(y)}$ 来表示左边只含 $y$ 的部分。方程变为:$$p(y) dy = g(x) dx$$

(2)积分

对等式的两边分别进行不定积分:$$\int p(y) dy = \int g(x) dx + C$$
其中 $C$ 是任意常数。

(3)得到通解

执行积分后,通常可以得到一个包含 $x, y$ 和常数 $C$ 的隐式解,即方程的通解。如果可能,也可以进一步解出 $y$ 关于 $x$ 的显式解。

\medskip

\begin{example} 求解 $\frac{dy}{dx} = \frac{y}{x}$(假设 $x \neq 0,\; y \neq 0$)。
    
    分离变量得 $\frac{dy}{y} = \frac{dx}{x}$。

    两边积分:
    \begin{align*}
        &\int \frac{dy}{y} = \int \frac{dx}{x} \\
        &\ln |y| = \ln |x| + C \\
        &e^{\ln |y|} = e^{\ln |x| + C} \\
        &|y| = e^C \cdot |x| \\
        &y = A x \qquad (A = \pm e^C,\quad \text{其中$A$ 为任意非零常数})
    \end{align*}
    
    方程通解:$y(x) = A x$。
\end{example}

这是一个比例增长模型,如人口增长率与人口成正比。

\begin{example} 求解微分方程:$\frac{dy}{dx} = x^2 y$。

    分离变量:假设 $y \neq 0$,将 $y$ 除到左边, $dx$ 乘到右边:$$\frac{1}{y} dy = x^2 dx$$

    积分:对两边分别积分:
    \begin{align*}
        &\int \frac{1}{y} \,dy = \int x^2 \,dx + C \\
        &\ln|y| = \frac{1}{3} x^3 + C
    \end{align*}

    得到通解(显式解):将通解转化为显式形式(解出 $y$):
    $$|y| = e^{\frac{1}{3} x^3 + C} = e^C e^{\frac{1}{3} x^3}$$

    令 $A = \pm e^C$($A$ 为任意非零常数),则:$$y = A e^{\frac{1}{3} x^3}$$

    如果考虑到 $y=0$ 也是原方程的一个解(即 $y \equiv 0$),并且 $y \equiv 0$ 可以包含在上述通解中(令 $A=0$),所以最终的通解为:
    $$y = A e^{\frac{1}{3} x^3} \quad (\text{其中 } A \text{ 是任意常数})$$
\end{example}

\medskip

\textbf{线性方程}

一阶线性微分方程的标准形式可以表示为:$$\frac{dy}{dx} + P(x) y = Q(x)$$
其中:

$y$ 是未知函数(因变量),它是 $x$ 的函数。

$\frac{dy}{dx}$ 是 $y$ 对 $x$ 的一阶导数。

$P(x)$ 和 $Q(x)$ 是只与自变量 $x$ 有关的已知函数(或常数)。

\medskip

求解步骤:

求解一阶线性微分方程的标准方法是利用积分因子法。

(1)确定积分因子 $I(x)$

积分因子 $I(x)$ 定义为:
\begin{align*}
    &I(x) = e^{\int P(x) \,dx} \\
    &I'(x) = e^{\int P(x) \,dx} \cdot \left(\int P(x) \,dx \right)' = I(x)P(x)
\end{align*}
注意: 在计算 $\int P(x) \,dx$ 时,可以不加任意常数 $C$,因为 $C$ 最终会被消去。

(2)求解过程

将标准形式的微分方程 $\frac{dy}{dx} + P(x) y = Q(x)$ 两边同乘以积分因子 $I(x)$:
$$I(x) \frac{dy}{dx} + I(x) P(x) y = I(x) Q(x)$$
关键在于,等式的左边正好是函数乘积 $I(x) y$ 的导数:
$$\frac{d}{dx} [I(x) y] = I(x) \frac{dy}{dx} + I(x) P(x) y$$
所以原方程变为:$$\frac{d}{dx} [I(x) y] = I(x) Q(x)$$

(3)积分并得到通解

对上式两边关于 $x$ 进行积分:$$I(x) y = \int I(x) Q(x) \,dx + C$$

其中 $C$ 是积分常数。

最后,解出 $y$,得到方程的通解:
$$y = \frac{1}{I(x)} \left[ \int I(x) Q(x) \,dx + C \right]$$

\medskip

\begin{example} 求解微分方程:$$\frac{dy}{dx} + \frac{2}{x} y = x$$

    确定 $P(x)$ 和 $Q(x)$:$P(x) = \frac{2}{x},\;Q(x) = x$

    计算积分因子 $I(x)$:
    \begin{align*}
        &\int P(x) \,dx = \int \frac{2}{x} \,dx = 2 \ln|x| = \ln(x^2) \\
        &I(x) = e^{\int P(x) dx} = e^{\ln(x^2)} = x^2
    \end{align*}

    应用公式或直接求解:将原方程乘以 $I(x) = x^2$:$$x^2 \frac{dy}{dx} + 2xy = x^3$$
    左边可以写成 $\frac{d}{dx} (x^2 y)$:$$\frac{d}{dx} (x^2 y) = x^3$$

    积分:
    \begin{align*}
        \int \frac{d}{dx} (x^2 y) \,dx &= \int x^3 \,dx \\
        x^2 y &= \frac{1}{4} x^4 + C
    \end{align*}

    得到通解:
    \begin{align*}
        &y = \frac{1}{x^2} \left( \frac{1}{4} x^4 + C \right) \\
        &y = \frac{1}{4} x^2 + \frac{C}{x^2}
    \end{align*}
\end{example}

一阶线性方程的求解公式是:
$$y = e^{-\int P(x) \,dx} \left[ \int Q(x) e^{\int P(x) \,dx} \,dx + C \right]$$
这个公式是通用且可靠的。

\medskip

\textbf{齐次方程}

一个一阶常微分方程可以写成:$$\frac{dy}{dx} = f(x, y)$$
如果函数 $f(x, y)$ 是一个零次齐次函数,它总是可以写成只依赖于 $\frac{y}{x}$ 的某个函数 $g\left(\frac{y}{x}\right)$。

因此,一阶齐次微分方程的标准形式可以写成:$$\frac{dy}{dx} = g\left(\frac{y}{x}\right)$$

一阶齐次微分方程通过代换 $u = \frac{y}{x}$ 转化为可分离变量方程来求解。

\medskip

求解步骤:

齐次微分方程的求解方法是利用一个巧妙的变量代换,将其转化为可分离变量方程来求解。

(1)变量代换

令一个新的变量 $u$ 为 $\frac{y}{x}$:$$u = \frac{y}{x} \quad \implies \quad y = ux$$

(2)转换微分

对 $y = ux$ 两边关于 $x$ 求导(使用乘积求导法则):
$$\frac{dy}{dx} = \frac{d}{dx} (ux) = \frac{du}{dx} x + u \frac{dx}{dx} = x \frac{du}{dx} + u$$

(3)代入原方程

将 $\frac{dy}{dx}$ 和 $\frac{y}{x}$ 代入原方程 $\frac{dy}{dx} = g\left(\frac{y}{x}\right)$:
$$x \frac{du}{dx} + u = g(u)$$

(4)转化为可分离变量方程

将 $u$ 移到等式右边:$$x \frac{du}{dx} = g(u) - u$$
假设 $g(u) - u \neq 0$,现在方程变成了关于 $x$ 和 $u$ 的可分离变量方程:
$$\frac{1}{g(u) - u} du = \frac{1}{x} dx$$

(5)积分求解

对两边积分:$$\int \frac{1}{g(u) - u} du = \int \frac{1}{x} dx + C$$
解出关于 $u$ 和 $x$ 的关系。

(6)还原变量

用 $u = \frac{y}{x}$ 替换回 $u$,得到原方程关于 $y$ 和 $x$ 的通解。

\medskip

\begin{example} 求解微分方程:$$\frac{dy}{dx} = \frac{y^2 + x^2}{xy}$$

    判断齐次性:
    $$\frac{dy}{dx} = \frac{y^2}{xy} + \frac{x^2}{xy} = \frac{y}{x} + \frac{x}{y}$$
    由于右边可以表示成 $\frac{y}{x}$ 的函数 $$g\left(\frac{y}{x}\right) = \frac{y}{x} + \frac{1}{\frac{y}{x}}$$
    所以是齐次方程。

    变量代换:令 $u = \frac{y}{x}$,则 $\frac{dy}{dx} = x \frac{du}{dx} + u$。

    代入方程:$$x \frac{du}{dx} + u = u + \frac{1}{u}$$

    分离变量:$u$ 被抵消:
    \begin{align*}
        x \frac{du}{dx} &= \frac{1}{u} \\
        u du &= \frac{1}{x} \,dx
    \end{align*}

    积分求解:
    \begin{align*}
        \int u \,du &= \int \frac{1}{x} \,dx + C \\
        \frac{1}{2} u^2 &= \ln|x| + C
    \end{align*}    
    
    还原变量:用 $u = \frac{y}{x}$ 替换:
    \begin{align*}
        \frac{1}{2} \left(\frac{y}{x}\right)^2 = \ln|x| + C \\
        \frac{y^2}{2x^2} = \ln|x| + C \\
        y^2 = 2x^2 (\ln|x| + C)
    \end{align*}
\end{example}
